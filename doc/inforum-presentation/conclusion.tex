\section{Conclusão e trabalho futuro}

\begin{frame}[fragile]{Conclusão e trabalho futuro}
  \textbf{Conclusão:}
  \begin{itemize}
  \item Linguagem concorrente e explicitamente tipificada
  \item Comunicação unicamente por troca de mensagens
  \item Canais síncronos descritos por tipos de sessão independentes do contexto
  \end{itemize}
  \textbf{Trabalho futuro:}
  \begin{itemize}
  \item Reduzir a verbosidade da linguagem
  \item Abreviar tipos: \lstinline{type SendInt = !Int}
  \item Inferência de tipos em alguns cenários (as aplicações de tipos \lstinline|e[T]|)
  \item Canais partilhados
  \item Operador de \lstinline|dualof|
  \end{itemize}
\end{frame}

\begin{frame}[fragile]{Motivação}
  \begin{tcolorbox}[colback=blue!5,colframe=blue!60!black,title=Facto,arc=2pt,outer arc=2pt]
    A linguagem produzida pela gramática que descreve um tipo de sessão é:
    \begin{itemize}
    \item Reconhecida por um autómato finito
    \item Uma linguagem ($\omega$-) regular
    \end{itemize}
  \end{tcolorbox}
  \begin{itemize}
  \item \textbf{Gramática das árvores é independente do contexto:} \lstinline"Tree ::= Leaf | Node int Tree Tree" 
  %\item Linguagem produzida pelo não terminal \lstinline|N| é independente do contexto
  \item \textbf{Consequência:} Os tipos de sessão tradicionais não podem descrever a transmissão destas estruturas 
  \item \textbf{Solução:} Tipos de sessão independentes do contexto propostos por Thiemann e Vasconcelos.
  \end{itemize}
\end{frame}

%%% Local Variables:
%%% mode: latex
%%% TeX-master: "cfst"
%%% End:
