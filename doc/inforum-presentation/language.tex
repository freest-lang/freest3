\section{A linguagem}

\begin{frame}[fragile]{A linguagem}

  \begin{itemize}
  \item Sintaxe semelhante à do Haskell
  \newline
  \item Acrescida de primitivas para
    \begin{itemize}
    \item Criação de canais
    \item Envio de valores nos canais
    \item Receção de valores nos canais
    \end{itemize}
 
  \item Primitivas de comunicação: troca de mensagens e escolhas
  \newline
  \item Canais de comunicação síncronos e bidirecionais
  \newline
  \item Os processos podem escrever numa das extremidades do canal ou ler na outra.
  \end{itemize}  
\end{frame}

\begin{frame}[fragile]{A linguagem}
  Os tipos disponíveis na linguagem são:
  \begin{figure}[h!]
  \begin{align*}
    \tcBase \grmeq & \inte \grmor \, \chare \grmor \, \boole \grmor \, \unite && \text{Tipos básicos}\\
    T \grmeq       & \tskip \grmor \tSemi{T}{T} \grmor \,\tOut{\tcBase} \grmor \,\tIn{\tcBase} && \text{Tipos}\\
    \grmor         & \tIChoice\{l_i\colon T_i\}_{i\in I} \grmor \tEChoice\{l_i\colon T_i\}_{i\in I} \\ 
    \grmor         & \tcBase \grmor \tUnFun{T}{T} \grmor \tLinFun{T}{T}\\   
    \grmor         & \tPair{T}{T} \grmor \tDatatype{l_i\colon T_i}_{i\in I} \grmor \tRec{\alpha}{T} \grmor \alpha\\
    \kindsch \grmeq & T \grmor \tForall{\alpha}{\kindsch}  && \text{Esquemas de tipos}
    % 
  \end{align*}
\end{figure}


%%% Local Variables:
%%% mode: latex
%%% TeX-master: "cfst"
%%% End:  
\end{frame}

\lstset{language=CFST, style=eclipse, frame=none, numbers=none}
\begin{frame}[fragile]{A linguagem - Exemplo}
  \begin{itemize}
  \item \textbf{Objetivo}: Transmitir uma árvore binária num canal
    \newline
  \item \textbf{Tipos de dados:}
\begin{lstlisting}
data Tree = Leaf | Node Int Tree Tree 
type TreeChannel =
  +{LeafC: Skip,
    NodeC: !Int; TreeChannel; TreeChannel}
\end{lstlisting}

  \item \textbf{Tipo da função que envia a árvore}
\begin{lstlisting}
sendTree :: forall a =>
  Tree -> (TreeChannel; a) -> a
\end{lstlisting}
  \end{itemize}
\end{frame}

\begin{frame}[fragile]{A linguagem - Envio de uma árvore}
  O código da função para enviar uma árvore é:
    \vskip 0.5cm
  \begin{lstlisting}
sendTree :: forall a =>
  Tree -> (TreeChannel; a) -> a
sendTree t c =
 case t of
   Leaf -> select LeafC c
   Node x l r ->
     let c1 = select NodeC c in
     let c2 = send x c1 in
     let c3 = sendTree[TreeChannel;a] l c2 in
     let c4 = sendTree[a] r c3 in
     c4

\end{lstlisting}
\end{frame}

\begin{frame}[fragile]{A linguagem - Receção de uma árvore}
  O código da função complementar que permite receber uma árvore é:
  \vskip 0.5cm
  \begin{lstlisting}
type TreeChannelR =
  &{LeafC: Skip, NodeC: ?Int; TreeChannelR ; TreeChannelR}
receiveTree :: forall a =>
  (TreeChannelR; a) -> (Tree, a)
receiveTree c =
  match c with
    LeafC c1 -> (Leaf, c1)
    NodeC c1 ->
      let x, c2 = receive c1 in
      let left, c3 = receiveTree [TreeChannelR;a] c2 in
      let right, c4 = receiveTree [a] c3 in
      (Node x left right, c4)
\end{lstlisting}
\end{frame}

\begin{frame}[fragile, shrink=20]{A linguagem - Sintaxe das expressões}
  \vskip 1cm
  A sintaxe das expressões disponíveis na linguagem é:
  \begin{figure}[ht]
  \begin{align*}
    e \grmeq & \unite \grmor \inte \grmor \chare \grmor \boole && \text{Expressões básicas}\\
    \grmor & \vare{x} \grmor \unlete{x}{e}{e} && \text{Variáveis}\\
    \grmor & \appe{e}{e} \grmor \tappe{e}{T} && \text{Applicações}\\
    \grmor & \conditionale{e}{e}{e} && \text{Condicional}\\
    \grmor & \paire{e}{e} \grmor \binlete{x}{y}{e}{e} && \text{Pares}\\
    %
    \grmor & \newe{T} \grmor \sende{e}{e} \grmor \recve{e} && \text{Operações de comunicação}\\
    \grmor & \selecte{e} \grmor \matche{e}{\{l_i\;\to\;e_i\}_{i\in I}} \\
    \grmor & \forke{e}  && \text{Fork}\\
    \grmor & \ctrcte \grmor \casee{e}{\{C_i\;\to\;e_i\}_{i\in I}} && \text{Tipos de dados}\\
    %
  \end{align*}
  \hrulefill
  \caption{Sintaxe das expressões}
  \label{fig:expressions}
\end{figure}


%%% Local Variables:
%%% mode: latex
%%% TeX-master: "cfst-inforum18"
%%% End:

\end{frame}

\begin{frame}[fragile]{A linguagem - Validação}
  \begin{itemize}
  \item Sistema de \textit{kinding}:
    \begin{itemize}
    \item Boa formação dos tipos
    \item Classifica os tipos nas categorias de tipos de sessão ou tipos gerais
    \item Associa multiplicidades (linear ou partilhado)
    \end{itemize}
  \item Exemplos:
    \begin{itemize}
    \item \lstinline|!Int|: Bem formado (tipo de sessão, linear) % $\kinds^\Lin$)
    \item \lstinline|(Int->Bool);Int| não é bem formado
    \item \lstinline|rec x.a;x|
      \begin{itemize}
      \item Mal formado se \lstinline|a| não estiver no ambiente de \textit{kinding}
      \item Mal formado se \lstinline|a| não for um tipo de sessão
      \end{itemize}
    \end{itemize}
    
  \item Verificação de tipos
  \end{itemize}
\end{frame}
  
\begin{frame}[fragile]{A linguagem - Geração de código}
  \begin{itemize}
  \item A linguagem alvo da geração de código é Haskell
  \item Quatro desafios do processo de tradução:
  \end{itemize}
  \begin{enumerate}
    \item \textit{Call-by-value} VS. \textit{Call-by-name}\\
      \textbf{Solução:} \textit{BangPatterns}\\      
      ex: \lstinline|fun x = e| quando traduzida fica \lstinline|fun !x = e|
   \newline
 \item Canais de comunicação implementadas através de duas \textit{MVar} por canal.
   \begin{itemize}
   \item \lstinline[Haskell]|putMVar| - Operação \lstinline|send|
   \item \lstinline[Haskell]|takeMVar| - Operação \lstinline|receive|
   \end{itemize}
  \end{enumerate}

\end{frame}

\begin{frame}[fragile]{A linguagem - Geração de código}
  \begin{enumerate}
    \setcounter{enumi}{2}    
  \item \textit{MVar} só têm um tipo que se mantém inalterado durante a computação
    \begin{itemize}
    \item Canais necessitam que o tipo possa variar (ex:
      \lstinline|!Int;?Bool| progride para \lstinline|?Bool| )
    \item Sistema de tipos do Haskell não pode verificar os tipos dos canais (\textit{unsafeCoerce})
    \end{itemize}
  \end{enumerate}
  \begin{lstlisting}[language=Haskell]

  _send x (m1, m2) = do
    putMVar m2 (unsafeCoerce x)
    return (m1, m2)

  _receive (m1, m2) = do
    a <- takeMVar m1
    return ((unsafeCoerce a), (m1, m2))
\end{lstlisting}
\end{frame}

\begin{frame}[fragile]{A linguagem - Geração de código}
  \begin{enumerate}
    \setcounter{enumi}{3}  
  \item Operações de comunicação (\lstinline|fork|, \lstinline|send|, \lstinline|receive| e \lstinline|new|) são operações sobre um monáde
    \begin{itemize}
    \item Quando traduzir uma expressão para código de um monáde?
    \item Anotação da árvore sintática com valores booleanos
    \end{itemize}
  \end{enumerate}
  Geramos código com base na seguinte tabela:
  \vskip 0.2cm
\resizebox{\textwidth}{!}{%
  \begin{tabular}[ht!]{| c | c | c |}
    \hline  
    \quad Valor esperado \quad&\quad Valor encontrado \quad&\quad Código gerado \quad\\
    \quad (anotação da árvore sintática) \quad&\quad (na função de tradução) \quad& (Haskell) \quad\\\hline
    \lstinline|False| & \lstinline|False| & \lstinline|e| \\
    \lstinline|True| & \lstinline|False| & \lstinline|return e| \\
    \lstinline|True| & \lstinline|True| & \lstinline|e| \\
    \lstinline|False| & \lstinline|True| & \lstinline|e >>= x -> x| \\
    \hline
  \end{tabular}}
\end{frame}

\begin{frame}[fragile]{A linguagem - Função traduzida}
  \textbf{Resultado:}
  \vskip 0.5cm
  \begin{lstlisting}[language=Haskell]
  sendTree !t !c =
    case t of 
      Leaf -> _send "LeafC" c 
      Node x l r ->
        _send "NodeC" c >>=
        \c1 -> _send x c1 >>=
        \c2 -> sendTree l c2 >>=
        \c3 -> sendTree r c3 >>=
        \c4 -> return c4 
   \end{lstlisting}
\end{frame}



%%% Local Variables:
%%% mode: latex
%%% TeX-master: "cfst"
%%% End:
