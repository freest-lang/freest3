\documentclass[unknownkeysallowed]{beamer}

\usepackage[compress]{beamerThemeLasige}
\usepackage[portuguese]{babel}
\usepackage{amsmath,amssymb,amsthm, MnSymbol}
\usepackage{mathtools}

\usepackage{listings,color}
\usepackage{alltt}
\usepackage{flushend}
\usepackage[utf8x]{inputenc} % for proper diacritics

\usepackage{graphicx}

\usepackage[T1]{fontenc}
\usepackage{microtype}

\usepackage{tikz}

\usepackage{times}

\usepackage{graphicx}

\usepackage{tcolorbox}
\usepackage{tabularx}

%%% Local Variables:
%%% mode: latex
%%% TeX-master: "cfst"
%%% End:

%%% Local Variables:
%%% mode: latex
%%% TeX-master: "cfst-inforum18"
%%% End:
      
 
% THEME
\newtcolorbox{mybox}{colback=orange!5!white,colframe=orange!75!black}
\newtcolorbox{myboxazul}{colback=teal!5!white,colframe=teal!75!black}

% TIKZ 
\usetikzlibrary{positioning}
\usetikzlibrary{shapes,arrows}
\usepgfplotslibrary{dateplot}
\tikzstyle{block} = [rectangle, draw, 
    text width=5cm, text centered, rounded corners, minimum height=4em]
\tikzstyle{block2} = [rectangle, draw, 
    text width=10cm, text centered, rounded corners, minimum height=4em]
\tikzstyle{line} = [draw, -latex']

\newenvironment<>{varblock}[2][.9\textwidth]{%
  \setlength{\textwidth}{#1}
  \begin{actionenv}#3%
    \def\insertblocktitle{#2}%
    \par%
    \usebeamertemplate{block begin}}
  {\par%
    \usebeamertemplate{block end}%
  \end{actionenv}}

\newenvironment{changemargin}[3]{%
\begin{list}{}{%
\setlength{\leftmargin}{#1}%
\setlength{\rightmargin}{#2}%
\setlength{\topmargin}{#3}%
}%
\item[]}
{\end{list}}

% session constructors
\newcommand{\intk}{\keyword{int}}
\newcommand{\skipk}{\keyword{skip}}

% The language
\newcommand{\freest}{\textsc{FreeST}}

% notes
\newcommand{\todo}[1]{[{\color{blue}\textbf{#1}}]}

% Keywords
\newcommand{\keyword}[1]{\mathsf{#1}}
\newcommand{\link}{\keyword{lin}}
\newcommand{\unk}{\keyword{un}}

% Kinds
\newcommand\prekind{\upsilon}
\newcommand{\stypes}{\mathcal S}
\newcommand\kinds{\stypes}
\newcommand{\types}{\mathcal T}
\newcommand\kindt{\types}
\newcommand\kindsch{\mathcal C}
\newcommand\kind{\kappa}

% Multiplicity
\newcommand\Un{\ensuremath{\mathbf{u}}} % \infty
\newcommand\Lin{\ensuremath{\mathbf{l}}} % 1 

% Grammars
\newcommand{\grmeq}{\; ::= \;}
\newcommand{\grmor}{\;\mid\;}

% type constructors
\newcommand\tcBase{B}
\newcommand\tcLolli\multimap
\newcommand\tcFun\to
\newcommand\tcBang{\mathop!}

% Keywords for types
\newcommand\kRec{\keyword{rec}}
\newcommand\kForall{\keyword{forall}}

% Types
\newcommand{\tskip}{\keyword{Skip}}
\newcommand\tSemi[2]{#1;#2}
\newcommand\tOut[1]{\tcBang#1}
\newcommand\tIn[1]{?#1}
\newcommand{\tMsg}[1]{\sharp{#1}}
\newcommand\tIChoice[1]{\oplus{#1}}
\newcommand\tEChoice[1]{\&{#1}}
\newcommand{\tChoice}[1]{\star{#1}}
\newcommand{\tData}[1]{[{#1}]}
\newcommand\tUnFun[2]{#1\tcFun#2}
\newcommand\tLinFun[2]{#1\tcLolli#2}
\newcommand\tPair[2]{(#1,\,#2)}
\newcommand\tDatatype[1]{{[#1]}}
\newcommand\tRec[2]{\kRec\,#1\,.\,#2}
%\newcommand\tForall[2]{\forall\,#1\,.\,#2}
%\newcommand\tForall[2]{\kForall\,#1\,=>\,#2}
\newcommand\tForall[2]{\forall\,#1\Rightarrow#2}
% Basic Types
\newcommand{\unite}{()}
\newcommand{\inte}{\keyword{Int}}
\newcommand{\chare}{\keyword{Char}}
\newcommand{\boole}{\keyword{Bool}}

\newcommand\tRecK[2]{\kRec\,#1\,.\,#2}
% Environments
\newcommand{\Empty}{\varepsilon}
\newcommand\emptyEnv{\Empty}
\newcommand\kindEnv{\Delta}
\newcommand\varEnv{\Gamma}

% Variables
\newcommand\vare[1]{#1}
\newcommand\unlete[3]{\keyword{let} \; #1 = #2 \; \keyword{in} \; #3} 

% Applications
\newcommand\appe[2]{#1#2}
\newcommand\tappe[2]{#1[#2]}

% Conditional
\newcommand\conditionale[3]{\keyword{if}\;#1\;\keyword{then}\;#2\;\keyword{else} \; #3}

% Goal
\newcommand\Goal{\vdash}

% Pairs
\newcommand\paire[2]{(#1,#2)}
\newcommand\binlete[4]{\keyword{let}\;#1, #2 = #3\;\keyword{in}\;#4}

% Session Types
\newcommand\newe[1]{\keyword{new}\;#1}
\newcommand\sende[2]{\keyword{send}\;#1\; #2}
\newcommand\sendce[1]{\keyword{send}\;#1}
\newcommand\recve[1]{\keyword{receive}\;#1}
\newcommand\selecte[2]{\keyword{select}\;#1\;{#2}}
\newcommand\matche[2]{\keyword{match}\;#1\;\keyword{with}\;#2}

% Fork
\newcommand\forke[1]{\keyword{fork}\;#1}

% Datatypes
\newcommand{\ctrcte}{C}
\newcommand\casee[2]{\keyword{case}\;#1\;\keyword{of}\;#2}

% Sequents
\newcommand{\isType}[3][\Delta]{{#1} \vdash {#2} : {#3}}
\newcommand{\algkindout}[3][\kindEnv]{{#1} \Alg {#2} \shortrightarrow{ #3}}
\newcommand{\algkindin}[3][\kindEnv]{{#1} \Alg {#2} \shortleftarrow {#3}}
\newcommand{\subkind}[2]{{#1} <: {#2}}
\newcommand{\algtypeout}[4][\kindEnv;\varEnv]{{#1} \Alg {#2} \shortrightarrow {#3};{#4}}
%\newcommand{\algtypein}[4][\kindEnv;\varEnv]{{#1} \Alg {#2}\colon {#3}\shortrightarrow {#4}}
\newcommand{\algtypein}[4][\kindEnv;\varEnv]{{#1} \Alg {#2}\shortleftarrow {#3}; {#4}}
\newcommand{\ctxequiv}[3][\kindEnv]{{#1} \vdash \Equiv{#2}{#3}}
\newcommand{\typeequiv}[3][\kindEnv]{{#1} \vdash \Equiv{#2}{#3}}
\newcommand{\isqualifier}[3][\kindEnv]{{#1} \vdash {#2}\colon{#3}}
\newcommand{\isLin}[2][\kindEnv]{\isqualifier[#1]{#2}\link}
\newcommand{\isUn}[2][\kindEnv]{\isqualifier[#1]{#2}\unk}
\newcommand{\contractive}[2][\kindEnv]{{#1} \vdash_{\textsf c} {#2}}
%\newcommand\Alg{\vdash_{\textsf a}}
\newcommand\Alg{\vdash}

% Operators
\newcommand\Extract[1]{\leadsto_{#1}}% \rightlsquigarrow}
\newcommand{\subs}[3]{[{#1}/{#2}]{#3}}
\newcommand\dual[1]{\overline{#1}}

% Predicates
%\newcommand\Equiv[2]{#1\,\thicksim\,#2}
\newcommand\Equiv[2]{#1\,\sim\,#2}

% Colour

\newcommand{\Blue}[1]{\textcolor{blue}{#1}}
\newcommand{\Red}[1]{\textcolor{red}{#1}}
\newcommand{\Brown}[1]{\textcolor{brown}{#1}}
\newcommand{\highlight}[1]{\Blue{#1}}

% ECLIPSE LOOK

\newcommand\Small{\small}
%\newcommand\Small{\fontsize{7.5}{8}\selectfont} 

\definecolor{darkviolet}{rgb}{0.5,0,0.4}
\definecolor{darkgreen}{rgb}{0,0.4,0.2} 
\definecolor{darkblue}{rgb}{0.1,0.1,0.9}
\definecolor{darkgrey}{rgb}{0.5,0.5,0.5}
\definecolor{lightblue}{rgb}{0.4,0.4,1}

\lstdefinestyle{eclipse}{
  breaklines=true,
  basicstyle=\sffamily\Small,
  emphstyle=\color{red}\bfseries, 
  keywordstyle=\color{darkviolet}\bfseries,
  commentstyle=\color{darkgreen},
  stringstyle=\color{darkblue},
  numberstyle=\color{darkgrey},%\lstfontfamily,
  emphstyle=\color{red},
  % get also javadoc style comments
  morecomment=[s][\color{lightblue}]{/**}{*/},
  %columns=fullflexible, %spaceflexible, %flexible, fullflexible             
  %  escapeinside=`',
  %  escapechar=@,
  showstringspaces=false,
  numbers=left,
  tabsize=2
}

\lstdefinestyle{eclipse-Haskell}{
  breaklines=true,
  basicstyle=\sffamily\Small,
  emphstyle=\color{red}\bfseries, 
  keywordstyle=\color{darkviolet}\bfseries,
  commentstyle=\color{darkgreen},
  stringstyle=\color{darkblue},
  emphstyle=\color{red},
  % get also javadoc style comments
  morecomment=[s][\color{lightblue}]{/**}{*/},
  %columns=fullflexible, %spaceflexible, %flexible, fullflexible             
  %  escapeinside=`',
  %  escapechar=@,
  showstringspaces=false,
  numbers=none,
  tabsize=2
}

\lstdefinelanguage{freest}{
  style=eclipse,
  morekeywords=[1]{Int, Char, Bool, Skip, type, dualof, forall, rec, let, in, if, then, else, new, send, receive, select, fork, case, of, data, match, with, True, False},
  sensitive=true,
  literate=
  {->}{$\rightarrow$}2
  {-o}{$\multimap$}2
  {=>}{$\Rightarrow$}2
  {alpha}{$\alpha$}1,
  breaklines=true,
  morecomment=[l]{--},%
  morecomment=[s]{{-}{-}},%
  morestring=[b]',%
  morestring=[b]",%
  morestring=[s]{`}{`},%
}

\lstset{
  language=freest,
  numbers=none
}
 
%%% Local Variables:
%%% mode: latex
%%% TeX-master: "main"
%%% End:


\title[A Programming Language with Context-Free Session Types]{A Programming Language with Context-Free Session Types}
\author{Bernardo Almeida and Vasco T. Vasconcelos}
\institute{LASIGE, Faculdade de Ciências, Universidade de Lisboa}
%\date{24 de Setembro de 2018}
\date{September 24th, 2018}

\begin{document}
\begin{frame}{\null}
  \titlepage 
\end{frame}

\begin{frame}[fragile]{Context}
  \begin{itemize}
  \item Static verification of concurrent programs that communicate through message passing
    %Static program verification where communication is achieved by message passing
%  \item Verificação estática da comunicação em programas concorrentes com troca de mensagens
    \newline
  \item Communication channels are described by session types
%  \item Canais de comunicação governados por tipos de sessão
  \end{itemize}
\end{frame}

\lstset{language=CFST, numbers=none}
\begin{frame}[fragile]{Motivation}

  \begin{itemize}
  \item How to transmit a list on a communication channel?
    %Transmitir uma lista num canal de comunicação
    \newline
    

    
\begin{lstlisting}  
data List = Nil | Cons Int List

type ListServer = &{
  Nil : end
  Cons : ?int . ListServer
}
\end{lstlisting}
  \end{itemize}
\end{frame}


\begin{frame}{Motivation}
  \begin{itemize}
    \item The sequence of operations on a channel is defined by a finite automaton
%    \item A sequência de operações num canal é definida por um autómato finito:
      \newline
      \usetikzlibrary{automata,positioning}
\tikzstyle{edge} = [draw,thick,->]
\begin{wrapfigure}{R}{0.35\textwidth}
%\begin{center}
  \begin{tikzpicture}[shorten >=1pt,node distance=2cm,on grid,auto] 
   \node[state,initial] (choice)   {$\&$}; 
   \node[state] (input) [above right=of choice] {$?$}; 
   \node[state,accepting] (end) [below right=of choice] {\textsf{end}}; 

   \path[edge] (choice) to[bend left] node {\textsf{Cons}} (input);
   \path[edge] (input) to[bend left] node {\textsf{int}} (choice);
   \path[edge] (choice) edge node {\textsf{Nil}} (end);
   
  \end{tikzpicture}
%\end{center}
\end{wrapfigure}

%%% Local Variables:
%%% mode: latex
%%% TeX-master: "cfst-inforum18"
%%% End:

    \item Regular expression: $(\textsf{\&Cons}\,\cdot\textsf{?Int})^*\cdot\textsf{\&Nil}$
      
  \end{itemize}

\end{frame}

\begin{frame}[fragile]{Goal}
  \begin{itemize}
  \item Type-safe streaming of a tree (on a single channel)?
  \newline
    \begin{center}
      \lstinline"data Tree = Leaf | Node Int Tree Tree"
    \end{center}
  \vspace{0.5cm}
  \item Serialize the tree into a stream of basic data
    \newline
  \item Send sequences of \lstinline|Node|, \lstinline|Leaf| and \lstinline|Int| values
    \newline
  \item And deserialize the result stream back into \lstinline|Tree|  
  \end{itemize}
\end{frame}

\begin{frame}[fragile]{Goal}
  \begin{itemize}
    \item Communication restricted to base types (\lstinline|Int|, \lstinline|Bool| and \lstinline|Char|)
    \newline
  \item First order session types
    \newline
  \item Session types must ensure that trees are well-formed\\
    \vspace{0.3cm}
    Example: \lstinline|Node 2 Leaf Node 5 Leaf Leaf|\\
    Non example: \lstinline|Node Leaf 2 Node Leaf|
    \newline
  \item \textbf{Context-Free session types}
  \end{itemize}
\end{frame}

% Language
\begin{frame}[fragile]{The language - FreeST}

  \begin{itemize}
  \item Haskell-like syntax
  \newline
  \item With primitives for:
    \begin{itemize}
    \item Creating channels
    \item Sending values on channels
    \item Receiving values on channels
    \item Forking new threads
    \end{itemize}
  \vspace{0.5cm}  
\item Communication through message passing
  \newline
\item Communication channels are synchronous and bidirectional 
  \end{itemize}  
\end{frame}

\begin{frame}[fragile]{Type Syntax}
  % \begin{figure}[t]
  \begin{align*}
    \tcBase \grmeq & \inte \grmor \, \chare \grmor \, \boole \grmor \, \unite  \grmor \, l && \text{Tipos básicos}\\
    T \grmeq       & \tskip \grmor \tChoice{T}{T} \grmor \tSemi{T}{T}  \grmor \,\tOut{\tcBase} \grmor \,\tIn{\tcBase} \grmor \tRec{x}{T} \grmor x && \text{Tipos} 
    % 
  \end{align*}
%   \hrulefill
%   \caption{Sintaxe dos tipos}
%   \label{fig:types}
% \end{figure}


%%% Local Variables:
%%% mode: latex
%%% TeX-master: "main"
%%% End:
  
\end{frame}


\begin{frame}[fragile]{Transmit a Tree on a channel}
  \begin{itemize}
  \item \textbf{Datatype:}
    \begin{lstlisting}
data Tree = Leaf | Node Int Tree Tree
      
type TreeChannel =
  +{LeafC: Skip,
    NodeC: !Int; TreeChannel; TreeChannel}
\end{lstlisting}
\vspace{0.3cm}
\item \textbf{Type of the sendTree function:}
  \lstinline|sendTree :: forall a => Tree -> (TreeChannel; a) -> a|
  \vspace{0.3cm}
\item \textbf{Top-level call:}\\
  \lstinline|start t c = ... sendTree[Skip] t c ...|
  \end{itemize}
\end{frame}

\begin{frame}[fragile]{Serialization of a Tree}
  \begin{lstlisting}
sendTree :: forall a => Tree -> (TreeChannel; a) -> a
sendTree t c =
 case t of
   Leaf -> select LeafC c
   Node x l r ->
     let c1 = select NodeC c in
                @\color{blue}{--c1:!Int;TreeChannel;TreeChannel;a}@
     let c2 = send x c1 in
                @\color{blue}{--c2:TreeChannel;TreeChannel;a}@
     let c3 = sendTree[TreeChannel;a] l c2 in
                @\color{blue}{--c3:TreeChannel;a}@
     let c4 = sendTree[a] r c3 in
                @\color{blue}{--c4: a}@
     c4
\end{lstlisting}
\end{frame}

\begin{frame}[fragile]{Deserialization of a Tree}
  \begin{lstlisting}
type TreeChannelR =
  &{LeafC: Skip,
    NodeC: ?Int ; TreeChannelR ; TreeChannelR}

recvTree :: forall a => (TreeChannelR;a) -> (Tree,a)
recvTree c =
 match c with
   LeafC c1 -> (Leaf, c1)
   NodeC c1 ->
     let x, c2 = receive c1 in
     let left, c3 = recvTree [TreeChannelR;a] c2 in
     let right, c4 = recvTree [a] c3 in
     (Node x left right, c4)
\end{lstlisting}
\end{frame}

\begin{frame}[fragile, shrink=20]{Expression Syntax}
  \begin{figure}[t]
  \begin{align*}
    e \grmeq & \unite \grmor x \grmor c \grmor \text{True} \grmor \text{False} && \text{Expressões básicas}\\
    \grmor & \vare{x} \grmor \unlete{x}{e}{e} && \text{Variáveis e let}\\
    \grmor & \appe{e}{e} \grmor \tappe{e}{T} && \text{Aplicações}\\
    \grmor & \conditionale{e}{e}{e} && \text{Condicional}\\
    \grmor & \paire{e}{e} \grmor \binlete{x}{y}{e}{e} && \text{Pares}\\
    %
    \grmor & \newe{T} \grmor \sende{e}{e} \grmor \recve{e} && \text{Operações de comunicação}\\
    \grmor & \selecte{e} \grmor \matche{e}{\{l_i\;\to\;e_i\}_{i\in I}} \\
    \grmor & \forke{e}  && \text{Fork}\\
    \grmor & \ctrcte \grmor \casee{e}{\{C_i\;\to\;e_i\}_{i\in I}} && \text{Tipos de dados}\\
    %
  \end{align*}
  \hrulefill
  \caption{Sintaxe das expressões}
  \label{fig:expressions}
\end{figure}


%%% Local Variables:
%%% mode: latex
%%% TeX-master: "cfst-inforum18"
%%% End:

\end{frame}

\begin{frame}[fragile]{Validation}
  \begin{itemize}
  \item Kinding system:
    \begin{itemize}
    \item Establishes what constitutes a valid type
    \item Distinguishes session types from general types
    \item Distinguishes linear from unrestricted types.
    \end{itemize}
  \vspace{0.3cm}  
  \item Examples:
    \begin{itemize}
    \item \lstinline|!Int|: Well-formed (linear session type)
    \item \lstinline|(Int->Bool);Int| not well-formed
    \item \lstinline|!Bool;x|
      \begin{itemize}
      \item Not well-formed if \lstinline|x| is not on the kinding environment $\kindEnv$
      \item Not well-formed if \lstinline|x| is not a session type
      \end{itemize}
    \end{itemize}
  \vspace{0.3cm}      
  \item Type checking
  \end{itemize}
\end{frame}

\begin{frame}[fragile]{Code Generation}
  \begin{itemize}
  \item Process that translates from FreeST to Haskell
  \item Four main challenges:
  \end{itemize}
  \begin{enumerate}
    \item \textit{Call-by-value} VS. \textit{Call-by-name}\\
      \textbf{Solution:} \textit{BangPatterns}\\      
      ex: $\llbracket$\lstinline|fun x = e|$\rrbracket \Rightarrow$ \lstinline|fun !x =| $\llbracket$\lstinline|e|$\rrbracket$ 
    \newline
    \item Communication channels were implemented with two \textit{MVar}      
   \begin{itemize}
   \item \lstinline[Haskell]|putMVar| -- \lstinline|send|
   \item \lstinline[Haskell]|takeMVar| -- \lstinline|receive|
   \end{itemize}
  \end{enumerate}
\end{frame}

\lstset{language=Haskell, style= eclipse}
\begin{frame}[fragile]{Code Generation}
  \begin{enumerate}
  \setcounter{enumi}{2}    
  \item An \lstinline|MVar t| is mutable location that is either empty or contains a value of type t (doesn't change during the computation)
% As \textit{MVar} Haskell só têm um tipo que se mantém inalterado durante a computação
    \begin{itemize}
    \item Channels require type \lstinline|t| to change (eg: \lstinline|!Int;?Bool| changes to \lstinline|?Bool|)
      %Canais necessitam que o tipo possa variar (ex:
      %\lstinline|!Int;?Bool| progride para \lstinline|?Bool| )
    \item Haskell type system can't verify the channel types (\textit{unsafeCoerce} primitive)
      %Sistema de tipos do Haskell não pode verificar os tipos dos canais (\textit{unsafeCoerce})
    \end{itemize}
  \end{enumerate}
  \pause
  \begin{columns}[onlytextwidth,T]
  \begin{column}{0.6\textwidth}
    \begin{lstlisting}
_new = do
  m1 <- newEmptyMVar
  m2 <- newEmptyMVar
  return ((m1, m2),(m2, m1))

_receive c = do
  a <- takeMVar (fst c)
  return (unsafeCoerce a, c)
    \end{lstlisting}
  \end{column}
  \begin{column}{0.4\textwidth}
    \begin{lstlisting}
_send x c = do
  putMVar (snd c)
    (unsafeCoerce x)
  return c
    \end{lstlisting}
  \end{column}
  \end{columns}
\end{frame}

\lstset{language=CFST, style=eclipse}
\begin{frame}[fragile]{Code Generation}
  \begin{enumerate}
    \setcounter{enumi}{3}  
  \item \lstinline|fork|, \lstinline|send|, \lstinline|receive| and \lstinline|new| are monad operations (IO)
    \begin{itemize}
    \item When should we translate an expression to monadic code?
      % Quando traduzir uma expressão para código de um mónade?
    \item Annotate the AST with boolean values
      %Anotação da árvore sintática com valores booleanos
    \end{itemize}
  \end{enumerate}
  Code generation is based on the following table:
%  Geramos código com base na seguinte tabela:
  \vskip 0.2cm
\resizebox{\textwidth}{!}{%
  \begin{tabular}[ht!]{| c | c | c |}
    \hline  
    \quad Expected value \quad&\quad Found value \quad&\quad Generated code \quad\\
    \quad (AST annotation) \quad&\quad (translation function) \quad& (Haskell) \quad\\\hline
    \lstinline|False| & \lstinline|False| & \lstinline|e| \\
    \lstinline|True| & \lstinline|False| & \lstinline[language=Haskell]|return e| \\
    \lstinline|True| & \lstinline|True| & \lstinline|e| \\
    \lstinline|False| & \lstinline|True| & \lstinline|e >>= x -> x| \\
    \hline
  \end{tabular}}
\end{frame}

\lstset{language=Haskell, style=eclipse}
\begin{frame}[fragile]{Translation result}
  \textbf{Haskell code:}
  \vskip 0.3cm
  \begin{lstlisting}
  sendTree !t !c =
    case t of 
      Leaf -> _send "LeafC" c 
      Node x l r ->
        _send "NodeC" c >>=
        \c1 -> _send x c1 >>=
        \c2 -> sendTree l c2 >>=
        \c3 -> sendTree r c3 >>=
        \c4 -> return c4 
      \end{lstlisting}
 \end{frame}

 \lstset{language=CFST, style=eclipse}
 \begin{frame}[fragile]{Conclusion and future work}
  \textbf{Conclusion:}
  \begin{itemize}
  \item Concurrent typed language
  \item Functional language (syntax inspired in Haskell)
  \item Communication only by message passing
  \item Synchronous channels described by \textbf{Context-Free Session Types}
  \end{itemize}
  \pause
  \textbf{Future work:}
  \begin{itemize}
  \item Implement type abbreviation: \lstinline{type SendInt = !Int}
  \item Type inference in some scenarios (eg: type applications \lstinline|e[T]|)
  \item Shared channels
  \item Introduce the \lstinline|dualof| operator
  \end{itemize}
\end{frame}


\end{document}

%%% Local Variables:
%%% mode: latex
%%% TeX-master: t
%%% End:


% \begin{column}{0.5\textwidth}
% \end{column}
% 
