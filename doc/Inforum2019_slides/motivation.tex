\lstset{language=freest, numbers=none, escapeinside=||}

\begin{frame}[fragile]{Motivação}
%\vspace*{5mm}

  Os tipos de sessão:
  \begin{itemize}
  \item Foram propostos para responder à necessidade de formalização de trocas de mensagens.
    \pause
  \item Permitem definir protocolos na forma de tipos que representam interações corretas do sistema.
    \pause
  \item Garantem propriedades tais como a inexistência de erros na comunicação e de situações de impasse.
  \end{itemize}
\end{frame}

\begin{frame}[fragile]{Motivação}
  Contudo, a expressividade dos tipos de sessão está ainda aquém de todos os desafios de comunicação que encontramos em sistemas concorrentes:
  \vspace{3mm}
  \begin{itemize}
  \item Não permitem, em particular, misturar \textit{input} e \textit{output} na mesma escolha.
  \end{itemize}
\end{frame}

\begin{frame}[fragile]{Exemplo -- Introdução}
%\vspace*{5mm}

  Considere-se o seguinte problema:
  
  Dado $k \in \mathbb{Z}$, qual o maior inteiro $n$ tal que, $\sum_{i=1}^{n} i < k$?
  \pause
  
  Imaginem-se dois processadores:
  \begin{itemize}
  \item Um capaz de gerar sequências de números inteiros, mas sem a capacidade de os somar.
    \pause
  \item Outro capaz de somar números inteiros, mas incapaz de os gerar.
  \end{itemize}  
\end{frame}

% \begin{frame}{Exemplo -- Considerações}
%   \begin{itemize}
%   \item Ambos os processadores estão aptos para comunicar através de canais.
%     \vspace*{2mm}
%     \pause
%   \item Este exemplo simplista retrata o caso em que a interação entre os processos se dá necessariamente nos dois sentidos.
%     \vspace*{5mm}
%     \pause
%   \item Os dois processos deverão estar corretamente tipados e responder às necessidades da comunicação.
%   \end{itemize}
% \end{frame}

\begin{frame}{Exemplo -- Solução \hfill \color{mLightBrown}Produtor}

  Tirando partido das restrições computacionais dos dois processadores, podemos tentar resolver o problema do seguinte modo:\\
  \pause
  \vspace*{3mm}
 O processo \textbf{produtor}:
  \begin{enumerate}
  \item Gera a sequência de números: 1,2,3,...
    \pause
  \item Envia-a num canal de comunicação.
    \pause
  \item Pára quando receber, no mesmo canal, uma notificação para terminar o envio.
    \pause
  \item A notificação vem na forma de uma marca: \textbf{EOS} (\textit{end-of-stream})
  \end{enumerate}

\end{frame}


\begin{frame}{Exemplo -- Solução \hfill \color{mLightBrown}Consumidor}

  O outro processo, o \textbf{consumidor}:
  \vspace{2mm}
  \begin{enumerate}
  \item Vai somando os números que recebe enquanto a sua soma for menor do que $n$.
    \pause
    \vspace{5mm}
  \item Neste ponto, envia a marca \textbf{EOS}.
        \pause
    \vspace{5mm}
  \item Devolve $n-1$.
  \end{enumerate}
\end{frame}

\begin{frame}[fragile]{Exemplo -- Implementação}
%  \vspace*{-1cm}
  O canal de comunicação, do lado do \textbf{produtor}, tem o tipo:
  \vspace*{4mm}
  \begin{lstlisting}
        type CanalInt = |\tikzmark{ineos}|?EOS + |\tikzmark{sendInt}|!Int;CanalInt|\tikzmark{recCall}|
\end{lstlisting} 
 
\begin{tikzpicture}[
  remember picture,
  overlay,
  expl/.style={draw=orange,fill=orange!30,rounded corners, text width=4cm},
  arrow/.style={red!80!black,ultra thick,->,>=latex}
]
\node<2-2>[expl] 
  (eosExpl) 
  at (3,-1cm)
  {Leitura (?) da marca \textbf{EOS}};

\node<3-4>[expl] 
  (sendIntExpl) 
  at (2.5,-1cm)
  {Escrita (!) de um inteiro};

\node<4-4>[expl] 
  (recCallExpl) 
  at (7.83,-1cm)
  {O protocolo volta ``ao início''};    

\draw<2-2>[arrow]
  (eosExpl) to[out=70,in=250] ([xshift=3ex,yshift=-1ex]pic cs:eos);  

\draw<3-4>[arrow]
(sendIntExpl) to[out=80,in=255] ([xshift=3ex,yshift=-1ex]pic cs:sendInt);

\draw<4-4>[arrow]
  (recCallExpl) to[out=90,in=270] ([xshift=-4.8ex,yshift=-1ex]pic cs:recCall);    
    
\end{tikzpicture}


%%% Local Variables:
%%% mode: latex
%%% TeX-master: "main"
%%% End:
 

\onslide<5-> Do lado do \textbf{consumidor} o canal tem o tipo obtido por troca das operações de leitura pelas de escrita, e vice-versa.
\vspace*{4mm} 
\onslide<6->
\begin{lstlisting}
|\hspace{0.1\linewidth}|type CanalIntDual = |\tikzmark{out}|!EOS + |\tikzmark{in}|?Int;CanalIntDual
\end{lstlisting}
\begin{tikzpicture}[
  remember picture,
  overlay,
  expl/.style={},
  arrow/.style={red!80!black,ultra thick,->,>=latex}
]
\node<7-7>[expl] 
  (outExpl) 
  at (2.9,-0.2cm)
  {};

\node<8-8>[expl] 
  (inExpl) 
  at (4.9,-0.2cm)
  {};

\draw<7-7>[arrow, draw=orange,ultra thick]
  (outExpl) to ([xshift=0.5ex,yshift=-0.5ex]pic cs:out);  

\draw<8-8>[arrow, draw=orange,ultra thick]
(inExpl) to ([xshift=0.6ex,yshift=-0.5ex]pic cs:in);

\end{tikzpicture}



%%% Local Variables:
%%% mode: latex
%%% TeX-master: "main"
%%% End:


\onslide<9->
\centering
\begin{tcolorbox}
  O tipo acima é \textbf{dual} de \lstinline|CanalInt|.

  Abreviamos esse tipo para \lstinline|dualof CanalInt|.
\end{tcolorbox}

\end{frame}

\begin{frame}[fragile]{Exemplo -- Implementação  \hfill \color{mLightBrown}Produtor}
  % \vspace*{-1cm}
  \begin{lstlisting}[xrightmargin=.15\textwidth]
type CanalInt = ?EOS + !Int;CanalInt

    
produtor : CanalInt -> Int -> ()
produtor c n = |\tikzmark{choice}|choice {
  |\tikzmark{chan}|(c, |\tikzmark{pattern}|EOS) = |\tikzmark{receive}|receive -> (),
         c = |\tikzmark{send}|send c n -> produtor c (n+1)
}
  \end{lstlisting}

\begin{tikzpicture}[
  remember picture,
  overlay,
  expl/.style={draw=orange,fill=orange!30,rounded corners},
  arrow/.style={red!80!black,ultra thick,->,>=latex}
  ]

  \node<2-2>[expl] (choiceExpl) at (3,-0.5cm) {Manipulação de escolhas};

  \node<3-3>[expl] (receiveExpl) at (8,-1cm) {Receção da marca \textbf{EOS}};  

  \node<5-5>[expl] (chanExpl) at (5,-1cm)  {Canal de continução};
  \node<4-4>[expl] (patternExpl) at (5,-1cm) {Permite \textit{pattern-matching}};
%  {\makecell[c]{A receção das marcas\\permite \textit{pattern-matching}}};

  \node<6-6>[expl] (sendExpl) at (5,-1cm) {Envio do inteiro \lstinline|n|};  

  \draw<2-2>[arrow, draw=orange,ultra thick] (choiceExpl)
  to ([xshift=3ex,yshift=-1ex]pic cs:choice);

  \draw<3-3>[arrow, draw=orange,ultra thick] (receiveExpl)
  to ([xshift=3ex,yshift=-1ex]pic cs:receive);

  \draw<4-4>[draw=orange,ultra thick] (pic cs:pattern) [xshift=0.35cm,yshift=.1cm]
  ellipse (0.45cm and 0.3cm);
   
  \draw<4-4>[arrow, draw=orange,ultra thick] (patternExpl)
  to ([xshift=3ex,yshift=-1ex]pic cs:pattern);

  \draw<5-5>[arrow, draw=orange,ultra thick] (chanExpl)
  to ([xshift=3ex,yshift=-1ex]pic cs:chan);
  
  \draw<6-6>[arrow, draw=orange,ultra thick] (sendExpl)
  to ([xshift=3ex,yshift=-1ex]pic cs:send);
\end{tikzpicture}  



%%% Local Variables:
%%% mode: latex
%%% TeX-master: "main"
%%% End:


\end{frame}


\begin{frame}[fragile]{Exemplo -- Implementação  \hfill \color{mLightBrown}Consumidor}
  % \vspace*{-1cm}
  \begin{lstlisting}[xleftmargin=-.04\textwidth]
consumidor : dualof CanalInt -> Int -> Int -> Int -> Int
consumidor c s n k =
  if n >= s
  then choice {
    c = |\tikzmark{eos}|send EOS -> n|$-$|1
  }
  else choice {
    (c, m) = |\tikzmark{rcv}|receive c -> consumidor c (s+m) (n+1) k
  }
\end{lstlisting}

\begin{tikzpicture}[
  remember picture,
  overlay,
  expl/.style={draw=orange,fill=orange!30,rounded corners},
  arrow/.style={red!80!black,ultra thick,->,>=latex}
]
  \node<2-2>[expl] (eosExpl) at (3,-0.5cm) {Envio da marca \textbf{EOS}};
  \draw<2-2>[arrow, draw=orange,ultra thick] (eosExpl) to ([xshift=3ex,yshift=-1ex] pic cs:eos);

  \node<3-3>[expl] (rcvExpl) at (3,-0.5cm) {Receção do número inteiro};
  \draw<3-3>[arrow, draw=orange,ultra thick] (rcvExpl) to ([xshift=3ex,yshift=-1ex] pic cs:rcv);
\end{tikzpicture}


%%% Local Variables:
%%% mode: latex
%%% TeX-master: "main"
%%% End:

 
\onslide<4-4>
% %\centering
\begin{tcolorbox}
  As escolhas dos ramos \lstinline|then| e \lstinline|else| são degeneradas (constituídas apenas por um componente).\\
  Poderíamos, neste caso, eliminar a palavra reservada \lstinline|choice|.
\end{tcolorbox}

\end{frame}

\begin{frame}[fragile]{Exemplo -- Implementação  \hfill \color{mLightBrown}Programa principal}
  % \vspace*{-1cm}
  \begin{lstlisting}[xleftmargin=.2\textwidth, xrightmargin=.15\textwidth]
main : Int
main = let
  k = 1000
  (p, c) = |\tikzmark{new}|new CanalInt in
  |\tikzmark{fork}|fork produtor p 1;
  consumidor c 0 0 k|\tikzmark{consumer}|
\end{lstlisting}
\begin{tikzpicture}[
  remember picture,
  overlay,
  expl/.style={draw=orange,fill=orange!30,rounded corners},
  arrow/.style={red!80!black,ultra thick,->,>=latex}
  ]

  \node<2-2>[expl] (newExpl) at (9.2,1.9cm) {Cria um novo canal};
  \draw<2-2>[arrow, draw=orange,ultra thick] (newExpl)
  to[bend right] ([xshift=2.5ex,yshift=1.5ex]pic cs:new);

  \node<3-3>[expl] (forkExpl) at (2,-1cm) {Cria uma nova \textit{thread}};
  \draw<3-3>[arrow, draw=orange,ultra thick] (forkExpl)
  to[bend left, in=90] ([xshift=-0.2ex, yshift=0.3ex]pic cs:fork);

  
  \node<4-4>[expl] (consumerExpl) at (7,-1cm) {\makecell[c]{Resultado do programa é o\\ resultado do processo consumidor}};
  \draw<4-4>[arrow, draw=orange,ultra thick] (consumerExpl)
  to ([xshift=1ex, yshift=0.8ex]pic cs:consumer);
 \end{tikzpicture}  



%%% Local Variables:
%%% mode: latex
%%% TeX-master: "main"
%%% End:

\onslide<5->
\begin{verbatim}
$ freestMC example.mc
$ runhaskell example.hs
44
\end{verbatim}


\end{frame}

% <- 0
% v 90
% -> 180
% ^ 270

%%% Local Variables:
%%% mode: latex
%%% TeX-master: "main"
%%% End:
 
