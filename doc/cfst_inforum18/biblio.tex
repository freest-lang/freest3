\bibliographystyle{splncs04}
\begin{thebibliography}{8}
% \bibitem{ref_article1}
% Author, F.: Article title. Journal \textbf{2}(5), 99--110 (2016)

\bibitem{ref-cfst}
  Thiemann, P., Vasconcelos, V.T.: Context-free Session Types. In: ICFP 2016, vol. 51, pp. 462--475. ACM (2016). \doi{10.1145/3022670.2951926}

\bibitem{ref-lang-primitives}
  Honda K., Vasconcelos V.T., Kubo M.: Language primitives and type discipline for structured communication-based programming. In: Programming Languages and Systems. ESOP 1998. Lecture Notes in Computer Science, vol 1381. Springer-Verlag, Heidelberg. \doi{10.1007/BFb0053567}

\bibitem{ref-sessions-pi}  
  Gay, S., Hole, M.: Subtyping for session types in the pi calculus. In: Acta Informatica (2005)
  42, pp.191–225. \doi{10.1007/s00236-005-0177-z}

\bibitem{ref-go}
  The Go programming language, \url{http://golang.org/}.

\bibitem{ref-CSP-Hoare}
  Hoare C.A.R.: Communicating sequential processes. Commun. ACM 21, 8 (1978), 666-677. \doi{10.1145/359576.359585}
  
\bibitem{ref-sepi}
    Franco J., Vasconcelos V.T.: A Concurrent Programming Language with Refined Session Types. In: Software Engineering and Formal Methods. SEFM 2013. Lecture Notes in Computer Science, vol 8368, pp. 33--42. Springer. \doi{10.1007/978-3-319-05032-4\_2}

% \bibitem{ref-lin-ref-st}
%   Baltazar, P., Mostrous, D., Vasconcelos, V.T.: Linearly Refined Session Types. In Linearity'11, EPTCS, vol. 101, pp. 38--49
%   \doi{10.4204/EPTCS.101.4}

%   Linearly Refined Session Types. Pedro Baltazar, Dimitris Mostrous, and Vasco Thudichum Vasconcelos. In Linearity'11, volume 101 of EPTCS, pages 38-49. 2012. 
  
\bibitem{Hewitt:StructuresAsPatternsOfPassingMessages}
   Hewitt, C.: Viewing control structures as patterns of passing messages. In Artificial Intelligence, 1977, vol. 8, pp. 323--364
  \doi{10.1016/0004-3702(77)90033-9}

\bibitem{Hewitt:ActorFormalismForAI}
   Hewitt, C., Bishop, P., Steiger, R.: A Universal Modular ACTOR Formalism for Artificial Intelligence. In IJCAI, 1973, vol. 8, pp.35--245

  
% \bibitem{Agha:Actors}
% Agha, G.:Actors: A Model of Concurrent Computation in Distributed Systems. MIT Press (1986)
  
\bibitem{Armstrong:ErlangBook}
Armstrong, J.: Programming Erlang: Software for a Concurrent World. Pragmatic Bookshelf (2007)

\bibitem{Hoare:CSP}
  Hoare, C.A.R.: Communicating Sequential Processes. In Commun. ACM, pp. 100--106. ACM, 1983
\doi{10.1145/357980.358021}  

\bibitem{bisimilarity}
  Jančar P., Moller F. (1999) Techniques for Decidability and Undecidability of Bisimilarity. In: CONCUR '99. Lecture Notes in Computer Science, vol 1664. Springer, Berlin, Heidelberg
  \doi{10.1007/3-540-48320-9\_5}

\bibitem{decidable-CFP-bisimilarity}
  Christensen S., Hüttel H., Stirling C. (1992) Bisimulation equivalence is decidable for all context-free processes. In: CONCUR '92. Lecture Notes in Computer Science, vol 630. Springer.
  \doi{10.1007/BFb0084788}  

% \bibitem{ref_lncs1}
% Author, F., Author, S.: Title of a proceedings paper. In: Editor,
% F., Editor, S. (eds.) CONFERENCE 2016, LNCS, vol. 9999, pp. 1--13.
% Springer, Heidelberg (2016). \doi{10.10007/1234567890}
  
% \bibitem{ref_book1}
% Author, F., Author, S., Author, T.: Book title. 2nd edn. Publisher,
% Location (1999)

% \bibitem{ref_proc1}
% Author, A.-B.: Contribution title. In: 9th International Proceedings
% on Proceedings, pp. 1--2. Publisher, Location (2010)

% \bibitem{ref_url1}
% LNCS Homepage, \url{http://www.springer.com/lncs}. Last accessed 4
% Oct 2017
\end{thebibliography}


%%% Local Variables:
%%% mode: latex
%%% TeX-master: "cfst-inforum18"
%%% End:
