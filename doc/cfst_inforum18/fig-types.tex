
\begin{figure}[t]
  \begin{align*}
    \tcBase \grmeq & \inte \grmor \, \chare \grmor \, \boole \grmor \, \unite && \text{Tipos básicos}\\
    T \grmeq       & \tskip \grmor \tSemi{T}{T} \grmor \,\tOut{\tcBase} \grmor \,\tIn{\tcBase} && \text{Tipos}\\
    \grmor         & \tIChoice\{l_i\colon T_i\}_{i\in I} \grmor \tEChoice\{l_i\colon T_i\}_{i\in I} \\ 
    \grmor         & \tcBase \grmor \tUnFun{T}{T} \grmor \tLinFun{T}{T}\\   
    \grmor         & \tPair{T}{T} \grmor \tDatatype{l_i\colon T_i}_{i\in I} \grmor \tRec{x}{T} \grmor x\\
    \kindsch \grmeq & T \grmor \tForall{x}{\kindsch}  && \text{Esquemas de tipos}
    % 
  \end{align*}
  \hrulefill
  \caption{Sintaxe dos tipos}
  \label{fig:types}
\end{figure}


%%% Local Variables:
%%% mode: latex
%%% TeX-master: "cfst-inforum18"
%%% End:
