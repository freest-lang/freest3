
\section{Trabalho relacionado}
\label{sec:rel-work}

Muitas linguagens de programação e outros formalismos foram sendo propostos ao longo do tempo para endereçar os aspetos de comunicação no software. Dois modelos comuns para lidar com a comunicação entre componentes numa computação concorrente são a memória partilhada e a troca de mensagens.

Neste artigo focamo-nos no modelo de troca de mensagens, destacando os aspetos principais da programação baseada em canais de comunicação \ref{sec:prog-chan} e da programação baseada em atores \ref{sec:actors}.

\subsection{Programação baseada em canais}
\label{sec:prog-chan}

\subsubsection{Tipos de sessão}
\label{sec:session-types} Os tipos de sessão têm como principal objetivo enriquecer a expressividade que os tipos tradicionais fornecem, tornando possível estruturar interações complexas numa computação
concorrente onde são trocadas muitas mensagens em canais de tipos heterogéneos. Apareceram, em primeiro lugar, numa variante do cálculo pi apresentando uma distinção sintática entre canais lineares e canais partilhados \cite{ref-lang-primitives}. Mais tarde, Gay e Hole \cite{ref-sessions-pi} introduziram o conceito de sub-tipos para os tipos de sessão o que permitiu que as especificações de um protocolo fossem estendidas levando a descrições mais ricas das interações.

% Session types[12] are by now a well-established methodology for typed, message-passing concurrent computations. By assigning session types to communication channels, and by checking programs against session type systems, a number important program properties can be established, including the absence of races in channel manipulation operations, and the guarantee that channels are used as prescribed by their types. 

\subsubsection{Go}
\label{sec:go}
\lstset{language=Golang}
Hoare apresentou, como alternativa ao uso de memória partilhada a linguagem CSP (Communicating Sequential Processes) \cite{ref-CSP-Hoare}, onde apenas existe uma primitiva: a comunicação síncrona. Deste modo, os processos comunicam através de canais cuja operação de envio bloqueia até que o recetor leia a mensagem, fornecendo assim um mecanismo de sincronização.

Go ou \textit{golang} \cite{ref-go}, é uma linguagem de programação concorrente desenvolvida pela Google que recorre ao uso de canais para enviar variáveis partilhadas. O uso de canais permite não só que duas rotinas comuniquem entre si, mas também que sincronizem as suas execuções. Antes de um canal ser utilizado é necessário inicializá-lo e definir o tipo de dados que transporta. Por exemplo, a seguinte inicialização representa \lstinline"ch := make (chan int)" um canal de inteiros.

\subsubsection{SePi}
\label{sec:sepi}
\lstset{language=Sepi}
Franco e Vasconcelos \cite{ref-sepi} apresentaram uma linguagem concorrente baseada no cálculo pi monádico onde as interações são feitas recorrendo a tipos que resultam de uma combinação entre tipos de sessão e tipos linearmente refinados.

Nesta linguagem, os canais de comunicação são síncronos e bi-direcionais. São descritos pelas suas duas extremidades, onde os processos podem ler ou escrever em qualquer parte dos programas e usam tipos para descrever o fluxo de mensagens que são escritas/lidas no canal.

Baltazar et al. introduziram um conceito em que combinam os tipos de sessão com alguns refinamentos originando assim os tipos linearmente refinados \cite{ref-lin-ref-st}. Estes tipos permitem ao programador acoplar formulas a tipos permitindo especificar propriedades nos tipos. Um exemplo de um tipo linearmente refinado é \{x : \lstinline"integer" $|$ A\}, no qual \textit{x} representa um inteiro que tem de respeitar a formula A. Estes tipos serviram de base para implementar linguagens como o SePi.

\subsection{Programação baseada em atores}
\label{sec:actors}

\todo{Worth??}




%%% Local Variables:
%%% mode: latex
%%% TeX-master: "cfst-inforum18"
%%% End:
