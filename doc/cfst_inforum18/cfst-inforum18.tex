% This is samplepaper.tex, a sample chapter demonstrating the
% LLNCS macro package for Springer Computer Science proceedings;
% Version 2.20 of 2017/10/04
%
\documentclass[runningheads]{llncs}
\usepackage[portuguese]{babel}
\usepackage[utf8]{inputenc}  % for proper diacritics

%
\usepackage{graphicx}
% Used for displaying a sample figure. If possible, figure files should
% be included in EPS format.
%
% If you use the hyperref package, please uncomment the following line
% to display URLs in blue roman font according to Springer's eBook style:
% \renewcommand\UrlFont{\color{blue}\rmfamily}

\begin{document}

\title{Uma linguagem de programação com tipos de sessão independentes do contexto}

\titlerunning{Abbreviated paper title}
% If the paper title is too long for the running head, you can set
% an abbreviated paper title here
%
\author{Bernardo Almeida e Vasco T. Vasconcelos}

%\authorrunning{F. Author et al.}

\institute{LASIGE, Faculdade de Ciências, Universidade de Lisboa, Portugal}
%
\maketitle

\begin{abstract}

  Os sistemas de software distribuídos têm uma comunicação bastante intensiva onde o elevado número de mensagens trocadas entre processos tende a tornar a codificação dos mesmos bastante complexa.
  
  Os tipos de sessão foram propostos para responder a esta necessidade, permitindo definir protocolos na forma de tipos que representam ``interações corretas'' do sistema e que garantem propriedades tais como a inexistência de erros na comunicação e de situações de impasse.
  
  Os tipos de sessão tradicionais são descritos por linguagens regulares, permitindo, por exemplo, a definição de protocolos na forma de lista mas não em forma de árvore.

  Neste artigo apresenta-se uma linguagem de programação concorrente, explicitamente tipificada, onde os processos comunicam exclusivamente por troca de mensagens cujos protocolos são definidos por tipos de sessão livres do contexto.

\keywords{Concorrência \and Troca de mensagens \and Tipos de sessão livres do contexto.}

\end{abstract}


% How to
% \section{First Section}
% \subsection{A Subsection Sample}
% \subsubsection{Sample Heading (Third Level)} % Only two levels of headings should be numbered.
% \paragraph{Sample Heading (Fourth Level)}
% The contribution should contain no more than four levels of % headings. Table~\ref{tab1} gives a summary of all heading levels.


% \noindent Displayed equations are centered and set on a separate
% line.
% \begin{equation}
% x + y = z
% \end{equation}
% Please try to avoid rasterized images for line-art diagrams and
% schemas. Whenever possible, use vector graphics instead (see
% Fig.~\ref{fig1}).

% \begin{figure}
% \includegraphics[width=\textwidth]{fig1.eps}
% \caption{A figure caption is always placed below the illustration.
% Please note that short captions are centered, while long ones are
% justified by the macro package automatically.} \label{fig1}
% \end{figure}

% \begin{theorem}
% This is a sample theorem. The run-in heading is set in bold, while
% the following text appears in italics. Definitions, lemmas,
% propositions, and corollaries are styled the same way.
% \end{theorem}

% \begin{proof}
% Proofs, examples, and remarks have the initial word in italics,
% while the following text appears in normal font.
% \end{proof}
% For citations of references, we prefer the use of square brackets
% and consecutive numbers. Citations using labels or the author/year
% convention are also acceptable. The following bibliography provides
% a sample reference list with entries for journal
% articles~\cite{ref_article1}, an LNCS chapter~\cite{ref_lncs1}, a
% book~\cite{ref_book1}, proceedings without editors~\cite{ref_proc1},
% and a homepage~\cite{ref_url1}. Multiple citations are grouped
% \cite{ref_article1,ref_lncs1,ref_book1},
% \cite{ref_article1,ref_book1,ref_proc1,ref_url1}.
% %
% % ---- Bibliography ----
% %
% % BibTeX users should specify bibliography style 'splncs04'.
% % References will then be sorted and formatted in the correct style.
% %
% % \bibliographystyle{splncs04}
% % \bibliography{mybibliography}
% %
% \begin{thebibliography}{8}
% \bibitem{ref_article1}
% Author, F.: Article title. Journal \textbf{2}(5), 99--110 (2016)

% \bibitem{ref_lncs1}
% Author, F., Author, S.: Title of a proceedings paper. In: Editor,
% F., Editor, S. (eds.) CONFERENCE 2016, LNCS, vol. 9999, pp. 1--13.
% Springer, Heidelberg (2016). \doi{10.10007/1234567890}

% \bibitem{ref_book1}
% Author, F., Author, S., Author, T.: Book title. 2nd edn. Publisher,
% Location (1999)

% \bibitem{ref_proc1}
% Author, A.-B.: Contribution title. In: 9th International Proceedings
% on Proceedings, pp. 1--2. Publisher, Location (2010)

% \bibitem{ref_url1}
% LNCS Homepage, \url{http://www.springer.com/lncs}. Last accessed 4
% Oct 2017
% \end{thebibliography}
\end{document}

%%% Local Variables:
%%% mode: latex
%%% TeX-master: t
%%% End:
