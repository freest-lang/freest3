% This is samplepaper.tex, a sample chapter demonstrating the
% LLNCS macro package for Springer Computer Science proceedings;
% Version 2.20 of 2017/10/04
%
\documentclass[runningheads]{llncs}

\usepackage[portuguese]{babel}
\usepackage[utf8]{inputenc}  % for proper diacritics
\usepackage[T1]{fontenc}
\usepackage {listings}
\usepackage{tikz}
% \usepackage{xcolor}
% Links
\usepackage{hyperref}

\usepackage{amsmath,amssymb}
%\usepackage{stmaryrd}
\usepackage{listings,color}
\usepackage{alltt}
\usepackage{flushend}



%\usepackage[T1]{fontenc}

% \lstdefinestyle{Go}{	
% 	keywordstyle=[1]\bfseries,
% 	basicstyle=\footnotesize\ttfamily,	
% 	numberstyle=\tiny,
% 	numbersep=5pt,
% 	breaklines=true,
% 	%prebreak=\raisebox{0ex}[0ex2][0ex]{\ensuremath{\hookleftarrow}},
% 	showstringspaces=false,
% 	upquote=true,
% 	tabsize=3,
% 	frame=tb,
% 	morekeywords={go,make,chan,int,import,main,func,for,select,case,string},
%       }

\lstdefinelanguage{Golang}%
  {morekeywords=[1]{package,import,func,type,struct,return,defer,panic,%
     recover,select,var,const,iota,},%
   morekeywords=[2]{string,uint,uint8,uint16,uint32,uint64,int,int8,int16,%
     int32,int64,bool,float32,float64,complex64,complex128,byte,rune,uintptr,%
     error,interface},%
   morekeywords=[3]{map,slice,make,new,nil,len,cap,copy,close,true,false,%
     delete,append,real,imag,complex,chan,},%
   morekeywords=[4]{for,break,continue,range,goto,switch,case,fallthrough,if,%
     else,default,},%
   morekeywords=[5]{Println,Printf,Error,Print,},%
   sensitive=true,%
   morecomment=[l]{//},%
   morecomment=[s]{/*}{*/},%
   morestring=[b]',%
   morestring=[b]",%
   morestring=[s]{`}{`},%
}

      
\lstdefinelanguage{SePi}%
  {morekeywords=[1]{type,integer,string,boolean,new, select, assume, assert},%
   sensitive=true,%
   morecomment=[l]{//},%
   morecomment=[s]{/*}{*/},%
   morestring=[b]',%
   morestring=[b]",%
   morestring=[s]{`}{`},%
 }

\lstdefinelanguage{CFST}%
{
  morekeywords=[1]{Int, Char, Bool, Skip, forall, rec, let, in, if, then, else, new, send, receive,
    select, fork, case, of, data, match, with},%  
  sensitive=true,%
  literate={->}{{$\rightarrow$}}1,%
   breaklines=true,
   morecomment=[l]{--},%
   morecomment=[s]{{-}{-}},%
   morestring=[b]',%
   morestring=[b]",%
   morestring=[s]{`}{`},%
 }

 

% notes
\newcommand{\todo}[1]{[{\color{blue}\textbf{#1}}]}

% Keywords
\newcommand{\keyword}[1]{\mathsf{#1}}

% Prekinds

\newcommand\prekind{\upsilon}

\newcommand{\stypes}{\mathcal S}
\newcommand\kinds{\stypes}

\newcommand{\types}{\mathcal T}
\newcommand\kindt{\types}

\newcommand\kindsch{\mathcal C}

% Multiplicity
\newcommand\Un{\ensuremath{\mathbf{u}}} % \infty
\newcommand\Lin{\ensuremath{\mathbf{l}}} % 1 

% Kinds
\newcommand\kind{\kappa}

% Grammars
\newcommand{\grmeq}{\; ::= \;}
\newcommand{\grmor}{\;\mid\;}

% type constructors
\newcommand\tcBase{B}
\newcommand\tcLolli\multimap
\newcommand\tcFun\to
\newcommand\tcBang{\mathop!}

% Keywords for types
\newcommand\kRec{\keyword{rec}}


% Types
\newcommand{\tskip}{\keyword{Skip}}
\newcommand\tSemi[2]{#1;#2}
\newcommand\tOut[1]{\tcBang#1}
\newcommand\tIn[1]{?#1}
\newcommand\tIChoice[1]{\oplus#1}
\newcommand\tEChoice[1]{\&#1}
\newcommand\tUnFun[2]{#1\tcFun#2}
\newcommand\tLinFun[2]{#1\tcLolli#2}
\newcommand\tPair[2]{#1\otimes#2}
\newcommand\tDatatype[1]{{[#1]}}
\newcommand\tRec[2]{\mu\,#1\,.\,#2}
\newcommand\tForall[2]{\forall\,#1\,.\,#2}

\newcommand\tRecK[2]{\kRec\,#1\,.\,#2}
% Environments
% \newcommand\emptyEnv{\cdot}
\newcommand\emptyEnv{\varepsilon}
\newcommand\kindEnv{\Delta}
\newcommand\varEnv{\Gamma}

% Language
% Expressions
% Language Types
\newcommand{\unite}{\keyword{Unit}}
\newcommand{\inte}{\keyword{Int}}
\newcommand{\chare}{\keyword{Char}}
\newcommand{\boole}{\keyword{Bool}}

% Variables
\newcommand\vare[1]{#1}
\newcommand\unlete[3]{\keyword{let} \; #1 = #2 \; \keyword{in} \; #3} 

% Applications
\newcommand\appe[2]{#1#2}
\newcommand\tappe[2]{#1[#2]}

% Conditional
\newcommand\conditionale[3]{\keyword{if}\;#1\;\keyword{then}\;#2\;\keyword{else} \; #3}

% Pairs
\newcommand\paire[2]{(#1,#2)}
\newcommand\binlete[4]{\keyword{let}\;#1, #2 = #3\;\keyword{in}\;#4}

% Session Types
\newcommand\newe[1]{\keyword{new}\;#1}
\newcommand\sende[2]{\keyword{send}\;#1\; #2}
\newcommand\recve[1]{\keyword{receive}\;#1}
\newcommand\selecte[1]{\keyword{select}\;#1}
\newcommand\matche[2]{\keyword{match}\;#1\;\keyword{with}\;#2}

% Fork
\newcommand\forke[1]{\keyword{fork}\;#1}

% Datatypes
\newcommand{\ctrcte}{C}
\newcommand\casee[2]{\keyword{case}\;#1\;\keyword{of}\;#2}

% Goal
\newcommand\Alg{\vdash_{a}}

% Equivalent
\newcommand\Equiv[2]{#1\,\thicksim\,#2}


%%% Local Variables:
%%% mode: latex
%%% TeX-master: "cfst-inforum18"
%%% End:
      

%
%\usepackage{graphicx}
% Used for displaying a sample figure. If possible, figure files should
% be included in EPS format.
%
% If you use the hyperref package, please uncomment the following line
% to display URLs in blue roman font according to Springer's eBook style:
% \renewcommand\UrlFont{\color{blue}\rmfamily}

\begin{document}

\title{Uma linguagem de programação com tipos de sessão independentes do contexto}

\titlerunning{Cresstie}%
%\titlerunning{Linguagem com tipos de sessão independentes do contexto}%
% If the paper title is too long for the running head, you can set
% an abbreviated paper title here
%
\author{Bernardo Almeida e Vasco T. Vasconcelos}

%\authorrunning{F. Author et al.}

\institute{LASIGE, Faculdade de Ciências, Universidade de Lisboa, Portugal}
%
\maketitle

\begin{abstract}

  Os sistemas de software distribuídos têm uma comunicação bastante intensiva onde o elevado número de mensagens trocadas entre processos tende a tornar a codificação dos mesmos bastante complexa.
  
  Os tipos de sessão foram propostos para responder a esta necessidade, permitindo definir protocolos na forma de tipos que representam ``interações corretas'' do sistema e que garantem propriedades tais como a inexistência de erros na comunicação e de situações de impasse.
  
  Os tipos de sessão tradicionais são descritos por linguagens regulares, permitindo, por exemplo, a definição de protocolos na forma de lista mas não em forma de árvore.

  Neste artigo apresenta-se uma linguagem de programação concorrente, explicitamente tipificada, onde os processos comunicam exclusivamente por troca de mensagens cujos protocolos são definidos por tipos de sessão livres do contexto.

\keywords{Concorrência \and Troca de mensagens \and Tipos de sessão livres do contexto.}

\end{abstract}

\section{Session types deserve to be free}

Session types have been long subject to the shackles of tail
recursion~\cite{DBLP:conf/concur/Honda93,DBLP:conf/esop/HondaVK98}. Regular
session-type languages bear the evident advantage of providing for
simple algorithms to check type equivalence and subtyping. Given two
types, a fixed-point construction algorithm such as the one introduced
by Gay and Hole builds, in polynomial time and space, a bisimulation
relating the two types, or decides that no such relation
exists~\cite{DBLP:journals/acta/GayH05}. The scenario darkens when one
decides to let go of tail recursion, for now the fixed-point
construction algorithm does not necessarily terminate.
%
This is one of the main reasons why session types have been confined
to ($\omega$-) regular languages for so many years.

The discipline of conventional (that is, regular) session types
provides guarantees not easily accessible to simpler languages such as
concurrent ML, where channels are unidirectional and transport values
of a fixed size~\cite{DBLP:conf/mcmaster/Reppy93}. Session types, in
turn, provide for the description of richer protocols, epitomised by
the math server~\cite{DBLP:journals/acta/GayH05}, which can be
rendered in the SePi language~\cite{DBLP:conf/sefm/FrancoV13} as
follows:
%
\begin{lstlisting}[morekeywords=end]
MathServer = +{
  Plus: !Int.!Int.?Int.MathServer,
  Eq: !Int.!Int.?Bool.MathServer,
  Done: end
}
\end{lstlisting}

Type \lstinline|MathServer| describes the client side of the protocol,
introducing three choices: \lstinline|Plus|, \lstinline|Eq|, and
\lstinline|Done|. A client that chooses the \lstinline|Plus| choice is
supposed to send two integer values and to receive a further integer
(possibly representing the sum of the former two), after which it goes
back to the beginning. If the same client chooses instead the
\lstinline|Eq| option, it must subsequently send two integer values
and expect a boolean result (possibly describing whether the two
integers are equal), after which it must go back to the beginning.
The \lstinline|Done| option terminates the protocol, as described by
type \lstinline[morekeywords=end]|end|.

The guarantees introduced by session type systems include the
adherence of the code to the protocol and the related absence of
runtime errors, including race
conditions~\cite{DBLP:conf/esop/HondaVK98}. Some systems further
guarantee progress~\cite{DBLP:conf/concur/CairesP10}. All this, under
a rather expressive type language, that of (regular) session types.

There is one further characteristic of session types that attest for
its flexibility: the ability send channels on channels, often called
delegation. This feature provides for the transmission of complex data
on channels in a typeful manner. Suppose we want to stream a tree
%
\begin{lstlisting}
data Tree = Leaf | Node Int Tree Tree
\end{lstlisting}
%
on a channel. One has to choose between a) using multiple channels for
sending the tree or b) incurring on runtime checks to check adherence
to the protocol. In the former scenario, trees are sent on channels of
type
%
\begin{lstlisting}[morekeywords=end]
type TreeChannelR = +{Leaf:end, Node:!Int.!TreeChannelR.!TreeChannelR.end}
\end{lstlisting}
%
and we see that two new channels must be created and exchanged for
each \lstinline|Node| in a tree, so that $2n+1$ channels are needed to
stream an $n$-\lstinline|Node| tree.
%
In the latter case, tree parts are sent on a single stream, but not
necessarily in a ``tree form''. A suitable channel type is
%
\begin{lstlisting}[morekeywords=end]
type TreeParts = +{Leaf: TreeParts, Node: !Int.TreeParts, EOS: end}
\end{lstlisting}
%
where \lstinline|EOS| represents the end of stream. In this case tree
\lstinline|Node 1 (Node 2 Leaf Leaf) Leaf| can be streamed as
\lstinline|Node 1 Node 2 Leaf Leaf Leaf EOS|, when the tree is visited
in a depth-first manner. It should be easy to see that type
\lstinline|TreeParts| allows streaming many different tree parts that
do not add up to a tree, hence the necessary runtime checks to look
over unexpected parts on the stream.

In 2016, Thiemann and Vasconcelos introduced the concept of
context-free session types and proved that type equivalence remains
decidable~\cite{DBLP:conf/icfp/ThiemannV16}.
%
Context-free session types appear as a natural extension of
conventional (regular) session-types. Syntactically, the changes are
minor: rather than dot (\lstinline|.|), the prefix operator, we use
semi-colon (\lstinline|;|), a new binary operator on types. We also
take the chance to replace \lstinline [morekeywords=end]|end| by
\lstinline|Skip| to make it clear that it does not necessarily ``end''
a session type, but else merely introduces a mark that can sometimes
be omitted. In fact \lstinline|Skip| is the identity element of
sequential composition, so that \freest{} enjoys the monoidal axioms:
\lstinline|Skip;T| $\equiv$ \lstinline|T;Skip| $\equiv$ \lstinline|T|,
for all types \lstinline|T|.
%
Using the syntax of \freest, a channel that streams a tree can be
written as follows:
%
\begin{lstlisting}
type TreeChannel = +{Leaf: Skip, Node: !Int;TreeChannel;TreeChannel}
\end{lstlisting}

The language \freest, described in the sequel, provides for the best
of both worlds: stream the tree on a \emph{single} channel,
\emph{without} extraneous runtime checks.

% Related work

There are a few experimental programming languages based on session
types and there are many proposals for encoding session types in
mainstream programming languages. We cannot cover them all in
this short abstract; the interested reader is referred to a 2016
survey on behavioural types in programming
languages~\cite{DBLP:journals/ftpl/AnconaBB0CDGGGH16}.  Here, we briefly
discuss a few prototypical programming languages using session types.

SePi is a programming language based on the pi-calculus whose channels
are governed by regular session types refined with uninterpreted
predicates~\cite{DBLP:conf/sefm/FrancoV13}. The concrete syntax we
choose for \freest{} is heavily influenced by SePi.

SILL is a functional language with session typed concurrency, based on
the Curry-Howard interpretation of intuitionistic linear
logic~\cite{DBLP:conf/concur/CairesP10}, further extended with
recursive types and
processes~\cite{Toninho:phd,DBLP:conf/esop/ToninhoCP13}.
%
C1 is an imperative language~\cite{Pfenning:C1} developed along the
lines of SILL, featuring types that express
sharing~\cite{DBLP:journals/pacmpl/BalzerP17}.
%
All these languages use regular session types. Compared to \freest,
they present stronger properties (including progress), at the expense
of imposing a particular form of programming (derived from the
Curry-Howard interpretation) whereby each process uses zero or more
channels and provides exactly one. At the time of this writing,
\freest{} does not include shared types.

Links~\cite{DBLP:conf/fmco/CooperLWY06} is a functional programming
language for the web, later extended with session types
primitives~\cite{Lindley.Morris_Lightweight.functional.session.types}. Very
much like SILL and C1, the base language is deadlock-free and
terminating. Links also include recursion and shared channels,
foregoing deadlock-freedom.

The only other implementation of context-free session types we are
aware of is that of Padovani~\cite{DBLP:conf/esop/Padovani17}, a
language that admits equivalence, subtyping and, of particular
interest, inference algorithms. It does however require a structural
alignment between the process code and the session types, enforced by
a \emph{resumption} process operator that explicitly breaks a type
$S;T$. The usage of resumptions requires additional annotations
from the programmer---which we would like to avoid in 
\freest---and, beyond that, it does not solve the type equivalence
problem since the monoidal rules proposed by Thiemann and
Vasconcelos cannot be used. Padovani proposes the use of 
\emph{explicit coercions} to fix this limitation, but this
would require a greater effort from the programmer. 
Any additional effort by the programmer is error-prone, hence
we have embedded a type equivalence checker into  
\freest\ so that it does not need to rely on any
additional annotation from the programmer.


%%% Local Variables:
%%% mode: latex
%%% TeX-master: "main"
%%% End:


\section{Trabalho relacionado}
\label{sec:rel-work}

Muitas linguagens de programação e outros formalismos foram sendo propostos ao longo do tempo para endereçar os aspetos de comunicação no software. Dois modelos comuns para lidar com a comunicação entre componentes numa computação concorrente são a memória partilhada e a troca de mensagens.

Neste artigo focamo-nos no modelo de troca de mensagens, destacando os aspetos principais da programação baseada em canais de comunicação e da programação baseada em atores.

\subsection{Programação baseada em canais}
\label{sec:prog-chan}

\subsubsection{Tipos de sessão}
\label{sec:session-types} Os tipos de sessão têm como principal objetivo enriquecer a expressividade que os tipos tradicionais fornecem, tornando possível estruturar interações complexas numa computação
concorrente onde são trocadas muitas mensagens em canais de tipos heterogéneos. Honda, Vasconcelos e Kubo \cite{ref-lang-primitives}, apresentaram uma variante do cálculo pi na qual é feita uma distinção sintática entre canais lineares e canais partilhados \cite{ref-lang-primitives}. Mais tarde, Gay e Hole \cite{ref-sessions-pi} introduziram o conceito de sub-tipos para os tipos de sessão o que permitiu que as especificações de um protocolo fossem estendidas levando a descrições mais ricas das interações.

% Session types[12] are by now a well-established methodology for typed, message-passing concurrent computations. By assigning session types to communication channels, and by checking programs against session type systems, a number important program properties can be established, including the absence of races in channel manipulation operations, and the guarantee that channels are used as prescribed by their types. 

\subsubsection{Go}
\label{sec:go}
\lstset{language=Golang}
Hoare apresentou, como alternativa ao uso de memória partilhada a linguagem CSP (Communicating Sequential Processes) \cite{ref-CSP-Hoare}, onde apenas existe uma primitiva: a comunicação síncrona. Deste modo, os processos comunicam através de canais cuja operação de envio bloqueia até que o recetor leia a mensagem, fornecendo assim um mecanismo de sincronização.

Go ou \textit{golang} \cite{ref-go}, é uma linguagem de programação concorrente desenvolvida pela Google que recorre ao uso de canais para enviar variáveis partilhadas. O uso de canais permite não só que duas rotinas comuniquem entre si, mas também que sincronizem as suas execuções. Antes de um canal ser utilizado é necessário inicializá-lo e definir o tipo de dados que transporta. Por exemplo, a seguinte inicialização representa \lstinline"ch := make (chan int)" um canal de inteiros.

\subsubsection{SePi}
\label{sec:sepi}
\lstset{language=Sepi}
Franco e Vasconcelos \cite{ref-sepi} apresentaram uma linguagem concorrente baseada no cálculo pi onde as interações são feitas recorrendo a tipos que resultam de uma combinação entre tipos de sessão e tipos linearmente refinados.
Nesta linguagem, os canais de comunicação são síncronos e bi-direcionais. São descritos pelas suas duas extremidades, onde os processos podem ler ou escrever em qualquer parte dos programas e usam tipos para descrever o fluxo de mensagens que são escritas/lidas no canal.

%Baltazar et al. introduziram um conceito em que combinam os tipos de sessão com alguns refinamentos originando assim os tipos linearmente refinados \cite{ref-lin-ref-st}. Estes tipos permitem ao programador acoplar formulas a tipos de sessão, permitindo especificar algumas propriedades. Estes, serviram de base para implementar linguagens como o SePi.
\paragraph{}Os tipos de sessão tradicionais têm limitações na sua estrutura que impedem a serialização de forma eficiente e com segurança de tipos de estruturas de dados organizadas em forma de árvore. A linguagem que propomos permite definir protocolos que descrevem eficientemente essas estruturas recorrendo a tipos de sessão independentes do contexto.
%O SePi é uma linguagem que usufrui dos tipos de sessão tradicionais definir protocolos que descrevam os dados que circulam nos canais de comunicação. No entanto, as limitações destes tipos impedem a serialização com segurança de tipos de estruturas em forma de árvore. A linguagem que propomos permite definir protocolos que descrevem eficientemente essas estruturas recorrendo a tipos de sessão independentes do contexto.


\subsection{Programação baseada em atores}
\label{sec:actors}
%Esta secção apresenta brevemente o modelo de atores e a linguagem Erlang que implementa este modelo. Descreve ainda o Akka que é um conjunto de ferramentas que também implementa o modelo de atores e que pode ser utilizado nas linguagens de programação Java e Scala.

\subsubsection{Modelo de atores}
\label{sec:actor-model}
O conceito de atores foi introduzido por Hewitt et al. \cite{Hewitt:StructuresAsPatternsOfPassingMessages} como um formalismo que se foca na relação entre eventos que é causada por um ator. Os atores são entidades concorrentes que trocam mensagens de forma assíncrona. De tal forma, os problemas tradicionais que estão inerentes à programação concorrente como as situações de impasse e as condições de corrida foram também tidos em consideração, de modo a que não seja necessário recorrer ao uso de semáforos para limitar o acesso a regiões críticas dos programas. %\cite{Agha:Actors}.

\subsubsection{Erlang}
O Erlang é uma linguagem de programação que foi desenhada com o objetivo de desenvolver software concorrente, em tempo-real e sistemas distribuídos tolerantes a faltas \cite{Armstrong:ErlangBook}. É a implementação mais conhecida do modelo de atores.

Nesta linguagem, programadores têm de especificar quais são as atividades que são representadas em processos paralelos e toda a comunicação existente entre os processos. Esta visão de concorrência é similar à do CSP \cite{Hoare:CSP} que visa obter o máximo desempenho compilando os programas para execução paralela e não para modelar a concorrência do mundo real.

O modelo de concorrência desta linguagem é baseado em processos com troca de mensagens assíncrona. Os mecanismos de concorrência são bastante simples e requerem pouco esforço computacional, visto que, os processos precisam de pouca memória e criar e destruir estes processos e comunicar.

\subsubsection{Akka}

Akka é um conjunto de ferramentas que foi desenhado com o intuito de construir sistemas escaláveis e resilientes.
Implementa o modelo de atores proposto por Hewitt  \cite{Hewitt:ActorFormalismForAI} com o objetivo de fornecer um nível de abstração que facilite a escrita de sistemas concorrentes, paralelos e distribuídos, este pode ser utilizado em Java ou Scala.

Em Akka, os atores são assíncronos e usam um sistema de comunicação baseado em troca de mensagens não bloqueantes, são considerados processos mais leves, são orientados a eventos, isto é, esperam por mensagens e de seguida reagem a essas mensagens. As mensagens que transitam entre atores não têm tipos apesar de existirem atores com tipos nas versões mais recentes.

Um ator tem estado, uma ``caixa de correio'' (\textit{mailbox}) e um determinado comportamento, pode ter como filhos outros atores por ele criados. % que têm de ser explicitamente terminados. O comportamento dos atores pode mudar ao longo do tempo (operações \lstinline"become" e \lstinline"unbecome").

\paragraph{}O modelo de atores foca-se no conceito de ator e na relação entre os eventos causada pelos atores. Tem em consideração aspetos relativos à programação concorrente como as situações de impasse e as condições de corrida que também são tidos em consideração na programação baseada em canais. A diferença em relação à linguagem que apresentamos é que os atores trocam mensagens de forma assíncrona enquanto que a nesta linguagem as mensagens são trocadas de forma síncrona.

%O modelo de atores foca-se no conceito de ator e na relação entre os eventos causada pelos atores. Tem em consideração aspetos relativos à programação concorrente como as situações de impasse e as condições de corrida que também são tidos em consideração na programação baseada em canais. Contudo, a comunicação entre atores é assíncrona recorrendo a uma ``caixa de correio'' para guardar as mensagens que ainda não foram entregues. A linguagem que apresentamos comunica de forma síncrona

%%% Local Variables:
%%% mode: latex
%%% TeX-master: "cfst-inforum18"
%%% End:

\section{A linguagem de programação}
\lstset{language=CFST, style=eclipse}

Esta secção introduz a linguagem apresentando exemplos, a sua sintaxe e semântica e o sistema de tipos. 

A linguagem que propomos é funcional e, como tal, apresenta uma sintaxe bastante semelhante à do Haskell acrescida com primitivas para criação de canais e de envio e receção de dados nos mesmos. Vamos considerar como exemplo ao longo desta secção, o envio de um tipo de dados estruturado em forma de árvore num canal.

A única primitiva de comunicação que a linguagem disponibiliza é troca de mensagens. As mensagens são trocadas em canais de comunicação síncronos e bidirecionais. Cada canal pode ser descrito pelas suas duas extremidades (\textit{endpoints}) e são caracterizados por tipos que descrevem a sequência de mensagens que passam no canal.
Os processos podem escrever numa das pontas do canal ou ler na outra.

Na seguinte figura \ref{fig:types} estão presentes os tipos que estão disponíveis na linguagem.

\begin{figure}[h!]
  \begin{align*}
    \tcBase \grmeq & \inte \grmor \, \chare \grmor \, \boole \grmor \, \unite && \text{Tipos básicos}\\
    T \grmeq       & \tskip \grmor \tSemi{T}{T} \grmor \,\tOut{\tcBase} \grmor \,\tIn{\tcBase} && \text{Tipos}\\
    \grmor         & \tIChoice\{l_i\colon T_i\}_{i\in I} \grmor \tEChoice\{l_i\colon T_i\}_{i\in I} \\ 
    \grmor         & \tcBase \grmor \tUnFun{T}{T} \grmor \tLinFun{T}{T}\\   
    \grmor         & \tPair{T}{T} \grmor \tDatatype{l_i\colon T_i}_{i\in I} \grmor \tRec{\alpha}{T} \grmor \alpha\\
    \kindsch \grmeq & T \grmor \tForall{\alpha}{\kindsch}  && \text{Esquemas de tipos}
    % 
  \end{align*}
\end{figure}


%%% Local Variables:
%%% mode: latex
%%% TeX-master: "cfst"
%%% End:

Os tipos básicos representados na figura \ref{fig:types} são alguns dos existentes no Haskell, inteiros, caracteres, booleanos e ainda o tipo Unit (``()''). Os restantes tipos são compostos pelo operador de sequenciação $\tSemi{\_}{\_}$, pela sua unidade $\tskip$, pelos tipos que representam o envio $\tOut{B}$, a receção $\tIn{B}$, escolhas internas e externas, $\tIChoice{\{l_i\colon T_i\}}_{i\in I}$ e $\tEChoice{\{l_i\colon T_i\}}_{i\in I}$ respetivamente, funções lineares e \textit{unrestricted}, $\tLinFun{T}{T}$ e $\tUnFun{T}{T}$ e ainda pares $\tPair{T}{T}$, tipos de dados $\tDatatype{l_i\colon T_i}_{i\in I}$, tipos recursivos $\tRecK{x}{T}$ e variáveis \textit{x}.

Deste modo, um tipo \lstinline"!Int;?Bool" é um tipo que descreve uma ponta de um canal que espera fazer sequencialmente as operações de enviar um inteiro, receber um booleano e de seguida termina sem mais nenhuma interação.

\subsection{Exemplo: Transmitir uma árvore binária num canal}
\label{sec:example}
Para realizar este exemplo, vamos definir o tipo de dados que representa uma árvore binária:

\begin{lstlisting}
  data Tree = Leaf | Node Int Tree Tree
\end{lstlisting}

O tipo de sessão que descreve o envio da árvore é \lstinline"rec x . +{Leaf: Skip, Node: !Int;x;x}". Este, apresenta uma escolha interna ($\tIChoice{\{l_i\colon T_i\}}_{i\in I}$) que contempla duas opções: \lstinline"Leaf" e \lstinline"Node". No ramo \lstinline"Leaf: Skip" temos o valor \lstinline"Skip" que representa a unidade (não há comunicação). No ramo \lstinline"Node: !Int;x;x" podemos observar a chamada recursiva (\lstinline"rec") do tipo para que seja possível enviar as duas subárvores que este define.

Assim sendo, a função que envia árvores binárias num canal tem o seguinte tipo:

\begin{lstlisting}
  sendTree :: forall a => Tree -> (rec x . +{LeafC : Skip, NodeC: !Int;x;x}); a -> a
\end{lstlisting}


No caso em que se envia uma \lstinline"Leaf", é necessário escolher o ramo através da operação \lstinline"select Leaf" que espera um canal que tenha tipo na forma $\tIChoice{\{l_i\colon T_i\}}_{i\in I}$.
Por outro lado, quando se envia um \lstinline"Node", selecionamos o ramo certo: \lstinline"select Node" e ficamos com o tipo $\tSemi{\tOut{\inte}}{\tSemi{x}{x}}$, assim sendo, primeiro enviamos o inteiro v no canal c ($\sende{v}{c}$) de seguida, é necessário fazer uma chamada recursiva à função para as duas subárvores.

\begin{lstlisting}
  sendTree[rec x . +{LeafC : Skip, NodeC: !Int;x;x}] l c1
  sendTree[Skip] r c2
\end{lstlisting}

O polimorfismo, presente no tipo da função raramente é considerado com tipos de sessão. No entanto, como se pode observar no exemplo, este aparece de forma bastante natural, visto que, o envio de uma árvore generaliza o envio de um único valor, que é polimórfico.

As chamadas a funções polimórficas, são na forma \lstinline"sendTree [Skip]" porque nesta linguagem é necessário especificar o tipo de que as variáveis (neste caso \textit{a}) vão adotar na chamada recursiva.

A função que recebe a árvore binária é análoga mas com o tipo de sessão dual \lstinline"rec x . &{Leaf: Skip, Node: ?Int;x;x}" que, em vez de impor a seleção de um dos ramos (\lstinline"select"), oferece uma escolha $\matche{tree}{\{l_i\,\to\,S_i\}_{i\in I}}$ dos mesmos.
\begin{lstlisting}  
  receiveTree c =
    match c with
      LeafC c1 -> (Leaf, c1)
      NodeC c1 ->
        let x, c2 = receive c1 in
        let left, c3 = receiveTree [rec x.&{LeafC: Skip, NodeC: ?Int;x;x}] c2 in
        let right, c4 = receiveTree [Skip] c3 in
        (Node x left right, c4)
\end{lstlisting}


\subsection{Expressões}
A figura \ref{fig:expressions} apresenta a sintaxe para as expressões da linguagem. As expressões básicas (para os tipos básicos) são inteiros (ex: 1), caracteres (ex: 'a'), booleanos (True e False) e ainda o tipo Unit (``()'').
As aplicações, operações de envio e receção em canais e expressões para escolhas foram brevemente descritas na secção \ref{sec:example}

Das restantes expressões, é importante realçar as operações de criação de canais e de fios de execução (\textit{threads}).
A operação de criação de canais \lstinline"new T", devolve um par com as duas extremidades do canal que são descritas pelo tipo T. Mais precisamente, a primeira extremidade (primeiro elemento do par) é descrita pelo tipo T e a segunda extremidade (segundo elemento) pelo seu dual (\textbf{dualof} T).
A expressão \lstinline"fork e" é responsável por criar um novo fio de execução onde a expressão e vai ser executada, esta operação devolve $\unite$.

\begin{figure}[ht]
  \begin{align*}
    e \grmeq & \unite \grmor \inte \grmor \chare \grmor \boole && \text{Expressões básicas}\\
    \grmor & \vare{x} \grmor \unlete{x}{e}{e} && \text{Variáveis}\\
    \grmor & \appe{e}{e} \grmor \tappe{e}{T} && \text{Applicações}\\
    \grmor & \conditionale{e}{e}{e} && \text{Condicional}\\
    \grmor & \paire{e}{e} \grmor \binlete{x}{y}{e}{e} && \text{Pares}\\
    %
    \grmor & \newe{T} \grmor \sende{e}{e} \grmor \recve{e} && \text{Operações de comunicação}\\
    \grmor & \selecte{e} \grmor \matche{e}{\{l_i\;\to\;e_i\}_{i\in I}} \\
    \grmor & \forke{e}  && \text{Fork}\\
    \grmor & \ctrcte \grmor \casee{e}{\{C_i\;\to\;e_i\}_{i\in I}} && \text{Tipos de dados}\\
    %
  \end{align*}
  \hrulefill
  \caption{Sintaxe das expressões}
  \label{fig:expressions}
\end{figure}


%%% Local Variables:
%%% mode: latex
%%% TeX-master: "cfst-inforum18"
%%% End:

\subsection{Validação}

Para garantir que o sistema está isento de erros a nossa fase de validação contempla com um sistema de \textit{kinding} para assegurar a boa formação dos tipos, uma verificação de tipos onde se verifica se todas as expressões têm o tipo esperado e uma parte que define quando é que dois tipos são equivalentes.

\subsubsection{Sistema de \textit{kinding}}
O sistema de \textit{kinding} tem como objetivo assegurar a boa formação dos tipos. Para além de verificar se os tipos são bem formados classifica-os nas categorias de tipos de sessão ou tipos gerais. Associa ainda multiplicidades aos tipos (linear ou \textit{unrestricted}).
Por exemplo, o tipo $\tOut{\inte}$ é bem formado e é um tipo de sessão linear, isto é, só pode ser utilizado uma vez. Por outro lado o tipo $\tRec{x}{\tSemi{a}{x}}$ é mal formado se a variável \textit{a} não estiver no ambiente de \textit{kinding} (onde estão todas as associações de variáveis com os respetivos \textit{kinds}).

\subsubsection{Verificação de tipos}
\label{sec:typecheck}
Para fazermos a verificação de tipos utilizámos um sistema bidirecional, isto é, distinguimos duas relações:
\begin{itemize}
\item Dado um contexto e uma expressão, sintetizar o tipo T: \begin{figure}[h!]
%\centering
  \begin{gather*}
    \frac{\overbrace{\text{...}}^{\text{Específico para cada expressão e}}}
    {\underbrace{\kindEnv;\varEnv \Goal \text{e}}_{\text{In}} \rightarrow
      \underbrace{T;\varEnv}_{\text{Out}}}
    %
  \end{gather*}
  %\hrulefill
%  \caption{Sintetizar um tipo T a partir de uma expressão e}
  \label{fig:synthesize}
\end{figure}



%%% Local Variables:
%%% mode: latex
%%% TeX-master: "cfst-inforum18"
%%% End:
\item Dado um contexto, uma expressão e um tipo verificar o tipo da expressão de encontro ao tipo esperado:
\begin{figure}[h!]
%\centering
  \begin{gather*}
    \frac{\kindEnv;\varEnv_1 \vdash U;\varEnv_2 \qquad \kindEnv \vdash \Equiv{U}{T}}
         {\underbrace{\kindEnv;\varEnv_1 \vdash \, e \colon T}_{\text{Input}} \rightarrow \underbrace{\varEnv_2}_{\text{Output}}}
         % 
  \end{gather*}
  %\hrulefill
  \caption{Verificar de encontro a um tipo}
  \label{fig:check-against}
\end{figure}


%%% Local Variables:
%%% mode: latex
%%% TeX-master: "cfst-inforum18"
%%% End:

\end{itemize}

\subsubsection{Equivalencia de tipos}

Determinar se dois tipos são equivalentes apresenta diversos desafios. O artigo apresentado por Christensen et. al \cite{decidable-CFP-bisimilarity} procura mostrar que a equivalência de tipos é decidível para processos independentes do contexto, contudo não define diretamente um algoritmo.

O algoritmo que utilizamos está baseado nas ideias de Jančar et. al \cite{bisimilarity} em que traduzimos um tipo de sessão numa gramática independente do contexto e onde geramos uma árvore de expansão. Este algoritmo, tal como está implementado ainda não é completo, visto que, há tipos que sabemos equivalentes mas que o algoritmo diz não serem. Este algoritmo é ainda trabalho que está em progresso.

% que o algoritmo tal como implementado ainda nao é completo (há tipos de sabemos equivalentes, mas q o algoritmo diz nao serem); que estamos a trabalhar no assunto. 

\subsection{Geração de código}

A linguagem que apresentamos é \textit{call-by-value} e o Haskell (código gerado) é \textit{call-by-name}, isto é, apenas computa as expressões que são passadas como argumento quando esta são utilizadas em vez de computar antes de chamar a função. Para resolver esta questão, tirámos partido da extensão da linguagem Haskell \textit{BangPatterns} que, se usar-mos um sinal de ! antes de cada parâmetro força a avaliação do mesmo. Uma função \lstinline"fun x = e" quando traduzida fica \lstinline"fun !x = e".

O código gerado relativo às operações de comunicação presentes na figura \ref{fig:expressions} foi implementado recorrendo a uma \textit{MVar} que é uma zona de memória mutável que apenas tem dois estados, ou vazio ou um valor do tipo t. Tem duas operações fundamentais \textit{putMVar} para escrever nessa zona de memória (operação de \lstinline"send") e \textit{takeMVar} para ler dessa zona (operação de \lstinline"receive").

Uma \textit{MVar} t têm um tipo associado (t) que é sempre o mesmo para cada \textit{MVar}. Como esta é utilizada para a implementação dos canais de comunicação necessitamos que o seu tipo possa variar para que seja possível enviar, por exemplo, um valor inteiro seguido de um booleano (\lstinline"!Int;!Bool"). Para que o sistema de tipos do Haskell não verifique os tipos que circulam nos canais utilizámos uma primitiva do Haskell que converte um valor de um tipo noutro. Apesar de insegura, esta primitiva não apresenta qualquer problema porque fazemos a nossa própria verificação de tipos (secção \ref{sec:typecheck}).

O código gerado está escrito na linguagem Haskell e derivado disso, as operações de envio ($\sende{v}{c}$), receção ($\recve{c}$), criação de canais ($\newe{T}$) e criação de fios de execução ($\forke{e}$) são forçosamente operações sobre um mónade, mais precisamente, são operações de IO.

Este facto representa uma dificuldade que é decidir quando traduzir as expressões para código de um mónade ou não. A cada momento da tradução ($[\![e]\!]$) apenas temos disponível uma expressão, que sabemos se é monádica ou não, mas não temos qualquer informação acerca das expressões anteriores (na mesma função). Se, num dado momento, encontramos uma expressão que origina código monádico, teríamos de retraduzir o código anterior para estar também na forma monádica.

A nossa aproximação foi resolver este problema antes de aplicar a função de tradução, isto é, anotar a árvore sintática com valores booleanos que representam o estado que é esperado de cada expressão. Antes disso, é necessário percorrer as expressões para saber que funções são monádicas (têm alguma expressão monádica), isto porque, as chamadas a funções são apenas aplicações de variáveis.

Assim sendo, geramos código para cada expressão com base na tabela \ref{tab:monad} que contempla o valor esperado e o encontrado pela função de tradução a cada momento da geração de código.

% Assim sendo, após a primeira iteração obtemos uma estrutura com as funções e um valor booleano associado que indica se são monádicas. Após esta fase, vamos anotar a árvore sintática e terminamos este processo com os valores esperados para cada nó da árvore.

% Neste momento, a função de tradução, para cada expressão, tem informação acerca do valor monádico esperado e do encontrado (na própria função de tradução).

% A tabela \ref{tab:monad} contempla as combinações dos valores monádicos encontrados e esperados assim como o código que deve ser gerado em cada um dos casos.

\begin{table}
\begin{center}
  \begin{tabular}[ht!]{| c | c | c |}
    \hline  
    \quad Valor esperado \quad&\quad Valor encontrado \quad&\quad Código gerado \quad\\\hline
    False & False & e \\
    True & False & \lstinline"return e" \\
    True & True & e \\
    False & True & \lstinline"e >>= x -> x" \\
    \hline
  \end{tabular}
  \vspace{0.2cm}
  \caption{Tabela para geração de código monádico}
  \label{tab:monad}
\end{center}
\end{table}

% Nos casos em o valor esperado e o valor encontrado são iguais a tradução deve ser literal, no caso em que o valor esperado é true e o valor encontrado é false a tradução deve ser \lstinline"return e" para tornar o valor que é suposto ser monádico num valor monádico. No último caso, em que o valor esperado é false e o valor encontrado é true devemos, ao traduzir, retirar o valor do monad: \lstinline"e >>= x -> x".
Por exemplo, o código fonte para o envio de uma árvore seria:

\begin{lstlisting}
  case tree of
    Node x l r ->
      let w1 = select NodeC tree in
      let w2 = send x w1 in
      let w3 = sendTree[rec x . +{LeafC : Skip, NodeC: !Int;x;x}] l w2 in
      let w4 = sendTree[Skip] r w3 in
      w4
\end{lstlisting}

e o código gerado:
\begin{lstlisting}
  case tree of
   Node x l r ->
    _send "NodeC" tree >>=
    \w1 -> _send x w1 >>=
    \w2 -> ((sendTree l) w2) >>=
    \w3 -> ((sendTree r) w3) >>=
    \w4 -> return w4 
\end{lstlisting}

Note-se que, como são enviados e recebidos valores o código encontra-se num monáde de IO. 


\subsection{Testes}

Foram realizados diversos tipos de testes, com diversas ferramentas para aferir a robustez do compilador. As ferramentas que utilizámos foram o \textit{HSpec}\footnote{\url{http://hspec.github.io/}} e o \textit{HUnit}\footnote{\url{https://github.com/hspec/HUnit}} para desenvolver os testes e o \textit{HPC}\footnote{\url{https://wiki.haskell.org/Haskell_program_coverage}} para aferir a cobertura dos testes. Foram feitos testes unitários para testar diversas funções isoladamente e ainda testes com programas, nos quais escrevemos um programa na linguagem apresentada e verificamos se o seu resultado é o esperado.



%%% Local Variables:
%%% mode: latex
%%% TeX-master: "cfst-inforum18"
%%% End:

\subsection{Testes}
\section{Conclusion and future work}

% How to
% \section{First Section}
% \subsection{A Subsection Sample}
% \subsubsection{Sample Heading (Third Level)} % Only two levels of headings should be numbered.
% \paragraph{Sample Heading (Fourth Level)}
% The contribution should contain no more than four levels of % headings. Table~\ref{tab1} gives a summary of all heading levels.


% \noindent Displayed equations are centered and set on a separate
% line.
% \begin{equation}
% x + y = z
% \end{equation}
% Please try to avoid rasterized images for line-art diagrams and
% schemas. Whenever possible, use vector graphics instead (see
% Fig.~\ref{fig1}).

% \begin{figure}
% \includegraphics[width=\textwidth]{fig1.eps}
% \caption{A figure caption is always placed below the illustration.
% Please note that short captions are centered, while long ones are
% justified by the macro package automatically.} \label{fig1}
% \end{figure}

% \begin{theorem}
% This is a sample theorem. The run-in heading is set in bold, while
% the following text appears in italics. Definitions, lemmas,
% propositions, and corollaries are styled the same way.
% \end{theorem}

% \begin{proof}
% Proofs, examples, and remarks have the initial word in italics,
% while the following text appears in normal font.
% \end{proof}
% For citations of references, we prefer the use of square brackets
% and consecutive numbers. Citations using labels or the author/year
% convention are also acceptable. The following bibliography provides
% a sample reference list with entries for journal
% articles~\cite{ref_article1}, an LNCS chapter~\cite{ref_lncs1}, a
% book~\cite{ref_book1}, proceedings without editors~\cite{ref_proc1},
% and a homepage~\cite{ref_url1}. Multiple citations are grouped
% \cite{ref_article1,ref_lncs1,ref_book1},
% \cite{ref_article1,ref_book1,ref_proc1,ref_url1}.
% %
% % ---- Bibliography ----
% %
% % BibTeX users should specify bibliography style 'splncs04'.
% % References will then be sorted and formatted in the correct style.
% %

\bibliographystyle{splncs04}
\begin{thebibliography}{8}
% \bibitem{ref_article1}
% Author, F.: Article title. Journal \textbf{2}(5), 99--110 (2016)

\bibitem{ref-cfst}
  Thiemann, P., Vasconcelos, V.T.: Context-free Session Types. In: ICFP 2016, vol. 51, pp. 462--475. ACM (2016). \doi{10.1145/3022670.2951926}

\bibitem{ref-lang-primitives}
  Honda K., Vasconcelos V.T., Kubo M.: Language primitives and type discipline for structured communication-based programming. In: Programming Languages and Systems. ESOP 1998. Lecture Notes in Computer Science, vol 1381. Springer-Verlag, Heidelberg. \doi{10.1007/BFb0053567}

\bibitem{ref-sessions-pi}  
  Gay, S., Hole, M.: Subtyping for session types in the pi calculus. In: Acta Informatica (2005)
  42, pp.191–225. \doi{10.1007/s00236-005-0177-z}

\bibitem{ref-go}
  The Go programming language, \url{http://golang.org/}.

\bibitem{ref-CSP-Hoare}
  Hoare C.A.R.: Communicating sequential processes. Commun. ACM 21, 8 (1978), 666-677. \doi{10.1145/359576.359585}
  
\bibitem{ref-sepi}
    Franco J., Vasconcelos V.T.: A Concurrent Programming Language with Refined Session Types. In: Software Engineering and Formal Methods. SEFM 2013. Lecture Notes in Computer Science, vol 8368, pp. 33--42. Springer. \doi{10.1007/978-3-319-05032-4\_2}

% \bibitem{ref-lin-ref-st}
%   Baltazar, P., Mostrous, D., Vasconcelos, V.T.: Linearly Refined Session Types. In Linearity'11, EPTCS, vol. 101, pp. 38--49
%   \doi{10.4204/EPTCS.101.4}

%   Linearly Refined Session Types. Pedro Baltazar, Dimitris Mostrous, and Vasco Thudichum Vasconcelos. In Linearity'11, volume 101 of EPTCS, pages 38-49. 2012. 
  
\bibitem{Hewitt:StructuresAsPatternsOfPassingMessages}
   Hewitt, C.: Viewing control structures as patterns of passing messages. In Artificial Intelligence, 1977, vol. 8, pp. 323--364
  \doi{10.1016/0004-3702(77)90033-9}

\bibitem{Hewitt:ActorFormalismForAI}
   Hewitt, C., Bishop, P., Steiger, R.: A Universal Modular ACTOR Formalism for Artificial Intelligence. In IJCAI, 1973, vol. 8, pp.35--245

  
% \bibitem{Agha:Actors}
% Agha, G.:Actors: A Model of Concurrent Computation in Distributed Systems. MIT Press (1986)
  
\bibitem{Armstrong:ErlangBook}
Armstrong, J.: Programming Erlang: Software for a Concurrent World. Pragmatic Bookshelf (2007)

\bibitem{Hoare:CSP}
  Hoare, C.A.R.: Communicating Sequential Processes. In Commun. ACM, pp. 100--106. ACM, 1983
\doi{10.1145/357980.358021}  

\bibitem{bisimilarity}
  Jančar P., Moller F. (1999) Techniques for Decidability and Undecidability of Bisimilarity. In: CONCUR '99. Lecture Notes in Computer Science, vol 1664. Springer, Berlin, Heidelberg
  \doi{10.1007/3-540-48320-9\_5}

\bibitem{decidable-CFP-bisimilarity}
  Christensen S., Hüttel H., Stirling C. (1992) Bisimulation equivalence is decidable for all context-free processes. In: CONCUR '92. Lecture Notes in Computer Science, vol 630. Springer.
  \doi{10.1007/BFb0084788}  

% \bibitem{ref_lncs1}
% Author, F., Author, S.: Title of a proceedings paper. In: Editor,
% F., Editor, S. (eds.) CONFERENCE 2016, LNCS, vol. 9999, pp. 1--13.
% Springer, Heidelberg (2016). \doi{10.10007/1234567890}
  
% \bibitem{ref_book1}
% Author, F., Author, S., Author, T.: Book title. 2nd edn. Publisher,
% Location (1999)

% \bibitem{ref_proc1}
% Author, A.-B.: Contribution title. In: 9th International Proceedings
% on Proceedings, pp. 1--2. Publisher, Location (2010)

% \bibitem{ref_url1}
% LNCS Homepage, \url{http://www.springer.com/lncs}. Last accessed 4
% Oct 2017
\end{thebibliography}


%%% Local Variables:
%%% mode: latex
%%% TeX-master: "cfst-inforum18"
%%% End:

\end{document}

%%% Local Variables:
%%% mode: latex
%%% TeX-master: t
%%% End:
