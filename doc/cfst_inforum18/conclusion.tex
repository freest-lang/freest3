\section{Conclusão e trabalho futuro}

Apresentámos uma linguagem de programação concorrente e explicitamente tipificada, onde os processos comunicam exclusivamente por troca de mensagens. As mensagens são trocadas em canais síncronos que são descritos por tipos de sessão independentes do contexto. Graças a estes tipos, a linguagem é capaz de serializar, por exemplo, tipos de dados recursivos, estruturados em forma de árvore com segurança de tipos, entre outros exemplos.

Como trabalho futuro, temos o objetivo de reduzir a verbosidade da linguagem tornando possível abreviar tipos (\textbf{type} A = ...), deste modo, reduz-se a necessidade de escrever tipos ao longo do código fonte, reduzindo a probabilidade de existirem erros nas declarações de tipos das funções e/ou nas aplicações de tipos. Existe ainda o objetivo de fazer inferência de tipos para tornar a mais fácil a escrita de programas, reduzindo mais uma vez, a verbosidade e a existência de erros.


%%% Local Variables:
%%% mode: latex
%%% TeX-master: "cfst-inforum18"
%%% End:
