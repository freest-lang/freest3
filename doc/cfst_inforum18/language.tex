\section{A Linguagem}
\lstset{language=CFST}
Esta secção introduz a linguagem apresentando exemplos, a sua sintaxe e semântica e o sistema de tipos. 

A linguagem que propomos é funcional e, como tal, apresenta uma sintaxe bastante semelhante à do Haskell acrescida com primitivas para criação de canais e de envio e receção de dados nos mesmos. Vamos considerar como exemplo ao longo desta secção, o envio de um tipo de dados estrururado em forma de árvore num canal.

A única primitiva de comunicação que a linguagem disponibiliza é troca de mensagens. As mensagens são trocadas em canais de comunicação síncronos e bidireccionais. Cada canal pode ser descrito pelas suas duas extremidades (\textit{endpoints}) e são caracterizados por tipos que descrevem a sequência de mensagens que passam no canal.
Os processos podem escrever numa das pontas do canal ou ler na outra.

Na seguinte figura \ref{fig:types} estão presentes os tipos que estão disponíveis na linguagem.

% \begin{figure}[t]
  \begin{align*}
    \tcBase \grmeq & \inte \grmor \, \chare \grmor \, \boole \grmor \, \unite  \grmor \, l && \text{Tipos básicos}\\
    T \grmeq       & \tskip \grmor \tChoice{T}{T} \grmor \tSemi{T}{T}  \grmor \,\tOut{\tcBase} \grmor \,\tIn{\tcBase} \grmor \tRec{x}{T} \grmor x && \text{Tipos} 
    % 
  \end{align*}
%   \hrulefill
%   \caption{Sintaxe dos tipos}
%   \label{fig:types}
% \end{figure}


%%% Local Variables:
%%% mode: latex
%%% TeX-master: "main"
%%% End:


Os tipos básicos representados na figura \ref{fig:types} são alguns dos existentes no Haskell, inteiros, caracteres, booleanos e ainda o tipo Unit (``()''). Os restantes tipos são compostos a partir do operador de sequenciação $\tSemi{\_}{\_}$ da sua unidade $\tskip$, os tipos que representam o envio $\tOut{B}$, a receção $\tIn{B}$, escolhas internas e externas, $\tIChoice{\{l_i\colon T_i\}}_{i\in I}$ e $\tEChoice{\{l_i\colon T_i\}}_{i\in I}$ respectivamente, funções lineares e \textit{unrestricted}, $\tLinFun{T}{T}$ e $\tUnFun{T}{T}$ e ainda pares $\tPair{T}{T}$, tipos de dados $\tDatatype{l_i\colon T_i}_{i\in I}$, tipos recursivos $\tRecK{x}{T}$ e variáveis \textit{x}.

Deste modo, um tipo \lstinline"!Int;?Bool;Skip" é um tipo que descreve uma ponta de um canal que espera fazer sequencialmente as operações de enviar um inteiro, receber um booleano e de seguida termina sem mais nenhuma interação.

\subsection{Exemplo: Enviar uma árvore binária num canal}
\label{sec:example}
Para realizar este exemplo, vamos definir o tipo de dados que representa uma árvore binária:

\begin{lstlisting}
  data Tree = Leaf | Node Int Tree Tree
\end{lstlisting}

O tipo de sessão que descreve o envio da árvore é \lstinline"rec x . +{Leaf: Skip, Node: !Int;x;x}". Este, apresenta uma escolha interna ($\tIChoice{\{l_i\colon T_i\}}_{i\in I}$) que contempla duas opções: \lstinline"Leaf" e \lstinline"Node". No ramo \lstinline"Leaf: Skip" temos o valor \lstinline"Skip" que representa a unidade (um canal vazio). No ramo \lstinline"Node: !Int;x;x" podemos observar a chamada recursiva (\lstinline"rec") do tipo para que seja possível enviar as duas subárvores que este define.

Assim sendo, a função que envia árvores binárias num canal tem o seguinte tipo:

\begin{lstlisting}
  sendTree :: forall a => Tree -> (rec x . +{LeafC : Skip, NodeC: !Int;x;x}); a -> a
\end{lstlisting}


No caso em que se envia uma \lstinline"Leaf", é necessário escolher o ramo através da operação \lstinline"select Leaf" que espera um canal que tenha tipo na forma $\tIChoice{\{l_i\colon T_i\}}_{i\in I}$.
Por outro lado, quando se envia um \lstinline"Node", selecionamos o ramo certo: \lstinline"select Node" e ficamos com o tipo $\tSemi{\tOut{\inte}}{\tSemi{x}{x}}$, assim sendo, primeiro enviamos o inteiro v no canal c ($\sende{v}{c}$) de seguida, é necessário fazer uma chamada recursiva à função para as duas subárvores.

\begin{lstlisting}
  sendTree[rec x . +{LeafC : Skip, NodeC: !Int;x;x}] l c1
  sendTree[Skip] r c2
\end{lstlisting}

O polimorfismo, presente no tipo da função raramente é considerado com tipos de sessão. No entanto, como se pode observar no exemplo, este aparece de forma bastante natural, visto que, o envio de uma árvore generaliza o envio de um único valor, que é polimorfico.

As chamadas a funções polimorficas, são na forma \lstinline"sendTree [Skip]" porque nesta linguagem é necessário especificar o tipo de que as variáveis (neste caso \textit{a}) vão adotar na chamada recursiva.

A função que recebe a árvore binária é análoga mas com o tipo de sessão dual \lstinline"rec x . &{Leaf: Skip, Node: ?Int;x;x}" que, em vez de impor a seleção de um dos ramos (\lstinline"select"), oferefece um escolha $\matche{tree}{\{l_i\,\to\,S_i\}_{i\in I}}$.
\begin{lstlisting}  
  receiveTree c =
    match c with
      LeafC c1 -> (Leaf, c1)
      NodeC c1 ->
        let x, c2 = receive c1 in
        let left, c3 = receiveTree [rec x.&{LeafC: Skip, NodeC: ?Int;x;x}] c2 in
        let right, c4 = receiveTree [Skip] c3 in
        (Node x left right, c4)
\end{lstlisting}


\subsection{Expressões}
A figura \ref{fig:expressions} apresenta a sintaxe para as expressões da linguagem. As expressões básicas (para os tipos básicos) são inteiros (ex: 1), caracteres (ex: 'a'), booleanos (True e False) e ainda o tipo Unit (``()'').
As aplicações, operações de envio e receção em canais e expressões para escolhas foram brevemente descritas na secção \ref{sec:example}

Das restantes expresões, é importante realçar as operações de criação de canais e de fios de execução (\textit{threads}).
A operação de criação de canais \lstinline"new T", devolve um par com as duas extremidades do canal que são descritas pelo tipo T. Mais precisamente, a primeira extremidade (primeiro elemento do par) é descrita pelo tipo T e a segunda extremidade (segundo elemento) pelo seu dual (\textbf{dualof} T).
A expressão \lstinline"fork e" é responsável por criar um novo fio de execução onde a expressão e vai ser executada, esta operação devolve $\unite$.

\begin{figure}[t]
  \begin{align*}
    e \grmeq & \unite \grmor x \grmor c \grmor \text{True} \grmor \text{False} && \text{Expressões básicas}\\
    \grmor & \vare{x} \grmor \unlete{x}{e}{e} && \text{Variáveis e let}\\
    \grmor & \appe{e}{e} \grmor \tappe{e}{T} && \text{Aplicações}\\
    \grmor & \conditionale{e}{e}{e} && \text{Condicional}\\
    \grmor & \paire{e}{e} \grmor \binlete{x}{y}{e}{e} && \text{Pares}\\
    %
    \grmor & \newe{T} \grmor \sende{e}{e} \grmor \recve{e} && \text{Operações de comunicação}\\
    \grmor & \selecte{e} \grmor \matche{e}{\{l_i\;\to\;e_i\}_{i\in I}} \\
    \grmor & \forke{e}  && \text{Fork}\\
    \grmor & \ctrcte \grmor \casee{e}{\{C_i\;\to\;e_i\}_{i\in I}} && \text{Tipos de dados}\\
    %
  \end{align*}
  \hrulefill
  \caption{Sintaxe das expressões}
  \label{fig:expressions}
\end{figure}


%%% Local Variables:
%%% mode: latex
%%% TeX-master: "cfst-inforum18"
%%% End:


\subsection{Verificação de tipos... (Validation)}
\todo{Kinding + typechecking + type equiv}

\subsection{CodeGen}




%%% Local Variables:
%%% mode: latex
%%% TeX-master: "cfst-inforum18"
%%% End:
