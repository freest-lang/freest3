%%% https://tacas.info/artifacts-20.php. This is exactly as the
%%% conference paper, except that the first line of abstract is a URL
%%% of the artifact

\documentclass[orivec,runningheads]{llncs}

\usepackage[utf8]{inputenc} % for proper diacritics: Universität, Hüttel, Luís
\usepackage[T1]{fontenc}
\usepackage{amsmath,amssymb}
\usepackage{stmaryrd}
\usepackage{listings,color}
\usepackage{alltt}
\usepackage{flushend}
\usepackage[dvipsnames]{xcolor}
% \usepackage{algorithm}
% \usepackage[noend]{algpseudocode}
\usepackage{graphicx}
\usepackage{tikz-cd}
\usepackage{multicol,enumitem}
\usepackage{tikz}
\usepackage{subfig}
\usepackage[font=small,labelfont=bf]{caption}



%%% Local Variables:
%%% mode: latex
%%% TeX-master: "cfst-inforum18"
%%% End:
      
 
% THEME
\newtcolorbox{mybox}{colback=orange!5!white,colframe=orange!75!black}
\newtcolorbox{myboxazul}{colback=teal!5!white,colframe=teal!75!black}

% TIKZ 
\usetikzlibrary{positioning}
\usetikzlibrary{shapes,arrows}
\usepgfplotslibrary{dateplot}
\tikzstyle{block} = [rectangle, draw, 
    text width=5cm, text centered, rounded corners, minimum height=4em]
\tikzstyle{block2} = [rectangle, draw, 
    text width=10cm, text centered, rounded corners, minimum height=4em]
\tikzstyle{line} = [draw, -latex']

\newenvironment<>{varblock}[2][.9\textwidth]{%
  \setlength{\textwidth}{#1}
  \begin{actionenv}#3%
    \def\insertblocktitle{#2}%
    \par%
    \usebeamertemplate{block begin}}
  {\par%
    \usebeamertemplate{block end}%
  \end{actionenv}}

\newenvironment{changemargin}[3]{%
\begin{list}{}{%
\setlength{\leftmargin}{#1}%
\setlength{\rightmargin}{#2}%
\setlength{\topmargin}{#3}%
}%
\item[]}
{\end{list}}

% session constructors
\newcommand{\intk}{\keyword{int}}
\newcommand{\skipk}{\keyword{skip}}

% The language
\newcommand{\freest}{\textsc{FreeST}}

% notes
\newcommand{\todo}[1]{[{\color{blue}\textbf{#1}}]}

% Keywords
\newcommand{\keyword}[1]{\mathsf{#1}}
\newcommand{\link}{\keyword{lin}}
\newcommand{\unk}{\keyword{un}}

% Kinds
\newcommand\prekind{\upsilon}
\newcommand{\stypes}{\mathcal S}
\newcommand\kinds{\stypes}
\newcommand{\types}{\mathcal T}
\newcommand\kindt{\types}
\newcommand\kindsch{\mathcal C}
\newcommand\kind{\kappa}

% Multiplicity
\newcommand\Un{\ensuremath{\mathbf{u}}} % \infty
\newcommand\Lin{\ensuremath{\mathbf{l}}} % 1 

% Grammars
\newcommand{\grmeq}{\; ::= \;}
\newcommand{\grmor}{\;\mid\;}

% type constructors
\newcommand\tcBase{B}
\newcommand\tcLolli\multimap
\newcommand\tcFun\to
\newcommand\tcBang{\mathop!}

% Keywords for types
\newcommand\kRec{\keyword{rec}}
\newcommand\kForall{\keyword{forall}}

% Types
\newcommand{\tskip}{\keyword{Skip}}
\newcommand\tSemi[2]{#1;#2}
\newcommand\tOut[1]{\tcBang#1}
\newcommand\tIn[1]{?#1}
\newcommand{\tMsg}[1]{\sharp{#1}}
\newcommand\tIChoice[1]{\oplus{#1}}
\newcommand\tEChoice[1]{\&{#1}}
\newcommand{\tChoice}[1]{\star{#1}}
\newcommand{\tData}[1]{[{#1}]}
\newcommand\tUnFun[2]{#1\tcFun#2}
\newcommand\tLinFun[2]{#1\tcLolli#2}
\newcommand\tPair[2]{(#1,\,#2)}
\newcommand\tDatatype[1]{{[#1]}}
\newcommand\tRec[2]{\kRec\,#1\,.\,#2}
%\newcommand\tForall[2]{\forall\,#1\,.\,#2}
%\newcommand\tForall[2]{\kForall\,#1\,=>\,#2}
\newcommand\tForall[2]{\forall\,#1\Rightarrow#2}
% Basic Types
\newcommand{\unite}{()}
\newcommand{\inte}{\keyword{Int}}
\newcommand{\chare}{\keyword{Char}}
\newcommand{\boole}{\keyword{Bool}}

\newcommand\tRecK[2]{\kRec\,#1\,.\,#2}
% Environments
\newcommand{\Empty}{\varepsilon}
\newcommand\emptyEnv{\Empty}
\newcommand\kindEnv{\Delta}
\newcommand\varEnv{\Gamma}

% Variables
\newcommand\vare[1]{#1}
\newcommand\unlete[3]{\keyword{let} \; #1 = #2 \; \keyword{in} \; #3} 

% Applications
\newcommand\appe[2]{#1#2}
\newcommand\tappe[2]{#1[#2]}

% Conditional
\newcommand\conditionale[3]{\keyword{if}\;#1\;\keyword{then}\;#2\;\keyword{else} \; #3}

% Goal
\newcommand\Goal{\vdash}

% Pairs
\newcommand\paire[2]{(#1,#2)}
\newcommand\binlete[4]{\keyword{let}\;#1, #2 = #3\;\keyword{in}\;#4}

% Session Types
\newcommand\newe[1]{\keyword{new}\;#1}
\newcommand\sende[2]{\keyword{send}\;#1\; #2}
\newcommand\sendce[1]{\keyword{send}\;#1}
\newcommand\recve[1]{\keyword{receive}\;#1}
\newcommand\selecte[2]{\keyword{select}\;#1\;{#2}}
\newcommand\matche[2]{\keyword{match}\;#1\;\keyword{with}\;#2}

% Fork
\newcommand\forke[1]{\keyword{fork}\;#1}

% Datatypes
\newcommand{\ctrcte}{C}
\newcommand\casee[2]{\keyword{case}\;#1\;\keyword{of}\;#2}

% Sequents
\newcommand{\isType}[3][\Delta]{{#1} \vdash {#2} : {#3}}
\newcommand{\algkindout}[3][\kindEnv]{{#1} \Alg {#2} \shortrightarrow{ #3}}
\newcommand{\algkindin}[3][\kindEnv]{{#1} \Alg {#2} \shortleftarrow {#3}}
\newcommand{\subkind}[2]{{#1} <: {#2}}
\newcommand{\algtypeout}[4][\kindEnv;\varEnv]{{#1} \Alg {#2} \shortrightarrow {#3};{#4}}
%\newcommand{\algtypein}[4][\kindEnv;\varEnv]{{#1} \Alg {#2}\colon {#3}\shortrightarrow {#4}}
\newcommand{\algtypein}[4][\kindEnv;\varEnv]{{#1} \Alg {#2}\shortleftarrow {#3}; {#4}}
\newcommand{\ctxequiv}[3][\kindEnv]{{#1} \vdash \Equiv{#2}{#3}}
\newcommand{\typeequiv}[3][\kindEnv]{{#1} \vdash \Equiv{#2}{#3}}
\newcommand{\isqualifier}[3][\kindEnv]{{#1} \vdash {#2}\colon{#3}}
\newcommand{\isLin}[2][\kindEnv]{\isqualifier[#1]{#2}\link}
\newcommand{\isUn}[2][\kindEnv]{\isqualifier[#1]{#2}\unk}
\newcommand{\contractive}[2][\kindEnv]{{#1} \vdash_{\textsf c} {#2}}
%\newcommand\Alg{\vdash_{\textsf a}}
\newcommand\Alg{\vdash}

% Operators
\newcommand\Extract[1]{\leadsto_{#1}}% \rightlsquigarrow}
\newcommand{\subs}[3]{[{#1}/{#2}]{#3}}
\newcommand\dual[1]{\overline{#1}}

% Predicates
%\newcommand\Equiv[2]{#1\,\thicksim\,#2}
\newcommand\Equiv[2]{#1\,\sim\,#2}

% Colour

\newcommand{\Blue}[1]{\textcolor{blue}{#1}}
\newcommand{\Red}[1]{\textcolor{red}{#1}}
\newcommand{\Brown}[1]{\textcolor{brown}{#1}}
\newcommand{\highlight}[1]{\Blue{#1}}

% ECLIPSE LOOK

\newcommand\Small{\small}
%\newcommand\Small{\fontsize{7.5}{8}\selectfont} 

\definecolor{darkviolet}{rgb}{0.5,0,0.4}
\definecolor{darkgreen}{rgb}{0,0.4,0.2} 
\definecolor{darkblue}{rgb}{0.1,0.1,0.9}
\definecolor{darkgrey}{rgb}{0.5,0.5,0.5}
\definecolor{lightblue}{rgb}{0.4,0.4,1}

\lstdefinestyle{eclipse}{
  breaklines=true,
  basicstyle=\sffamily\Small,
  emphstyle=\color{red}\bfseries, 
  keywordstyle=\color{darkviolet}\bfseries,
  commentstyle=\color{darkgreen},
  stringstyle=\color{darkblue},
  numberstyle=\color{darkgrey},%\lstfontfamily,
  emphstyle=\color{red},
  % get also javadoc style comments
  morecomment=[s][\color{lightblue}]{/**}{*/},
  %columns=fullflexible, %spaceflexible, %flexible, fullflexible             
  %  escapeinside=`',
  %  escapechar=@,
  showstringspaces=false,
  numbers=left,
  tabsize=2
}

\lstdefinestyle{eclipse-Haskell}{
  breaklines=true,
  basicstyle=\sffamily\Small,
  emphstyle=\color{red}\bfseries, 
  keywordstyle=\color{darkviolet}\bfseries,
  commentstyle=\color{darkgreen},
  stringstyle=\color{darkblue},
  emphstyle=\color{red},
  % get also javadoc style comments
  morecomment=[s][\color{lightblue}]{/**}{*/},
  %columns=fullflexible, %spaceflexible, %flexible, fullflexible             
  %  escapeinside=`',
  %  escapechar=@,
  showstringspaces=false,
  numbers=none,
  tabsize=2
}

\lstdefinelanguage{freest}{
  style=eclipse,
  morekeywords=[1]{Int, Char, Bool, Skip, type, dualof, forall, rec, let, in, if, then, else, new, send, receive, select, fork, case, of, data, match, with, True, False},
  sensitive=true,
  literate=
  {->}{$\rightarrow$}2
  {-o}{$\multimap$}2
  {=>}{$\Rightarrow$}2
  {alpha}{$\alpha$}1,
  breaklines=true,
  morecomment=[l]{--},%
  morecomment=[s]{{-}{-}},%
  morestring=[b]',%
  morestring=[b]",%
  morestring=[s]{`}{`},%
}

\lstset{
  language=freest,
  numbers=none
}
 
%%% Local Variables:
%%% mode: latex
%%% TeX-master: "main"
%%% End:


\begin{document}

\title{Deciding the bisimilarity of context-free session types
  % \thanks{This work was supported by FCT via project Confident
  % (PTDC/EEI-CTP/ 4503/2014) and the LASIGE Research Unit
  % (UID/CEC/00408/2019).} % isto está nos ACK.
}

%optional, please use if title is longer than one line
%\titlerunning{Deciding the bisimilarity of context-free session types}

\author{Bernardo Almeida\inst{1} \and
Andreia Mordido\inst{1} \and
Vasco T. Vasconcelos\inst{1}}

%\author{Bernardo Almeida}{LASIGE, Faculdade de Ciências, Universidade de Lisboa, Portugal}{balmeida@lasige.di.fc.ul.pt}{[funding]}{https://orcid.org/0000-0001-5398-6529}
%
%\author{Andreia Mordido}{LASIGE, Faculdade de Ciências, Universidade de Lisboa, Portugal}{afmordido@fc.ul.pt}{[funding]}{https://orcid.org/0000-0002-1547-0692}
%
%\author{Vasco T. Vasconcelos}{LASIGE, Faculdade de Ciências, Universidade de Lisboa, Portugal}{vmvasconcelos@fc.ul.pt}{[funding]}{https://orcid.org/0000-0002-9539-8861}

\institute{LASIGE, Faculdade de Ciências, Universidade de Lisboa, Portugal}

%TODO mandatory, please use full name; only 1 author per \author macro; first two parameters are mandatory, other parameters can be empty. Please provide at least the name of the affiliation and the country. The full address is optional

\authorrunning{B. Almeida, A. Mordido and V.\,T. Vasconcelos}
%
%\category{}%optional, e.g. invited paper
%
%
%\supplement{}%optional, e.g. related research data, source code, ... 
%hosted on a repository like zenodo, figshare, GitHub, ...

%\acknowledgements{I want to thank \dots}%optional

\maketitle

\begin{abstract}
  URL of the artifact: \url{http://www.di.fc.ul.pt/~vv/}
  \par
  We present an algorithm to decide the equivalence of context-free
  session types, practical to the point of being incorporated in a
  compiler. We prove its soundness and completeness. We further
  evaluate its behaviour in practice. In the process, we introduce an
  algorithm to decide the bisimilarity of simple grammars.

  % mandatory; please add comma-separated list of keywords
  \keywords{Types, Type equivalence, Bisimulation, Algorithm} 
\end{abstract}

\section{Introduction}
\label{sec:introduction}

Session types came to enhance the expressivity of traditional types, enabling to express structured communication on heterogeneously typed channels~\cite{DBLP:conf/concur/Honda93,DBLP:conf/parle/TakeuchiHK94}. Session types have been extended to deal with many realistic situations, e.g., multi-party session types~\cite{DBLP:conf/popl/HondaYC08}, session types for distributed object-oriented programming~\cite{DBLP:conf/popl/GayVRGC10}, session types for programming web services~\cite{DBLP:journals/toplas/CarboneHY12}. Thiemann and Vasconcelos~\cite{thiemann2016context} proposed {\it context-free session types} as an extension of session types by allowing nested protocols that are not restricted to tail recursion. Context-free session types capture the type-safe serialization of recursive datatypes and enable the type-safe implementation of remote operations on recursive datatypes. 

Inspired by the context-free session types' framework, Almeida and
Vasconcelos~\cite{bernardo} proposed a functional programming language
with context-free session types. The usability of such a programming
language is highly dependent on an algorithm to decide type
equivalence. We have developed and implemented an algorithm to decide
type equivalence. Our work capitalizes on the metatheory of
context-free session types proposed by Thiemann and
Vasconcelos~\cite{thiemann2016context}, where type equivalence was
proved to be decidable. Although the decidability of equivalence on
context-free session types has been addressed in the
literature~\cite{DBLP:journals/iandc/ChristensenHS95,janvcar1999techniques,thiemann2016context},
to the best of our knowledge, an algorithm was not yet implemented and
a possible implementation was not obvious.

This document summarizes an ongoing work whose main contributions
stand on the proposal and implementation of an algorithm to decide
type equivalence of context-free session types, prove the soundness of
the algorithm against the declarative definition
in~\cite{thiemann2016context}, study and possibly prove the
completeness of the algorithm, and validate the algorithm on several
meaningful examples.


%%% Local Variables:
%%% mode: latex
%%% TeX-master: "main"
%%% End:

\section{Context-free session types}
\label{sec:contextfreesession}

This section briefly introduces context-free session types, based on
Thiemann and Vasconcelos~\cite{thiemann2016context}.
%
The types we consider build upon a denumerable set of \emph{variables}
denoted by $X,Y,Z$ and a set \emph{choice labels} denoted by
$\ell$. We assume given a set of base types denoted by~$B$.
% Further base types could include integer and boolean types, or any
% other type for which equality is decidable.
The syntax of types is given by the grammar below.
%
\begin{gather*}
  S,T \grmeq \skipk \grmor \sharp B \grmor 
  \star\{\ell_i\colon S_i\}_{i\in I} \grmor S;T \grmor \mu X.T \grmor X
  \\
  \sharp \grmeq {}! \grmor {}? 
  \qquad \qquad
  \star  \grmeq \oplus \grmor {}\&
  \qquad \qquad
  a \grmeq \sharp B \grmor \star l %\grmor \alpha
\end{gather*}

In type $\mu X.T$, variable $X$ is bound in the subterm~$T$. The sets
of bound and free variables in a given type are defined
accordingly. Notation $\subs{T}{X}S$ denotes the resulting of
substituting the (free) occurrences of $X$ by $T$ in $S$.

Judgement $\DONE S$ characterizes \emph{terminated} types:
context-free session types that exhibit no further
action~\cite{DBLP:journals/jacm/AcetoH92}.

\noindent The terminated predicate\hfill\fbox{$\DONE T$} 
%
\begin{gather*}
  \DONE{\skipk}
  \qquad
  \frac{\DONE{S} \quad \DONE{T}}{\DONE{S; T}}
  \qquad
  \frac{\DONE S}{\DONE{\mu X.S}}
\end{gather*}

We are not interested in all types generated by the above grammar.
%
If $\Delta$ is a list of pairwise distinct variables, then judgement
$\isType T$ characterises the types of interest: the
\emph{well-formed} types.

\noindent The type formation system \hfill\fbox{$\isType T$}
%
\begin{gather*}
  \frac{} 
  {
    \isType \skipk
  }
  \qquad
  \frac{}
  {
    \isType {\sharp B}
  }
  \qquad
  \frac{
    X \in \Delta
  }{
    \isType{X}
  }
  \\
  \frac{
    \isType{S}
    \quad
    \isType{T}
  }{
    \isType{S;T}
  }
  \qquad
  \frac{
    \isType{T_i}
  }{
    \isType{\star 
      \{ l_i \colon T_i\}}
  }
  \qquad
  \frac{
    % \isContr T
    \neg\DONE T
    \quad 
    \isType [\Delta, X] T
  }{
    \isType {\mu X. T}
  }
\end{gather*}
%
Type formation serves two main purposes: that all variables introduced
by $\mu$-types are pairwise distinct, and that types underneath
a~$\mu$ are not terminated. This can be clearly seen by formation rule
for $\mu$-types, where notation $\Delta,X$ requires $X\notin\Delta$.
%
In the sequel we assume that all types are such that $\isType[] T$.

Denote by $\mathcal{T}$ the set of well-formed types.
%
The \emph{labelled transition system} (LTS) for context-free session
types is given by set $\mathcal{T}$ as \emph{states}, the set
generated by the grammar for~$a$ as \emph{actions}, and the
\emph{transition relation} $S\LTSderives T$ defined by the rules below.

\noindent The labelled transition system\hfill\fbox{$S \LTSderives T$}
%
\begin{gather*}
  \sharp B \LTSderives[\sharp B] \skipk
  \qquad
  \star\{l_i\colon S_i\}_{i\in I} \LTSderives[\star l_j] S_j\quad
  (j\in I)
  \\
  \frac{S \LTSderives S'}{S; T \LTSderives S';T}
  \qquad
  \frac{\DONE{S} \quad T \LTSderives T'}{S; T \LTSderives T'}
  \qquad
  \frac{\subs{\mu X.S}{X}S \LTSderives T}{\mu X.S \LTSderives T}
\end{gather*}

\emph{Type bisimulation}, $\TypeEquiv$, is defined in the usual way from the
labelled transition system~\cite{sangiorgi2014introduction}.
%
We say that a type relation $\mathcal R$ is a \emph{bisimulation} if,
whenever $S\mathcal RT$, for all~$a$ we have:
%
\begin{itemize}
\item for all $S'$ with $S \LTSderives S'$, there is $T'$ such that $T
  \LTSderives T'$ and $S'\mathcal RT'$, and
\item for all $T'$ with $T \LTSderives T'$, there is $S'$ such that $S
  \LTSderives S'$ and $S'\mathcal RT'$.
\end{itemize}
%
We say that two types are bisimilar, written $S \bisim T$, if there
is a bisimulation~$\mathcal R$ with $P\mathcal RQ$.

%%% Local Variables:
%%% mode: latex
%%% TeX-master: "main"
%%% End:

\section{An algorithm to decide type bisimilarity}
\label{sec:algorithm}

% Recall the type bisimulation problem for context-free session types.

% \begin{quote}
%   Given context-free session types $S$ and $T$, the type equivalence
%   problem consists in deciding if types $S$ and $T$ are equivalent,
%   i.e., $S \TypeEquiv T$.
% \end{quote}

This section presents an algorithm to decide whether two types are in
a bisimulation relation. In the process we also provide an algorithm
to decide the equivalence of simple context-free languages.
%
The algorithm comprises three stages. It
starts by converting types into grammars and then streamlines the
grammar by pruning unreachable symbols in productions. The last stage
explores an expansion tree, alternating between simplification and
expansion operations, until either finding an empty node---case in
which it decides positively---or failing to expand a node---case in
which it decides negatively.

\subsection{Converting types to grammars}
\label{subsec:typeToGrammar}

A context-free grammar in Greibach normal form is a pair $(X,P)$
where~$X$ is the \emph{start symbol} and~$P$ a \emph{set of
  productions} of the form $Y \rightarrow a\vec Z$ (we do not allow
productions of the form $X \rightarrow\varepsilon$). Type variables
are the \emph{non-terminal symbols} and LTS labels the \emph{terminal
  symbols}. We call \emph{words} to sequences of type
variables~$\vec X$, and denote by~$\varepsilon$ the empty word.
%
The grammars we are interested in are \emph{simple}: for each
non-terminal symbol~$X$ and each terminal symbol~$a$, there is at most
one production of the form
$X \rightarrow a\vec Y$~.

Grammars in Greibach Normal Form naturally induce an LTS by taking
sequences of non-terminal symbols~$\vec X$ as states, terminal
symbols~$a$ as the set of actions, and the transition relation
$\LTSderivesP$ defined as $X\vec Y\LTSderivesP \vec Z\vec Y$ when
$X \rightarrow a\vec Z \in\productions$.
% \vv{reference here}
The associated bisimulation is denoted by $\ProdEquiv$.

Given a context-free session type $S$, the algorithm starts by
inserting an initial production of the form
$X_S \rightarrow \,\initialProd\,(\toGrammarf\,S)$ in the set of
productions.
%
Function \lstinline|toGrammar| (Listing~\ref{lst:toGrammar}) returns a
sequence of non-terminal symbols, while computing the remaining
productions.
%
It uses a predicate \lstinline|isChecked| $S$ to determine whether $S$
is terminated, that is whether \DONE S.
%
The algorithm keeps the set of productions and an integer (to generate
fresh non-terminal symbols) in the monadic state
\lstinline{TransState}. It uses the following functions to manipulate
state.
%
\begin{itemize}
\item \lstinline{freshVar} returns a fresh non-terminal symbol (a type
  variable);
\item \lstinline{addProduction}$\,X\,a\,\vec Y$ updates the state by inserting
  the production $X\rightarrow a\vec Y$;
% \item \lstinline{insertVisited} marks a non-terminal symbol as visited;
% \item \lstinline{isVisited} identifies whether the given non-terminal symbol
%   was previously visited;
% \item \lstinline{subst} $X\,T\,S$ replaces the occurrences of~$X$
%   by~$T$ in~$S$;
\item \lstinline|getTransitions|$\,X$ retrives the transitions
  from~$X$ (a map from non-terminal symbols~$a$ to sequences~$\vec Y$
  of type variables).
\end{itemize}

% In Listing~\ref{lst:toGrammar}, type $\skipk$ is represented by
% \lstinline{Skip}, \lstinline{Message p b} stands for types of the form
% $\sharp B$,
% %\lstinline{VarLabel} stands for variables,
% \lstinline{Semi} represents sequential composition,
% \lstinline{Rec} represents recursive types, and
% \lstinline{Choice} stands for the choice operators $\oplus$ and
% $\&$. \vv{cut? looks obvious, and there is many more related things to say}

\begin{lstlisting}[
  caption={Haskell code for stage 1: the conversion of types into grammars},
  label={lst:toGrammar},
  captionpos=b]
type Transitions = Map.Map LTSLabel [TypeVar]
type Productions = Map.Map TypeVar Transitions
type TransState = State (Productions, Int)
type Substitution = (Type, TypeVar)

convertToGrammar :: TypeEnv -> [Type] -> Grammar
convertToGrammar tEnv ts = Grammar xs (productions state)
  where (xs, state) = runState (mapM typeToGrammar ts) (initial tEnv)

typeToGrammar :: Type -> TransState Word
typeToGrammar t = collect [] t >> toGrammar t

toGrammar :: Type -> TransState Word
toGrammar (Skip _) =
  return []
toGrammar (Semi _ t u) = do
  xs <- toGrammar t
  ys <- toGrammar u
  return $ xs ++ ys
toGrammar (Message _ p b) = do
  y <- getProd $ Map.singleton (MessageLabel p b) []
  return [y]
toGrammar (Choice _ p m) = do
  ms <- tMapM toGrammar m
  y <- getProd $ Map.mapKeys (ChoiceLabel p) ms
  return [y]
toGrammar (TypeVar _ x) = do      
  y <- getProd $ Map.singleton (VarLabel x) []
  return [y]
toGrammar (Rec _ (TypeVarBind _ x _) _) =
  return [x]
  
collect :: [Substitution] -> Type -> TransState ()
collect s (Semi _ t u) = collect s t >> collect s u
collect s (Choice _ _ m) = tMapM_ (collect s) m
collect s t@(Rec _ (TypeVarBind _ x _) u) = do
  let s' = (t, x) : s
  let u' = Substitution.subsAll s' u
  (z:zs) <- toGrammar (normalise Map.empty u')
  m <- getTransitions z
  addProductions x (Map.map (++ zs) m)
  collect s' u
collect _ _ = return ()

addProductions :: TypeVar -> Transitions -> TransState ()
addProductions x m =
  modify $ \s -> s {productions = Map.insert x m (productions s)}

getProd :: Transitions -> TransState TypeVar
getProd ts = do
  ps <- getProductions
  case reverseLookup ts ps of
    Nothing -> do
      y <- freshVar
      addProductions y ts
      return y
    Just x ->
      return x
  where fold x ts' acc = if prodExists ts' ts then Just x else acc
\end{lstlisting}

%%% Local Variables:
%%% mode: latex
%%% TeX-master: "main"
%%% End:

% Old version:
% toGrammar (Rec x t) =
%   | isChecked (Rec x t) = return []
%   | otherwise = do
%     zs <- toGrammar t
%     m <- getTransitions (head zs)
%     addProductions x (Map.map (++ tail zs) m)
%     return [x]


Notice that function \lstinline|toGrammar| terminates on all inputs and
that the resulting set of productions is finite, because recursion is
always on subterms.
%
Furthermore, due to the deterministic nature of the LTS,
\lstinline|toGrammar| returns a simple grammar.
%
% , context-free session types are always converted into \emph{simple
%   grammars}~\cite{baeten1993decidability}, i.e., context-free grammars
% in Greibach normal form such that, for each non-terminal symbol $X$
% and terminal symbol $a$, there is at most one production of the form
% $X\rightarrow a \enspace \vec Y$.
%
One can obtain a unique set of productions for two types by
ensuring that fresh variables do not overlap.

\begin{example}
\label{ex:productions}
Consider the following context-free session types:
%
\begin{equation*}
\begin{array}{lll}
    S & \triangleq & (\mu x . \&\{n: x;x;?\intk, \ell: ?\intk\});(\mu z . !\intk ; z;z)\\
    T & \triangleq & (\mu y . \&\{n: y;y, \ell: ?\skipk\};?\intk);(\mu w. !\intk ; w)
\end{array}
\end{equation*}
%
Function \lstinline{toGrammar}, when applied to $S$ and $T$, produces
the following productions.
\begin{center}
  \begin{tabular}{l l}
    \multicolumn{2}{c}{Productions for type $S$}\\ \hline
    $X_S \rightarrow \,\initialProd\, X_1 X_4$ &$X_2 \rightarrow \,? \intk$\\
    $X_1 \rightarrow \& n\, X_1 X_1 X_2$&$X_3 \rightarrow \,? \intk$\\
    $X_1 \rightarrow \& \ell\, X_3$ &$X_4 \rightarrow \,!\intk\, X_4 X_4$\\
  \end{tabular} \qquad
  \begin{tabular}{l l}
    \multicolumn{2}{c}{Productions for type $T$}\\ \hline
    $Y_T \rightarrow \,\initialProd\, Y_1 Y_3 $&$Y_2 \rightarrow \,? \intk$\\
    $Y_1 \rightarrow \& n\, Y_1 Y_1 Y_2 $&$Y_3 \rightarrow \,!\intk\, Y_3$\\
    $Y_1 \rightarrow \& \ell \,Y_2 $ &
  \end{tabular}
\end{center}
\end{example}

\subsection{Pruning unnormed productions}
\label{subsec:prune}

For $\vec a$ a sequence of non-terminal symbols $a_1,\ldots, a_k$
($k\ge1$), write $\vec Y \LTSderivesP[\vec a] \vec Z$ when
$\vec Y \LTSderivesP[a_1] \cdots \LTSderivesP[a_k] \vec Z$.
%
We say that $\vec Y$ is \emph{normed} when
$\vec Y \LTSderivesP[\vec a] \varepsilon$ for some~$\vec a$, and that
$\vec Y$ is \emph{unnormed} otherwise.
%
When $\vec Y$ is normed, the \emph{minimal path} of $\vec Y$ is the
shortest~$\vec a$ such that $\vec Y \LTSderivesP[\vec a]
\varepsilon$.
%
In this case, the \emph{norm} of $\vec Y$, denoted by $|\vec Y|$, is
the length of~$\vec a$.

% Using the same approach as~\cite{DBLP:journals/iandc/ChristensenHS95} on
% the definition of (un)normed processes, we say that a sequence of symbols
% $\vec Y$ is \emph{normed} if there are labels $a_1,\ldots, a_k$
% such that:
% \begin{equation}
% \label{eq:path}
% 	\vec Y \rightarrow a_1\enspace Y_1 \rightarrow \cdots \rightarrow a_k
% 	\rightarrow \varepsilon.
% \end{equation}
% $\vec Y$ is said to be \emph{unnormed} when it is not normed. If $\vec Y$
% is normed, we define its \emph{norm} as:
% \[ | \vec Y | = \underset{k}{\mathsf{min}} \{\vec Y \rightarrow a_1\enspace Y_1
% \rightarrow \cdots \rightarrow a_k \rightarrow \varepsilon \}.\]
% A \emph{minimal path} for a normed sequence of symbols $\vec Y$ is a sequence of
% labels $\vec a = a_1,\ldots,a_k$ as in~\eqref{eq:path} such that
% $k = | \vec Y |$. To ease notation, we will also represent~\eqref{eq:path} as
% $\vec Y \xrightarrow{\vec a} \varepsilon$.

As observed by Christensen H\"uttel, and
Stirling~\cite{DBLP:journals/iandc/ChristensenHS95}, any unnormed
words $\vec Y$ is bisimilar to its concatenation with any other
nonterminal symbols, that is, if $\vec Y$ is unnormed, then
$\vec Y \ProdEquiv \vec YX$.
%
% \begin{equation}
%\label{unnormed}
%\text{ if } \vec Y \text{ is unnormed, then } \vec Y \sim \vec Y X.
%\end{equation}
%
We use this fact
% Upon the definition of the productions underlying the context-free session types,
% our algorithm builds upon~\eqref{unnormed}
to prune out unreachable symbols in unnormed sequences of symbols. The
code is in Listing~\ref{lst:prune}.
%
%\vv{normedWord needs to be rewritten}

\begin{lstlisting}[
  caption={Haskell code for stage 2: pruning unnormed productions},
  label={lst:prune},
  captionpos=b
  ]
prune :: Productions -> Productions
prune p = Map.map (Map.map (pruneWord p)) p

pruneWord :: Productions -> [TypeVar] -> [TypeVar]
pruneWord p = foldr (\x ys -> if normed p x then x:ys else [x]) []

normed :: Productions -> TypeVar -> Bool
normed p x = normedWord p Set.empty [x]

normedWord :: Productions -> Visited -> [TypeVar] -> Bool
normedWord _ _ []     = True
normedWord p v (x:xs) =
  x `Set.notMember` v &&
  any (normedWord p v') (Map.elems (transitions p (x:xs)))
  where v' = if any (x `elem`) (Map.elems (transitions p [x]))
               then Set.insert x v else v
\end{lstlisting}

\begin{example}
  \label{ex:prune}
  Recall Example~\ref{ex:productions} and notice that both
  $X_S$ and
  $Y_T$ are unnormed. We can easily see that the last occurrence of
  $X_4$ in the last production for
  $S$ is unreachable. Hence, by pruning the productions for
  $S$ we get:
  %
  \begin{center}
    \begin{tabular}{l l l}
      \multicolumn{3}{c}{Pruned productions for type $S$}\\ \hline
      $X_S \rightarrow \,\initialProd\, X_1 X_4$ &
      $X_1 \rightarrow \& \ell\, X_3$  &
      $X_1 \rightarrow \& n\, X_1 X_1 X_2$
      \\
      $X_2 \rightarrow \,? \intk$ &
      $X_3 \rightarrow \,? \intk$ &
      $X_4 \rightarrow \,!\intk\, X_4$
    \end{tabular}
  \end{center}
\end{example}

\subsection{Building expansion trees}
\label{subsec:expand}

% We recall that, given two context-free session types $S$ and $T$, our main goal
% is to decide whether these types are equivalent or not. For this purpose,
% the algorithm we propose starts by applying algorithm presented in
% Listing~\ref{lst:toGrammar} to convert $S$ and $T$ into a grammar containing
% the productions derived from them. Afterwards, the algorithm in
% Listing~\ref{lst:prune} is used to streamline the grammars, by pruning
% unnormed sequences of symbols. Throughout this section we focus on the
% third and last step of the algorithm.

We base the third stage of the algorithm on the notion of
\emph{expansion tree} proposed by Jan{\v{c}}ar and
Moller~\cite{janvcar1999techniques}, an adaption of an idea by
Hirshfeld~\cite{hirshfeld1996bisimulation}. We say a set $N'$ of pairs
of words is an \emph{expansion} of $N$ if $N'$ is a minimal set such
that: for every pair $(\vec X, \vec Y) \in N$,
\begin{itemize}
\item if $\vec X \rightarrow a\vec X'$ then
  $\vec Y \rightarrow a\vec Y'$ with $(\vec X',\vec Y')\in N'$;
\item if $\vec Y \rightarrow a\vec Y'$ then
  $\vec X \rightarrow a \vec X'$ with $(\vec X',\vec Y')\in N'$.
\end{itemize}

An \emph{expansion tree} is built from nodes. Children nodes are
obtained by expansion from its parent node. Jan{\v{c}}ar and Moller
observed that expansion alone often leads to infinite trees. We then
alternate between expansion and simplification operations, until
either finding an empty node---case in which we decide equivalence
positively---or failing to expand a node---case in which we decide
equivalence negatively.
%
% The na\"ive proposal for an expansion tree considers that any
% children node is obtained by expansion from its parent
% node. Nevertheless, as Jan{\v{c}}ar and Moller observed, this would
% often lead to infinite expansion trees. Hence, we follow the
% proposal in~\cite{janvcar1999techniques} and let the expansion tree
% alternate between simplification and expansion operations until
% either finding an empty node---case in which we decide equivalence
% positively---or failing to expand a node---case in which we decide
% equivalence negatively.
%
We say that a branch is \emph{successful} if it is infinite or
finishes in an empty node, otherwise it is said to be
\emph{unsuccessful}.

%\paragraph*{Expansion step}
In the \emph{expansion step}, each node $N$ derives a single children
node, obtained as an expansion of $N$. As we are dealing with simple
grammars, no branching is expected in the expansion tree at this
step.
%
% \paragraph*{Simplification step}
The \emph{simplification step} consists on the application of the
following rules:
%
\begin{description}
\item[Reflexive rule:] Omit from a node any reflexive pair;
\item[Congruence rule:] Omit from a node $N$ any pair that belongs to
  the least congruence containing the ancestors of $N$;
  % \item {\bf Basic Process Algebra rules:} create sibling nodes for $N$
  % according to the following rules:
  % \begin{description}
\item[BPA1 rule:] If $(X_0 \vec X, Y_0 \vec Y)$ is in
  $N$ and $(X_0 \vec {X'}, Y_0 \vec {Y'})$ belongs to the ancestors of
  $N$, then create a sibling node for $N$ replacing
  $(X_0 \vec X, Y_0 \vec Y)$ by $(\vec X, \vec {X'})$ and
  $(\vec Y, \vec {Y'})$;
\item[BPA2 rule:] If $(X_0 \vec X, Y_0 \vec Y)$ is in $N$
  and $X_0$ and $Y_0$ are normed, then:
  \begin{description}
  \item[Case] $|X_0| \leq |Y_0|$: Let $\vec a$ be a minimal path
    for $X_0$ and $\vec Z$ a word such that
    $ Y_0 \LTSderivesP[\vec a] \vec Z$. Add a sibling node for
    $N$ including the pairs $(X_0 \vec Z, Y_0)$ and
    $(\vec X, \vec Z \vec Y)$ in place of $(X_0 \vec X, Y_0 \vec Y)$;
  \item[Case] $|X_0| > |Y_0|$: Let $\vec a$ be a minimal path for
    $Y_0$ and $\vec Z$ a word such that
    $ X_0 \LTSderivesP[\vec a] \vec Z$. Add a sibling node for $N$
    including the pairs $(X_0 , Y_0 \vec Z )$ and
    $(\vec Z\vec X, \vec Y)$ in place of $(X_0 \vec X, Y_0 \vec Y)$.
  \end{description}
%		  \end{description}
%	\item {\bf Filtering rule:} remove any node containing a pair
%	       $(\vec X, \vec Y)$ such that $|\vec X|\neq |\vec Y|$.
\end{description}

Contrarily to expansion and to the reflexive and congruence
simplifications, \BPA\ rules promote branching in the expansion
tree. The number of children nodes generated by these rules is
finite~\cite{DBLP:journals/iandc/ChristensenHS95}.
%
Notice that the sibling nodes do not exclude the (often) infinite
branch resulting from successive expansions.

\subsection{Checking the bisimilarity of context-free session types}

% The algorithm to decide the equivalence of context-free session
% types capitalizes on the previous algorithms.

Given two context-free session types, function \lstinline|bisimilar|
(in Listing~\ref{lst:algorithm}) starts by converting the two session
types into a grammar, which is then pruned. Function
\lstinline|convertToGrammar| (not shown) builds the initial monadic
state, and runs the algorithm of section~\ref{subsec:typeToGrammar} to
convert the session types given as parameters.
%
An expansion tree is computed afterwards, through an alternation of
expansion of children nodes and their simplification, using the
reflexive, congruence, and \BPA\ rules.
%
To avoid getting stuck in an infinite branch of the expansion tree, we
use a breadth-first search on the expansion tree. Upcoming nodes are
stored in a queue.
%
The simplification stage distinguishes the case where all type
variables are normed, in which case \BPA1 is not required to decide
equivalence~\cite{caucal1986decidabilite,DBLP:journals/iandc/ChristensenHS95},
from the case where some type variables might be unnormed. The
recursive procedure terminates as soon as all nodes fail to expand
and, thus, the queue is empty, case in which the algorithm returns
\lstinline|False|, or an empty node is reached, case is which the
algorithm returns \lstinline|True|.

\begin{lstlisting}[
  caption={Haskell code for checking the bisimilarity of context-free
    session types},
  label={lst:algorithm},
  captionpos=b
  ]
type Node = Set.Set ([TypeVar], [TypeVar])
type Ancestors = Node
type NodeQueue = Queue.Queue (Node, Ancestors)
type NodeTransformation = Productions -> Ancestors -> Node -> Set.Set

bisimilar :: Type -> Type -> Bool
bisimilar t u = expand (prune p) [x] [y]
  where Grammar [x, y] p = convertToGrammar [t, u]

expand :: Productions -> NodeQueue -> Bool
expand ps q
  | Queue.null q = False 
  | Set.null n   = True
  | otherwise    = case expandNode ps n of
      Nothing -> expand ps (Queue.dequeue q)
      Just n' -> expand ps (simplify ps n' (Set.union a n) (Queue.dequeue q))
  where (n,a) = Queue.front q

simplify :: Productions -> Node -> Ancestors -> NodeQueue -> NodeQueue
simplify ps n a q = foldr Queue.append q nas'
  where nas  = Set.singleton (n,a)
        nas' = if allNormed ps
               then foldr (apply ps) nas [reflex,congruence,bpa2]
               else foldr (apply ps) nas [reflex,congruence,bpa1,bpa2]
\end{lstlisting}

\begin{example}
  The expansion tree for our running example is
  in Figure~\ref{fig:expansionTree}. Once a successful
  branch is reached (the $\checkmark$ in the figure), the algorithm in
  Listing~\ref{lst:algorithm} decides that $S\TypeEquiv T$.
\end{example}

\begin{figure}[h]
\centering
	\includegraphics[width=8.5cm]{expansionTree}
	\caption{Expansion tree for the context-free session types $S$ and $T$
	introduced in Example~\ref{ex:productions}}
	\label{fig:expansionTree}
\end{figure}

%%% Local Variables:
%%% mode: latex
%%% TeX-master: "main"
%%% End:

\section{Soundness and Completeness}
\label{sec:soundness}

In this section we prove that our algorithm is sound and complete 
with respect to the meta-theory of context-free session types proposed 
by Thiemann and Vasconcelos~\cite{thiemann2016context}.

We start by showing that the bisimulation relation proposed by 
Thiemann and Vasconcelos, $\TypeEquiv$, is equivalent to the 
bisimulation relation obtained from the productions, $\ProdEquiv$. 
Then, based on results from Christensen, H{\"{u}}ttel, 
Stirling~\cite{DBLP:journals/iandc/ChristensenHS95}, Jan{\v{c}}ar 
and Moller~\cite{janvcar1999techniques}, we conclude that our algorithm 
is sound and complete.

\subsection{The bisimilarities coincide}

To ease notation, we will represent $\mathsf{toGrammar}
(\mathsf{freshVar}(\,),S)$ as $\mathsf{toGrammar}(S)$ whenever 
the fresh variable is not relevant in the context.

In the following lemma we prove that the initial (dummy) production 
does not affect equivalence checking.
\begin{lemma}
	Given two session types $S_1, S_2$, 
	\[ X_{S_1} \sim X_{S_2}  \text{ if and only if } 
	\mathsf{toGrammar}(S_1) \sim \mathsf{toGrammar}(S_2).\]
\end{lemma}

\begin{proof}
	By~\eqref{initial_prod}, we have:
	\[ X_{S_1}  \rightarrow \enspace \initialProd\enspace \mathsf{toGrammar}(S_1) 	\]
    \[ X_{S_2}  \rightarrow \enspace \initialProd\enspace \mathsf{toGrammar}(S_2) 	\]

Hence, $X_{S_1}\sim X_{S_2}$ if and only if 
$\mathsf{toGrammar}(S_1) \sim \mathsf{toGrammar}(S_2)$.
\end{proof}

Now we prove that any transition in the \LTS\ has a 
corresponding transition derived from the set of productions.

\begin{lemma}
Given session types $S,S'$ and a label $\ell$,
	\[ \text{if } S \LTSderives[\ell] S' \text{ then } 
	\mathsf{toGrammar}(S) \rightarrow \enspace \ell \enspace \vec Y, \]
	where $\mathsf{toGrammar}(S')$ is prefix of $\vec Y$.
\end{lemma}

\begin{proof}
The proof is done by induction on the structure of the labelled 
transition system (Fig.~\ref{lts}):
\begin{itemize}
	\item If $S\triangleq \skipk$, then $S$ is a terminated 
	      session type, with no further transition. Similarly, 
	      $\mathsf{toGrammar}(S)$ returns $\varepsilon$ and 
	      does not add any production.
	\item If $S\triangleq A$, where $A$ ranges over $!B$ and $?B$, 
	      then $S  \LTSderives[A] \skipk$. On the other hand, 
	      $\mathsf{toGrammar}(X,S)$ inserts a production $X\rightarrow A$, 
	      i.e., $X\rightarrow A \enspace  \mathsf{toGrammar}(\skipk).$  
	\item If $S \triangleq \alpha$, then $S   \LTSderives[\alpha] \skipk$. 
	      On the other hand, since $\alpha$ does not contain non-terminal 
	      symbols, $\mathsf{toGrammar}(X,S)$ inserts a production 
	      $X\rightarrow \alpha$, i.e.,  $X\rightarrow \alpha \enspace 
	      \mathsf{toGrammar}(\skipk).$
	\item If $S\triangleq \star\{l_i\colon S_i\}_{i\in I}$ then, for each 
          $j\in I$, $S \LTSderives[\star l_j] S_j$. On the other hand, 
          for each $j\in I$, $\mathsf{toGrammar}(X,S)$ inserts a production 
          $X\rightarrow \star l_j \enspace \mathsf{toGrammar}(S_j)$.
	\item If $S\triangleq \mu x.S$ and $S[\mu x.S/x] \LTSderives[a] S'$, 
	      then $\mu x.S \LTSderives[a] S'$. By induction hypothesis, 
	      $\mathsf{toGrammar}(X,\mathsf{subst}(x,X,S))$ adds the production 
	      $X \rightarrow a\enspace  \mathsf{toGrammar}(S')$.
	\item If $S\triangleq T;U$ and $T \LTSderives[a] T'$ then $S \LTSderives[a] T';U$. 
	      By induction hypothesis, $\mathsf{toGrammar}(X,T)$ adds the production 
	      $X\rightarrow a \enspace \mathsf{toGrammar}(T')$.
		Denoting by 
		\begin{align*}
			\vec Y_T &\leftarrow \mathsf{toGrammar}(X,T)\\
			\vec Y_U &\leftarrow \mathsf{toGrammar}(Y,U)
		\end{align*}
		we notice that $\vec Y_T = X \enspace \vec Y_T'$ for some 
		sequence of non-terminal symbols $\vec Y_T'$. By congruence, we have: 
		\[\mathsf{toGrammar}(S) = X \enspace \vec Y_T' \enspace \vec Y_U' 
		\rightarrow a \enspace \mathsf{toGrammar}(T') \enspace \vec  Y_T' 
		\enspace \vec Y_U.\]
	\item If $S\triangleq T;U$, $T$ is a terminated session, with no further 
	      action, and $U \LTSderives[a] U'$ then, by induction hypothesis, 
	
	    \begin{tabular}{lll}
			$\mathsf{toGrammar}(T)$ & $= \varepsilon$\\
			$\mathsf{toGrammar}(X,U)$ & adds a production 
			$X \rightarrow a \enspace \mathsf{toGrammar}(U')$
		\end{tabular}\\\\
		then, considering $\vec X_U \leftarrow \mathsf{toGrammar}(X,U)$, 
		we notice that $X_U = X \enspace \vec X_U'$ for some sequence of 
		non-terminal symbols $\vec X_U'$. Hence, by congruence, we have a 
		transition 
		\[\mathsf{toGrammar}(S) = \enspace \varepsilon \enspace X \enspace 
		\vec X_U' \rightarrow \enspace a \enspace \mathsf{toGrammar}(U') \enspace 
		\vec X_U'.\]
\end{itemize}
\end{proof}

Conversely, we prove that any transition derived from the productions 
has a corresponding labelled transition in the \LTS.

\begin{lemma}
Given a session type $S$ and a label $\ell$,
	\[ \text{if } \mathsf{toGrammar}(S) \rightarrow \enspace \ell \enspace 
	 \vec Y \text{ then } S \LTSderives[\ell] S', \text{for some $S'$}.\]
\end{lemma}

\begin{proof}
	The proof is by induction on the structure of $S$:
	\begin{description}
		\item[Case $S \triangleq \skipk$:] $\mathsf{toGrammar}(S)$ does not 
		     add any production, and \DONE{S}.
		\item[Case $S \triangleq A$:] $\mathsf{toGrammar}(Y,S)$ adds the 
		     production $Y\rightarrow A$ and, in the LTS, $S \LTSderives[A] 
		     \skipk$, where $A$ ranges over $!B$, $?B$, and $\alpha$.
		\item[Case $S\triangleq \star \{\ell_i : S_i\}_{i\in I}$:] for each 
		     $j\in I$, $\mathsf{toGrammar}(Y,S)$ adds a production 
		     \linebreak $Y \rightarrow \star \ell_j \enspace 
		     \mathsf{toGrammar}(S_j)$. In the \LTS\ we also have 
		     $S_1 \LTSderives[\star \ell_j] S_j$, for each $j\in I$.
		\item[Case $S\triangleq \mu x.S$:] $\mathsf{toGrammar}(\_,S)$ adds, 
		     recursively, all productions from \linebreak$\mathsf{toGrammar}
		     (\mathsf{subst}(x,Y,S))$. Analogously, in the \LTS\ side, any transition 
		     $S[\mu x.S/x] \LTSderives[a] S'$ leads to a transition 
		     $\mu x.S \LTSderives[a] S'$.
		\item[Case $S \triangleq T;U$:] $\mathsf{toGrammar}(S)$ recursively adds 
		     all productions from  $\mathsf{toGrammar}(T)$ and $\mathsf{toGrammar}(U)$.
		     By induction hypothesis, all these productions have corresponding 
		     transitions in the \LTS.
	\end{description}
\end{proof}

Having proved that any labelled transition in the LTS has a corresponding
transition in the grammars and vice-versa, the following theorem is now 
immediate.

\begin{theorem}
\label{cfst_vs_grammar}
	Given two context-free session types $S_1, S_2$,
	\[ S_1 \TypeEquiv S_2 \text{ if and only if } X_{S_1} \ProdEquiv X_{S_2}. \]
\end{theorem}

\subsection{\textit{Unnormedness} is preserved}

To prove that the pruning stage is in accordance with the results 
from Christensen et al., we now observe that unnormed types have 
unnormed non-terminal symbols and vice-versa. These results follow 
immediately from the previous results. 

\begin{corollary}
	Given a context-free session type $S$, $|S| = |\mathsf{toGrammar}(\_,S)|$.
\end{corollary}

\begin{corollary}
	A context-free session type $S$ is unnormed if and only if 
	$X_S$ is unnormed.
\end{corollary}

\subsection{The expansion tree is correct}

Let us start by proving a small lemma, whose ultimate purpose 
stands on proving that all nodes excluded by the filtering rule
would lead to unsuccessful branches.

\begin{lemma}
	Let $(\vec X, \vec Y)$ be a pair in node $N$ of the expansion tree. 
	If $|\vec X| \neq |\vec Y|$ then  $\vec X \not\ProdEquiv \vec Y$.
\end{lemma}

\begin{proof}
	Assume that $n = |\vec X| < |\vec Y|$. This means that 
	\[\vec X \rightarrow \ell_1 \vec X_1 \rightarrow \cdots 
	\rightarrow \ell_n \rightarrow \varepsilon.\]
	Now assume that $\vec Y$ has an expansion sequence whose 
	labels coincide with those from $\vec X$. The derivation for $\vec Y$ 
	should be of the form
	\[\vec Y \rightarrow \ell_1 \vec Y_1 \rightarrow \cdots 
	\rightarrow \ell_n \vec Y_n\rightarrow \ell_{n+1} \vec Y_n \rightarrow \cdots\]
	The $(n+1)$-th expansion of $\vec X$ with label $\ell_{n+1}$ would fail, 
	hence $\vec X \not\ProdEquiv \vec Y$.
\end{proof}

The expansion and simplification operations we have presented in 
subsection~\ref{subsec:expand} were proved to preserve the safeness 
property~\cite{janvcar1999techniques}:

\begin{proposition} [Safeness Property~\cite{janvcar1999techniques}]
\label{safeness}
	$\vec X \ProdEquiv \vec Y$ if and only if the expansion tree rooted 
	at $\{(\vec X, \vec Y)\}$ has a successful branch.
\end{proposition}

\begin{proof}
	The reflexive, congruence and \BPA\ rules were already proved to 
	preserve the safeness property~\cite{janvcar1999techniques}.
	It remains to prove that the filtering rule also preserves 
	the safeness property: \todo{AM}{complete}
\end{proof}

The results on the soundness of the algorithm are now straightforward. 

\begin{theorem}
	If Algorithm~\ref{lst:algorithm} returns \textsf{true} on input 
	$(S_1,S_2)$, then $X_{S_1} \ProdEquiv X_{S_2}$.
\end{theorem}

\begin{proof}
	The algorithm returns \textsf{true} on input $(S_1,S_2)$ whenever 
	it reaches a finite successful branch in the expansion tree rooted 
	at $\{(X_{S_1}, X_{S_2})\}$. Since all rules preserve the safeness 
	property, if $\{(X_{S_1}, X_{S_2})\}$ has a (finite) successful 
	branch then $X_{S_1} \ProdEquiv X_{S_2}$.
\end{proof}

Using Theorem~\ref{cfst_vs_grammar}, the soundness of our algorithm is 
immediate:

\begin{theorem}
	Algorithm~\ref{lst:algorithm} is sound with respect to the meta-theory 
	of context-free session types, i.e., if it returns \textsf{true} then $S_1 \TypeEquiv S_2$.
\end{theorem}

Having observed that the safeness property was paramount for soundness, 
we now notice that the \emph{finite witness property} is of utmost 
importance to prove completeness.

\begin{proposition} [Finite Witness Property~\cite{janvcar1999techniques}]
\label{finite_witness}
	If $\vec X \ProdEquiv \vec Y$, then the expansion tree rooted at 
	$\{(\vec X, \vec Y)\}$ has a finite successful branch.
\end{proposition}

\begin{proof}
	The reflexive, congruence and \BPA\ rules were already proved to 
	preserve the finite witness property~\cite{janvcar1999techniques}.
	It remains to prove that the filtering rule also preserves 
	it: \todo{AM}{complete}
\end{proof}

\begin{theorem}
	Algorithm~\ref{lst:algorithm} is complete with respect to the meta-theory 
	of context-free session types, i.e., if $S_1 \TypeEquiv S_2$ then 
	the algorithm returns \textsf{true}.
\end{theorem}

\begin{proof}
	Assuming that $S_1 \TypeEquiv S_2$, by Theorem~\ref{cfst_vs_grammar}, we 
	have $X_{S_1} \ProdEquiv X_{S_2}$. Hence, the finite witness property 
	ensures the existence of a finite successful branch on the expansion 
	tree rooted at $\{(X_{S_1},  X_{S_2})\}$. Since our algorithm traverses 
	the expansion tree using breadth-first search we will, eventually, 
	reach the finite successful branch and conclude the equivalence positively.
\end{proof}

%\section{Optimizations}
\label{sec:optimisations}

Armed with the results in Section~\ref{sec:algorithm}, we decided to
benchmark the algorithm on a test suite of carefully crafted pair of
types (more on this in Section~\ref{sec:evaluation}). During this 
process we came across a pair of types,
\begin{equation}
\label{ex:chaotic}
\begin{aligned}
  S &\triangleq \mu x . \&\{ \mathsf{Add}\colon x;x; !\,\intk,
  \mathsf{Const}\colon ?\,\intk;!\intk,
  \mathsf{Mult}\colon x;x;!\,\intk\}
  \\
  T &\triangleq \mu x . \&\{ \mathsf{Add}\colon x;x,
  \mathsf{Const}\colon ?\,\intk,
  \mathsf{Mult}\colon x;x\}; !\,\intk
\end{aligned}
\end{equation}
%
on which function \lstinline|bisimilar| took 4379.98 seconds (that is
one hour and forty minutes) to terminate. This is certainly not a
reasonable running time for an algorithm to be included in a
compiler. Hence we looked into ways to improve the running time. Among
the different optimisations that we tried, two stand out:
\begin{enumerate}
\item Iterate the simplification stage until a fixed point is reached;
\item Use a double-ended queue where promising children are prepended
  rather than appended.
\end{enumerate}

If, on the one hand, we believed that the computation of the expansion
tree could be speeded up by extending the simplification phase, on the
other hand we suspected that a double-ended queue would allow
prioritizing nodes with potential to reach an empty node faster.
%
Iterating the simplification procedure on a given node $N$, the
algorithm computes the simplest possible children nodes derived from
$N$. Of course, we need to make sure that a fixed-point exists, which
we do with Theorem~\ref{thm:fixed_point}.
%
Using a double-ended queue, the algorithm prepends (rather than
appends) nodes that are already empty or whose pairs $(\vec X, \vec Y)$
are such that $|\vec X|\leq 1$ and $|\vec Y| \leq 1$.
%
The revised \lstinline|simplify| function is in Listing~\ref
{lst:enhanced}.

The next theorem shows that the simplification function that consists
in applying the reflexive, congruence and \BPA\ rules has a fixed
point.  The result applies regardless of whether all nonterminal
symbols symbols are normed or not.

\begin{theorem}
  \label{thm:fixed_point}
  The simplification function that results from applying the
  reflexive, congruence, and \BPA\ rules, has a fixed point in the
  complete partial ordered set
  {\upshape\lstinline|Set (Node, Ancestors)|}, where the set of
  ancestors is supposed to be fixed. % and equal to $A$.
\end{theorem}

\begin{proof}
  Throughout the proof we abuse notation and denote the application
  of simplification rules to nodes and to elements of
  $\text{\lstinline{Set (Node, Ancestors)}}$ similarly, when no
  ambiguity arises.
%
  Consider the order $\leqSets$, defined on
  $\text{\lstinline{Set (Node, Ancestors)}} \times
  \text{\lstinline{Set (Node, Ancestors)}}$, as $S_1 \leqSets S_2$ if
  $|S_1| \leq |S_2|$ and there exists an injective map
  $\sigma : S_1 \rightarrow S_2$ s.t.\ $\sigma(N_1,A) = (N_2,A)$ with
  $N_2\subseteq N_1$.
  % 
  \begin{itemize}
  \item $\leqSets$ is a partial order. The proof that $\leqSets$ is
    reflexive and transitive is straightforward.  To prove that it is
    antisymmetric, assume that $S_1\leqSets S_2$ and
    $S_2 \leqSets S_1$.  This means that $|S_1|=|S_2|$ and,
    furthermore, the maps $\sigma_1 : S_1 \rightarrow S_2$ and
    $\sigma_2 : S_2 \rightarrow S_1$ are bijective. Notice that
    $\sigma_1\circ \sigma_2$ is the identity map, otherwise we could
    consider $(N,A)\in S_2$ where $N$ is minimal w.r.t.\ inclusion and
    s.t.\ $(\sigma_1\circ \sigma_2)(N,A) \neq (N,A)$, i.e.,
    $(\sigma_1\circ \sigma_2)(N,A) = (N',A)$ with $N'\subseteq N$ for
    some $(N',A)\in S_2$; due to the minimality of $N$, we would have
    $(\sigma_1\circ \sigma_2)(N',A) = (N',A)$, which would contradict
    the injectivity of $\sigma_1\circ \sigma_2$. Since
    $\sigma_1 (N,A) = (N',A)$ is such that $N'\subseteq N$, we shall
    have $\sigma_1 (N,A) = (N,A)$. Hence, $S_1=S_2$.
  \item The simplification function is order-preserving.
	To prove that the reflexive rule preserves the order, let
        $S_1$ and $S_2$ be s.t.\ $S_1\leqSets S_2$ and let us prove
        that
        $\text{\lstinline{reflex}}
        S_1\leqSets\text{\lstinline{reflex}}S_2$.
	% Start noticing that $|S| = |\text{\lstinline{reflex}} S|$,
	% hence
	% the number of children nodes is preserved by using the
	% reflexive
	% rule, let us analyze each case.
	Let $(N,A)\in \text{\lstinline{reflex}} S_1$ and notice that
        there exists $(N_1,A)\in S_1$, such that
        $\text{\lstinline{reflex}} N_1 = N $, and so, in $S_2$ there
        is $(N_2,A)=\sigma(N_1,A)$ s.t. $N_2\subseteq N_1$.  Since
        $N_2\subseteq N_1$, we have
        $\text{\lstinline{reflex}} N_2 \subseteq
        \text{\lstinline{reflex}} N_1 = N$.  The same reasoning
        applies to prove that if $S_1\leqSets S_2$ then
        $\text{\lstinline{congruence} }
        S_1\leqSets\text{\lstinline{congruence} }S_2$.
%
	To prove that \lstinline{bpa1} preserves the order,
	note that 
	$S \subseteq \text{\lstinline{bpa1} }S$. Assume that 
	$S_1\leqSets S_2$, let $(N,A)\in \text{\lstinline{bpa1} }S_1$, 
	and denote by $(N_1,A)\in S_1$ and $(\vec X,\vec Y)\in N_1$  
	the node and the pair whose simplification 
	led to $(N,A)$. We know that exists $(N_2,A)\in S_2$
	s.t.\ $N_2 \subseteq N_1$. If $(\vec X,\vec Y)\in N_2$,
	then the \lstinline{bpa1} simplification of $N_2$ with
	the pair $(\vec X,\vec Y)$ generates 
	$(N',A)\in \text{\lstinline{bpa1} }S_2$ such that 
	$N'\subseteq N$. On the other hand, if 
	$(\vec X,\vec Y)\not \in N_2$, then $N_2\subseteq N$ 
	and, since  $S_2 \subseteq \text{\lstinline{bpa1} }S_2$,
	$(N_2,A)\in \text{\lstinline{bpa1} }S_2$ is such that
	$N_2\subseteq N$.
	The same reasoning applies to \lstinline{bpa2}. 
      \end{itemize}
      
      Having proved that each simplification function preserves the
      order, and since the simplification procedure results from the
      successive application of these rules, we have proved that the
      simplification function also preserves the order.\smallskip
      %
      \begin{itemize}
      \item $(\text{\lstinline{Set (Node, Ancestors)}}, \leqSets)$ is
        a lattice. Given
        $S_1, S_2\in \text{\lstinline{Set (Node, Ancestors)}}$,
        $S_1 \cup S_2$ is an upper bound and $S_1 \cap S_2$ is a lower
        bound for $S_1$ and $S_2$.
      \item $(\text{\lstinline{Set (Node, Ancestors)}}, \leqSets)$ is
        a complete lattice. Given
      $\mathcal{B}\subseteq \text{\lstinline{Set (Node, Ancestors)}}$:
      $\bigcup_{S\in \mathcal{B}} S$ is an upper bound and
      $\bigcap_{S\in \mathcal{B}} S$ is a lower bound for the sets in
      $\mathcal{B}$.
    \end{itemize}
    Using Tarski's fixed point theorem~\cite{tarski1955lattice}, we
    conclude that the simplification function has a fixed point in
    $\text{\lstinline{Set (Node, Ancestors)}}$.
\end{proof}

Having proved that the fixed point exists, we can now adapt the
simplification phase to, on the one hand, iterate the simplification
rules until reaching a fixed point and, on the other hand, identify
and prepend promising nodes. An improved version of the
\lstinline|simplify| function (Listing~\ref{lst:algorithm}) is in
Listing~\ref{lst:enhanced}.

\begin{lstlisting}[
  caption={Haskell code for the improved simplification step (replaces
    function \lstinline|simplify| in Listing~\ref{lst:algorithm})},
  label={lst:enhanced},
  captionpos=b]
simplify :: Productions -> Node -> Ancestors -> NodeQueue -> NodeQueue
simplify ps n a q = foldr enqueueNode (Queue.dequeue q) nas
  where nas = findFixedPoint ps (Set.singleton (n,a))

enqueueNode :: (Node,Ancestors) -> NodeQueue -> NodeQueue
enqueueNode (n,a) q
 | maxLength n <= 1 = Queue.prepend (n,a) q
 | otherwise        = Queue.append (n,a) q

findFixedPoint :: Productions -> Set.Set (Node,Ancestors) -> 
                    Set.Set (Node,Ancestors)
findFixedPoint ps nas
  | nas == nas' = nas
  | otherwise   = findFixedPoint ps nas'
  where nas' = if allNormed ps
               then foldr (apply ps) nas [reflex,congruence,bpa2]
               else foldr (apply ps) nas [reflex,congruence,bpa1,bpa2]
\end{lstlisting}

The optimisations we propose aim at improving the performance of the
algorithm, however the branching nature of the expansion tree promotes
an exponential complexity: each simplification step (potentially)
generates a polynomial number of nodes, each of which with linear size
on the size of the input.  In turn, the same simplification phase may,
in the worst case, be iterated a linear number of times on the size of
the input.  For these reasons the complexity turns out to be (at least)
exponential.  Nevertheless, these heuristics seem to work quite well
in practice, as we show in the next section.

%%% Local Variables:
%%% mode: latex
%%% TeX-master: "main"
%%% End:
 
\section{Evaluation}
\label{sec:evaluation}

In this section, we analyze the performance of our algorithm
to check the equivalence of context-free session types. 
To do so, we have implemented the algorithm we have presented
thus far (sketched in Listings~\ref{lst:toGrammar}, 
\ref{lst:prune}, \ref{lst:algorithm}) 
in Haskell, and used Haskel compiler 
Glasgow Haskell Compiler, GHC version 8.6.3, from which we have 
obtained the running times we present throughout this section.
The machine used for the evaluation was a Mac mini at 3,6 GHz Intel 
Core i3 with 8 GB of memory. 

At this stage, we were aiming to analyze the performance of
our type equivalence checking algorithm, when we came across 
a particular test whose context-free session types where:
\begin{align}
	S \triangleq \mu x . \&\{ \mathsf{add}: x;x; !\,\intk,
							  \mathsf{const}: ?\,\intk;!\in\intk,
							  \mathsf{mult}: x;x;!\,\intk\}

\end{align}
\newpage
\section{The bright future of \freest{}}
\label{sec:conclusion}

We have developed a basic compiler for \freest, a concurrent
functional programming with context-free session types, based on the
ideas of Thiemann and Vasconcelos~\cite{DBLP:conf/icfp/ThiemannV16}.
%
There are many possible extensions to the language. We discuss a
few.
%
Support for linear pairs and linear datatypes, as well for polymorphic
datatypes, should not be difficult to incorporate.
%
Because \freest{} compiles to Haskell, a better interoperability is
called for. We plan to add primitive support for lists, and for some
functions in Haskell's prelude, namely rank-1 functions that we will
have to annotate with \freest{} types.
%
We have chosen a buffered semantics with buffers of size one, for ease
of implementation, but we plan to experiment with buffered channels of
arbitrary size by simply replacing the back-end.
%
The original proposal of context-free sessions is based on a
call-by-value operational semantics and we kept that strategy in
\freest. We however plan to experiment with call-by-need, taking
advantage of the back-end in Haskell. 
%
Shared channels allow for multiple readers and multiple writers,
thus introducing (benign) races. There are several proposals in the
literature~\cite{DBLP:journals/pacmpl/BalzerP17,
  DBLP:conf/sefm/FrancoV13,Lindley.Morris_Lightweight.functional.session.types,DBLP:journals/iandc/Vasconcelos12}
on which we may base this extension.
%
The \lstinline|dualof| type operator is present in the SePi
language~\cite{DBLP:conf/sefm/FrancoV13}. Its incorporation in
\freest{} may be complicated by the presence of polymorphic type
variables.

We also plan to extend the expressivity of \freest{} by allowing
messages to convey arbitrary types, as opposed to basic types only.
% session types to be used to send or receive other session types.  For
% this purpose, we intend to enable message operators to be applied not
% only to basic types but to any functional or session type. 
In this wider scope, the type equivalence algorithm for context-free
session types must be intertwined with the type equivalence algorithm
for functional types, for now, the labels of the labeled-transition
system are types themselves.

Last but not least, we plan to incorporate type inference on type
applications in order to allow the automatic identification of the
unifier matching a polymorphic type against a given type. This
unification process should recognize the unifiers of two types up to
type equivalence. However, dealing with type inference on type
applications with recursive types might be challenging, as observed by
Hosoya and Pierce~\cite{DBLP:journals/toplas/PierceT00}.
%in~\cite{hosoya1999good}.

%%% Local Variables:
%%% mode: latex
%%% TeX-master: "main"
%%% End:


\paragraph{Acknowledgements}
We thank Alcides Fonseca for conducting the testing process and
preparing the diagrams in Section~\ref{sec:evaluation}, and Filipe
Casal for comments and discussions.
%
This work was supported by FCT through the LASIGE Research Unit, ref.\
UID/CEC/00408/2019 and by Cost Action CA15123 EUTypes.

\bibliographystyle{plain}
\bibliography{biblio}

%\newpage 
%\appendix
%\appendix
\section{Transforming recursive types}
\label{sec:transf-recurs-types}


\begin{figure}[t]
  \begin{align*}
    a & ::= \omega && \text{infinite words, only} \\
    & \mid \infty && \text{finite and infinite words} \\
    & \omega \sqsubset \infty \\
    n & ::= 0 \mid 1 && \text{minimum length of trace} \\
    A & ::= \cdot \mid A, x:a && \text{environments} \\
    C & ::= \cdot \mid C, (x,n) && \text{depth of guardedness}
  \end{align*}
  \begin{gather*}
    \frac{}{A \vdash \skipk : \infty; 0; C }
    \quad
    \frac{}{A \vdash {!B} : \infty; 1; C}
    \quad
    \frac{}{A \vdash {?B} : \infty; 1; C}
    \\
    \frac{A \vdash S_1 : a_1;n_1;C_1 \quad A \vdash S_2 : a_2;n_2;C_2}{
      A \vdash (S_1;S_2) : a_1 \sqcap a_2;\seq(a_1)(a_2)}
    \\
    \frac{(\forall i\in I)~A \vdash S_i : a_i; n_i ; C_i}{
      A \vdash \star\{l_i\colon S_i\}_{i\in I} :
      \bigsqcup_i a_i; 
      1; \bigsqcap_i C_i \uparrow 1}
%     \quad
%     \frac{(\forall i\in I)~A \vdash S_i : a_i}{
%       A \vdash \&\{l_i\colon S_i\}_{i\in I} :
%       \bigsqcup_i a_i}
    \\
    \frac{}{A, x : a \vdash x : 1 ; C,(x,0)} \quad
    \frac{A, x : a \vdash S : a; n ; C, (x,1) }{A \vdash \mu x.S : a; n; C}
  \end{gather*}
  \begin{align*}
    \seq (n_1; C_1) (n_2; C_2) & = (n_1 + n_2 ; C_1 \sqcap C_2 \uparrow n_1
    ) \\
    (x, i) \uparrow n &= (x, i + n)
  \end{align*}
  \caption{Inference for infiniteness (and contractivity)}
  \label{fig:inference-infiniteness}
\end{figure}
Figure~\ref{fig:inference-infiniteness} contains an inference system for detecting finite words. It
classifies session types $S$ according to their trace languages. The intent is as follows: If all traces of $S$ are infinite
words in $\Sigma^\omega$, then $A \vdash S : \omega; n; C$ is derivable. If $S$ may admit a finite trace
in $\Sigma^\infty$, then $A \vdash S : \infty; n; C$ is derivable. The result $n$ indicates the
minimum length of a trace generated/accepted by $S$. The $C$ contains pairs of the form $(x,n)$
which indicates a lower bound on the trace produced before $x$ is mentioned recursively. The
formation rule for $\mu x.S$ requires $(x,1)$ which is equivalent to contractivity in $x$.


We write $A \models \rho$ if $\dom (A) = \dom (\rho)$ and for all $x :a \in A$, let $R_x = \rho (x)$
where $R_x \ne \emptyset$, $R_x \subseteq \Sigma^a$, and $\min\{ |t| \mid t\in R_x \}\ge1$.

\begin{lemma}
  If $A \vdash S : a; n; C$ and $A \models \rho$, then $T = \TR (S) \rho \subseteq \Sigma^a$ and
  $\min\{ |t| \mid t \in T\} \ge n$.
\end{lemma}
\begin{proof}
  Induction on $S$. Relies on $\TR (S) \rho \ne \emptyset$, for all $S$.

  \textbf{Case }$\skipk$, $!B$, $?B$: Immediate.

  \textbf{Case }$(S_1;S_2)$: By inversion, $A \vdash S_1: a_1;n_1;C_1$, $A\vdash S_2:a_2;n_2;C_2$, and $a = a_1
  \sqcap a_2$, $(n;C) = \seq (n_1;C_1) (n_2;C_2)$.
  If $a=\omega$, then there are two non-exclusive cases $a_1=\omega$ or $a_2 = \omega$.

  \textbf{Subcase }$a_1 = \omega$: By induction $\TR (S_1) \rho \subseteq \Sigma^\omega$, hence $\TR
  (S_1;S_2)\rho = \TR (S_1)\rho \cdot \TR (S_2)\rho = \TR (S_1)\rho \subseteq \Sigma^\omega$ because $\TR (S_2)\rho$
  is not empty. The condition on $n$ holds trivially because there are no finite traces.

  \textbf{Subcase }$a_1 = \infty$ and $a_2 = \omega$: By induction $\TR (S_2) \rho \subseteq \Sigma^\omega$, hence  $\TR
  (S_1;S_2)\rho = \TR (S_1)\rho \cdot \TR (S_2)\rho \subseteq \Sigma^\omega$ because $\TR (S_1)\rho$
  is not empty. Again, the condition on $n$ holds trivially.

  \textbf{Subcase }$a_1 = a_2 = \omega$: The condition on $n$ holds because $|v\cdot w| = |v|+|w|$.

  \textbf{Case }$\star\{l_i\colon S_i\}$: By inversion, $A \vdash S_i : a_i; n_i; C_i$ and $a = \bigsqcup_i
  a_i$, $n=1$, and $C = \bigsqcap_i C_i \uparrow 1$.
  If $a=\omega$, then $a_i = \omega$, for all $i$. Hence, by induction $\TR (\oplus\{l_i\colon
  S_i\})\rho = \bigcup_i \{L_i\}\cdot\TR (S_i)\rho \subseteq \Sigma^\omega$. The condition on $n$ is
  trivial because each trace has length $\ge1$.

  \textbf{Case }$x$: Immediate by assumption $A \models \rho$.

  \textbf{Case }$\mu x.S$: By inversion, $A, x:a \vdash S:a;n;C,(x,1)$. If $a=\infty$, then any (non-empty)
  extension of $\rho$ satisfies $A, x:\infty \models \rho[x\mapsto Y]$ and the result is immediate by
  induction. 

  If $a= \omega$, then consider
  \begin{align*}
    \TR (\mu x.S)\rho & = \GFP\, \lambda Y. \TR (S)\rho[x\mapsto Y] \\
    & = \TR (S)\rho[x\mapsto \GFP\, \lambda Y. \TR (S)\rho[x\mapsto Y]]
  \end{align*}

  \textbf{Subsidiary lemma:}
  Let $F (Y) = \TR (S)\rho[x\mapsto Y]$ and show that an $F$-consistent set
  cannot contain finite words.

  That is, suppose that $Y \subseteq \TR (S)\rho[x\mapsto Y]$ and there exists some $v \in Y$ of 
  length $|v| = k < \infty$. We show that there must be some $v' \in Y$ with $|v'|\le k-m$ where
  $(x,m)$ is the minimal prefix length for $x$ in $C$.

  Given that result, we argue as follows. Choose $v \in Y$ of finite minimal length $k$. From the
  inversion, we know that $m=1$. Hence, there exists some $v' \in Y$ with length $\le k-1$. From
  this contradiction it follows that $Y$ contains no finite elements.
  
  Proof by induction on the derivation of $A, x:\omega \vdash S : \omega; n; C,(x,m)$.

  \textbf{Case }$\skipk$, $!B$, $?B$: contradiction because they do not derive $\omega$.

  \textbf{Case }$(S_1;S_2)$: Observe that $v \in \TR (S_1;S_2)\rho[x\mapsto Y]$ implies that $v = v_1v_2$ with
  $v_1 \in \TR (S_1)\rho[x\mapsto Y]$ and $v_2 \in \TR (S_2)\rho[x\mapsto Y]$ where $|v_1|, |v_2| \le |v| \le k$.
  By inversion, there are two subcases:

  \textbf{Subcase }$A \vdash S_1:\omega;n_1;C_1,(x,m_1)$ where $m_1\sqsupseteq m$: By induction,
  there exists some $v'\in Y$ with $|v'| \le |v_1| - m_1 \le |v| - m$.

  \textbf{Subcase }$A \vdash S_1:\infty;n_1;C_1,(x,m_1)$ and $A \vdash S_2:\omega;n_2;C_2,(x,m_2)$
  where  $m_1\sqsupseteq m$ and  $m_2 + n_1 \sqsupseteq m$:
  By induction, there exists some $v'\in Y$ with $|v'| \le  |v_2| - m_2 \le |v| - m$ (if $n_1=0$).
  If $n_1=1$, then $|v_2| < |v|$ and we can exploit the inductive hypothesis similarly.

  \textbf{Case }$\star\{l_i\colon S_i\}$: Inversion yields $A \vdash S_i: \omega; n_i;C_i,(x,m_i)$
  and $m =1$.
  In this case, $v = L_iv_i$ (for some $i$) and $v_i \in \TR
  (S_i)\rho[x\mapsto Y]$. By induction, there exists $v'\in Y$ with $|v'|\le |v_i| - m_i \le |v_i|+1
  -1 = |v| -1$.

  \textbf{Case }$x$: In this case, $m=0$ and $v \in \rho (x) = Y$ and $|v|\le |v|-0$.

  \textbf{Case }$x'\ne x$: By inversion, it must be that $x':\omega \in A$. By $A \models \rho$, it
  must be that $v \in \rho (A) \subseteq \Sigma^\omega$, a contradiction.

  \textbf{Case }$\mu x'.S$: By the outer induction, this implies that $v\in\Sigma^\omega$, a contradiction.
\end{proof}

\begin{lemma}
  For each $S$, there is a minimal derivation, where derivations are ordered pointwise on judgments;
  judgments are ordered pointwise on environments $A$ and the $a$-component of the result.
\end{lemma}

\begin{lemma}
  If $A \vdash (S_1; S_2) : \omega; n; C$ is derivable with subderivation $A \vdash S_1 : \omega;
  n_1; C_1$, then $(S_1;S_2) \TypeEquiv S_1$.
\end{lemma}

This lemma enables the direct syntactic transformation of $\mu x.{!B};x;x$ to the equivalent $\mu
x.{!B}; x$. It even applies to proving $\mu x.{!B}; x; S \TypeEquiv \mu x.{!B}; x$ for any $S$.


\section{Obsolete type stuff}
\subsection{Type unfolding}
\label{sec:type-unfolding}

We establish a notion of type unfolding as a step towards
defining type equivalence. The idea of unfolding is to expose the
first nontrivial action of a session type by squeezing out sequences
of $\skipk$s.
Let $A$ range over $\alpha$, $!B$, and $?B$; let $\star$ range over
$\oplus$ and $\&$. 
%
Define the unfolding of a type $T$,  $\Unfold(T)$, recursively by cases on the
structure of~$T$ as follows. 
\begin{enumerate}
\item $\Unfold(\mu x.T) = \Unfold(T\subs{\mu x.T}x)$
\item $\Unfold (S;S') = \left\{%
  \begin{array}{ll}
    \Unfold(S') & \Unfold(S) = \skipk
    \\
    (A; \Norm(S')) & \Unfold(S) = A
    \\
    (S_3; (S_4\fatsemi S')) & \Unfold(S) = (S_3;S_4)
    \\
    \star\{l_i\colon S_i\fatsemi S'\}  & \Unfold (S) = \star\{l_i\colon S_i\}
  \end{array}
  \right.
$
\item $\Unfold(T) = T$, otherwise
\end{enumerate}
We assume two auxiliary definitions
\begin{align*}
  \Norm (S) &=
              \begin{cases}
                S_1 \fatsemi S_2 & S = (S_1; S_2) \\
                S & \text{otherwise}
              \end{cases}
  \\
  S_1 \fatsemi S_2 &=
                     \begin{cases}
                       S_1' \fatsemi (S_1'' \fatsemi S_2) & S_1 =
                       (S_1'; S_1'') \\
                       (S_1; S_2) & \text{otherwise}
                     \end{cases}
\end{align*}

The function $\Unfold$ is well-defined and terminating because we
assume that the body of a recursive type is contractive. The auxiliary
functions $\Norm$ and $\fatsemi$ are both terminating.
The following
lemmas establish well-definedness of $\Unfold$.

\begin{lemma}\label{lemma:app:guarded=skip}
  Suppose that $\GEnv$ does not bind recursion variables and that
  $\sigma$ is a substitution of recursion variables by recursive types.
  If $\GEnv \Contr S : \Guarded$, then $\Unfold (S\sigma) = \skipk$.
\end{lemma}
\begin{proof}
  Induction on $\GEnv \Contr S : \Guarded$.

  \textbf{Case }$\GEnv \Contr \skipk : \Guarded$. Immediate.

  \textbf{Case }$\GEnv \Contr (S_1;S_2) : \Guarded$ because $\GEnv
  \Contr S_1 : \Guarded$ and $\GEnv \Contr S_2 : \Guarded$. By
  induction, $\Unfold (S_1\sigma) = \skipk$, hence $\Unfold ((S_1;S_2)\sigma) =
  \Unfold (S_2\sigma) = \skipk$ by induction.

  \textbf{Case }$\GEnv \Contr \mu x.S : \Guarded$ because $\GEnv
  \Contr S : \Guarded$. In this case, $\Unfold ((\mu x.S)\sigma) = \Unfold
  (S[\mu x.S/x]\sigma) = \skipk$ by induction (for $\sigma' = [\mu x.S/x]\sigma$).
\end{proof}

\begin{definition}
  A \emph{guarded} type has one of the forms below.
  % where $A$ ranges over $\alpha$, $!B$, and $?B$.
\begin{gather*}
  \skipk \quad A \quad A;S' \quad
  \star\{l_i\colon S_i\} %\quad \&\{l_i\colon S_i\}
  \\
  T \to T' \quad T \multimap T' \quad T \otimes T' \quad [l_i : T_i]
  \quad B 
\end{gather*}
\end{definition}
% We say that %
% %
% \footnote{To me $T$ is only a type if $\Theta \vdash T \isOk$ holds,
%   for some $\Theta$ that does not bind recursion variables. To
%   $(B\rightarrow B;\skipk)$ I call a piece of \emph{junk syntax}. This
%   means that ``there is no life outside types'' (the intuitionistic
%   approach). Pragmatically, it means that, whenever we talk of $T$, we
%   need not keep saying ``s.t.\ $\Theta \vdash T \isOk$ for some
%   $\Theta$ that does not bind recursion variables.''. In this , a
%   session type $S$ is a type such that $\Delta \vdash S: \kinds^m$,
%   for some $\Delta$ that does not bind recursion variables.}
% %


\begin{lemma}[Characterization of $\Unfold$]
  \label{lem:app:unfold-yields-guarded-types}
  Suppose that $\GEnv$ does not bind recursion variables and that
  $\GEnv \vdash T :: \kind$ for $\kind \le \kindt^\Unrestricted$, then $\Unfold (T)$ is defined and yields a
  guarded type.

  % result that has one of the following forms.
  % \begin{enumerate}
  % \item $ \skipk$,
  % \item $(\alpha;S')$, $(!B; S')$, $(?B; S')$ for some $\Delta \vdash S' \isOk$,
  % \item $\oplus\{l_i\colon S_i\}$,  $\&\{l_i\colon S_i\}$ for some
  %   $\Delta \vdash S_i \isOk$.
  % \end{enumerate}

  % Furthermore, if $\GEnv \Contr S : \gamma$, then $\Unfold (\mu
  % x.S)$ is defined and yields a guarded type.
\end{lemma}
\begin{proof}
  Induction on the derivation of  $\GEnv \vdash T :: \kind$.

  \textbf{Case }$\GEnv \vdash \skipk :: \kinds^\Unrestricted$.
  In this case, $\Unfold (\skipk) = \skipk$.

  \textbf{Case }$\GEnv \vdash A :: \kind$. $\Unfold (A) = {A}$.

  % \textbf{Case }$\GEnv \vdash !B :: \kinds^\Linear$. $\Unfold (!B) = {!B}$.
  % \textbf{Case }$\GEnv \vdash ?B :: \kinds^\Linear$. $\Unfold (?B) = {?B}$.

  \textbf{Case }$\GEnv \vdash (S_1;S_2) :: \kinds^m$.

  Inversion yields $\GEnv \vdash S_1 :: \kinds^{m_1}$ and  $\GEnv
  \vdash S_2 :: \kinds^{m_2}$. By induction, $S_1' = \Unfold (S_1)$ is
  guarded.

  \textbf{Subcase }$S_1' = \skipk$. In this case, $\Unfold (S_1;S_2) =
  \Unfold (S_2)$ which is guarded by induction.

  \textbf{Subcase }$S_1' = A$. Then $\Unfold (S_1;S_2) = (A; S_2)$
  which is guarded.

  \textbf{Subcase }$S_1' = (A; S_3)$. Hence, $\Unfold (S_1;S_2) = (A; (S_3; S_2))$ is guarded.

  \textbf{Subcase }$S_1' = \star\{l_i\colon S_i\}$: $\Unfold
  (S_1;S_2) = \star\{l_i\colon (S_i; S_2) \}$.

  % \textbf{Subcase }$S_1' = \&\{l_i\colon S_i\}$: $\Unfold
  % (S_1;S_2) = \&\{l_i\colon (S_i; S_2) \}$.

  \textbf{Case }$\GEnv \vdash \star\{l_i\colon S_i\}
  :: \kinds^\Linear$. Immediate.

  % \textbf{Case }$\GEnv \vdash \&\{l_i\colon S_i\}
  % :: \kinds^\Linear$. Immediate.

  % \textbf{Case }$\GEnv \vdash \alpha :: \kind$. Immediate: $\Unfold
  % (\alpha) = \alpha$.

  \textbf{Case }$\GEnv \vdash \mu x. T :: \kind$.
%
  Inversion yields $\GEnv, x:\kind \vdash T :: \kind$ and
  $\GEnv \Contr T : \gamma$. Observe that
  $\Unfold (\mu x.T) = \Unfold (T[\mu x.T/x])$ and
  $\GEnv \Contr T[\mu x.T/x] : \gamma$. By the second claim 
  $\Unfold (T[\mu x.T/x])$ is defined and yields a guarded type using
  $\sigma = [\mu x.T/x]$.

  \textbf{All other cases}: $\Unfold (T) = T$ is guarded.
  
  \textbf{Second claim.}
  % It holds that $\Unfold (\mu x.S) = \Unfold (S[\mu x.S/x])$.
  % Further $\GEnv\setminus x \Contr S : \gamma$ implies that
  % $\GEnv \Contr S[\mu x.S/x] : \gamma$.
  Suppose that  $\GEnv \Contr T :  \gamma$ where $\GEnv$ does not bind
  recursion variables and that  $\sigma$ is a substitution on recursion variables.
  Then  $\Unfold (T\sigma)$ is defined and yields a guarded type.

  The proof is by induction on the derivation of $\GEnv \Contr T :
  \gamma$.

  \textbf{Case }$\skipk$. Immediate.

  \textbf{Case }$A \in \{ {!B}, {?B}, \alpha\}$. $\Unfold (A\sigma) = A$ which is guarded.

  \textbf{Case }$x$ cannot occur because $\GEnv$ 
  contains no assumptions about recursion variables.

  \textbf{Case }$\star\{l_i\colon S_i\}$. $\Unfold
  ((\star\{l_i\colon S_i\})\sigma) = \Unfold (\star\{l_i\colon
  S_i\sigma\}) = \star\{l_i\colon S_i\sigma\}$ which is guarded.

  % \textbf{Case }$\&\{l_i\colon S_i\}$. Analogously.

  \textbf{Case }$(S_1;S_2) : \Productive$ because $S_1 :
  \Productive$. By induction $S_1' = \Unfold (S_1)$ is
  guarded. Proceed by subcases on $S_1'$.

  \textbf{Subcase }$\skipk$. Contradicts $S_1 : \Productive$.

  \textbf{Subcase }$A$. Here, $\Unfold ((S_1;S_2)\sigma) = (A;
  S_1)\sigma$, which is guarded.

  \textbf{Subcase }$(A; S_3)$. $\Unfold
  ((S_1;S_2)\sigma) = (A; (S_3; S_2))\sigma$, which is guarded. 

  \textbf{Subcase }$\star\{l_i\colon S_i\}$. $\Unfold
  ((S_1;S_2)\sigma) = \star\{l_i\colon (S_i; S_2)\sigma\}$ is guarded.

  % \textbf{Subcase }$\&\{l_i\colon S_i\}$. Similar.

  \textbf{Case }$(S_1;S_2) : \Productive$ because $S_1 :
  \Guarded$ and $S_2 : \Productive$. In this case, $\Unfold (S_1) =
  \skipk$ by Lemma~\ref{lemma:guarded=skip} so that the result is
  $\Unfold (S_2\sigma)$, which is guarded by induction on $S_2 : \Productive$.

  \textbf{Case }$\GEnv \Contr \mu x.T : \gamma$ because $\GEnv \Contr
  T : \gamma$. Hence, $\Unfold ((\mu x.T)\sigma) = \Unfold (T\sigma[\mu
  x.T\sigma/x])$. The result follows by induction using $\sigma' = \sigma[\mu
  x.T\sigma/x]$.

  % \textbf{Case }$\GEnv, \alpha : \gamma \Contr \alpha :
  % \gamma$. Immediate.

  \textbf{All remaining cases}: Immediate.
\end{proof}

Next, we consider invariance of kinding and contractivity under
unfolding of recursion anywhere in the type.

\begin{lemma}[Weakening]\label{lemma:app:weakening-kind}
  If $\Delta \vdash T :: \kind$, then
  $\Delta, x\colon \gamma \vdash T :: \kind$ for some $x$ not in
  $\Delta$.
\end{lemma}


\begin{lemma}[Unfolding preserves kinding]
  If $\Delta \vdash T :: \kind$ then $\Delta \vdash \Unfold(T) :: \kind$.
\end{lemma}
%
\begin{proof}
  We only consider the case for a recursive type as the other cases
  are straightforward.
  
  % (Needs adjustment)
  If  $\GEnv \vdash \mu x.T :: \kinds^m$, it must be because  $\GEnv,
  x:\kinds^m \vdash T :: \kinds^m$ and $\GEnv \Contr T : \gamma$.
  We prove by induction on $\GEnv, x:\kinds^m \vdash T :: \kinds^m$ that
  $\GEnv \vdash T[\mu  x. T/x] :: \kinds^m$.

  There are two interesting cases. In the first case, we encounter the
  recursion variable $\GEnv, x::\kinds^m, \GEnv' \vdash x
  :: \kinds^m$. At this point, we have to return $\GEnv, \GEnv' \vdash \mu
  x.T :: \kinds^m$, which is derivable by the initial assumption and
  weakening (Lemma~\ref{lemma:weakening-kind}).

  The other case is a different $\mu$ operator in a judgment $\GEnv,
  x::\kinds^m, \GEnv' \vdash \mu x'. T' :: \kinds^{m'}$. Inversion yields $\GEnv,
  x::\kinds^m, \GEnv', x'::\kinds^{m'} \vdash T'  :: \kinds^{m'}$ and $\GEnv,
  x:\gamma, \GEnv' \Contr T' : \gamma'$. The first part
  can be handled by induction, but the second part requires an
  auxiliary induction to prove that $(\GEnv, \GEnv')
  \Contr T'[\mu x.T/x] : \gamma'$. For this auxiliary induction it is
  sufficient to observe that a successful derivation never reaches a
  recursion variable, so the unrolling does not matter. 
\end{proof}

\begin{lemma}
  \label{lemma:app:unfold-fixpoints}
  If $T$ is a guarded type, then $\Unfold (T) = T$.
\end{lemma}
\begin{proof}
  Cases on $T$.

  \textbf{Case }$\skipk$: Obvious.

  \textbf{Case }$A$: $\Unfold (A) = A$.

  \textbf{Case }$(A; S')$:
  $\Unfold (A;S') = (A; S')$ as $\Unfold (A) = A$.

  \textbf{Case }$\star\{l_i\colon S_i\}$: Immediate.

  \textbf{Remaining cases}: Immediate.
\end{proof}

\begin{lemma}
  \label{lemma:app:unfold-idempotent}
  % For all well-formed types, $
  $\Unfold (\Unfold (T)) = \Unfold (T).$
\end{lemma}
\begin{proof}
  By Lemma~\ref{lem:unfold-yields-guarded-types}, $\Unfold (T)$ is guarded and hence a fixpoint of
  $\Unfold$ by Lemma~\ref{lemma:unfold-fixpoints}.
\end{proof}

\subsection{Type equivalence}
\label{sec:type-equivalence}

We want to define a notion of type equivalence for session types that
only depends on the communication behavior of a process with that
type. To this end, we first define a (weak) labelled transition system
$(\stypes, \Sigma, \LTSderives)$ that captures this behavior. The set
of states is  $\stypes = \{ S \mid \GEnv \vdash S :: \kinds^m \}$
where $\GEnv$ is an arbitrary, fixed kinding environment that binds no
recursion variables. The
actions in this system are drawn from the set $\Sigma = \{ {!B}, {?B}
\mid  B \in \btypes \} \uplus \Tyvars \uplus \{ {!l}, {?l} \mid l
\in \Labels\} $. We use the label $\Silent$ for the \emph{silent
  action} that exhibits no externally observable behavior. The transition
relation $\LTSderives$ is defined by the rule set in Figure~\ref{fig:type-behavior}. We write
$\Wderives$ for the reflexive transitive closure of $\LTSderives[\Silent]$ and $\Wderives[\xi]$ for
the composition $\Wderives \circ \LTSderives[\xi] \circ \Wderives$.

\begin{figure}[tp]
  \begin{gather*}
    {A \LTSderives[A] \skipk }
    \qquad
    {\star\{\overline{l_n:S_n}\} \LTSderives[\star l_i] S_i}
    \\
    \frac{S_1 \LTSderives[\xi] S_1'}{(S_1; S_2) \LTSderives[\xi]
      (S_1';S_2) }
    \qquad
    {(\skipk; S) \LTSderives[\Silent] S}
    \\
    {((S_1;S_2); S_3) \LTSderives[\Silent] (S_1; (S_2; S_3))}
    \\
    {(\star\{\overline{l_n:S_n}\}; S) \LTSderives[\Silent]
      \star\{\overline{l_n:(S_n; S)}\}}
    \\
    { \mu x.S \LTSderives[\Silent] S[\mu x.S/x]}
  \end{gather*}
  \caption{Behavior of a type}
  \label{fig:type-behavior}
\end{figure}
\begin{lemma}\label{lemma:app:unfold-silent}
  If $S' = \Unfold (S)$ is defined, then $S \Wderives S'$.
\end{lemma}
\begin{proof}
  By induction on the number of recursive calls to compute $\Unfold
  (S)$.

  \textbf{Case }$\Unfold (\mu x.S) = \Unfold (S[\mu x.S/x])$: By
  definition, $\mu x.S \LTSderives[\Silent] S[\mu x.S/x] $ and by
  induction $S[\mu x.S/x] \Wderives S'$.

  \textbf{Case }$\Unfold (S_1;S_2)$.

  \textbf{Subcase }$\Unfold (S_1) = \skipk$: By induction, $S_1
  \Wderives \skipk$. By the context rule for behaviors,
  $(S_1; S_2) \Wderives (\skipk; S_2 )$ and by the skip
  rule: $(\skipk; S_2 ) \LTSderives[\Silent] S_2$. Proceed by another
  induction on $S_2$.

  \textbf{Subcase }$\Unfold (S_1) = A$. By induction and the context
  rule.

  \textbf{Subcase }$\Unfold (S_1) = (S_1'; S_1'')$. By induction and
  the context rule, we obtain $(S_1;S_2) \Wderives
  ((S_1'; S_1''); S_2) \LTSderives[\Silent] (S_1'; (S_1''; S_2))$
  where the last step is an application of associativity.

  \textbf{Subcase }$\Unfold (S_1) = \star\{\overline{l_n:S_n} \}$. By
  induction and distributivity.

  \textbf{Remaining cases}. No silent transition needed.
\end{proof}

\begin{lemma}\label{lemma:app:silent-unfold-compatible}
  If $S\LTSderives[\Silent] S'$, then $\Unfold (S) = \Unfold (S')$.
\end{lemma}
\begin{proof}

  \textbf{Case }$\mu x.S \LTSderives[\Silent] S[\mu x.S/x]$:
  $\Unfold (\mu x.S) =  \Unfold (S[\mu x.S/x])$ by definition.

  \textbf{Case }$    {(\star\{\overline{l_n:S_n}\}; S) \LTSderives[\Silent]
    \star\{\overline{l_n:(S_n; S)}\}}$: Immediate by definition of
  $\Unfold$.
  
  \textbf{Case }${((S_1;S_2); S_3) \LTSderives[\Silent] (S_1; (S_2;
    S_3))}$: \\
  Case analysis on the possible outcomes of $\Unfold (S_1)$ and $\Unfold (S_2)$. 

  \textbf{Subcase } $\Unfold (S_1) = \skipk$:
  
  \textbf{Subsubcase }$\Unfold (S_2) = \skipk$:
  
  $\Unfold  (S_1; (S_2;  S_3)) = \Unfold (S_2; S_3) = \Unfold (S_3)$
  and
  $\Unfold ((S_1; S_2); S_3) = \Unfold (S_3)$.

  \textbf{Subsubcase }$\Unfold (S_2) = A$.

  $\Unfold  (S_1; (S_2;  S_3)) = \Unfold (S_2; S_3) = (A; S_3)$
  and
  $\Unfold ((S_1; S_2); S_3) = (A; S_3)$.

  \textbf{Subsubcase }$\Unfold (S_2) = (S_2'; S_2'')$.

  $\Unfold  (S_1; (S_2;  S_3)) = \Unfold (S_2; S_3) = (S_2' ;( S_2''; S_3))$
  and
  $\Unfold ((S_1; S_2); S_3) = (S_2'; (S_2''; S_3))$.

  \textbf{Subsubcase }$\Unfold (S_2) =
  \star\{\overline{l_n:S_n}\}$.

  $\Unfold  (S_1; (S_2;  S_3)) = \Unfold (S_2; S_3) =
  \star\{\overline{l_n:(S_n; S_3)} $
  and
  $\Unfold ((S_1; S_2); S_3) = \star\{\overline{l_n:(S_n; S_3)}$.

  \textbf{Subcase }$\Unfold (S_1) = A$.

  $\Unfold  (S_1; (S_2;  S_3)) =(A; (S_2; S_3))$
  and
  $\Unfold ((S_1; S_2); S_3) = (A; (S_2; S_3))$.

  \textbf{Subcase }$\Unfold (S_1) = (S_1';S_1'')$.

  $\Unfold  (S_1; (S_2;  S_3)) =(S_1'; (S_1''; (S_2; S_3)))$
  and
  $\Unfold ((S_1; S_2); S_3) = (S_1'; (S_1''; (S_2; S_3)))$.

  \textbf{Subcase }$\Unfold (S_1) = \star\{\overline{l_n:S_n}\}$.

  $\Unfold  (S_1; (S_2;  S_3)) =\star\{\overline{l_n:(S_n;
    (S_2\fatsemi S_3))}\} $
  and
  $\Unfold ((S_1; S_2); S_3) = \star\{\overline{l_n:(S_n; (S_2\fatsemi
    S_3)))}\}$.

  \textbf{Case }${(\skipk; S) \LTSderives[\Silent] S}$:
  $\Unfold (\skipk;S) = \Unfold (S) = S'$.

  \textbf{Case }$(S_1; S_2) \LTSderives[\Silent] (S_1';S_2)$
  because $S_1 \LTSderives[\Silent] S_1'$: By induction, $\Unfold
  (S_1) = \Unfold(S_1')$, so that $\Unfold (S_1; S_2) = \Unfold
  (S_1';S_2)$ by definition of unfolding.
\end{proof}
\begin{lemma}\label{lemma:app:silent-unfold}
  If $S \Wderives S'$ and $S'$ is guarded, then $S' = \Unfold (S)$.
\end{lemma}
\begin{proof}
  By induction on the number of silent steps.

  \textbf{Case }$0$. If $S=S'$ is already guarded, then $S$ is a
  fixpoint of $\Unfold$  by Lemma~\ref{lemma:unfold-fixpoints}.

  \textbf{Case }$n>0$. In this case, there is some $S''$ such that $S
  \LTSderives[\Silent] S''$ and $S'' \Wderives S'$ in less than $n$
  steps. Now, $S' = \Unfold (S'') = \Unfold (S)$, the former by
  induction  and the latter by
  Lemma~\ref{lemma:silent-unfold-compatible}.
\end{proof}

% A relation $R \subseteq \stypes \times \stypes$ is a \emph{type
%   simulation} if $(S_1,S_2)\in R$ implies the following conditions:
% %
% \begin{enumerate}
% \item If $\Unfold(S_1) = \skipk$ then $\Unfold(S_2) = \skipk$. 
% \item If $\Unfold(S_1) = (\alpha; S_1')$ then $\Unfold(S_2) =
%   (\alpha;S_2')$ and $(S_1', S_2') \in R$. 
% \item If $\Unfold(S_1) = (!B;S_1')$ then $\Unfold(S_2) = (!B;S_2')$  and $(S_1', S_2') \in R$. 
% \item If $\Unfold(S_1) = (?B;S_1')$ then $\Unfold(S_2) = (?B;S_2')$  and $(S_1', S_2') \in R$. 
% % \item If $\Unfold(S_1) = S_1';S_1''$ then $\Unfold(S_2) = S_2';S_2''$
% %   and both $(S_1',S_2')$  and $(S_1'',S_2'')$ are in $R$.
% \item If $\Unfold(S_1) = \oplus\{l_i\colon S_{1,i}'\}_{i\in I}$ then
%   $\Unfold(S_2) = \oplus\{l_i\colon S_{2,i}'\}_{i\in I}$ and
%   $(S_{1,i}',S_{2,i}')\in R$, for all $i\in I$.
% \item If $\Unfold(S_1) = \&\{l_i\colon S_{1,i}'\}_{i\in I}$ then
%   $\Unfold(S_2) = \&\{l_i\colon S_{2,i}'\}_{i\in I}$ and
%   $(S_{1,i}',S_{2,i}')\in R$, for all $i\in I$.
% \end{enumerate}

\begin{definition}
Define a monotone function~$F$ on $\stypes\times\stypes$ as 
follows. 
%
\begin{align*}
  F (R) &= \{ (S_1, S_2) \mid \Unfold (S_1) = \Unfold (S_2) = \skipk \}
  \\
        &\cup \{ (S_1, S_2) \mid \Unfold (S_1) = \Unfold (S_2) = A \}
  \\
        &\cup \{ (S_1, S_2) \mid
          \begin{array}[t]{@{}l}
            \Unfold (S_1) = (A; S_1'),\\
            \Unfold (S_2) = (A; S_2'), \\
            (S_1', S_2')  \in R \}
          \end{array}
  \\
        &\cup \{ (S_1, S_2) \mid
          \begin{array}[t]{@{}l}
            \Unfold (S_1) = \star\{l_i\colon S_{1,i}'\}_{i\in I}, \\
            \Unfold (S_2) = \star\{l_i\colon S_{2,i}'\}_{i\in I}, \\
            \forall i: (S'_{1,i}, S'_{2,i}) \in R \}
          \end{array}
  % \\
  %       &\cup \{ (S_1, S_2) \mid
  %         \begin{array}[t]{@{}l}
  %           \Unfold (S_1) = \oplus\{l_i\colon S_{1,i}'\}_{i\in I}, \\
  %           \Unfold (S_2) = \oplus\{l_i\colon S_{2,i}'\}_{i\in I}, \\
  %           \forall i: (S'_{1,i}, S'_{2,i}) \in R \}
  %         \end{array}
  % \\
  %       &\cup \{ (S_1, S_2) \mid
  %         \begin{array}[t]{@{}l}
  %           \Unfold (S_1) = \&\{l_i\colon S_{1,i}'\}_{i\in I}, \\
  %           \Unfold (S_2) = \&\{l_i\colon S_{2,i}'\}_{i\in I}, \\
  %           \forall i: (S'_{1,i}, S'_{2,i}) \in R \}
  %         \end{array}
\end{align*}

This function helps define (weak) bisimularity $\TypeEquiv$ for the labelled transition system
$(\stypes, \Sigma, \LTSderives)$ on well-formed session types as a greatest fixpoint: ${\TypeEquiv}
= \GFP (F)$. The 
definition relies on the $\Unfold$ function instead of using silent transitions which is sanctioned
by Lemmas~\ref{lemma:unfold-silent} and~\ref{lemma:silent-unfold}.
\end{definition}

Our goal is to use weak bisimilarity for type
equivalence. To this end, we need to establish that it is indeed an
equivalence relation.\footnote{This development can be extended to a
  relation on $\kindt$, but the extension is entirely standard.}
Reflexivity and symmetry follow directly from the definition. Transitivity requires some work.



\begin{lemma}
  Type  bisimilarity $\TypeEquiv$ is reflexive.
\end{lemma}
% \begin{proof}
%   Obvious.
% \end{proof}

\begin{lemma}
  Type bisimilarity $\TypeEquiv$ is symmetric.
\end{lemma}
% \begin{proof}
%   Obvious.
% \end{proof}

\begin{lemma}
  Type bisimilarity $\TypeEquiv$ is transitive.
\end{lemma}
\begin{proof}
  Let $R  = {\TypeEquiv} \circ {\TypeEquiv}$. Show that $R \subseteq F(R)$, which implies that $R
  \subseteq {\TypeEquiv}$. Observe that ${\TypeEquiv}
  \subseteq R$ because $\TypeEquiv$ is reflexive.

  Suppose that $S_1 \TypeEquiv S_2$ and $S_2 \TypeEquiv S_3$ so that $(S_1, S_3) \in R$.

  \textbf{Case }$\Unfold (S_1) = \skipk$ implies $\Unfold (S_2) =\skipk$, which in turn implies $\Unfold (S_3) =
  \skipk$. Hence, $(S_1, S_3) \in F (\emptyset)$.

  \textbf{Case }$\Unfold (S_1) = A$. Then it must be the case that
  $\Unfold (S_2) = A$ and also $\Unfold (S_3)=A$. Hence, $(S_1, S_3)
  \in F (\emptyset)$.

  \textbf{Case }$\Unfold (S_1) = (A; S_1')$ with $A$ either $\alpha$, $!B$, or $?B$. It must be the
  case that $\Unfold (S_2) = (A; S_2')$ with $(S_1', S_2') \in {\TypeEquiv}$ and further
  $\Unfold (S_3) = (A; S_3')$ with $(S_2', S_3') \in {\TypeEquiv}$. But then $(S_1', S_3') \in R =
  {\TypeEquiv} \circ {\TypeEquiv}$ and hence $(S_1, S_3) \in F (R)$.

  \textbf{Case }$\Unfold (S_1) = \star\{l_i\colon S_{1,i}'\}_{i\in I}$. Then it must be that
  $\Unfold (S_2) = \star\{l_i\colon S_{2,i}'\}_{i\in I}$ with $S_{1,i}' \TypeEquiv S_{2,i}'$, for all
  $i$, and $\Unfold (S_1) = \star\{l_i\colon S_{1,i}'\}_{i\in I}$ with  $S_{2,i}' \TypeEquiv S_{3,i}'$, for all
  $i$. Hence, $(S_{1,i}', S_{3,i}') \in R $, for all $i$, so that $(S_1, S_3) \in F (R)$.
  % \textbf{Case }$\Unfold (S_1) = \&\{l_i\colon S_{1,i}'\}_{i\in I}$. Analogously.
\end{proof}

\begin{lemma}
  \label{lem:app:unfold-type-sim}
  $S \TypeEquiv \Unfold(S)$.
\end{lemma}
\begin{proof}
  Straightforward application of coinduction. We show that $\{ (S, \Unfold (S)) \} \subseteq
  F (\TypeEquiv)$ because 
  of idempotence of $\Unfold$ (Lemma \ref{lemma:unfold-idempotent}) and reflexivity of $\TypeEquiv$.
\end{proof}

\begin{lemma}
  \label{lem:app:skip-elim}
  $\skipk;S_1 \TypeEquiv \skipk;S_2$ iff $S_1 \TypeEquiv S_2$.
\end{lemma}
\begin{proof}
  Immediate because $\Unfold (\skipk; S) = \Unfold (S)$.
\end{proof}

Outside this section, we write $\GEnv \vdash S_1 \TypeEquiv S_2$ to fix the environment $\GEnv$
needed for the formation of $S_1$ and $S_2$. As usual, $\GEnv$ must
not bind recursion variables.


\subsection{Translation to BPA}
\label{sec:translation-bpa}

We define a variant of the unfolding
function for a session type $S$,  $\Unravel(S)$, recursively by cases on the
structure of~$S$. The difference to $\Unfold (S)$ is that the
structure of $S$ is left intact as much as possible.
\begin{enumerate}
\item $\Unravel(\mu x.S) = \Unravel(S[\mu x.S/x])$
\item $\Unravel (S;S') = \left\{%
  \begin{array}{ll}
    \Unravel(S') & \Unravel(S) = \skipk
    \\
    (\Unravel(S); S') & \Unravel(S) \ne \skipk
  \end{array}
  \right.
$
\item $\Unravel (S) = S$ for all other cases
\end{enumerate}


To define the translation to BPA,
we need to show that, for a well-formed session type $S$, $\Unravel
(S)$ is always guarded. That is the output of $\Unravel (S)$ is either
$\skipk$ or it has one
of the following forms:
\begin{align*}
  O &::= A \mid \star\{\overline{l_i:S_i}\} \mid (O; S)
\end{align*}

Now we define the translation of well-formed $S$ to a BPA as
follows. Assume that all recursion variable bindings are unique in the
sense that the set $\{ \mu x_1.S_1, \dots, \mu x_n.S_n\}$ contains all $\mu$-subterms
of $S$ with $S_i : \Productive$ modulo $x_i \equiv \mu x_i.S_i$. 
Define the BPA process equations for $S$ by 
\begin{align*}
  \toBPATop{S} &= \{ \\
  x_0 &= \toBPA{S}, \\
               x_1&= \toBPA{\Unravel (S_1[\mu x_1.S_1/x_1])}, &
                                                                &\dots &
                                                     & \}
\end{align*}
\begin{align*}
  \toBPA{\skipk} &= \varepsilon \\
  \toBPA{A} &= A\\
  \toBPA{S_1;S_2} &= \toBPA{S_1} ; \toBPA{S_2} \\
  \toBPA{\star\{\overline{l_n:S_n}\}} &= (\star l_1;
                                        \toBPA{S_1} + \dots + \star
                                        l_n; \toBPA{S_n}) \\
  \toBPA{\mu x.S} &=
                    \begin{cases}
                      x & S : \Productive \\
                      \varepsilon & S : \Guarded
                    \end{cases}
                   \\
 \toBPA{x} &= x
\end{align*}
It is deliberate that we do \textbf{not} unfold the top-level type $S$
in the defining equation for $x_0$. This equation need not be guarded
because $x_0$ does not appear on the right-hand side of any equation.

Alternative: one could also define the equation extraction by
induction on the kind derivation for $S$ and the right-hand side
extraction by induction on the contractivity judgment. 

\begin{lemma}
  If $\mu x.S$ is a subterm of well-formed $S_0$ with $\Delta \Contr S:\Productive$, then
  $\toBPA{\Unravel (S)}$ is guarded with respect to $\toBPATop{S_0}$. 
\end{lemma}

\begin{proof}
  By Lemma~\ref{lem:unfold-yields-guarded-types}, we know that
  $\Unravel (S)$ yields a term of the form $A$, $A;S'$ or
  $\star\{\overline{l_n:S_n}\}$. Clearly, the translation of a term of
  one of these forms is guarded.

  If $\mu x.S$ has a free variable $x' \ne x$, then
  $\toBPA{\Unravel (S)}$ may have the form $x'$ or $x';S'$.
\end{proof}

It remains to show that $S$ is bisimilar to its
translation. Essentially, we want to prove that the function
$\toBPA{\cdot}$ is a bisimilation when considered as a relation.

\begin{lemma}\label{lemma:app:skip-implies-done}
  If $\Unravel (S) = \skipk$, then $\DONE{S}$.
\end{lemma}
\begin{proof}
  Induction on the number $n$ of recursive calls to $\Unravel$.

  \textbf{Case }$n=0$. $S=\skipk$ and $\DONE{\skipk}$.

  \textbf{Case }$n>0$.

  \textbf{Subcase }$\mu x.S$. $\Unravel (\mu x.S) = \skipk$ 
  because $\Unravel (S[\mu x.S/x]) = \skipk$. By induction,
  $\DONE{S[\mu x.S/x]}$ and by applying the mu-DONE rule $\DONE{\mu
    x.S}$.

  \textbf{Subcase }$S_1;S_2$.
  $\Unravel (S_1;S_2) = \skipk$ because $\Unravel (S_1) =
  \Unravel (S_2) = \skipk$. By induction $\DONE{S_1}$ and
  $\DONE{S_2}$. By rule seq-DONE $\DONE{S_1;S_2}$.
\end{proof}

\begin{lemma}\label{lemma:app:s=unr-s}
  Let $S$ be closed, well-formed. \\
  Then
  $\toBPA{S} \TypeEquiv \toBPA{\Unravel (S)}$.
\end{lemma}
\begin{proof}
  Induction on the number $n$ of recursive calls to $\Unravel$.

  \textbf{Case }$n=0$. In this case, $S$ must be $\skipk$, $A$, or
  $\star\{\overline{l_i:S_i}\}$ and the claim is immediate.

  \textbf{Case }$n>0$. There are two subcases.

  \textbf{Subcase }$\mu x.S$. Then $\toBPA{\mu x.S} = x$ and there is
  an equation $x = \toBPA{\Unravel(S[\mu x.S])}$. Now, $x$ is
  obviously bisimilar to $\toBPA{\Unravel(S[\mu x.S])}$.

  \textbf{Subcase }$S_1;S_2$. If $\Unravel (S_1) = \skipk$, then
  $\DONE{S_1}$ and hence $\DONE{\toBPA{S_1}}$. Furthermore, $\Unravel
  (S_1;S_2) = \Unravel (S_2)$ and, by induction, $\toBPA{S_2}
  \TypeEquiv \toBPA{\Unravel(S_2)}$. The result follows because
  $\toBPA{S_1;S_2} = \toBPA{S_1};\toBPA{S_2} \TypeEquiv \toBPA{S_2}$
  and $\toBPA{\Unravel(S_2)} = \toBPA{\Unravel (S_1;S_2)}$.

  If $\Unravel (S_1) =: S_u \ne \skipk$, then $\Unravel (S_1;S_2) =
  S_u; S_2$.
  By induction, we know that $\toBPA{S_1} \TypeEquiv \toBPA{S_u}$ and
  as bisimilarity is a congruence  $\toBPA{S_1;S_2} \TypeEquiv
  \toBPA{S_u}; \toBPA{S_2} = \toBPA{\Unravel (S_1;S_2)}$.
\end{proof}

\begin{lemma}\label{lemma:app:bisimulation-BPA-forwards}
  Suppose $S$ is a well-formed closed session type.
  If $S \LTSderives S'$, then $\toBPATop{S}
  \LTSderives \toBPATop{S'}$.
\end{lemma}
\begin{proof}
  By induction on  $S \LTSderives S'$.

  \textbf{Case }$A \LTSderives[A] \skipk$.
  In this case $\toBPATop{A} = \{ x_0 = A \} \LTSderives[A] \{ x_0 =
  \varepsilon \} = \toBPATop{\skipk}$.

  \textbf{Case }$\star\{\overline{l_i:S_i}\} \LTSderives[\star l_i]
  S_i$.
  In this case $\toBPATop{\star\{\overline{l_i:S_i}\}} = \{ x_0 =
  (\dots+ \star l_i; \toBPA{S_i} + \dots) \} \LTSderives [\star l_i]
  \{ x_0 = \toBPA{S_i} \} =  \toBPATop{S_i}$.

  \textbf{Case }$\frac{S_1 \LTSderives S_1'}{S_1; S_2
    \LTSderives S_1';S_2}$.
  In this case $\toBPATop{S_1;S_2} = \{ x_0 = E_1;E_2, \dots \}$ where
  $E_i = \toBPA{S_i}$ for $i=1,2$.
  Because $S_1 \LTSderives S_1'$,
  we obtain by induction that $\toBPATop{S_1} = \{ x_0 = E_1, \dots \}
  \LTSderives \toBPATop{S_1'} = \{ x_0 = E_1', \dots
  \}$. Therefore, $\{ x_0 = E_1;E_2, \dots \} \LTSderives \{ x_0 =
  E_1';E_2, \dots \} = \toBPATop{S_1'; S_2}$.

  \textbf{Case }$\frac{\DONE{S_1} \quad S_2 \LTSderives S_2'}{S_1; S_2
    \LTSderives S_2'}$.
  In this case $\toBPATop{S_1;S_2} = \{ x_0 = E_1;E_2, \dots \}$ where
  $E_i = \toBPA{S_i}$ for $i=1,2$.
  It is easy to see that $\DONE{S_1}$ implies $\DONE{\toBPA{S_1}}$,
  that is, $\DONE{E_1}$.
  Because $S_2 \LTSderives S_2'$,
  we obtain by induction that $\toBPATop{S_2} = \{ x_0 = E_2, \dots \}
  \LTSderives \toBPATop{S_2'} = \{ x_0 = E_2', \dots
  \}$. 
  Therefore, $\{ x_0 = E_1;E_2, \dots \} \LTSderives \{ x_0 =
  E_2', \dots \} = \toBPATop{S_2'}$.

  \textbf{Case }$\frac{S[\mu x.S/x] \LTSderives S'}{\mu x.S
    \LTSderives S'}$.
  In this case
  $\toBPATop{\mu x.S} = \{ x_0 = x, x = E, \dots \}$ with $E = \toBPA{\Unravel
    (S[\mu x.S/x])}$.
  By induction, $\toBPATop{S[\mu x.S/x]} \LTSderives
  \toBPATop{S'}$.
  Now $\toBPATop{S[\mu x.S/x]} = \{ x_0 = \toBPA{S[\mu x.S/x]}, \dots
  \}$ which proves the claim because $x_0 \TypeEquiv E$ by
  Lemma~\ref{lemma:s=unr-s}. 
\end{proof}

\clearpage
\begin{lemma}\label{lemma:app:bpa-unr-s}
  Suppose that $\toBPA{\Unravel
    (S)} \LTSderives E'$. Then $S \LTSderives S'$
  and $E' = \toBPATop{S'}$.
\end{lemma}
\begin{proof}
  By induction on the number $n$ of recursive calls of $\Unravel$.

  \textbf{Case }$n=0$.

  \textbf{Subcase }$S=\skipk$. Contradictory.

  \textbf{Subcase }$S=A$. Then $a=A$, $E'=\varepsilon$, and $S' =
  \skipk$.

  \textbf{Subcase }$S = \star\{\overline{l_i:S_i}\}$. Then $a = \star
  l_i$, $E' = \toBPA{S_i}$, and $S' = S_i$.

  \textbf{Case }$n>0$.

  \textbf{Subcase }$S = S_1;S_2$.
  If $\Unravel (S_1) = \skipk$, then $\Unravel (S) = \Unravel
  (S_2)$ with less than $n$ calls. As $\toBPA{\Unravel (S_2)} \LTSderives E'$, induction yields
  that $S_2 \LTSderives S'$ and $E' = \toBPATop{S'}$.
  As $\Unravel (S_1) = \skipk$, we know that $\DONE{S_1}$. Hence,
  ${S_1;S_2} \LTSderives S'$ and $E' = \toBPATop{S'}$.

  If $\Unravel (S_1)\ne \skipk$, then consider $\toBPA{\Unravel({S_1});
    S_2} \LTSderives E'$ because $\toBPA{\Unravel({S_1})} \LTSderives
  E_1'$, so that induction yields some $S_1'$ such that $S_1
  \LTSderives S_1'$ and $E_1' = \toBPATop{S_1'}$.

  \textbf{Subcase }$\mu x.S$.
  $\Unravel (\mu x.S) = \Unravel (S[\mu x.S/x])$ with one less
  invocation. As $\toBPA{\Unravel (S[\mu x.S/x])}
  \LTSderives E'$, induction yields that  $S[\mu x.S/x] \LTSderives
  S'$ with $E' = \toBPATop{S'} $.
\end{proof}

\begin{lemma}\label{lemma:app:bisimulation-BPA-backwards}
  Suppose that $S$ is well-formed and let $\BPAprocess = \toBPATop{S}$
  and $\BPAprocess \LTSderives \BPAprocess'$.

  There is some $S'$ such that $S \LTSderives S'$ and $\BPAprocess' = \toBPATop{S'}$.
\end{lemma}
\begin{proof}
  By induction on $S$.

  \textbf{Case }$\skipk$. Contradictory.

  \textbf{Case }$A$. For $\BPAprocess$, ${A \LTSderives \varepsilon
  }$. Choose $S' = \skipk$.

  \textbf{Case }$\star\{\overline{l_i:S_i}\}$. For $\BPAprocess$,
  ${\sum \overline{\star l_i; \toBPA{S_i}} \LTSderives[\star l_i]
    \toBPA{S_i} }$. Choose $S' = S_i$.

  \textbf{Case }$S_1;S_2$.
  If $\toBPA{S_1} \LTSderives E_1'$,
  then $S_1 \LTSderives S_1'$ and $E_1' = \toBPA{S_1'}$, by induction.
  Now, $\toBPA{S_1;S_2} \LTSderives E_1'; \toBPA{S_2} = \toBPA{S_1';
    S_2}$. Choose $S' = S_1';S_2$.

  If $\DONE{\toBPA{S_1}}$ and $\toBPA{S_2} \LTSderives E_2'$,
  then $\DONE{S_1}$ and $S_2 \LTSderives S_2'$ and $E_2' =
  \toBPA{S_2'}$, by induction. Now, $\toBPA{S_1;S_2} \LTSderives E_2' = \toBPA{S_2'}$. Choose $S' = S_2'$.

  \textbf{Case }$\mu x.S$.
  $\BPAprocess = \toBPATop{\mu x.S} = \{ x_0 = x, x = \toBPA{\Unravel
    (S[\mu x. S/x])} \}$. If $\BPAprocess \LTSderives \BPAprocess'$,
  then it must be because $\toBPA{\Unravel
    (S[\mu x. S/x])} \LTSderives E'$. Use Lemma~\ref{lemma:bpa-unr-s}
  to establish the claim.
\end{proof}

\begin{theorem}
  Suppose that $S$ is well-formed and let $\BPAprocess = \toBPATop{S}$.
  \begin{enumerate}
  \item If $ S \LTSderives[a] S'$, then $\BPAprocess \BPAderives[a]
    \BPAprocess'$ with $\BPAprocess' = \toBPATop{S'}$.
  \item If $\BPAprocess \BPAderives[a] \BPAprocess'$, then $S
    \LTSderives[a] S'$ with  $\BPAprocess' = \toBPATop{S'}$.
  \end{enumerate}
\end{theorem}
\begin{proof}
  By Lemmas~\ref{lemma:bisimulation-BPA-forwards} and~\ref{lemma:bisimulation-BPA-backwards}.
\end{proof}

%%% Local Variables:
%%% mode: latex
%%% TeX-master: "main"
%%% End:


%%% Local Variables:
%%% mode: latex
%%% TeX-master: "main"
%%% End:

%\newpage
%\section{Haskell code (excerpt)}
%\lstinputlisting{Bisimulation.hs}
%\lstinputlisting{TypeToGrammar.hs}
%\lstinputlisting{Norm.hs}
%\lstinputlisting{Grammar.hs}

\end{document}


%%% Local Variables:
%%% mode: latex
%%% TeX-master: t
%%% End:
