\begin{abstract}
  We present an algorithm to decide the equivalence of context-free
  session types, practical to the point of being incorporated in a
  compiler. We prove its soundness and completeness. We further study
  its behaviour in practice. In the process, we introduce an algorithm
  to decide the equivalence of simple grammars.


  % mandatory; please add comma-separated list of keywords
  \keywords{Types, Type equivalence, Bisimulation, Algorithm} 
\end{abstract}

  % Session types are paramount to describe the structured interaction
  % of processes via typed communication channels; however, they lack an
  % efficient description for serialization of tree-structured data in a
  % type-safe way.

  % Context-free session types were proposed as an extension of session
  % types able to capture the type-safe serialization of recursive
  % datatypes. For the sake of a practical usage, namely in the
  % definition of new programming languages, there is an urgent need for
  % an algorithm to decide type equivalence. In this work, we propose an
  % algorithm to decide type equivalence on context-free session
  % types. We prove its soundness and completeness, and validate the
  % algorithm on several examples.

%%% Local Variables:
%%% mode: latex
%%% TeX-master: "main"
%%% End:
