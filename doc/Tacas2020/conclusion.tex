\section{Conclusion}
\label{sec:conclusion}

Context-free session types are a promising tool to describe protocols
in concurrent programs. In order to be incorporated in programming
languages and effectively used in compilers, a practical algorithm to
decide bisimulation is called for.
%
Taking advantage of a process algebra graph representation of types to
decide
bisimulation~\cite{DBLP:journals/tcs/HirshfeldJM96,DBLP:conf/concur/HirshfeldM94},
we developed one such algorithm and proved it correct. The algorithm
is incorporated in a compiler for a concurrent functional language
equipped with context-free session
types~\cite{almeida.etal_freest-functional-language}.

Possible extensions to this work include
% the incorporation of subtyping~\cite{DBLP:journals/acta/GayH05}, and
addressing higher-order session types.  We also plan to extend the
implementation of the algorithm to cope with context-free grammars in
Greibach Normal Form that are not necessarily deterministic.


% Motivated by the need of a practical algorithm to decide type
% equivalence on context-free session types, we have proposed an
% algorithm that takes advantage of a process algebra graph
% representation of types to decide type equivalence.

%%% Local Variables:
%%% mode: latex
%%% TeX-master: "main"
%%% End:
