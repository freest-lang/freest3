\section{Context-free session types}
\label{sec:contextfreesession}

This section briefly introduces context-free session types, based on
the work of Thiemann and Vasconcelos~\cite{thiemann2016context}.
%
The types we consider build upon a denumerable set of \emph{variables}
and a set of \emph{choice labels}.  Metavariables $X,Y,Z$ range over
types and $\ell$ over labels.
%
We assume given a set of base types denoted by~$B$.
% Further base types could include integer and boolean types, or any
% other type for which equality is decidable.
The syntax of types is given by the grammar below.
%
\begin{gather*}
  S,T \grmeq \skipk \grmor \sharp B \grmor 
  \star\{\ell_i\colon T_i\}_{i\in I} \grmor S;T \grmor \mu X.T \grmor X
  \\
  \sharp \grmeq {}! \grmor {}? 
  \qquad \qquad
  \star  \grmeq \oplus \grmor {}\&
  % \qquad \qquad
  % a \grmeq \sharp B \grmor \star l \grmor X
\end{gather*}

In type $\mu X.T$, variable $X$ is bound in the subterm~$T$. The sets
of bound and free variables in a given type are defined
accordingly. Notation $\subs{T}{X}S$ denotes the resulting of
substituting T for the (free) occurrences of $X$ in $S$.

Judgement $\DONE S$ characterizes \emph{terminated} types:
context-free session types that exhibit no further
action~\cite{DBLP:journals/jacm/AcetoH92}.

\noindent The terminated predicate:\hfill\fbox{$\DONE T$} 
%
\begin{gather*}
  \DONE{\skipk}
  \qquad 
  \DONE X
  \qquad
  \frac{\DONE{S} \quad \DONE{T}}{\DONE{S; T}}
  \qquad
  \frac{\DONE T}{\DONE{\mu X.T}}
\end{gather*}
%
Notice that all types of the form $\mu X. \mu X_1\dots\mu X_n.X$, for
$n\ge0$, are terminated.

We are not interested in all types generated by the above grammar.
%
If $\Delta$ is a list of pairwise distinct variables, then judgement
$\isType T$ characterises the types of interest: the
\emph{well-formed} types.

\noindent The type formation system: \hfill\fbox{$\isType T$}
%
\begin{gather*}
  \frac{} 
  {
    \isType \skipk
  }
  \enspace\;
  \frac{}
  {
    \isType {\sharp B}
  }
  \enspace\;
  \frac{
    X \in \Delta
  }{
    \isType{X}
  }
  \enspace\;
  \frac{
    \isType{S}
    \enspace
    \isType{T}
  }{
    \isType{S;T}
  }
  \enspace\;
  \frac{
    \isType{T_i}
    \enspace
    (\forall i\in I)
  }{
    \isType{\star 
      \{ \ell_i \colon T_i\}_{i\in I}}
  }
  \enspace\;
  \frac{
    % \isContr T
    \neg\DONE T
    \enspace 
    \isType [\Delta, X] T
  }{
    \isType {\mu X. T}
  }
\end{gather*}

Terminated processes have a simple characterisation---types comprising
$\skipk$, $\mu$ and $;$---which justifies the inclusion of
$\neg\DONE T$ in the rules for type formation (Vasconcelos and
Thiemann~\cite{thiemann2016context} introduce a contractive judgement
for the effect).
%
Type formation serves two main purposes: ensuring that all variables
introduced by $\mu$-types are pairwise distinct and that types
underneath a~$\mu$ are not terminated. This can be clearly seen by
formation rule for $\mu$-types, where notation $\Delta,X$ is
understood as requiring $X\notin\Delta$.
%
In the sequel we assume that all types are such that $\isType[] T$ and
denote by $\mathcal{T}$ the set such types.

The set of \emph{actions} is generated by the following grammar.
\begin{equation*}
  a \grmeq \sharp B \grmor \star \ell
\end{equation*}

The \emph{labelled transition system} (LTS) for context-free session
types is given by $\mathcal{T}$ as the set of \emph{states}, the
language of \emph{actions}, and the \emph{transition relation}
$S\LTSderives T$ defined by the rules below.

\noindent The labelled transition system:\hfill\fbox{$S \LTSderives T$}
%
\begin{gather*}
  \sharp B \LTSderives[\sharp B] \skipk
  \qquad
  % X \LTSderives[X] \skipk
  % \qquad
  \star\{\ell_i\colon T_i\}_{i\in I} \LTSderives[\star \ell_j] T_j\quad
  (j\in I)
  \\
  \frac{S \LTSderives S'}{S; T \LTSderives S';T}
  \qquad
  \frac{\DONE{S} \quad T \LTSderives T'}{S; T \LTSderives T'}
  \qquad
  \frac{\subs{\mu X.S}{X}S \LTSderives T}{\mu X.S \LTSderives T}
\end{gather*}

\emph{Type bisimulation} is defined in the usual way from the labelled
transition system~\cite{sangiorgi2014introduction}.
%
We say that a type relation $\mathcal R$ is a \emph{bisimulation} if,
whenever $S\mathcal RT$, for all~$a$ we have:
%
\begin{itemize}
\item for each $S'$ with $S \LTSderives S'$, there is $T'$ such that $T
  \LTSderives T'$ and $S'\mathcal RT'$, and
\item for each $T'$ with $T \LTSderives T'$, there is $S'$ such that $S
  \LTSderives S'$ and $S'\mathcal RT'$.
\end{itemize}
%
We say that two types are bisimilar, written $S \bisim T$, if there
is a bisimulation~$\mathcal R$ with $S\mathcal RT$.

%%% Local Variables:
%%% mode: latex
%%% TeX-master: "main"
%%% End:
