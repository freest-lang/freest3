\section{Context-free session types}
\label{sec:contextfreesession}

The types we consider build upon a denumerable set of \emph{type
  variables} denoted by $X,Y,Z$, and a set \emph{choice labels}
denoted by $\ell$. We assume given a set of base types $B$ that
include the $\unitk$ type. Further base types could include integer
and boolean types, or any other type for which equality is
decidable. The syntax of types is given by the grammar below.
%
\begin{gather*}
  S,T \grmeq \skipk \grmor \sharp B \grmor 
  \star\{\ell_i\colon S_i\}_{i\in I} \grmor S;T \grmor \mu X.S \grmor X
  \\
  \sharp \grmeq {}! \grmor {}? 
  \qquad \qquad
  \star  \grmeq \oplus \grmor {}\&
  \qquad \qquad
  a \grmeq \sharp B \grmor \star l %\grmor \alpha
\end{gather*}

Let us denote by $\mathcal{T}$ the set of types.
We assume that all occurrences of variables in a type are introduced
by some $\mu$-binder, thus precluding free variables in types.
%
%We further assume that types are renamed so that all variables
%introduced by~$\mu$ are distinct.
%%
%Finally, we require types to be contractive, thus forbidding subterms
%of the form
%$\mu X_1.\mu X_2 \dots \mu
%X_n. X_1$~\cite{DBLP:journals/tcs/Courcelle83,thiemann2016context}.
%
For simplicity we removed polymorphic type variables (not bound by
$\mu$) from the grammar of session types.
% they can be treated similarly to $\sharp B$ -- This is not true:
% poly vars of un type are terminated; the others not.

The labelled transition system (LTS) for context-free session types is
given by the set of types as \emph{states}, the set of \emph{actions}
ranged over by $a$, and the \emph{transition relation} $\LTSderives$
defined by the rules below, taken from Thiemann and
Vasconcelos~\cite{thiemann2016context}.  Notation $\subs{T}{X}S$
denotes the resulting of substituting the (free) occurrences of $X$ by
$T$ in $S$. The transition relation makes use of an auxiliary judgment
$\DONE{S}$ that characterizes terminated states: session types that
exhibit no further action~\cite{DBLP:journals/jacm/AcetoH92} .
%
\begin{gather*}
  \DONE{\skipk}
  \quad
  \frac{\DONE{S} \quad \DONE{T}}{\DONE{S; T}}
  \quad
  \frac{\DONE S}{\DONE{\mu X.S}}
  \qquad \qquad \qquad
  % \alpha \LTSderives[\alpha] \skipk
  \sharp B \LTSderives[\sharp B] \skipk
  \quad
  \star\{l_i\colon S_i\} \LTSderives[\star l_j] S_j
  \\
  \frac{S \LTSderives S'}{S; T \LTSderives S';T}
  \quad
  \frac{\DONE{S} \quad T \LTSderives T'}{S; T \LTSderives T'}
  \quad
  \frac{\subs{\mu X.S}{X}S \LTSderives T}{\mu X.S \LTSderives T}
\end{gather*}

\emph{Type bisimulation}, $\TypeEquiv$, is defined in the usual way from the
labelled transition system~\cite{sangiorgi2014introduction}.
%

To define the type formation rules, we need the notion of contractivity, 
forbidding subterms of the form 
$\mu X_1.\mu X_2 \dots \mu X_n. X_1$. The contractivity and type
formation
rules are given in Figure~\ref{fig:contractivity_type_form_rules}.

\begin{figure}
  The is-contractive predicate \hfill\fbox{$\isContr S$}
  %
  \begin{gather*}
    \frac{
      \isContr{S_1}
    }{
      \isContr{S_1;S_2}
    }
    \qquad
    \frac{
      \DONE{S_1}
      \quad
      \isContr{S_2}
    }{
      \isContr{S_1;S_2}
    }
    \qquad
    \frac{
      \isContr S
    }{
      \isContr{\mu X.S}
    }
    \qquad
    \frac{}{
      \isContr[\Delta,X] X
    }
    \\
    \frac{}
    {
      \isContr {\sharp B}
    }
    \qquad
    \frac{}
    {
      \isContr {\star \{ l_i \colon T_i\}_{i\in I}}
    }
  \end{gather*}
  
  Type formation, $\Delta ::= \varepsilon \grmor \Delta, X$ \hfill\fbox{$\isType {S}$}
  %
  \begin{gather*}
    \frac{}
    {
      \isType \skipk
    }
    \qquad
    \frac{}
    {
      \isType {\sharp B}
    }
    \qquad
    \frac{
      X \in \Delta
    }{
      \isType{X}
    }
    \\\\
    \frac{
      \isType{S_1}
      \quad
      \isType{S_2}
    }{
      \isType{S_1;S_2}
    }
    \quad
    \frac{
      \isType{S_i}
    }{
      \isType{\star 
      \{ l_i \colon S_i\}}
    }
    \quad
    \frac{
      \isContr T
      \quad 
      \isType [\Delta, X] T
    }{
      \isType {\mu X. T}
    }
  \end{gather*}
  \caption{Contractivity and type formation rules.}
  \label{fig:contractivity_type_form_rules}
\end{figure}



\begin{lemma}
\label{lemma:contr_has_transition}
If $\isContr S$, then $S \LTSderives T$ for some action $a$  and type $T$. 
\end{lemma}

\begin{lemma}
	If $\isType S$, then $\isContr S$.
\end{lemma}
The converse does not hold, take: $S = (\mu X. \skipk); !\boolk$.
\
%%% Local Variables:
%%% mode: latex
%%% TeX-master: "main"
%%% End:
