\documentclass[10pt]{beamer}

\usetheme{metropolis}

\usepackage{booktabs}
\usepackage{array}
\usepackage[utf8]{inputenc}
\usepackage{pgfplots}
\usepackage{stmaryrd}
\usepackage{mathrsfs}
\usepackage{amsmath}
\usepackage{stmaryrd}
\usepackage{amsfonts}
\usepackage{movie15}
\usepackage{amssymb}
\usepackage{accsupp}
\usepackage{amsthm}
\usepackage{verbatim,cprotect}
\usepackage{listings}
\usepackage{hyperref}
\usepackage{enumerate}
\usepackage{tikz}
\usepackage[percent]{overpic}
\usepackage{tcolorbox}
\usepackage{color}
\usepackage{xspace}
\usepackage{alltt}
\usepackage{flushend}
\usepackage{graphicx}
\usepackage{wrapfig}
\usepackage[underline=false]{pgf-umlsd}
\usepackage{xcolor}

%%% Local Variables:
%%% mode: latex
%%% TeX-master: "cfst-inforum18"
%%% End:
      
 
% THEME
\newtcolorbox{mybox}{colback=orange!5!white,colframe=orange!75!black}
\newtcolorbox{myboxazul}{colback=teal!5!white,colframe=teal!75!black}

% TIKZ 
\usetikzlibrary{positioning}
\usetikzlibrary{shapes,arrows}
\usepgfplotslibrary{dateplot}
\tikzstyle{block} = [rectangle, draw, 
    text width=5cm, text centered, rounded corners, minimum height=4em]
\tikzstyle{block2} = [rectangle, draw, 
    text width=10cm, text centered, rounded corners, minimum height=4em]
\tikzstyle{line} = [draw, -latex']

\newenvironment<>{varblock}[2][.9\textwidth]{%
  \setlength{\textwidth}{#1}
  \begin{actionenv}#3%
    \def\insertblocktitle{#2}%
    \par%
    \usebeamertemplate{block begin}}
  {\par%
    \usebeamertemplate{block end}%
  \end{actionenv}}

\newenvironment{changemargin}[3]{%
\begin{list}{}{%
\setlength{\leftmargin}{#1}%
\setlength{\rightmargin}{#2}%
\setlength{\topmargin}{#3}%
}%
\item[]}
{\end{list}}

% session constructors
\newcommand{\intk}{\keyword{int}}
\newcommand{\skipk}{\keyword{skip}}

% The language
\newcommand{\freest}{\textsc{FreeST}}

% notes
\newcommand{\todo}[1]{[{\color{blue}\textbf{#1}}]}

% Keywords
\newcommand{\keyword}[1]{\mathsf{#1}}
\newcommand{\link}{\keyword{lin}}
\newcommand{\unk}{\keyword{un}}

% Kinds
\newcommand\prekind{\upsilon}
\newcommand{\stypes}{\mathcal S}
\newcommand\kinds{\stypes}
\newcommand{\types}{\mathcal T}
\newcommand\kindt{\types}
\newcommand\kindsch{\mathcal C}
\newcommand\kind{\kappa}

% Multiplicity
\newcommand\Un{\ensuremath{\mathbf{u}}} % \infty
\newcommand\Lin{\ensuremath{\mathbf{l}}} % 1 

% Grammars
\newcommand{\grmeq}{\; ::= \;}
\newcommand{\grmor}{\;\mid\;}

% type constructors
\newcommand\tcBase{B}
\newcommand\tcLolli\multimap
\newcommand\tcFun\to
\newcommand\tcBang{\mathop!}

% Keywords for types
\newcommand\kRec{\keyword{rec}}
\newcommand\kForall{\keyword{forall}}

% Types
\newcommand{\tskip}{\keyword{Skip}}
\newcommand\tSemi[2]{#1;#2}
\newcommand\tOut[1]{\tcBang#1}
\newcommand\tIn[1]{?#1}
\newcommand{\tMsg}[1]{\sharp{#1}}
\newcommand\tIChoice[1]{\oplus{#1}}
\newcommand\tEChoice[1]{\&{#1}}
\newcommand{\tChoice}[1]{\star{#1}}
\newcommand{\tData}[1]{[{#1}]}
\newcommand\tUnFun[2]{#1\tcFun#2}
\newcommand\tLinFun[2]{#1\tcLolli#2}
\newcommand\tPair[2]{(#1,\,#2)}
\newcommand\tDatatype[1]{{[#1]}}
\newcommand\tRec[2]{\kRec\,#1\,.\,#2}
%\newcommand\tForall[2]{\forall\,#1\,.\,#2}
%\newcommand\tForall[2]{\kForall\,#1\,=>\,#2}
\newcommand\tForall[2]{\forall\,#1\Rightarrow#2}
% Basic Types
\newcommand{\unite}{()}
\newcommand{\inte}{\keyword{Int}}
\newcommand{\chare}{\keyword{Char}}
\newcommand{\boole}{\keyword{Bool}}

\newcommand\tRecK[2]{\kRec\,#1\,.\,#2}
% Environments
\newcommand{\Empty}{\varepsilon}
\newcommand\emptyEnv{\Empty}
\newcommand\kindEnv{\Delta}
\newcommand\varEnv{\Gamma}

% Variables
\newcommand\vare[1]{#1}
\newcommand\unlete[3]{\keyword{let} \; #1 = #2 \; \keyword{in} \; #3} 

% Applications
\newcommand\appe[2]{#1#2}
\newcommand\tappe[2]{#1[#2]}

% Conditional
\newcommand\conditionale[3]{\keyword{if}\;#1\;\keyword{then}\;#2\;\keyword{else} \; #3}

% Goal
\newcommand\Goal{\vdash}

% Pairs
\newcommand\paire[2]{(#1,#2)}
\newcommand\binlete[4]{\keyword{let}\;#1, #2 = #3\;\keyword{in}\;#4}

% Session Types
\newcommand\newe[1]{\keyword{new}\;#1}
\newcommand\sende[2]{\keyword{send}\;#1\; #2}
\newcommand\sendce[1]{\keyword{send}\;#1}
\newcommand\recve[1]{\keyword{receive}\;#1}
\newcommand\selecte[2]{\keyword{select}\;#1\;{#2}}
\newcommand\matche[2]{\keyword{match}\;#1\;\keyword{with}\;#2}

% Fork
\newcommand\forke[1]{\keyword{fork}\;#1}

% Datatypes
\newcommand{\ctrcte}{C}
\newcommand\casee[2]{\keyword{case}\;#1\;\keyword{of}\;#2}

% Sequents
\newcommand{\isType}[3][\Delta]{{#1} \vdash {#2} : {#3}}
\newcommand{\algkindout}[3][\kindEnv]{{#1} \Alg {#2} \shortrightarrow{ #3}}
\newcommand{\algkindin}[3][\kindEnv]{{#1} \Alg {#2} \shortleftarrow {#3}}
\newcommand{\subkind}[2]{{#1} <: {#2}}
\newcommand{\algtypeout}[4][\kindEnv;\varEnv]{{#1} \Alg {#2} \shortrightarrow {#3};{#4}}
%\newcommand{\algtypein}[4][\kindEnv;\varEnv]{{#1} \Alg {#2}\colon {#3}\shortrightarrow {#4}}
\newcommand{\algtypein}[4][\kindEnv;\varEnv]{{#1} \Alg {#2}\shortleftarrow {#3}; {#4}}
\newcommand{\ctxequiv}[3][\kindEnv]{{#1} \vdash \Equiv{#2}{#3}}
\newcommand{\typeequiv}[3][\kindEnv]{{#1} \vdash \Equiv{#2}{#3}}
\newcommand{\isqualifier}[3][\kindEnv]{{#1} \vdash {#2}\colon{#3}}
\newcommand{\isLin}[2][\kindEnv]{\isqualifier[#1]{#2}\link}
\newcommand{\isUn}[2][\kindEnv]{\isqualifier[#1]{#2}\unk}
\newcommand{\contractive}[2][\kindEnv]{{#1} \vdash_{\textsf c} {#2}}
%\newcommand\Alg{\vdash_{\textsf a}}
\newcommand\Alg{\vdash}

% Operators
\newcommand\Extract[1]{\leadsto_{#1}}% \rightlsquigarrow}
\newcommand{\subs}[3]{[{#1}/{#2}]{#3}}
\newcommand\dual[1]{\overline{#1}}

% Predicates
%\newcommand\Equiv[2]{#1\,\thicksim\,#2}
\newcommand\Equiv[2]{#1\,\sim\,#2}

% Colour

\newcommand{\Blue}[1]{\textcolor{blue}{#1}}
\newcommand{\Red}[1]{\textcolor{red}{#1}}
\newcommand{\Brown}[1]{\textcolor{brown}{#1}}
\newcommand{\highlight}[1]{\Blue{#1}}

% ECLIPSE LOOK

\newcommand\Small{\small}
%\newcommand\Small{\fontsize{7.5}{8}\selectfont} 

\definecolor{darkviolet}{rgb}{0.5,0,0.4}
\definecolor{darkgreen}{rgb}{0,0.4,0.2} 
\definecolor{darkblue}{rgb}{0.1,0.1,0.9}
\definecolor{darkgrey}{rgb}{0.5,0.5,0.5}
\definecolor{lightblue}{rgb}{0.4,0.4,1}

\lstdefinestyle{eclipse}{
  breaklines=true,
  basicstyle=\sffamily\Small,
  emphstyle=\color{red}\bfseries, 
  keywordstyle=\color{darkviolet}\bfseries,
  commentstyle=\color{darkgreen},
  stringstyle=\color{darkblue},
  numberstyle=\color{darkgrey},%\lstfontfamily,
  emphstyle=\color{red},
  % get also javadoc style comments
  morecomment=[s][\color{lightblue}]{/**}{*/},
  %columns=fullflexible, %spaceflexible, %flexible, fullflexible             
  %  escapeinside=`',
  %  escapechar=@,
  showstringspaces=false,
  numbers=left,
  tabsize=2
}

\lstdefinestyle{eclipse-Haskell}{
  breaklines=true,
  basicstyle=\sffamily\Small,
  emphstyle=\color{red}\bfseries, 
  keywordstyle=\color{darkviolet}\bfseries,
  commentstyle=\color{darkgreen},
  stringstyle=\color{darkblue},
  emphstyle=\color{red},
  % get also javadoc style comments
  morecomment=[s][\color{lightblue}]{/**}{*/},
  %columns=fullflexible, %spaceflexible, %flexible, fullflexible             
  %  escapeinside=`',
  %  escapechar=@,
  showstringspaces=false,
  numbers=none,
  tabsize=2
}

\lstdefinelanguage{freest}{
  style=eclipse,
  morekeywords=[1]{Int, Char, Bool, Skip, type, dualof, forall, rec, let, in, if, then, else, new, send, receive, select, fork, case, of, data, match, with, True, False},
  sensitive=true,
  literate=
  {->}{$\rightarrow$}2
  {-o}{$\multimap$}2
  {=>}{$\Rightarrow$}2
  {alpha}{$\alpha$}1,
  breaklines=true,
  morecomment=[l]{--},%
  morecomment=[s]{{-}{-}},%
  morestring=[b]',%
  morestring=[b]",%
  morestring=[s]{`}{`},%
}

\lstset{
  language=freest,
  numbers=none
}
 
%%% Local Variables:
%%% mode: latex
%%% TeX-master: "main"
%%% End:

%\lstset{identifierstyle=\color{identifierColor}}

\title[Deciding the bisimilarity of context-free session types]{Deciding the bisimilarity of context-free session types}

\date{
\vspace*{1cm}
\begin{center}
	March 2021
\end{center}}
\author[B.Almeida, A.Mordido, V.Vasconcelos]{\underline{Bernardo Almeida}, Andreia Mordido, and Vasco T. Vasconcelos}
\institute[LASIGE, Faculdade de Ci\^encias, ULisboa]{LASIGE, Faculdade de Ci\^encias, Universidade de Lisboa \\\\
}
%\titlegraphic{\hfill\includegraphics[height=1.5cm]{logo.png}}

\begin{document}
\lstset{language=Haskell}

\maketitle

\begin{frame}[fragile]{Motivation}
\vspace*{5mm}

% \begin{lstlisting}[language=CFST]
% data Tree = Leaf | Node Int Tree Tree
% \end{lstlisting}

% \begin{lstlisting}[language=CFST]
% sendTree Leaf c =
%   select Leaf c
% sendTree (Node x l r) c =
%   let c1 = select Node c
%       c2 = send x c1
%       c3 = sendTree l c2
%       c4 = sendTree r c3
%    in c4
%\end{lstlisting}

\begin{itemize}
\item Context-free session types from Thiemann and Vasconcelos
  \vskip 0.4cm
\item Context-free session types VS. regular session types
  \vskip 0.4cm
\item Thiemann and Vasconcelos proved that type equivalence is decidable
  \vskip 0.4cm
\item Developed an algorithm that can be used in a compiler
\end{itemize}

\vspace*{0.6cm}
\hfill \includegraphics[height=4cm]{img/tree}
\end{frame}

\begin{frame}
  \frametitle{Context-Free Session Types}
  \begin{align*}
  S,T &\grmeq \skipk \grmor \sharp B \grmor 
  \star\{\ell_i\colon T_i\}_{i\in I} \grmor S;T \grmor \mu X.T \grmor X
    \\&\\
  \sharp &\grmeq {}! \grmor {}? 
  \\&\\
  \star  &\grmeq \oplus \grmor {}\&
  % \qquad \qquad
  % a \grmeq \sharp B \grmor \star l \grmor X
\end{align*}

\end{frame}

%\begin{frame}{Table of contents}
%    \frametitle{Layout}
%    \setbeamertemplate{section in toc}[sections numbered]
%  \tableofcontents[hideallsubsections]
%\end{frame}

%\section{Algorithm for checking the equivalence of context-free session types}

\begin{frame}{Algorithm for checking the equivalence of CFST \hfill {\color{mLightBrown}Main stages}}
	\begin{center}
    	\begin{tikzpicture}[node distance = 2.5cm, auto]
      		% Place nodes
          \node [block2] (typeToGrammar) {{\bf Convert types to a simple
              grammar}\\ Translates types into a (finite) set of productions};
          \node [block2,below=5mm of typeToGrammar] (prune) {{\bf Prune
              unreachable symbols}\\ Streamlines the grammar by removing
            unreachable symbols in unnormed sequences of non-terminal symbols};%
          \node [block2,below=5mm of prune] (simplifyExpand) {{\bf Simplify and
              expand}\\ Alternates between simplification and expansion
            operations, until reaching a successful branch in the expansion tree
            or concluding that all branches are unsuccessful };
      		% Draw edges
      		\path [line] (typeToGrammar) -- (prune);
      		\path [line] (prune) -- (simplifyExpand);
   		\end{tikzpicture}
  	\end{center}
\end{frame}


% \begin{frame}{Convert types to a grammar}  
%   	\vspace*{-1cm}
%   	\hspace*{-3mm}\parbox{8cm}{
%   	\vspace*{-2.5cm}
%   	  	We consider a finite set of productions $X \rightarrow a \vec Y$ where:
%    	\begin{itemize}
%      	\item $X, Y$ represent \emph{non-terminal symbols}
%      	\item $a$ is a \emph{terminal symbol} (a \emph{label} in the labelled transition system)
%    	\end{itemize}
% 	}\hfill\includegraphics[height=2.8cm]{img/typeToGrammar.png}
% 	Context-free session types are seen as \emph{simple grammars}:
%   	\begin{itemize}
% 		\item context-free grammars in \emph{Greibach normal form} 
% 		\item s.t. for each $X$ and $a$ there is at most one production $X \rightarrow a \vec Y$
%   	\end{itemize}
% \end{frame}


\begin{frame}{Convert types to a grammar}

  	\hfill\includegraphics[height=2.8cm]{img/typeToGrammar.png}

  	\vspace*{-3cm}
  	{\color{teal} Example} \\
  
  	\begin{tabular} {l l }
    %   $S$ & $\triangleq (\mu X_1. \&\{n: X_1;X_1;?\intk, \ell: ?\intk\});(\mu X_2. !\intk ; X_2;X_2)$\\\smallskip 
    % $T$ & $\triangleq (\mu Y_1. \&\{n: Y_1;Y_1, \ell: \skipk\};?\intk);(\mu Y_2. !\intk ; Y_2)$
  		$S$ &$\triangleq (\mu x. \&\{n\colon x;x;?\,\intk,
      \ell\colon ?\,\intk \});( \mu z. !\,\intk; z;z )$\\\smallskip  
        $T$ &$\triangleq (\mu y. \&\{n\colon y;y,
      \ell\colon \skipk \}; ?\,\intk); (\mu z. !\,\intk; z)$\\
  	\end{tabular}
  
  	\pause \vspace*{5mm}
  	\begin{tabular}{l l l |l l l}
    	&Productions for $S$& & & Productions for $T$& \\ \hline
    	&\uncover<3-> {{\color{gray}$X_{} \rightarrow \,X_1 X_4$}} & & &\uncover<3-> {{\color{gray}$Y_{} \rightarrow \,Y_1 Y_3 $} }&\\
    	&$X_1 \rightarrow \& n\, X_1 X_1 X_2$ & & &$Y_1 \rightarrow \& n\, Y_1 Y_1 Y_2 $ &\\
    	&$X_1 \rightarrow \& \ell\, X_3$ &&& $Y_1 \rightarrow \& \ell \,Y_2 $&\\
    	&$X_2 \rightarrow \,? \intk$&&& $Y_2 \rightarrow \,? \intk$&\\
    	&$X_3 \rightarrow \,? \intk$&&& $Y_3 \rightarrow \,!\intk\, Y_3$&\\
    	&$X_4 \rightarrow \,!\intk\, X_4 X_4$ &&&&\\
  	\end{tabular}	
\end{frame}

% \begin{frame}{Algorithm for checking the equivalence of CFST \hfill {\color{mLightBrown}Main stages}}

%   	\begin{center}
%     	\begin{tikzpicture}[node distance = 2.8cm, auto]
%       		% Place nodes
%       		\node [block2] (typeToGrammar) {{\bf Convert types to a grammar}\\ 
%       		{\color{gray}\texttt{LOHaskellCode $\sim$ 100}}\\
%       		Soundness {\color{olive}\checkmark}
%       		\begin{enumerate}
%     	 		\item Conversion of types into BPAs is sound\footnote{ Thiemann and Vasconcelos. Context-free session types. 2016.}.
%     	 		\item Conversion of BPAs to Greibach Normal Form without altering the solution is sound\footnote{ Baeten et al. Decidability of bisimulation equivalence for process generating context-free languages.  1993.}.
%       		\end{enumerate}
%       		}; 
%       		\node [block2, below=5mm of typeToGrammar] (prune) {{\bf Prune unnormed productions}};
%       		\node [block2, below=5mm of prune] (simplifyExpand) {{\bf Simplify and expand} };
%       		\path [line] (typeToGrammar) -- (prune);
%       		\path [line] (prune) -- (simplifyExpand);
%     	\end{tikzpicture}
%   	\end{center}
% \end{frame}

\begin{frame} {Prune unnormed productions}
	\hfill\includegraphics[height=2.8cm]{img/prune.png}

	\vspace*{-2.8cm}
	\metroset{block=fill}
	\begin{varblock}[7.8cm]{Definition ((Un)normed symbols)}
			\smallskip 
		A sequence of symbols $\alpha$ is \emph{normed} if there is a path $\alpha \rightarrow \cdots \rightarrow \varepsilon$. 
		Otherwise, $\alpha$ is said to be \emph{unnormed}.\\
	\end{varblock}

	Christensen, Huttel, and Stirling noted that:\\\smallskip
	\hspace*{1cm}	$\alpha \sim \alpha \beta$, whenever $\alpha$ is unnormed.

	\pause
	\vspace*{-1mm}
	{\color{teal}\rule{3cm}{2pt}}\\
   	{\color{teal} Example} \\\smallskip
	\vspace*{-7mm}

	$$S \triangleq (\mu x. \&\{n\colon x;x;?\,\intk,
      \ell\colon ?\,\intk \});( \mu z. !\,\intk; z;z )$$

	\vspace*{-2mm}
	\hspace*{5mm}\begin{tabular}{l l l }
 		&Productions for $S$&  \\ \hline
 	 	&$X_1 \rightarrow \& n\, X_1 X_1 X_2$ &\\
 	 	&$X_1 \rightarrow \& \ell\, X_3$ &\\
 	 	&$X_2 \rightarrow \,? \intk$&\\
 	 	&$X_3 \rightarrow \,? \intk$&\\
        &$X_4 \rightarrow \,!\intk\, X_4$ {\color{red} $X_4$ } &\\                   
	\end{tabular}
	\hspace*{1cm} \pause
	\begin{tabular}{l l l }
 		&Productions for \emph{pruned} $S$&  \\ \hline
  		&\hspace*{4mm}$X_1 \rightarrow \& n\, X_1 X_1 X_2$ & \\
  		&\hspace*{4mm}$X_1 \rightarrow \& \ell\, X_3$ &\\
 		&\hspace*{4mm}$X_2 \rightarrow \,? \intk$&\\
  		&\hspace*{4mm}$X_3 \rightarrow \,? \intk$&\\
  		&\hspace*{4mm}$X_4 \rightarrow \,!\intk\, X_4 $ &\\
	\end{tabular}
 	\vspace*{2mm}
\end{frame}

%%% COMMENT
% \begin{frame}{Algorithm for checking the equivalence of CFST \hfill {\color{mLightBrown}Main stages}}

% 	\begin{center}
% 		\begin{tikzpicture}[node distance = 2.8cm, auto]
%    			% Place nodes
%     		\node [block2] (typeToGrammar) {{\bf Convert types to a grammar}\\ 
%     		{\color{gray}\texttt{LOHaskellCode $\sim$ 100}}\\
%    			Soundness {\color{olive}\checkmark}
%     		}; 
%     		\node [block2, below=5mm of typeToGrammar] (prune) {{\bf Prune unnormed productions} \\ 
%     		{\color{gray}\texttt{LOHaskellCode $\sim$ 30}}\\
%     		Soundness {\color{olive}\checkmark} 
%     		\begin{enumerate}
%     			\item Pruning unnormed productions is sound\footnote{ Christensen, Huttel, and Stirling. Bisimulation equivalence is decidable for all \\ context-free processes. 1995}
%     		\end{enumerate}
%     		};
%     		\node [block2, below=5mm of prune] (simplifyExpand) {{\bf Simplify and expand} };
%     		% Draw edges
%     		\path [line] (typeToGrammar) -- (prune);
%     		\path [line] (prune) -- (simplifyExpand);
% 		\end{tikzpicture}
% 	\end{center}
% \end{frame}


\begin{frame} {Simplify and expand}
	\hfill\includegraphics[height=2.8cm]{img/simplifyExpand.png}
	
	\vspace*{-3cm}
	Following the ideas from Hirshfeld, Jan{\v{c}}ar, Moller, \\
	a bisimulation is seen as an {\bf expansion tree}\\
    that alternates between:
	\begin{itemize}
		\item { expansion operations}
		\item { simplification operations	}
	\end{itemize}
	\begin{center}
		\hspace*{-2mm}\includegraphics[height=4cm]{img/expand_simplify}
	\end{center}	
\end{frame}


\begin{frame} {Simplify and expand}
	An expansion tree alternates between:
	\vspace*{-1mm}
	\begin{itemize} 
		\item {\color{mLightBrown} \bf expansion operations} - a single derived node results from the expansion of the parent node. \pause
		\item {\color{teal}\bf simplification operations	}
		\begin{itemize}
			\item {\bf reflexive rule}: omit from a node $N$ any reflexive pair.
			\item {\bf congruence rule}: omit from a node $N$ any pair that belongs to the least congruence containing the ancestors of $N$. \pause
			\item {\bf basic process algebra rules}\\
			\vspace*{-2mm}
			\hspace*{-2cm}\includegraphics[height=4cm]{img/bpa_new}\smallskip
		\end{itemize}
	\end{itemize}
\end{frame}


% \begin{frame}{Simplify and Expand}
% 	All these transformation rules preserve the \emph{safeness property}:
% 	\metroset{block=fill}
% 	\begin{block}{Safeness property\footnote{Jan{\v{c}}ar, Moller. Techniques for decidability and undecidability of bisimilarity. 1999}}
% 		\smallskip
% 		$S\sim T$ iff the expansion tree rooted at $\{(S,T\,)\}$ has a successful branch.	
% 	\end{block}
	
% 	The finite witness property holds\footnotemark[\value{footnote}]:
	
% 	\begin{block}{Finite witness property\footnotemark[\value{footnote}]\footnote{ Christensen et al. Bisimulation equivalence is decidable for all CF processes. 1995}}
% 		\smallskip
% 		If $S\sim T$, then there exists a {\bf finite successful branch} in the expansion tree.
% 	\end{block}
% \end{frame}

\newcommand{\typeA}{\semiT {(\recT x \choiceExtT {n\colon {\semiT x {\semiT
          x{\msgInT\intk}}}, \ell \colon \msgInT\intk})} {(\recT z
    {\semiT{\msgOutT\intk}{\semiT zz}})}}
\newcommand{\typeB}{\semiT {(\semiT {\recT y {\choiceExtT {n\colon\semiT yy,
          \ell\colon\skipk}}}{\msgInT\intk)}}{(\recT z \semiT{\msgOutT\intk}z)}}

\newcommand{\header}[2]{
    \footnotesize
    \setlength\tabcolsep{4pt}
    \vskip 0.2cm
	\begin{tabular} {l l l l l }    
      {\color{teal}\rule{3cm}{2pt}} &\hfill\hspace{1.5cm} & $S$ & $\triangleq$ &  $\typeA$\\
      {\color{teal} #1}             &\hfill & $T$ & $\triangleq$ &  $\typeB$\\
      {\color{teal} #2}             &\hfill &     &              &\\
	\end{tabular}
	\vspace*{2mm}
    \normalsize
}

\begin{frame}
    \header{Simplify \& expand --}{An example:}
	\includegraphics[width=11.5cm]{img/exemplo-9}\smallskip
\end{frame}

\begin{frame}
	\header{Simplify \& expand --}{An example:}
	\includegraphics[width=11.5cm]{img/exemplo-8}\smallskip
\end{frame}


\begin{frame}
	\header{Simplify \& expand --}{An example:}
	\includegraphics[width=11.5cm]{img/exemplo-7}\smallskip
\end{frame}

\begin{frame}
	\header{Simplify \& expand --}{An example:}
	\includegraphics[width=11.5cm]{img/exemplo-6}\smallskip
\end{frame}

\begin{frame}
	\header{Simplify \& expand --}{An example:}
	\includegraphics[width=11.5cm]{img/exemplo-5}\smallskip
\end{frame}

\begin{frame}
	\header{Simplify \& expand --}{An example:}
	\includegraphics[width=11.5cm]{img/exemplo-4}\smallskip
\end{frame}

\begin{frame}
	\header{Simplify \& expand --}{An example:}
	\includegraphics[width=11.5cm]{img/exemplo-3}\smallskip
\end{frame}

\begin{frame}
	\header{Simplify \& expand --}{An example:}
	\includegraphics[width=11.5cm]{img/exemplo-2}\smallskip
\end{frame}

\begin{frame}
	\header{Simplify \& expand --}{An example:}
	\includegraphics[width=11.5cm]{img/exemplo-1}\smallskip
\end{frame}

\begin{frame}
  \header{Full example}{}
  \vskip -0.9cm
  \includegraphics[width=11.5cm]{img/example_expansion_tree}\smallskip
\end{frame}

% \begin{frame}{Algorithm for checking the equivalence of CFST \hfill {\color{mLightBrown}Main stages}}
% 	\begin{center}
% 		\begin{tikzpicture}[node distance = 2.8cm, auto]
%     		% Place nodes
%     		\node [block2] (typeToGrammar) {{\bf Convert types to a grammar}\\ 
%     		{\color{gray}\texttt{LOHaskellCode $\sim$ 100}}\\
%     		Soundness {\color{olive}\checkmark}
%     		}; 
%     		\node [block2, below=5mm of typeToGrammar] (prune) {{\bf Prune unnormed productions} \\ 
%     		{\color{gray}\texttt{LOHaskellCode $\sim$ 30}}\\
%     		Soundness {\color{olive}\checkmark} 
%     		};
%     		\node [block2, below=5mm of prune] (simplifyExpand) {{\bf Simplify and expand}\\ 
%     		{\color{gray}\texttt{LOHaskellCode $\sim$ 170}}\\
%     		Soundness {\color{olive}\checkmark} 
%     		\begin{enumerate}
%     			\item The \emph{safeness property ensures soundness}\footnote{Jan{\v{c}}ar, Moller. Techniques for decidability and undecidability of bisimilarity.\\ 1999}
%    	 		\end{enumerate}
%  			};
%     	% Draw edges
%     		\path [line] (typeToGrammar) -- (prune);
%     		\path [line] (prune) -- (simplifyExpand);
% 		\end{tikzpicture}
% 	\end{center}
% \end{frame}

\begin{frame}
  \frametitle{Optimisations}
  To enhance the performance of the algorithm, we implemented the following optimisations
  \begin{enumerate}
  \item Eliminating redundant productions in the grammar
    \vskip 0.5cm
  \item Using a filter rule that removes nodes with hopeless pairs
    \vskip 0.5cm    
  \item Using a double-ended queue to prepend promising children
  \end{enumerate}
\end{frame}

\begin{frame}{Evaluation}
  \begin{itemize}
  \item Started with a carefully crafted, manually produced, suite of tests.
  \smallskip
  \item They were small, lacking diversity
  \smallskip
  \item We went for automated test generation: QuickCheck
  \smallskip
  \item Generating valid (well-formed) types is easy, but the probability of
    generating two types that are bisimilar is extremely low.
    \smallskip
  \item We generate arbitrary pairs of types that are \textbf{bisimilar}
    by construction (based on properties of type bisimilarity).
    \smallskip
  \item For \textbf{non-bisimilar pairs} we add five anti-axioms to the previous list.
  \end{itemize}
     
\end{frame}

% \begin{frame}{Generating valid types}
% \textbf{Properties of type bisimilarity (a subset)}
%   \begin{enumerate} 
%     % Congruence 
%   \item $\skipk \sim_\types \skipk$ and $\sharp B \sim_\types \sharp B$;
%   \item $S;T \sim_\types U;V$ if $S \sim_\types U$ and $T \sim_\types V$;
%   \item $\mu X.S \sim_\types \mu X.T$ if $S \sim_\types T$;
%   \item $\star\{\ell_i\colon S_i\}_{i\in I}\sim_\types
%     \star\{\ell_i\colon T_i\}_{i\in I}$ if $(S_i \sim_\types T_i)_{i\in
%       I}$;
%     % Laws for sequential composition
%   \item $S\sim_\types T;\skipk$ and $S\sim_\types \skipk;T$ if $S \sim_\types T$;
%   \item $\star\{\ell_i\colon S_i\}_{i\in I};U\sim_\types
%     \star\{\ell_i\colon T_i;V\}_{i\in I}$ if $(S_i \sim_\types T_i)_{i\in
%       I}$ and $U \sim_\types V$;
% <    \item \dots
%   \end{enumerate}
  
% \end{frame}

% \begin{frame}
%   \frametitle{Generating invalid types}
%   For \textbf{non-bisimilar pairs} we add five anti-axioms to the previous list

%   \begin{enumerate}
%   \item $\skipk \not\sim_\types \sharp B$;
%   \item $?B \not\sim_\types !B$;
%   \item $\skipk \not\sim_\types \star\{\ell_i\colon S_i\}_{i\in I}$;
%   \item $\oplus\{\ell_i\colon S_i\}_{i\in I} \not\sim_\types \&\{\ell_i\colon S_i\}_{i\in I}$;
%   \item $\star\{\ell_i\colon S_i\}_{i\in I} \not\sim_\types \star\{\ell_j\colon S_j\}_J$
%   \end{enumerate}
% \end{frame}



\begin{frame}
  \frametitle{Results (I)}
  \begin{itemize}
  \item Two test suites comprise 2000 entries. (number of
    nodes, in the syntax tree, ranging from 1 to 200).    
  \item Tests were run under a timeout of 2 minutes.
  \item We tested all the optimisations and their combinations.    
  \end{itemize}
  \pause
  \includegraphics[scale=0.20]{img/distribution_boxplot.pdf}
\end{frame}

\begin{frame}
  \frametitle{Results (II)}
  \includegraphics[width=.55\textwidth]{img/nodes_time.pdf}%
  \includegraphics[width=.55\textwidth]{img/timeouts.pdf}

  Based on these results, we have variant B1 in production: redundant
  productions are eliminated before exploring the expansion tree
\end{frame}

\begin{frame}
  \frametitle{Final remarks}

  \begin{itemize}
  \item We developed an algorithm to decide the bisimilarity of context-free
    session types and proved it \emph{sound} and \emph{complete}.
  \item The algorithm is incorporated in a compiler for a concurrent functional
    language equipped with context-free session types.
    
  \item Despite being exponential, the algorithm behaves well to types up to 200 nodes.
  \item \textbf{Future work:} Provide a counter-example when two types are not equivalent.
  \end{itemize}
 \end{frame}




 
\plain{Thank you!}


\end{document}

%%% Local Variables:
%%% mode: latex
%%% TeX-master: t
%%% End:
