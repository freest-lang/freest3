\section{Expressions, statics, and dynamics}
\label{sec:definitions}

% PROCESSES 
\subsection{Processes}
\label{sec:processes}

\begin{figure}[t]
  \begin{align*}
    v \grmeq& \sendk %&& \text{Values}\\
    \grmor \recvk
    \grmor ()
%    \grmor& \text{(other literals)}\\
    \grmor \lambda a.e
    \grmor (v,v)
    \grmor \inject lv
    \\
    e \grmeq& v %&& \text{Expressions}\\
    \grmor a
    \grmor \newk
    \grmor ee
    \grmor \fix ae
    \grmor (e,e)
    \grmor \ink\,l\,e\\
    \grmor& \letin{a,b} e e
    \grmor \forkk\, e
    \grmor \match e {l_i\rightarrow e_i}_{i\in I}\\
%    \grmor& \letin aee\\
    \grmor& \select le
    \grmor \casek\,e\,\ofk\,\{l_i\rightarrow e_i\}_{i\in I}
    \\
    p \grmeq& e %&& \text{Processes}\\
    \grmor p \PAR p
    \grmor (\nu a,b)p
  \end{align*}
  \caption{Values, expressions, and processes}
  \label{fig:processes}
\end{figure}

%%% Local Variables:
%%% mode: latex
%%% TeX-master: "main"
%%% End:


Values, expressions, and processes in Figure~\ref{fig:processes}.

% REDUCTION
\begin{figure}[t]
  \begin{gather*}
    % FUNCTIONAL
    (\lambda a.e)v \reduces e\subs va
    \qquad
    \letin{a,b}{(u,v)}{e} \reduces e\subs ua\subs vb
    \\
    \match{(\ink\,l_j\,v)}{l_i\rightarrow e_i}  \reduces e_jv
    \qquad
    \fix ae \reduces e \subs{\fix ae}{a}
    \\
    \frac{e_1 \reduces e_2}{E[e_1] \reduces E[e_2]}
    \qquad
    E[\fork e] \reduces E[()] \PAR e
    \\
    E[\newk] \reduces (\nu a,b)E[(a,b)]
    \\ % PROCESSES
    (\nu a,b)(E_1[\send va] \PAR E_2[\recv b])
    \reduces
    (\nu a,b)(E_1[a] \PAR E_2[(v,b)])
    \\
    (\nu a,b)(E_1[\selectk\,l_j\,a] \PAR E_2[\casek\,b\,\ofk\,\{l_i\rightarrow e_i\}])
    \reduces  \qquad \qquad  \qquad\\
    \qquad \qquad \qquad \qquad \qquad \qquad \qquad \qquad (\nu a,b)(E_1[a] \PAR E_2[e_jb])
    \\
    \frac{p \reduces p'}{p\PAR q \reduces p'\PAR q}
    \quad
    \frac{p \reduces p'}{(\new a,b)p \reduces (\new a,b)p'}
    \quad
    \frac{p \equiv q\quad q \reduces q'}{p \reduces q'}
  \end{gather*}
  Context $E_1$ (resp.~$E_2$, resp.~$E$) does not bind~$a$ (resp.~$b$,
  resp.~$a$ and~$b$).
  \\
  Dual $(\nu b,a)$ rules for $\sendk/\recvk$ and $\selectk/\casek$
  omitted.
  \caption{Reduction relation}
  \label{fig:reduction}
\end{figure}

%%% Local Variables:
%%% mode: latex
%%% TeX-master: "main"
%%% End:


Structural congruence, $\equiv$, is the smallest relation that
includes the commutative monoid rules (for $\mid$ and $()$) and scope
extrusion. Reduction in Figure~\ref{fig:reduction}.

% RUNTIME ERRORS

Runtime errors. The \emph{subject} of an expression~$e$, $\subj(e)$,
is~$x$ in the following cases and undefined in all other cases.
%
\begin{equation*}
  \send \_x \qquad
  \recv x \qquad
  \select \_x \qquad
  \case x\_
\end{equation*}

Two expressions \emph{agree} in the following cases.
%
\begin{align*}
  \agree^{xy}(\send \_x, \recv y) &\qquad
  \agree^{xy}(\recv x,\send \_y)\\
  \agree^{xy}(\select \_x, \case y\_) &\qquad
  \agree^{xy}(\case x\_,\select \_y)
\end{align*}

A process is an \emph{error} if it is structurally congruent to some
process that contains a subexpression or subprocess of one of the
following forms:
%
\begin{enumerate}
\item $\letin{xy}{v}{e}$ and $v$ is not a pair;
\item $\match{v}{l_i\rightarrow e_i}_{i\in I}$ and $v\ne(\ink\,l_j\,v')$ for some $v'$ and
  $j\in I$;
\item $E_1[e_1] \mid E_2[e_2]$ and $\subj(e_1) = \subj(e_2) = x$ where
  neither $E_1$ nor $E_2$ bind~$x$;
\item $(\new xy)(E_1[e_1] \mid E_2[e_2] \mid p)$ and $\subj(e_1)=x$
  and $\subj(e_2)=y$ and $\neg\agree^{xy}(e_1,e_2)$ where $E_1$ does not
  bind~$x$ and $E_2$ does not bind~$y$;
\item $  (\new xy)(E_1[\selectk\,l_j\,x] \mid
  E_2[\casek\,y\,\ofk\,\{l_i\rightarrow e_i\}_{i\in I}] \mid p) $
  where $E_1$ does not bind $x$ and $E_2$ does not bind $y$ and
  $j\notin I$ (or the same with $x$ and $y$ swapped).
\end{enumerate}

% TYPES

% BEGIN - Moved to the section on types
% \subsection{Types}
% \label{sec:types}


% Base sets: \emph{recursion variables}~$x$; \emph{type
%   variables}~$\alpha$; \emph{primitive types}~$B$ including for
% example $\unitk$ and $\intk$. Grammars for session types, types, and
% type schemes in Figure~\ref{fig:types}.

% % WELL FORMED TYPE EXPRESSIONS
% Not all types in the language generated by the grammar are of
% interest; as customary, we consider recursive types like $\mu x. \mu y.x$ or
% $\mu x.(x;x)$ meaningless as they only have trivial interpretations. We
% follow MacQueen at al.~\cite{MacQueen198695} and require recursive
% types to be contractive.
% %
% %\begin{figure}[tp]
  \begin{gather*}
    \frac{}{\GEnv \Contr T_1 \to T_2}% : \Productive}
    \qquad
    \frac{}{\GEnv \Contr T_1 \multimap T_2} % : \Productive}
    \qquad
    \frac{}{\GEnv \Contr T_1 \otimes T_2} % : \Productive}
    \\
    \frac{}{\GEnv \Contr [l_i:T_i]_{i\in I}} % : \Productive}
    \qquad
    \frac{}{\GEnv \Contr B} % : \Productive}
    \qquad
    \frac{}{\GEnv \Contr {!B}} % : \Productive}
    \qquad
    \frac{}{\GEnv \Contr {?B}} % : \Productive}
    \\
    \frac{}{\GEnv \Contr \oplus\{l_i\colon T_i\}_{i\in I}} % : \Productive}
    \qquad
    \frac{}{\GEnv \Contr \&\{l_i\colon T_i\}_{i\in I}} % : \Productive}
    \\
    \frac{}{\GEnv \Contr \skipk} % : \Guarded}
    \qquad
    \frac{\GEnv \Contr T_1 }{ % : \Productive}{
      \GEnv \Contr T_1;T_2} % : \Productive}
    \qquad
    \frac{\GEnv \Contr T_1 % : \Guarded
      \qquad \GEnv \Contr T_2}{ % : \gamma}{
      \GEnv  \Contr (T_1;T_2)} % : \gamma}
    \\
    \frac{\GEnv \Contr T}{\GEnv \Contr \mu x.T } % : \gamma}
    \quad
    \frac{ }{\GEnv, x\colon\kind \Contr x } % : \Productive }
    \quad
    \frac{\GEnv \Contr T}{\GEnv \Contr \forall\alpha::\kind.T } % : \gamma}
    \quad
    \frac{ }{\GEnv, \alpha\colon\kind \Contr \alpha } % : \Productive }
  \end{gather*}
  \caption{Contractivity, $\GEnv \Contr T$} %:\gamma$}
  \label{fig:contractivity}
\end{figure}

%%% Local Variables: 
%%% mode: latex
%%% TeX-master: "main"
%%% End: 

% %
% A type $T$ is \emph{contractive in $x$} if $\GEnv_x \Contr T : \gamma$ is
% derivable with the rules in Figure~\ref{fig:contractivity} for
% some $\GEnv_x$ that contains \textbf{no guardedness assumption about $x$}. The metavariable
% $\gamma$ ranges over $\Guarded$ and $\Productive$. 
% The intuitive reading is that any (recursive) use of $x$ must be preceded
% by a nontrivial type construction, where nontrivial means different
% from $\skipk$. For example, a session type like $\mu x. (\skipk; x)$
% or $\mu x.(x; !B)$ must be disallowed whereas $\mu x.(!B; x)$ should
% be allowed.  If $\GEnv \Contr T : \Guarded$ is derivable, but not $\GEnv
% \Contr T : \Productive$, then $T$ is essentially composed of
% $\skipk$s. If $\GEnv \Contr T : \Productive$ is derivable, then $T$
% describes a nontrivial interaction.

% The interaction between $\mu$-operators is also nontrivial. For
% example, the type $\mu x'. \mu x. (x'; x) $ is ruled out because $x'$
% would not be guarded after unrolling $\mu x$ once. However, the type
% $\mu x'. (!B; \mu x. (x'; x))$ is accepted because unrolling $\mu x'$
% reveals that the recursive occurrence of $x$ is guarded: $(!B; \mu
% x. (\mu x'. (!B; \mu x. (x'; x)); x))$. Hence, an essential
% requirement of well-formedness of types is preservation of
% well-formedness under arbitrary unrolling of $\mu$ operators. 

% The second judgment $ \GEnv \vdash T \isOk$ in Figure~\ref{fig:contractivity}
% specifies the set of \emph{well-formed} type expressions where
% $\GEnv$ contains guardedness assumptions about the recursion
% variables that may occur in $T$. The obvious
% inductive rules that require well-formedness of every subexpression
% are omitted.

% This definition incorporates type variables $\alpha$ by assuming that they
% are always replaced by productive types.
% Type variables must be restricted in this way because contractivity of $\mu
% x. (\alpha; x)$ requires the $\alpha \Contr \alpha : \Productive$ to be derivable so that
% $\alpha \Contr \mu x. (\alpha; x) : \Productive$ is derivable. 
% %
% Finally, a type scheme $\forall \alpha_1\dots\forall\alpha_n. T$ is
% \emph{well-formed} if $\alpha_1, \dots, \alpha_n \vdash T \isOk$ is
% derivable. 

% \begin{lemma}
%   If $\GEnv \vdash T \isOk$, then $\GEnv, x : \gamma \vdash T \isOk$
%   for some $x$ not in $\GEnv$.
% \end{lemma}

% \begin{lemma}
%  If $\GEnv \vdash \mu x.T \isOk$, then $\GEnv \vdash T[\mu
%   x. T/x] \isOk$.
% \end{lemma}
% \begin{proof}
%   If  $\GEnv \vdash \mu x.T \isOk$, it must be because  $\GEnv,
%   x:\gamma \vdash T \isOk$ and $\GEnv\setminus x \Contr T : \gamma$.
%   We prove by induction on $\GEnv, x:\gamma \vdash T \isOk$ that
%   $\GEnv \vdash T[\mu  x. T/x] \isOk$.

%   There are two interesting cases. In the first case, we encounter the
%   recursion variable $\GEnv, x:\gamma, \GEnv' \vdash x
%   \isOk$. At this point, we have to return $\GEnv, \GEnv' \vdash \mu
%   x.T \isOk$. But is derivable by the initial assumption and
%   weakening.

%   The other case is a different $\mu$ operators in a judgment $\GEnv,
%   x:\gamma, \GEnv' \vdash \mu x'. T' \isOk$. Inversion yields $\GEnv,
%   x:\gamma, \GEnv', x':\gamma' \vdash T' \isOk$ and $(\GEnv,
%   x:\gamma, \GEnv') \setminus x' \Contr T' : \gamma'$. The first part
%   can be handled by induction, but the second part requires an
%   auxiliary induction to prove that $(\GEnv, \GEnv') \setminus x'
%   \Contr T'[\mu x.T/x] : \gamma'$. For this auxiliary induction it is
%   sufficient to observe that a successful derivation never reaches a
%   recursion variable, so the unrolling does not matter. 
% \end{proof}

% END - Moved to the section on types 



% PJT: Too limiting, unfortunately.
% This definition rules out \mu x.(!B;x)  --- which should be allowed.
% This definition admits \mu x.(\mu x.skip; x) --- which should not be
% allowed.
% \begin{itemize}
% \item $T$ has one of the forms $T \rightarrow T$, $T \multimap T$, $T
%   \otimes T$, $[l_i\colon T_i]$, $B$, $\skipk$, $!B$, $?B$,
%   $\oplus\{l_i\colon S_i\}$, $\&\{l_i\colon S_i\}$, or $\alpha$.
% \item $T$ has the form $S_1;S_2$ with both $S_1$ and $S_2$ contractive
%   in~$x$.
% \item $T$ has the form $\mu x'.T'$ with either $x=x'$ or $T'$
%   contractive in~$x$.
% \end{itemize}

% TYPE EQUIVALENCE 

%\subsection{Type unfolding}
\label{sec:type-unfolding}

We establish a notion of type unfolding as a step towards
defining type equivalence. The idea of unfolding is to expose the
first nontrivial action of a session type by squeezing out sequences
of $\skipk$s.
Let $A$ range over $\alpha$, $!B$, and $?B$; let $\star$ range over
$\oplus$ and $\&$. 
%
Define the unfolding of a type $T$,  $\Unfold(T)$, recursively by cases on the
structure of~$T$ as follows. 
\begin{enumerate}
\item $\Unfold(\mu x.T) = \Unfold(T\subs{\mu x.T}x)$
\item $\Unfold (S;S') = \left\{%
  \begin{array}{ll}
    \Unfold(S') & \Unfold(S) = \skipk
    \\
    (A; \Norm(S')) & \Unfold(S) = A
    \\
    (S_3; (S_4\fatsemi S')) & \Unfold(S) = (S_3;S_4)
    \\
    \star\{l_i\colon S_i\fatsemi S'\}  & \Unfold (S) = \star\{l_i\colon S_i\}
  \end{array}
  \right.
$
\item $\Unfold(T) = T$, otherwise
\end{enumerate}
We assume two auxiliary definitions
\begin{align*}
  \Norm (S) &=
              \begin{cases}
                S_1 \fatsemi S_2 & S = (S_1; S_2) \\
                S & \text{otherwise}
              \end{cases}
  \\
  S_1 \fatsemi S_2 &=
                     \begin{cases}
                       S_1' \fatsemi (S_1'' \fatsemi S_2) & S_1 =
                       (S_1'; S_1'') \\
                       (S_1; S_2) & \text{otherwise}
                     \end{cases}
\end{align*}

The function $\Unfold$ is well-defined and terminating because we
assume that the body of a recursive type is contractive. The auxiliary
functions $\Norm$ and $\fatsemi$ are both terminating.
The following
lemmas establish well-definedness of $\Unfold$.

\begin{lemma}\label{lemma:app:guarded=skip}
  Suppose that $\GEnv$ does not bind recursion variables and that
  $\sigma$ is a substitution of recursion variables by recursive types.
  If $\GEnv \Contr S : \Guarded$, then $\Unfold (S\sigma) = \skipk$.
\end{lemma}
\begin{proof}
  Induction on $\GEnv \Contr S : \Guarded$.

  \textbf{Case }$\GEnv \Contr \skipk : \Guarded$. Immediate.

  \textbf{Case }$\GEnv \Contr (S_1;S_2) : \Guarded$ because $\GEnv
  \Contr S_1 : \Guarded$ and $\GEnv \Contr S_2 : \Guarded$. By
  induction, $\Unfold (S_1\sigma) = \skipk$, hence $\Unfold ((S_1;S_2)\sigma) =
  \Unfold (S_2\sigma) = \skipk$ by induction.

  \textbf{Case }$\GEnv \Contr \mu x.S : \Guarded$ because $\GEnv
  \Contr S : \Guarded$. In this case, $\Unfold ((\mu x.S)\sigma) = \Unfold
  (S[\mu x.S/x]\sigma) = \skipk$ by induction (for $\sigma' = [\mu x.S/x]\sigma$).
\end{proof}

\begin{definition}
  A \emph{guarded} type has one of the forms below.
  % where $A$ ranges over $\alpha$, $!B$, and $?B$.
\begin{gather*}
  \skipk \quad A \quad A;S' \quad
  \star\{l_i\colon S_i\} %\quad \&\{l_i\colon S_i\}
  \\
  T \to T' \quad T \multimap T' \quad T \otimes T' \quad [l_i : T_i]
  \quad B 
\end{gather*}
\end{definition}
% We say that %
% %
% \footnote{To me $T$ is only a type if $\Theta \vdash T \isOk$ holds,
%   for some $\Theta$ that does not bind recursion variables. To
%   $(B\rightarrow B;\skipk)$ I call a piece of \emph{junk syntax}. This
%   means that ``there is no life outside types'' (the intuitionistic
%   approach). Pragmatically, it means that, whenever we talk of $T$, we
%   need not keep saying ``s.t.\ $\Theta \vdash T \isOk$ for some
%   $\Theta$ that does not bind recursion variables.''. In this , a
%   session type $S$ is a type such that $\Delta \vdash S: \kinds^m$,
%   for some $\Delta$ that does not bind recursion variables.}
% %


\begin{lemma}[Characterization of $\Unfold$]
  \label{lem:app:unfold-yields-guarded-types}
  Suppose that $\GEnv$ does not bind recursion variables and that
  $\GEnv \vdash T :: \kind$ for $\kind \le \kindt^\Unrestricted$, then $\Unfold (T)$ is defined and yields a
  guarded type.

  % result that has one of the following forms.
  % \begin{enumerate}
  % \item $ \skipk$,
  % \item $(\alpha;S')$, $(!B; S')$, $(?B; S')$ for some $\Delta \vdash S' \isOk$,
  % \item $\oplus\{l_i\colon S_i\}$,  $\&\{l_i\colon S_i\}$ for some
  %   $\Delta \vdash S_i \isOk$.
  % \end{enumerate}

  % Furthermore, if $\GEnv \Contr S : \gamma$, then $\Unfold (\mu
  % x.S)$ is defined and yields a guarded type.
\end{lemma}
\begin{proof}
  Induction on the derivation of  $\GEnv \vdash T :: \kind$.

  \textbf{Case }$\GEnv \vdash \skipk :: \kinds^\Unrestricted$.
  In this case, $\Unfold (\skipk) = \skipk$.

  \textbf{Case }$\GEnv \vdash A :: \kind$. $\Unfold (A) = {A}$.

  % \textbf{Case }$\GEnv \vdash !B :: \kinds^\Linear$. $\Unfold (!B) = {!B}$.
  % \textbf{Case }$\GEnv \vdash ?B :: \kinds^\Linear$. $\Unfold (?B) = {?B}$.

  \textbf{Case }$\GEnv \vdash (S_1;S_2) :: \kinds^m$.

  Inversion yields $\GEnv \vdash S_1 :: \kinds^{m_1}$ and  $\GEnv
  \vdash S_2 :: \kinds^{m_2}$. By induction, $S_1' = \Unfold (S_1)$ is
  guarded.

  \textbf{Subcase }$S_1' = \skipk$. In this case, $\Unfold (S_1;S_2) =
  \Unfold (S_2)$ which is guarded by induction.

  \textbf{Subcase }$S_1' = A$. Then $\Unfold (S_1;S_2) = (A; S_2)$
  which is guarded.

  \textbf{Subcase }$S_1' = (A; S_3)$. Hence, $\Unfold (S_1;S_2) = (A; (S_3; S_2))$ is guarded.

  \textbf{Subcase }$S_1' = \star\{l_i\colon S_i\}$: $\Unfold
  (S_1;S_2) = \star\{l_i\colon (S_i; S_2) \}$.

  % \textbf{Subcase }$S_1' = \&\{l_i\colon S_i\}$: $\Unfold
  % (S_1;S_2) = \&\{l_i\colon (S_i; S_2) \}$.

  \textbf{Case }$\GEnv \vdash \star\{l_i\colon S_i\}
  :: \kinds^\Linear$. Immediate.

  % \textbf{Case }$\GEnv \vdash \&\{l_i\colon S_i\}
  % :: \kinds^\Linear$. Immediate.

  % \textbf{Case }$\GEnv \vdash \alpha :: \kind$. Immediate: $\Unfold
  % (\alpha) = \alpha$.

  \textbf{Case }$\GEnv \vdash \mu x. T :: \kind$.
%
  Inversion yields $\GEnv, x:\kind \vdash T :: \kind$ and
  $\GEnv \Contr T : \gamma$. Observe that
  $\Unfold (\mu x.T) = \Unfold (T[\mu x.T/x])$ and
  $\GEnv \Contr T[\mu x.T/x] : \gamma$. By the second claim 
  $\Unfold (T[\mu x.T/x])$ is defined and yields a guarded type using
  $\sigma = [\mu x.T/x]$.

  \textbf{All other cases}: $\Unfold (T) = T$ is guarded.
  
  \textbf{Second claim.}
  % It holds that $\Unfold (\mu x.S) = \Unfold (S[\mu x.S/x])$.
  % Further $\GEnv\setminus x \Contr S : \gamma$ implies that
  % $\GEnv \Contr S[\mu x.S/x] : \gamma$.
  Suppose that  $\GEnv \Contr T :  \gamma$ where $\GEnv$ does not bind
  recursion variables and that  $\sigma$ is a substitution on recursion variables.
  Then  $\Unfold (T\sigma)$ is defined and yields a guarded type.

  The proof is by induction on the derivation of $\GEnv \Contr T :
  \gamma$.

  \textbf{Case }$\skipk$. Immediate.

  \textbf{Case }$A \in \{ {!B}, {?B}, \alpha\}$. $\Unfold (A\sigma) = A$ which is guarded.

  \textbf{Case }$x$ cannot occur because $\GEnv$ 
  contains no assumptions about recursion variables.

  \textbf{Case }$\star\{l_i\colon S_i\}$. $\Unfold
  ((\star\{l_i\colon S_i\})\sigma) = \Unfold (\star\{l_i\colon
  S_i\sigma\}) = \star\{l_i\colon S_i\sigma\}$ which is guarded.

  % \textbf{Case }$\&\{l_i\colon S_i\}$. Analogously.

  \textbf{Case }$(S_1;S_2) : \Productive$ because $S_1 :
  \Productive$. By induction $S_1' = \Unfold (S_1)$ is
  guarded. Proceed by subcases on $S_1'$.

  \textbf{Subcase }$\skipk$. Contradicts $S_1 : \Productive$.

  \textbf{Subcase }$A$. Here, $\Unfold ((S_1;S_2)\sigma) = (A;
  S_1)\sigma$, which is guarded.

  \textbf{Subcase }$(A; S_3)$. $\Unfold
  ((S_1;S_2)\sigma) = (A; (S_3; S_2))\sigma$, which is guarded. 

  \textbf{Subcase }$\star\{l_i\colon S_i\}$. $\Unfold
  ((S_1;S_2)\sigma) = \star\{l_i\colon (S_i; S_2)\sigma\}$ is guarded.

  % \textbf{Subcase }$\&\{l_i\colon S_i\}$. Similar.

  \textbf{Case }$(S_1;S_2) : \Productive$ because $S_1 :
  \Guarded$ and $S_2 : \Productive$. In this case, $\Unfold (S_1) =
  \skipk$ by Lemma~\ref{lemma:guarded=skip} so that the result is
  $\Unfold (S_2\sigma)$, which is guarded by induction on $S_2 : \Productive$.

  \textbf{Case }$\GEnv \Contr \mu x.T : \gamma$ because $\GEnv \Contr
  T : \gamma$. Hence, $\Unfold ((\mu x.T)\sigma) = \Unfold (T\sigma[\mu
  x.T\sigma/x])$. The result follows by induction using $\sigma' = \sigma[\mu
  x.T\sigma/x]$.

  % \textbf{Case }$\GEnv, \alpha : \gamma \Contr \alpha :
  % \gamma$. Immediate.

  \textbf{All remaining cases}: Immediate.
\end{proof}

Next, we consider invariance of kinding and contractivity under
unfolding of recursion anywhere in the type.

\begin{lemma}[Weakening]\label{lemma:app:weakening-kind}
  If $\Delta \vdash T :: \kind$, then
  $\Delta, x\colon \gamma \vdash T :: \kind$ for some $x$ not in
  $\Delta$.
\end{lemma}


\begin{lemma}[Unfolding preserves kinding]
  If $\Delta \vdash T :: \kind$ then $\Delta \vdash \Unfold(T) :: \kind$.
\end{lemma}
%
\begin{proof}
  We only consider the case for a recursive type as the other cases
  are straightforward.
  
  % (Needs adjustment)
  If  $\GEnv \vdash \mu x.T :: \kinds^m$, it must be because  $\GEnv,
  x:\kinds^m \vdash T :: \kinds^m$ and $\GEnv \Contr T : \gamma$.
  We prove by induction on $\GEnv, x:\kinds^m \vdash T :: \kinds^m$ that
  $\GEnv \vdash T[\mu  x. T/x] :: \kinds^m$.

  There are two interesting cases. In the first case, we encounter the
  recursion variable $\GEnv, x::\kinds^m, \GEnv' \vdash x
  :: \kinds^m$. At this point, we have to return $\GEnv, \GEnv' \vdash \mu
  x.T :: \kinds^m$, which is derivable by the initial assumption and
  weakening (Lemma~\ref{lemma:weakening-kind}).

  The other case is a different $\mu$ operator in a judgment $\GEnv,
  x::\kinds^m, \GEnv' \vdash \mu x'. T' :: \kinds^{m'}$. Inversion yields $\GEnv,
  x::\kinds^m, \GEnv', x'::\kinds^{m'} \vdash T'  :: \kinds^{m'}$ and $\GEnv,
  x:\gamma, \GEnv' \Contr T' : \gamma'$. The first part
  can be handled by induction, but the second part requires an
  auxiliary induction to prove that $(\GEnv, \GEnv')
  \Contr T'[\mu x.T/x] : \gamma'$. For this auxiliary induction it is
  sufficient to observe that a successful derivation never reaches a
  recursion variable, so the unrolling does not matter. 
\end{proof}

\begin{lemma}
  \label{lemma:app:unfold-fixpoints}
  If $T$ is a guarded type, then $\Unfold (T) = T$.
\end{lemma}
\begin{proof}
  Cases on $T$.

  \textbf{Case }$\skipk$: Obvious.

  \textbf{Case }$A$: $\Unfold (A) = A$.

  \textbf{Case }$(A; S')$:
  $\Unfold (A;S') = (A; S')$ as $\Unfold (A) = A$.

  \textbf{Case }$\star\{l_i\colon S_i\}$: Immediate.

  \textbf{Remaining cases}: Immediate.
\end{proof}

\begin{lemma}
  \label{lemma:app:unfold-idempotent}
  % For all well-formed types, $
  $\Unfold (\Unfold (T)) = \Unfold (T).$
\end{lemma}
\begin{proof}
  By Lemma~\ref{lem:unfold-yields-guarded-types}, $\Unfold (T)$ is guarded and hence a fixpoint of
  $\Unfold$ by Lemma~\ref{lemma:unfold-fixpoints}.
\end{proof}

\subsection{Type equivalence}
\label{sec:type-equivalence}

We want to define a notion of type equivalence for session types that
only depends on the communication behavior of a process with that
type. To this end, we first define a (weak) labelled transition system
$(\stypes, \Sigma, \LTSderives)$ that captures this behavior. The set
of states is  $\stypes = \{ S \mid \GEnv \vdash S :: \kinds^m \}$
where $\GEnv$ is an arbitrary, fixed kinding environment that binds no
recursion variables. The
actions in this system are drawn from the set $\Sigma = \{ {!B}, {?B}
\mid  B \in \btypes \} \uplus \Tyvars \uplus \{ {!l}, {?l} \mid l
\in \Labels\} $. We use the label $\Silent$ for the \emph{silent
  action} that exhibits no externally observable behavior. The transition
relation $\LTSderives$ is defined by the rule set in Figure~\ref{fig:type-behavior}. We write
$\Wderives$ for the reflexive transitive closure of $\LTSderives[\Silent]$ and $\Wderives[\xi]$ for
the composition $\Wderives \circ \LTSderives[\xi] \circ \Wderives$.

\begin{figure}[tp]
  \begin{gather*}
    {A \LTSderives[A] \skipk }
    \qquad
    {\star\{\overline{l_n:S_n}\} \LTSderives[\star l_i] S_i}
    \\
    \frac{S_1 \LTSderives[\xi] S_1'}{(S_1; S_2) \LTSderives[\xi]
      (S_1';S_2) }
    \qquad
    {(\skipk; S) \LTSderives[\Silent] S}
    \\
    {((S_1;S_2); S_3) \LTSderives[\Silent] (S_1; (S_2; S_3))}
    \\
    {(\star\{\overline{l_n:S_n}\}; S) \LTSderives[\Silent]
      \star\{\overline{l_n:(S_n; S)}\}}
    \\
    { \mu x.S \LTSderives[\Silent] S[\mu x.S/x]}
  \end{gather*}
  \caption{Behavior of a type}
  \label{fig:type-behavior}
\end{figure}
\begin{lemma}\label{lemma:app:unfold-silent}
  If $S' = \Unfold (S)$ is defined, then $S \Wderives S'$.
\end{lemma}
\begin{proof}
  By induction on the number of recursive calls to compute $\Unfold
  (S)$.

  \textbf{Case }$\Unfold (\mu x.S) = \Unfold (S[\mu x.S/x])$: By
  definition, $\mu x.S \LTSderives[\Silent] S[\mu x.S/x] $ and by
  induction $S[\mu x.S/x] \Wderives S'$.

  \textbf{Case }$\Unfold (S_1;S_2)$.

  \textbf{Subcase }$\Unfold (S_1) = \skipk$: By induction, $S_1
  \Wderives \skipk$. By the context rule for behaviors,
  $(S_1; S_2) \Wderives (\skipk; S_2 )$ and by the skip
  rule: $(\skipk; S_2 ) \LTSderives[\Silent] S_2$. Proceed by another
  induction on $S_2$.

  \textbf{Subcase }$\Unfold (S_1) = A$. By induction and the context
  rule.

  \textbf{Subcase }$\Unfold (S_1) = (S_1'; S_1'')$. By induction and
  the context rule, we obtain $(S_1;S_2) \Wderives
  ((S_1'; S_1''); S_2) \LTSderives[\Silent] (S_1'; (S_1''; S_2))$
  where the last step is an application of associativity.

  \textbf{Subcase }$\Unfold (S_1) = \star\{\overline{l_n:S_n} \}$. By
  induction and distributivity.

  \textbf{Remaining cases}. No silent transition needed.
\end{proof}

\begin{lemma}\label{lemma:app:silent-unfold-compatible}
  If $S\LTSderives[\Silent] S'$, then $\Unfold (S) = \Unfold (S')$.
\end{lemma}
\begin{proof}

  \textbf{Case }$\mu x.S \LTSderives[\Silent] S[\mu x.S/x]$:
  $\Unfold (\mu x.S) =  \Unfold (S[\mu x.S/x])$ by definition.

  \textbf{Case }$    {(\star\{\overline{l_n:S_n}\}; S) \LTSderives[\Silent]
    \star\{\overline{l_n:(S_n; S)}\}}$: Immediate by definition of
  $\Unfold$.
  
  \textbf{Case }${((S_1;S_2); S_3) \LTSderives[\Silent] (S_1; (S_2;
    S_3))}$: \\
  Case analysis on the possible outcomes of $\Unfold (S_1)$ and $\Unfold (S_2)$. 

  \textbf{Subcase } $\Unfold (S_1) = \skipk$:
  
  \textbf{Subsubcase }$\Unfold (S_2) = \skipk$:
  
  $\Unfold  (S_1; (S_2;  S_3)) = \Unfold (S_2; S_3) = \Unfold (S_3)$
  and
  $\Unfold ((S_1; S_2); S_3) = \Unfold (S_3)$.

  \textbf{Subsubcase }$\Unfold (S_2) = A$.

  $\Unfold  (S_1; (S_2;  S_3)) = \Unfold (S_2; S_3) = (A; S_3)$
  and
  $\Unfold ((S_1; S_2); S_3) = (A; S_3)$.

  \textbf{Subsubcase }$\Unfold (S_2) = (S_2'; S_2'')$.

  $\Unfold  (S_1; (S_2;  S_3)) = \Unfold (S_2; S_3) = (S_2' ;( S_2''; S_3))$
  and
  $\Unfold ((S_1; S_2); S_3) = (S_2'; (S_2''; S_3))$.

  \textbf{Subsubcase }$\Unfold (S_2) =
  \star\{\overline{l_n:S_n}\}$.

  $\Unfold  (S_1; (S_2;  S_3)) = \Unfold (S_2; S_3) =
  \star\{\overline{l_n:(S_n; S_3)} $
  and
  $\Unfold ((S_1; S_2); S_3) = \star\{\overline{l_n:(S_n; S_3)}$.

  \textbf{Subcase }$\Unfold (S_1) = A$.

  $\Unfold  (S_1; (S_2;  S_3)) =(A; (S_2; S_3))$
  and
  $\Unfold ((S_1; S_2); S_3) = (A; (S_2; S_3))$.

  \textbf{Subcase }$\Unfold (S_1) = (S_1';S_1'')$.

  $\Unfold  (S_1; (S_2;  S_3)) =(S_1'; (S_1''; (S_2; S_3)))$
  and
  $\Unfold ((S_1; S_2); S_3) = (S_1'; (S_1''; (S_2; S_3)))$.

  \textbf{Subcase }$\Unfold (S_1) = \star\{\overline{l_n:S_n}\}$.

  $\Unfold  (S_1; (S_2;  S_3)) =\star\{\overline{l_n:(S_n;
    (S_2\fatsemi S_3))}\} $
  and
  $\Unfold ((S_1; S_2); S_3) = \star\{\overline{l_n:(S_n; (S_2\fatsemi
    S_3)))}\}$.

  \textbf{Case }${(\skipk; S) \LTSderives[\Silent] S}$:
  $\Unfold (\skipk;S) = \Unfold (S) = S'$.

  \textbf{Case }$(S_1; S_2) \LTSderives[\Silent] (S_1';S_2)$
  because $S_1 \LTSderives[\Silent] S_1'$: By induction, $\Unfold
  (S_1) = \Unfold(S_1')$, so that $\Unfold (S_1; S_2) = \Unfold
  (S_1';S_2)$ by definition of unfolding.
\end{proof}
\begin{lemma}\label{lemma:app:silent-unfold}
  If $S \Wderives S'$ and $S'$ is guarded, then $S' = \Unfold (S)$.
\end{lemma}
\begin{proof}
  By induction on the number of silent steps.

  \textbf{Case }$0$. If $S=S'$ is already guarded, then $S$ is a
  fixpoint of $\Unfold$  by Lemma~\ref{lemma:unfold-fixpoints}.

  \textbf{Case }$n>0$. In this case, there is some $S''$ such that $S
  \LTSderives[\Silent] S''$ and $S'' \Wderives S'$ in less than $n$
  steps. Now, $S' = \Unfold (S'') = \Unfold (S)$, the former by
  induction  and the latter by
  Lemma~\ref{lemma:silent-unfold-compatible}.
\end{proof}

% A relation $R \subseteq \stypes \times \stypes$ is a \emph{type
%   simulation} if $(S_1,S_2)\in R$ implies the following conditions:
% %
% \begin{enumerate}
% \item If $\Unfold(S_1) = \skipk$ then $\Unfold(S_2) = \skipk$. 
% \item If $\Unfold(S_1) = (\alpha; S_1')$ then $\Unfold(S_2) =
%   (\alpha;S_2')$ and $(S_1', S_2') \in R$. 
% \item If $\Unfold(S_1) = (!B;S_1')$ then $\Unfold(S_2) = (!B;S_2')$  and $(S_1', S_2') \in R$. 
% \item If $\Unfold(S_1) = (?B;S_1')$ then $\Unfold(S_2) = (?B;S_2')$  and $(S_1', S_2') \in R$. 
% % \item If $\Unfold(S_1) = S_1';S_1''$ then $\Unfold(S_2) = S_2';S_2''$
% %   and both $(S_1',S_2')$  and $(S_1'',S_2'')$ are in $R$.
% \item If $\Unfold(S_1) = \oplus\{l_i\colon S_{1,i}'\}_{i\in I}$ then
%   $\Unfold(S_2) = \oplus\{l_i\colon S_{2,i}'\}_{i\in I}$ and
%   $(S_{1,i}',S_{2,i}')\in R$, for all $i\in I$.
% \item If $\Unfold(S_1) = \&\{l_i\colon S_{1,i}'\}_{i\in I}$ then
%   $\Unfold(S_2) = \&\{l_i\colon S_{2,i}'\}_{i\in I}$ and
%   $(S_{1,i}',S_{2,i}')\in R$, for all $i\in I$.
% \end{enumerate}

\begin{definition}
Define a monotone function~$F$ on $\stypes\times\stypes$ as 
follows. 
%
\begin{align*}
  F (R) &= \{ (S_1, S_2) \mid \Unfold (S_1) = \Unfold (S_2) = \skipk \}
  \\
        &\cup \{ (S_1, S_2) \mid \Unfold (S_1) = \Unfold (S_2) = A \}
  \\
        &\cup \{ (S_1, S_2) \mid
          \begin{array}[t]{@{}l}
            \Unfold (S_1) = (A; S_1'),\\
            \Unfold (S_2) = (A; S_2'), \\
            (S_1', S_2')  \in R \}
          \end{array}
  \\
        &\cup \{ (S_1, S_2) \mid
          \begin{array}[t]{@{}l}
            \Unfold (S_1) = \star\{l_i\colon S_{1,i}'\}_{i\in I}, \\
            \Unfold (S_2) = \star\{l_i\colon S_{2,i}'\}_{i\in I}, \\
            \forall i: (S'_{1,i}, S'_{2,i}) \in R \}
          \end{array}
  % \\
  %       &\cup \{ (S_1, S_2) \mid
  %         \begin{array}[t]{@{}l}
  %           \Unfold (S_1) = \oplus\{l_i\colon S_{1,i}'\}_{i\in I}, \\
  %           \Unfold (S_2) = \oplus\{l_i\colon S_{2,i}'\}_{i\in I}, \\
  %           \forall i: (S'_{1,i}, S'_{2,i}) \in R \}
  %         \end{array}
  % \\
  %       &\cup \{ (S_1, S_2) \mid
  %         \begin{array}[t]{@{}l}
  %           \Unfold (S_1) = \&\{l_i\colon S_{1,i}'\}_{i\in I}, \\
  %           \Unfold (S_2) = \&\{l_i\colon S_{2,i}'\}_{i\in I}, \\
  %           \forall i: (S'_{1,i}, S'_{2,i}) \in R \}
  %         \end{array}
\end{align*}

This function helps define (weak) bisimularity $\TypeEquiv$ for the labelled transition system
$(\stypes, \Sigma, \LTSderives)$ on well-formed session types as a greatest fixpoint: ${\TypeEquiv}
= \GFP (F)$. The 
definition relies on the $\Unfold$ function instead of using silent transitions which is sanctioned
by Lemmas~\ref{lemma:unfold-silent} and~\ref{lemma:silent-unfold}.
\end{definition}

Our goal is to use weak bisimilarity for type
equivalence. To this end, we need to establish that it is indeed an
equivalence relation.\footnote{This development can be extended to a
  relation on $\kindt$, but the extension is entirely standard.}
Reflexivity and symmetry follow directly from the definition. Transitivity requires some work.



\begin{lemma}
  Type  bisimilarity $\TypeEquiv$ is reflexive.
\end{lemma}
% \begin{proof}
%   Obvious.
% \end{proof}

\begin{lemma}
  Type bisimilarity $\TypeEquiv$ is symmetric.
\end{lemma}
% \begin{proof}
%   Obvious.
% \end{proof}

\begin{lemma}
  Type bisimilarity $\TypeEquiv$ is transitive.
\end{lemma}
\begin{proof}
  Let $R  = {\TypeEquiv} \circ {\TypeEquiv}$. Show that $R \subseteq F(R)$, which implies that $R
  \subseteq {\TypeEquiv}$. Observe that ${\TypeEquiv}
  \subseteq R$ because $\TypeEquiv$ is reflexive.

  Suppose that $S_1 \TypeEquiv S_2$ and $S_2 \TypeEquiv S_3$ so that $(S_1, S_3) \in R$.

  \textbf{Case }$\Unfold (S_1) = \skipk$ implies $\Unfold (S_2) =\skipk$, which in turn implies $\Unfold (S_3) =
  \skipk$. Hence, $(S_1, S_3) \in F (\emptyset)$.

  \textbf{Case }$\Unfold (S_1) = A$. Then it must be the case that
  $\Unfold (S_2) = A$ and also $\Unfold (S_3)=A$. Hence, $(S_1, S_3)
  \in F (\emptyset)$.

  \textbf{Case }$\Unfold (S_1) = (A; S_1')$ with $A$ either $\alpha$, $!B$, or $?B$. It must be the
  case that $\Unfold (S_2) = (A; S_2')$ with $(S_1', S_2') \in {\TypeEquiv}$ and further
  $\Unfold (S_3) = (A; S_3')$ with $(S_2', S_3') \in {\TypeEquiv}$. But then $(S_1', S_3') \in R =
  {\TypeEquiv} \circ {\TypeEquiv}$ and hence $(S_1, S_3) \in F (R)$.

  \textbf{Case }$\Unfold (S_1) = \star\{l_i\colon S_{1,i}'\}_{i\in I}$. Then it must be that
  $\Unfold (S_2) = \star\{l_i\colon S_{2,i}'\}_{i\in I}$ with $S_{1,i}' \TypeEquiv S_{2,i}'$, for all
  $i$, and $\Unfold (S_1) = \star\{l_i\colon S_{1,i}'\}_{i\in I}$ with  $S_{2,i}' \TypeEquiv S_{3,i}'$, for all
  $i$. Hence, $(S_{1,i}', S_{3,i}') \in R $, for all $i$, so that $(S_1, S_3) \in F (R)$.
  % \textbf{Case }$\Unfold (S_1) = \&\{l_i\colon S_{1,i}'\}_{i\in I}$. Analogously.
\end{proof}

\begin{lemma}
  \label{lem:app:unfold-type-sim}
  $S \TypeEquiv \Unfold(S)$.
\end{lemma}
\begin{proof}
  Straightforward application of coinduction. We show that $\{ (S, \Unfold (S)) \} \subseteq
  F (\TypeEquiv)$ because 
  of idempotence of $\Unfold$ (Lemma \ref{lemma:unfold-idempotent}) and reflexivity of $\TypeEquiv$.
\end{proof}

\begin{lemma}
  \label{lem:app:skip-elim}
  $\skipk;S_1 \TypeEquiv \skipk;S_2$ iff $S_1 \TypeEquiv S_2$.
\end{lemma}
\begin{proof}
  Immediate because $\Unfold (\skipk; S) = \Unfold (S)$.
\end{proof}

Outside this section, we write $\GEnv \vdash S_1 \TypeEquiv S_2$ to fix the environment $\GEnv$
needed for the formation of $S_1$ and $S_2$. As usual, $\GEnv$ must
not bind recursion variables.


\subsection{Translation to BPA}
\label{sec:translation-bpa}

We define a variant of the unfolding
function for a session type $S$,  $\Unravel(S)$, recursively by cases on the
structure of~$S$. The difference to $\Unfold (S)$ is that the
structure of $S$ is left intact as much as possible.
\begin{enumerate}
\item $\Unravel(\mu x.S) = \Unravel(S[\mu x.S/x])$
\item $\Unravel (S;S') = \left\{%
  \begin{array}{ll}
    \Unravel(S') & \Unravel(S) = \skipk
    \\
    (\Unravel(S); S') & \Unravel(S) \ne \skipk
  \end{array}
  \right.
$
\item $\Unravel (S) = S$ for all other cases
\end{enumerate}


To define the translation to BPA,
we need to show that, for a well-formed session type $S$, $\Unravel
(S)$ is always guarded. That is the output of $\Unravel (S)$ is either
$\skipk$ or it has one
of the following forms:
\begin{align*}
  O &::= A \mid \star\{\overline{l_i:S_i}\} \mid (O; S)
\end{align*}

Now we define the translation of well-formed $S$ to a BPA as
follows. Assume that all recursion variable bindings are unique in the
sense that the set $\{ \mu x_1.S_1, \dots, \mu x_n.S_n\}$ contains all $\mu$-subterms
of $S$ with $S_i : \Productive$ modulo $x_i \equiv \mu x_i.S_i$. 
Define the BPA process equations for $S$ by 
\begin{align*}
  \toBPATop{S} &= \{ \\
  x_0 &= \toBPA{S}, \\
               x_1&= \toBPA{\Unravel (S_1[\mu x_1.S_1/x_1])}, &
                                                                &\dots &
                                                     & \}
\end{align*}
\begin{align*}
  \toBPA{\skipk} &= \varepsilon \\
  \toBPA{A} &= A\\
  \toBPA{S_1;S_2} &= \toBPA{S_1} ; \toBPA{S_2} \\
  \toBPA{\star\{\overline{l_n:S_n}\}} &= (\star l_1;
                                        \toBPA{S_1} + \dots + \star
                                        l_n; \toBPA{S_n}) \\
  \toBPA{\mu x.S} &=
                    \begin{cases}
                      x & S : \Productive \\
                      \varepsilon & S : \Guarded
                    \end{cases}
                   \\
 \toBPA{x} &= x
\end{align*}
It is deliberate that we do \textbf{not} unfold the top-level type $S$
in the defining equation for $x_0$. This equation need not be guarded
because $x_0$ does not appear on the right-hand side of any equation.

Alternative: one could also define the equation extraction by
induction on the kind derivation for $S$ and the right-hand side
extraction by induction on the contractivity judgment. 

\begin{lemma}
  If $\mu x.S$ is a subterm of well-formed $S_0$ with $\Delta \Contr S:\Productive$, then
  $\toBPA{\Unravel (S)}$ is guarded with respect to $\toBPATop{S_0}$. 
\end{lemma}

\begin{proof}
  By Lemma~\ref{lem:unfold-yields-guarded-types}, we know that
  $\Unravel (S)$ yields a term of the form $A$, $A;S'$ or
  $\star\{\overline{l_n:S_n}\}$. Clearly, the translation of a term of
  one of these forms is guarded.

  If $\mu x.S$ has a free variable $x' \ne x$, then
  $\toBPA{\Unravel (S)}$ may have the form $x'$ or $x';S'$.
\end{proof}

It remains to show that $S$ is bisimilar to its
translation. Essentially, we want to prove that the function
$\toBPA{\cdot}$ is a bisimilation when considered as a relation.

\begin{lemma}\label{lemma:app:skip-implies-done}
  If $\Unravel (S) = \skipk$, then $\DONE{S}$.
\end{lemma}
\begin{proof}
  Induction on the number $n$ of recursive calls to $\Unravel$.

  \textbf{Case }$n=0$. $S=\skipk$ and $\DONE{\skipk}$.

  \textbf{Case }$n>0$.

  \textbf{Subcase }$\mu x.S$. $\Unravel (\mu x.S) = \skipk$ 
  because $\Unravel (S[\mu x.S/x]) = \skipk$. By induction,
  $\DONE{S[\mu x.S/x]}$ and by applying the mu-DONE rule $\DONE{\mu
    x.S}$.

  \textbf{Subcase }$S_1;S_2$.
  $\Unravel (S_1;S_2) = \skipk$ because $\Unravel (S_1) =
  \Unravel (S_2) = \skipk$. By induction $\DONE{S_1}$ and
  $\DONE{S_2}$. By rule seq-DONE $\DONE{S_1;S_2}$.
\end{proof}

\begin{lemma}\label{lemma:app:s=unr-s}
  Let $S$ be closed, well-formed. \\
  Then
  $\toBPA{S} \TypeEquiv \toBPA{\Unravel (S)}$.
\end{lemma}
\begin{proof}
  Induction on the number $n$ of recursive calls to $\Unravel$.

  \textbf{Case }$n=0$. In this case, $S$ must be $\skipk$, $A$, or
  $\star\{\overline{l_i:S_i}\}$ and the claim is immediate.

  \textbf{Case }$n>0$. There are two subcases.

  \textbf{Subcase }$\mu x.S$. Then $\toBPA{\mu x.S} = x$ and there is
  an equation $x = \toBPA{\Unravel(S[\mu x.S])}$. Now, $x$ is
  obviously bisimilar to $\toBPA{\Unravel(S[\mu x.S])}$.

  \textbf{Subcase }$S_1;S_2$. If $\Unravel (S_1) = \skipk$, then
  $\DONE{S_1}$ and hence $\DONE{\toBPA{S_1}}$. Furthermore, $\Unravel
  (S_1;S_2) = \Unravel (S_2)$ and, by induction, $\toBPA{S_2}
  \TypeEquiv \toBPA{\Unravel(S_2)}$. The result follows because
  $\toBPA{S_1;S_2} = \toBPA{S_1};\toBPA{S_2} \TypeEquiv \toBPA{S_2}$
  and $\toBPA{\Unravel(S_2)} = \toBPA{\Unravel (S_1;S_2)}$.

  If $\Unravel (S_1) =: S_u \ne \skipk$, then $\Unravel (S_1;S_2) =
  S_u; S_2$.
  By induction, we know that $\toBPA{S_1} \TypeEquiv \toBPA{S_u}$ and
  as bisimilarity is a congruence  $\toBPA{S_1;S_2} \TypeEquiv
  \toBPA{S_u}; \toBPA{S_2} = \toBPA{\Unravel (S_1;S_2)}$.
\end{proof}

\begin{lemma}\label{lemma:app:bisimulation-BPA-forwards}
  Suppose $S$ is a well-formed closed session type.
  If $S \LTSderives S'$, then $\toBPATop{S}
  \LTSderives \toBPATop{S'}$.
\end{lemma}
\begin{proof}
  By induction on  $S \LTSderives S'$.

  \textbf{Case }$A \LTSderives[A] \skipk$.
  In this case $\toBPATop{A} = \{ x_0 = A \} \LTSderives[A] \{ x_0 =
  \varepsilon \} = \toBPATop{\skipk}$.

  \textbf{Case }$\star\{\overline{l_i:S_i}\} \LTSderives[\star l_i]
  S_i$.
  In this case $\toBPATop{\star\{\overline{l_i:S_i}\}} = \{ x_0 =
  (\dots+ \star l_i; \toBPA{S_i} + \dots) \} \LTSderives [\star l_i]
  \{ x_0 = \toBPA{S_i} \} =  \toBPATop{S_i}$.

  \textbf{Case }$\frac{S_1 \LTSderives S_1'}{S_1; S_2
    \LTSderives S_1';S_2}$.
  In this case $\toBPATop{S_1;S_2} = \{ x_0 = E_1;E_2, \dots \}$ where
  $E_i = \toBPA{S_i}$ for $i=1,2$.
  Because $S_1 \LTSderives S_1'$,
  we obtain by induction that $\toBPATop{S_1} = \{ x_0 = E_1, \dots \}
  \LTSderives \toBPATop{S_1'} = \{ x_0 = E_1', \dots
  \}$. Therefore, $\{ x_0 = E_1;E_2, \dots \} \LTSderives \{ x_0 =
  E_1';E_2, \dots \} = \toBPATop{S_1'; S_2}$.

  \textbf{Case }$\frac{\DONE{S_1} \quad S_2 \LTSderives S_2'}{S_1; S_2
    \LTSderives S_2'}$.
  In this case $\toBPATop{S_1;S_2} = \{ x_0 = E_1;E_2, \dots \}$ where
  $E_i = \toBPA{S_i}$ for $i=1,2$.
  It is easy to see that $\DONE{S_1}$ implies $\DONE{\toBPA{S_1}}$,
  that is, $\DONE{E_1}$.
  Because $S_2 \LTSderives S_2'$,
  we obtain by induction that $\toBPATop{S_2} = \{ x_0 = E_2, \dots \}
  \LTSderives \toBPATop{S_2'} = \{ x_0 = E_2', \dots
  \}$. 
  Therefore, $\{ x_0 = E_1;E_2, \dots \} \LTSderives \{ x_0 =
  E_2', \dots \} = \toBPATop{S_2'}$.

  \textbf{Case }$\frac{S[\mu x.S/x] \LTSderives S'}{\mu x.S
    \LTSderives S'}$.
  In this case
  $\toBPATop{\mu x.S} = \{ x_0 = x, x = E, \dots \}$ with $E = \toBPA{\Unravel
    (S[\mu x.S/x])}$.
  By induction, $\toBPATop{S[\mu x.S/x]} \LTSderives
  \toBPATop{S'}$.
  Now $\toBPATop{S[\mu x.S/x]} = \{ x_0 = \toBPA{S[\mu x.S/x]}, \dots
  \}$ which proves the claim because $x_0 \TypeEquiv E$ by
  Lemma~\ref{lemma:s=unr-s}. 
\end{proof}

\clearpage
\begin{lemma}\label{lemma:app:bpa-unr-s}
  Suppose that $\toBPA{\Unravel
    (S)} \LTSderives E'$. Then $S \LTSderives S'$
  and $E' = \toBPATop{S'}$.
\end{lemma}
\begin{proof}
  By induction on the number $n$ of recursive calls of $\Unravel$.

  \textbf{Case }$n=0$.

  \textbf{Subcase }$S=\skipk$. Contradictory.

  \textbf{Subcase }$S=A$. Then $a=A$, $E'=\varepsilon$, and $S' =
  \skipk$.

  \textbf{Subcase }$S = \star\{\overline{l_i:S_i}\}$. Then $a = \star
  l_i$, $E' = \toBPA{S_i}$, and $S' = S_i$.

  \textbf{Case }$n>0$.

  \textbf{Subcase }$S = S_1;S_2$.
  If $\Unravel (S_1) = \skipk$, then $\Unravel (S) = \Unravel
  (S_2)$ with less than $n$ calls. As $\toBPA{\Unravel (S_2)} \LTSderives E'$, induction yields
  that $S_2 \LTSderives S'$ and $E' = \toBPATop{S'}$.
  As $\Unravel (S_1) = \skipk$, we know that $\DONE{S_1}$. Hence,
  ${S_1;S_2} \LTSderives S'$ and $E' = \toBPATop{S'}$.

  If $\Unravel (S_1)\ne \skipk$, then consider $\toBPA{\Unravel({S_1});
    S_2} \LTSderives E'$ because $\toBPA{\Unravel({S_1})} \LTSderives
  E_1'$, so that induction yields some $S_1'$ such that $S_1
  \LTSderives S_1'$ and $E_1' = \toBPATop{S_1'}$.

  \textbf{Subcase }$\mu x.S$.
  $\Unravel (\mu x.S) = \Unravel (S[\mu x.S/x])$ with one less
  invocation. As $\toBPA{\Unravel (S[\mu x.S/x])}
  \LTSderives E'$, induction yields that  $S[\mu x.S/x] \LTSderives
  S'$ with $E' = \toBPATop{S'} $.
\end{proof}

\begin{lemma}\label{lemma:app:bisimulation-BPA-backwards}
  Suppose that $S$ is well-formed and let $\BPAprocess = \toBPATop{S}$
  and $\BPAprocess \LTSderives \BPAprocess'$.

  There is some $S'$ such that $S \LTSderives S'$ and $\BPAprocess' = \toBPATop{S'}$.
\end{lemma}
\begin{proof}
  By induction on $S$.

  \textbf{Case }$\skipk$. Contradictory.

  \textbf{Case }$A$. For $\BPAprocess$, ${A \LTSderives \varepsilon
  }$. Choose $S' = \skipk$.

  \textbf{Case }$\star\{\overline{l_i:S_i}\}$. For $\BPAprocess$,
  ${\sum \overline{\star l_i; \toBPA{S_i}} \LTSderives[\star l_i]
    \toBPA{S_i} }$. Choose $S' = S_i$.

  \textbf{Case }$S_1;S_2$.
  If $\toBPA{S_1} \LTSderives E_1'$,
  then $S_1 \LTSderives S_1'$ and $E_1' = \toBPA{S_1'}$, by induction.
  Now, $\toBPA{S_1;S_2} \LTSderives E_1'; \toBPA{S_2} = \toBPA{S_1';
    S_2}$. Choose $S' = S_1';S_2$.

  If $\DONE{\toBPA{S_1}}$ and $\toBPA{S_2} \LTSderives E_2'$,
  then $\DONE{S_1}$ and $S_2 \LTSderives S_2'$ and $E_2' =
  \toBPA{S_2'}$, by induction. Now, $\toBPA{S_1;S_2} \LTSderives E_2' = \toBPA{S_2'}$. Choose $S' = S_2'$.

  \textbf{Case }$\mu x.S$.
  $\BPAprocess = \toBPATop{\mu x.S} = \{ x_0 = x, x = \toBPA{\Unravel
    (S[\mu x. S/x])} \}$. If $\BPAprocess \LTSderives \BPAprocess'$,
  then it must be because $\toBPA{\Unravel
    (S[\mu x. S/x])} \LTSderives E'$. Use Lemma~\ref{lemma:bpa-unr-s}
  to establish the claim.
\end{proof}

\begin{theorem}
  Suppose that $S$ is well-formed and let $\BPAprocess = \toBPATop{S}$.
  \begin{enumerate}
  \item If $ S \LTSderives[a] S'$, then $\BPAprocess \BPAderives[a]
    \BPAprocess'$ with $\BPAprocess' = \toBPATop{S'}$.
  \item If $\BPAprocess \BPAderives[a] \BPAprocess'$, then $S
    \LTSderives[a] S'$ with  $\BPAprocess' = \toBPATop{S'}$.
  \end{enumerate}
\end{theorem}
\begin{proof}
  By Lemmas~\ref{lemma:bisimulation-BPA-forwards} and~\ref{lemma:bisimulation-BPA-backwards}.
\end{proof}

%%% Local Variables:
%%% mode: latex
%%% TeX-master: "main"
%%% End:


% \begin{figure}[tp]
  \begin{align*}
    % & F (R) \subseteq R \qquad R \subseteq F (R) \\
    F &:  \Rangeof{S} \times \Rangeof{S} \to \Rangeof{S} \times \Rangeof{S} \\
    F (R) &= \{ (\skipk, \skipk) \} \\
    & \cup \{ (!B, !B) \} \\
    & \cup \{ (?B, ?B) \} \\
    & \cup \{ ((S_1; S_2), (S_1'; S_2')) \mid (S_1, S_1'), (S_2, S_2') \in R \} \\
    & \cup \{ (\oplus\{l_i\colon S_i\}_{i\in I}, \oplus\{l_i\colon S_i'\}_{i\in I}) \mid \{ (S_i, S_i') \mid i\in I\} \subseteq R \} \\
    & \cup \{ (\&\{l_i\colon S_i\}_{i\in I}, \&\{l_i\colon S_i'\}_{i\in I}) \mid \{ (S_i, S_i') \mid i\in I\} \subseteq R \} \\
    & \cup \{ (\mu x.S, S') \mid (S[\mu x.S/x], S') \in R \} \\
    & \cup \{ (S, \mu x.S') \mid (S, S'[\mu x.S'/x]) \in R \} \\
            &\\
    & \cup \{ ((S_1; S_2), S_2') \mid (S_2, S_2') \in R, (S_1, \skipk) \in R \} \\
    & \cup \{ ((S_1; S_2), S_1') \mid (S_1, S_1') \in R, (S_2, \skipk) \in R \} \\
    & \cup \{ (S_1, (S_1'; S_2')) \mid (S_1, S_1') \in R, (S_2', \skipk) \in R \} \\
    & \cup \{ (S_2, (S_1'; S_2')) \mid (S_2, S_2') \in R, (S_1', \skipk) \in R \} \\
    \\
      & \cup \{ ((S;S_3),(S_1';S'))
        \begin{array}[t]{ll}
          \mid
          & ((S_1;S_2), S) \in R,\\
          & (S',(S_2';S_3')) \in R, \\
          & (S_1, S_1') \in R,\\
          & (S_2, S_2') \in R,\\
          & (S_3, S_3') \in R\}
        \end{array}
    \\
      & \cup \{ ((S_1';S'),(S;S_3))
        \begin{array}[t]{ll}
          \mid
          & ((S_1;S_2), S) \in R,\\
          & (S',(S_2';S_3')) \in R, \\
          & (S_1, S_1') \in R,\\
          & (S_2, S_2') \in R,\\
          & (S_3, S_3') \in R\}
        \end{array}
    \\
    % \\
    % & \cup \{ (((S_1;S_2);S_3),S_4) \mid ((S_1;(S_2;S_3)), S_4) \in R\}\\
    % & \cup \{ ((S_1;(S_2;S_3)),S_4) \mid (((S_1;S_2);S_3), S_4) \in R\}\\
    % & \cup \{ (S_1,((S_2;S_3);S_4)) \mid  (S_1,(S_2;(S_3;S_4))) \in R\}\\
    % & \cup \{ (S_1,(S_2;(S_3;S_4))) \mid  (S_1,((S_2;S_3);S_4)) \in
    % R\}\\
    \\
    & \cup
    \begin{array}[t]{ll}
      \{ & ((\oplus\{l_i\colon S_i\}_{i\in I}; S), \oplus\{l_i\colon
      S_i'\}_{i\in I}),
      \\ & ((\&\{l_i\colon S_i\}_{i\in I}; S), \&\{l_i\colon S_i'\}_{i\in I})
      \\ \mid & \qquad \{ ((S_i; S), S_i') \mid i\in I\} \subseteq R \}
    \end{array}
    \\
    & \cup
    \begin{array}[t]{ll}
      \{ & ((\oplus\{l_i\colon S_i\}_{i\in I}), \oplus\{l_i\colon S_i'\}_{i\in I}; S)
      \\ & ((\&\{l_i\colon S_i\}_{i\in I}), \&\{l_i\colon S_i'\}_{i\in I}; S)
      \\\mid& \qquad \{ (S_i, (S_i';S)) \mid i\in I\} \subseteq R \}
    \end{array}
    \\
    \\
    \approx &\subseteq \Rangeof{S} \times \Rangeof{S} \\
    \approx & = \GFP (F)
  \end{align*}
  % \begin{align*}
% %     \cup & \{ ((\skipk;S_1), S_2) \mid (S_1, S_2) \in R\} \\
% %     \cup & \{ ((S_1;\skipk), S_2) \mid (S_1, S_2) \in R\} \\
% %     \cup & \{ (S_1,(\skipk;S_2)) \mid (S_1, S_2) \in R\} \\
% %     \cup & \{ (S_1,(S_2;\skipk)) \mid (S_1, S_2) \in R\}\\
%   \end{align*}
  \caption{Session type equivalence}
  \label{fig:type-equivalence}
\end{figure}

%%% Local Variables:
%%% mode: latex
%%% TeX-master: "main"
%%% End:

% Type equivalence in Figure~\ref{fig:type-equivalence}.
% Given a binary relation $R$,  we write
% $R^{-1}$ for the inverse relation $\{(y,x)\mid (x,y)\in R\}$ and $R^s$
% for the symmetric closure $R \cup R^{-1}$.

% ALGORITHMIC TYPE EQUIVALENCE  _ DEPRECATED

% \newpage
\subsection{Properties of session type equivalence}

\subsubsection{Contractivity}
\label{sec:contractivity}


Consider the (illformed) type $W = \mu x.(\skipk; x)$ and the rule
\begin{equation}
  F(R) = \dots {}\cup{} \{ ((\skipk;S_1), S_2) \mid (S_1, S_2) \in R \}\label{eq:1}
\end{equation}
Because the set $R = \{(W,T), ((\skipk; W), T) \mid \cdot \vdash T :: S\}$ is $F$-consistent, every session type would
be equivalent to $W$! With symmetry and transitivity, every session type would be equivalent to
every other.

Fortunately, contractivity rules out types like $W$ so that the $\skipk$-canceling rules like~\eqref{eq:1} can only be
applied a finite number of times between non-skip rules.

\begin{lemma}\label{lemma:S=skip}
  If $S \approx \skipk$, then $S \grmeq \skipk \grmor (S; S) \grmor
  \mu x.S$.
\end{lemma}
\begin{proof}
  The proof rests on the observation that the derivation of $S \approx
  \skipk$ is finite because of contractivity. Thus, induction on the
  derivation yields the result.
\end{proof}

\begin{lemma}\label{lemma:trans-skip}
  If $S_1 \approx \skipk$ and $\skipk \approx S_3$, then $S_1 \approx S_3$.
\end{lemma}
\begin{proof}
  Construct a derivation for $S_1\approx S_3$ by induction on $S_1$
  using its structure according to Lemma~\ref{lemma:S=skip}.

  \textbf{Case $\skipk$.} Immediate by assumption.

  \textbf{Case $\mu x.S_1'$.} By  Lemma~\ref{lemma:S=skip}, $S_1'[\mu
  x.S_1'/x] = S_1'$ so inversion of the $\mu$-left rule yields $S_1'
  \approx \skipk$. By induction, there is a derivation for $S_1'
  \approx S_3$ which can be completed by the $\mu$-left rule to $S_1
  \approx S_3$.

  \textbf{Case $(S_1';S_1'')$ where $S_1' \approx \skipk$ and $S_1''
    \approx \skipk$.} We need to perform an auxiliary induction on
  $S_3$.

  %% This proof step requires the more general skip rules.
  \textbf{Subcase $\skipk$.} Apply rule skip-left-left.

  \textbf{Subcase $(S_3'; S_3'')$ where $\skipk \approx S_3'$ and
    $\skipk \approx S_3''$.} Apply the semicolon rule to the
  inductively constructed proofs for $S_1' \approx S_3'$ and $S_1''
  \approx S_3''$.

  \textbf{Subcase $\mu x. S_3'$.} Use the $\mu$-right rule. Analogous
  to the case for the $\mu$-left rule.
\end{proof}

\begin{lemma}\label{lemma:S=B}
  If $S \approx {!B}$, then $S \grmeq {!B} \grmor (S^{\skipk};
  S)\grmor (S; S^{\skipk}) \grmor \mu x.S$ where $S^{\skipk}$ is
  described in Lemma~\ref{lemma:S=skip}.
\end{lemma}

\begin{lemma}
  If $S_1 \approx {!B}$ and ${!B} \approx S_3$, then $S_1 \approx S_3$.  
\end{lemma}
\begin{proof}
  Construct a derivation for $S_1 \approx S_3$ by induction on $S_1$
  using its structure according to Lemma~\ref{lemma:S=B}.

  \textbf{Case $!B$.} Immediate by assumption.

  \textbf{Case $(S^{\skipk}; S)$ where $S^{\skipk}\approx \skipk$ and
    $S \approx{!B}$.} By induction $S \approx S_3$. Applying rule
  skip-l-l yields $S_1 \approx S_3$.

  \textbf{Case $(S; S^{\skipk})$ where $S^{\skipk}\approx \skipk$ and
    $S \approx{!B}$.} Similar.

  \textbf{Case $\mu x.S_1'$.} By Lemma~\ref{lemma:S=B}, $S_1'[\mu
  x.S_1'/x] = S_1'$, so inversion of the $\mu$-left rule yields $S_1'
  \approx {!B}$. By induction, there is a derivation for $S_1'
  \approx S_3$ which can be completed by the $\mu$-left rule to $S_1
  \approx S_3$.
\end{proof}

\begin{lemma}\label{lemma:S=oplus}
  If $S \approx {\oplus\{l_i\colon S_i\}_{i\in I}}$, then $S \grmeq {\oplus\{l_i\colon S_i'\}_{i\in I}} \grmor (S^{\skipk};
  S)\grmor (S; S^{\skipk}) \grmor \mu x.S$ where $S_i \approx S_i'$ and  $S^{\skipk}$ is
  described in Lemma~\ref{lemma:S=skip}.
\end{lemma}

\begin{lemma}
  If $S_1 \approx {\oplus\{l_i\colon S_i\}_{i\in I}}$ and
  ${\oplus\{l_i\colon S_i\}_{i\in I}} \approx S_3$, then $S_1 \approx
  S_3$.
\end{lemma}

\subsubsection{Reflexivity}
\label{sec:reflexivity}

Let $R = \{ (T, T) \mid \cdot \vdash T :: \kinds \} \cup \{(T'[\mu
x.T' / x], \mu x.T') \mid \cdot \vdash T' :: \kinds \}$. Show that $R \subseteq F(R)$.

Obvious, except for $(\mu x.T', \mu x.T') \in R$, but in this case
$(\mu x.T', \mu x.T') \in F (\{(T'[\mu x.T' / x], \mu x.T')\}) \subseteq F (R)$.

For $(T'[\mu x.T' / x], \mu x.T') \in R$, observe that
$$(T'[\mu x.T' / x], \mu x.T') \in F (\{(T'[\mu x.T' / x], T'[\mu x.T'/x])\}) \subseteq F (R).$$

\subsubsection{Symmetry}
\label{sec:symmetry}

Let $R = \{ (T_2, T_1) \mid T_1 \approx T_2 \}$. Show that $R \subseteq F(R)$.

To consider an exemplary case,
suppose that $((S_1'; S_2'), (S_1; S_2)) \in R$ because $((S_1; S_2) \approx (S_1'; S_2'))$ because
$(S_1 \approx S_1')$ and $(S_2 \approx S_2')$. From the latter two assumptions, we obtain that
$(S_1', S_1) \in R$ and $(S_2', S_2) \in R$, so that $((S_1'; S_2'), (S_1; S_2)) \in F(R)$.

Suppose that $(S', \mu x.S) \in R$ because $\mu x.S \approx S'$ because $S[\mu x.S/x] \approx
S'$. From the latter assumption, we obtain that $(S', S[\mu x.S/x]) \in R$ and hence $(S', \mu x.S)
\in F (R)$. The other case involving $\mu$ is analogous.

\subsubsection{Transitivity}
\label{sec:transitivity}

Let $R = \{ (T, T'') \mid \exists T'. T \approx T' \wedge T' \approx T''\}$. 
Show that $R \subseteq F (R)$.

Observe that, by reflexivity of $\approx$, ${\approx} \subseteq R$.

Suppose that $(\mu x. S, T) \in R$ because $(\mu x.S, T) \in
{\approx}$ and $(T, T) \in {\approx}$ (by reflexivity). Now  $(\mu x.S, T) \in
{\approx}$ because $(S[\mu x.S/x], T) \in {\approx}$ and hence $(S[\mu
x.S/x], T) \in R$, so that $(\mu x. S, T) \in F(R)$. 

More cases omitted.

Consider $(S_1, (\skipk; S_2)) \in {\approx}$ (because  $S_1\approx S_2$)
and $((\skipk; S_2), S_3) \in {\approx}$ (because $S_2 \approx S_3$).
Hence $(S_1, S_3) \in R$. Must show that $(S_1, S_3) \in F (R)$!

But it might be the case that $S_2 = (\skipk; S_2')$ so that, by inverting the same rule, $S_1
\approx S_2'$  and $S_2' \approx S_3$. Fortunately, contractivity tells us that $S_2$ cannot contain 
unguarded recursion, so that an auxiliary induction on $S_2$ or on the proof of contractivity of $S_2$ can be
applied. 

\textbf{Case $S_2 = \skipk$.} In this case, we have $S_1 \approx
\skipk$ and $\skipk \approx S_3$ and by Lemma~\ref{lemma:trans-skip} we
obtain $(S_1, S_3) \in {\approx} = F ({\approx}) \subseteq F (R)$. 

% \textbf{Subcase $S_1 = \skipk$, $S_3 = \skipk$.} Trivally in $F(R)$.

% \textbf{Subcase $S_1 = \skipk$, $S_3 = \mu x.S'$.} It must be that $\skipk \approx S'[\mu x.S'/x]$,
% so $(\skipk, S'[\mu x.S'/x]) \in R$, so $(\skipk, \mu x.S') \in F(R)$.

% \textbf{Subcase $S_1 = \skipk$, $S_3 = (\skipk; S_3')$.} It must be that $(\skipk, S_3') \in
% {\approx} \subseteq R$, so $(\skipk, S_3) \in F (R)$.

% \textbf{Subcase $S_1 = \skipk$, $S_3 = (S_3'; \skipk)$.} Similar.

% \textbf{Subcase $S_1 = \skipk$, $S_3 = ((S_2';S_3'); S_4')$.} It must be that $(\skipk,
% (S_2';(S_3'; S_4'))) \in {\approx} \subseteq R$, so $(\skipk, ((S_2';S_3'); S_4')) \in F(R)$.

% \textbf{Subcase $S_1 = \skipk$, $S_3 = (S_2';(S_3'; S_4'))$.} Similar.

% \textbf{Subcase $S_1 = \mu x.S_1'$, $S_3 = \mu x.S_3'$.}
% It must be that $S_1'[\mu x.S_1'/x] \approx \skipk$ and
% $\skipk \approx S_3'[\mu x.S_3'/x]$.
% By Lemma~\ref{lemma:S=skip}, $x$ does not occur in $S_1'$ or $S_3'$ so
% we have $S_1' \approx \skipk$ and $\skipk \approx S_3'$.

\textbf{Case $S_2 = !B$.} 
In this case, we have $S_1 \approx {!B}$ and ${!B} \approx S_3$ and by
Lemma~\ref{lemma:S=B} we obtain  $(S_1, S_3) \in {\approx} = F
({\approx}) \subseteq F (R)$.

\textbf{Case $S_2 = ?B$.} Analogous.

\newpage
%%% Local Variables:
%%% mode: latex
%%% TeX-master: "main"
%%% End:


% \begin{figure}[t]
  \begin{gather*}
    \frac{
      S_1 \equiv S_2 \in \Sigma
    }{
      \Sigma \vdash S_1 \equiv S_2
    }
    \quad
    \frac{}{
      \Sigma \vdash \skipk \equiv \skipk
    }
    \\
    \frac{
      \Sigma \vdash S_1 \equiv S_2
    }{
      \Sigma \vdash \alpha;S_1 \equiv \alpha;S_2
    }
    \;\,
    \frac{
      \Sigma \vdash S_1 \equiv S_2
    }{
      \Sigma \vdash \;!B;S_1 \equiv \;!B;S_2
    }
    \;\,
    \frac{
      \Sigma \vdash S_1 \equiv S_2
    }{
      \Sigma \vdash \;?B;S_1 \equiv \;?B;S_2
    }
    \\
    \frac{
      \Sigma \vdash S_i \equiv S_i'
      \quad
      \forall i \in I
    }{
      \Sigma \vdash \oplus\{l_i\colon S_i\}_{i\in I} \equiv \oplus\{l_i\colon S'_i\}_{i\in I} 
    }
    \\
    \frac{
      \Sigma \vdash S_i \equiv S_i'
      \quad
      \forall i \in I
    }{
      \Sigma \vdash \&\{l_i\colon S_i\}_{i\in I} \equiv \&\{l_i\colon S'_i\}_{i\in I} 
    }
    \\
    \frac{
      \Sigma, S_1 \equiv S_2 \vdash \Unfold(S_1) \equiv S_2
      \quad
      S_1\,\unguarded
    }{
      \Sigma \vdash S_1 \equiv S_2
    }
    \\
    \frac{
      \Sigma, S_1 \equiv S_2 \vdash S_1 \equiv \Unfold(S_2)
      \quad
      S_2\,\unguarded
    }{
      \Sigma \vdash S_1 \equiv S_2
    }
  \end{gather*}
  Rules should be tried in order
  \caption{Algorithmic type equivalence}
  \label{fig:alg-type-equiv}
\end{figure}

%%% Local Variables:
%%% mode: latex
%%% TeX-master: "main"
%%% End:


% Algorithmic type equivalence is in
% Figure~\ref{fig:alg-type-equiv}.
% %
% For example, letting $S_1= \mu x.!B;x$, $S_2 = \;!B;S_2'$, and
% $S_2' = \mu y.((\skipk;!B);y)$, for an initial goal of the form
% $\cdot \vdash S_1 \equiv S_2$, a run of the algorithm generates the
% following subgoals:
% %
% \begin{align*}
%   S_1 \equiv S_2 \vdash& \;!B;S_1 \equiv \;!B;S_2'
%   \\
%   S_1 \equiv S_2 \vdash&\; S_1\equiv S_2'
%   \\
%   S_1 \equiv S_2, S_1\equiv S_2' \vdash& \;!B;S_1 \equiv !B;(\skipk;S_2')
%   \\
%   S_1 \equiv S_2, S_1\equiv S_2' \vdash& S_1 \equiv \skipk;S_2'
%   \\
%   S_1 \equiv S_2, S_1\equiv S_2', !B;S_1 \equiv \skipk;S_2' \vdash& \;!B;S_1 \equiv \;!B;(\skipk;S_2')
%   \\
%   S_1 \equiv S_2, S_1\equiv S_2', !B;S_1 \equiv \skipk;S_2' \vdash& S_1 \equiv \skipk;S_2'
% \end{align*}

% \begin{lemma}[Termination]
  \label{lem:alg-terminates}
  The type equivalence algorithm always terminates on closed session
  types.
\end{lemma}
%
\begin{proof}
  Define the set of \emph{subterms} of a session type $S$,
  $\subterms(S)$, as follows.
  \begin{equation*}
    \{S,\Unfold(S)\} \cup\left\{
      \begin{array}{ll}
        \emptyset & \text{if } \Unfold(S) = \skipk
        \\
        \{S_1\}\cup\subterms(S_2) & \text{if } \Unfold(S) = S_1;S_2
        \\
        \{S_i\mid i\in I\}  & \text{if } \Unfold(S) = \mathsf{\_}\{l_i\colon S_i\}_{i\in I}
      \end{array}
      \right.
  \end{equation*}

  Given an initial goal $\cdot\vdash S^0_1\equiv S^0_2$, let $N$ be
%  the natural number
  $\cardinality{\subterms(S^0_1)} \times
  \cardinality{\subterms(S^0_2)}$.
  % 
  Define the \emph{measure} $\measure$ of an arbitrary goal
  $\Sigma \vdash S_1 \equiv S_2$ as the pair $(N-n,m)$ where~$n$ is
  the number of assumptions in~$\Sigma$ and~$m$ is the sum of the
  nesting of type constructors in~$S_1$ and~$S_2$. Assume $\measure$
  equipped with the usual \emph{lexicographic ordering}.%
  % 
  \footnote{$(a,b)<(a',b')$ if either $a<a'$ or $a=a'$ and $b<b'$.}
  %
  It is straightforward to show that each application of a rule
  strictly decreases $\measure$.

  It remains to show that $\measure$ is well-founded; this follows
  from the fact that $N$ is finite, that $N\ge n$ (below) and that
  $m>0$ (the depth of a term is a positive number). That $N\ge n$ is
  straightforward for all rules (where~$n$ is invariant) except the
  last. For the last rule we have to show that, for each goal
  $\Sigma \vdash S_0\equiv S_1$ arising in the execution of
  $\cdot\vdash S^0_1\equiv S^0_2$, we have:
  %
  \begin{equation*}
    (S_1,S_2) \in \subterms(S_1^0) \times \subterms(S_2^0)
  \end{equation*}
  (To be completeted)
\end{proof}

Now for soundness. We again follow Gay and
Hole~\cite{DBLP:journals/acta/GayH05}.
%
Say that a goal
$S_1 \equiv S'_1,\dots,S_{n-1}\equiv S_{n-1}' \vdash S_n \equiv S_n'$
is \emph{sound} when $S_i \TypeSim S_i'$ for all $1\le i\le n$.

\begin{lemma}
  \label{lem:alg-subgoals}
  If a goal is sound then the conclusion of one of the rules in
  figure~\ref{fig:alg-type-equiv} matches the goal and the new
  subgoals corresponding to the hypotheses are all sound goals.
\end{lemma}
%
\begin{proof}
  Let $\Sigma \vdash S_1 \equiv S_2$ be a sound goal. If
  $S_1 \equiv S_2 \in \Sigma$, then the first axiom applies and there
  are no subgoals.
  
  If $S_i = \alpha;S_i'$ with $i=1,2$, we know by hypothesis that
  $\alpha;S_1' \TypeSim \alpha;S_2'$, and by definition that
  $(\alpha;S_1', \alpha;S_2') \in R$, and by unfolding that
  $(\skipk;S_1',\skipk;S_2') \in R$. The result follows from
  Lemma~\ref{lem:skip-elim}.
  %
  The cases of $S_i = \;!;S_i'$ and $S_i = \;?;S_i'$ are similar.

  In the cases for choice, soundness of the new subgoals follows from
  the definition of~$\TypeSim$.

  The case for the last rule follows from
  Lemma~\ref{lem:unfold-type-sim}.
\end{proof}

\begin{lemma}
  \label{lem:alg-not-false}
  If $S_1 \TypeSim S_2$ then the type equivalence algorithm does not
  return false when applied to $\cdot \vdash S_1 \equiv S_2$.
\end{lemma}
%
\begin{proof}
  Consider all the subgoals produced by the algorithm when given
  $\cdot\vdash S_1\equiv S_2$. From the hypothesis we know that the
  initial goal is sound; by Lemma~\ref{lem:alg-subgoals} all of the
  generated subgoals are sound.  By the same lemma, the algorithm
  either proceeds or returns true.
\end{proof}

\begin{theorem}[Soundness of algorithmic type equivalence]
  \label{thm:alg-soundness}
  If $S_1 \TypeSim S_2$ then $\cdot \vdash S_1 \equiv S_2$  
\end{theorem}
%
\begin{proof}
  By Lemma~\ref{lem:alg-terminates} the algorithm terminates. By
  Lemma~\ref{lem:alg-not-false} the algorithm does not return
  false. Therefore it must return true.
\end{proof}

Now for completeness.

\begin{lemma}
  \label{lem:unfold-preserves-alg-equiv}
  If $\cdot\vdash S_1 \equiv S_2$ then
  $\cdot\vdash \Unfold(S_1) \equiv \Unfold(S_2)$.
\end{lemma}

\begin{theorem}[Completeness of algorithmic type equivalence]
  If $\cdot\vdash S_1 \equiv S_2$ then $S_1 \TypeSim S_2$.
\end{theorem}
%
\begin{proof}
  By Lemma~\ref{lem:unfold-preserves-alg-equiv} it is sufficient to
  consider the case when $S_1$ and $S_2$ are guarded types. We show
  that
  %
  \begin{equation*}
    R = \{(S_1,S_2) \mid \cdot\vdash S_1 \equiv S_2 \text{ and } S_1
    \text{ and } S_2 \text{ are guarded}\}
  \end{equation*}
  %
  is a type simulation.

  Consider case 6 (external choice, $\&$), and assume
  $(S_1,S_2) \in R$ and
  $\Unfold(S_1) = \;\&\{l_i\colon S_i\}_{i\in I}$. Since $S_1$ is
  guarded we have $S_1=\;\&\{l_i\colon S_i\}_{i\in I}$. The only rule
  in the algorithm that applies is the \&-rule, and we get that
  $S_2 = \;\&\{l_j\colon S'_j\}_{j\in J}$. The rule ensures that $I=J$
  and that $\cdot\vdash S_i \equiv S'_i$, for all $i \in I$. Then
  Lemma~\ref{lem:unfold-preserves-alg-equiv} ensures that
  $\cdot\vdash \Unfold(S_i) \equiv \Unfold(S'_i)$, and
  Lemma~\ref{lem:unfold-yields-guarded-types} that $S_i$ and $S'_i$
  are guarded, hence $(S_i',S_i') \in R$ as required.  
\end{proof}

\begin{corollary}
  $\cdot \vdash S_1 \equiv S_2$ if and only if $\S_1 \TypeSim S_2$.
\end{corollary}

%%% Local Variables:
%%% mode: latex
%%% TeX-master: "main"
%%% End:


% END OF DEPRECATED

% THE UN PREDs

The $\un(T)$ predicate for types is an abbreviation of judgment
$\vdash T :: \kind^\Linear$. For contexts, predicate
$\un(x_1\colon T_1,\dots, x_n\colon T_n)$ is true when all
$\un(T_1),\dots,\un(T_n)$ hold.

% The $\un$ predicate is true of the following session types and
% types. The predicate is not defined for type schema.
% %
% \begin{gather*}
%   \un(\skipk) \quad \un(S_1;S_2) \text{ if } \un(S_1) \text{ and } \un(S_2) 
%   \\
%   \un (\unitk)
%   \quad
%   \un([l_i\colon T_i]) \text{ if } \un(T_1) \text{ for all } i
%   \\ \un(T_1\rightarrow T_2)
%   \quad \un(B) \quad \un(\mu x.T) \text{ if } \un(T) \quad \un (x)
% \end{gather*}

% DUALITY

The duality function on session types, $\dual S$, is defined as
follows.
%
\begin{gather*}
  \dual\skipk = \skipk
  \qquad
  \dual{!B} = \;?B
  \qquad
  \dual{?B} = \;!B
  \qquad
  \dual{S_1;S_2} = \dual{S_1};\dual{S_2}
  \\
  \dual{\&\{l_i\colon S_i\}} = \oplus\{l_i\colon \dual{S_i}\}
  \qquad
  \dual{\oplus\{l_i\colon S_i\}} = \&\{l_i\colon \dual{S_i}\}
  \\
  \dual{\mu x.S} = \mu x.\dual S
  \qquad
  \dual x = x
\end{gather*}
%
This simple definition is justified by the fact that the types we
consider are first order, hence the complication known to arise
in presence of higher-order
recursion~\cite{bernardi.hennessy:using-contracts-model-session-types}
does occur.

To check whether $S_1$ is dual to $S_2$ we compute $S_3 = \dual{S_1}$
and check $S_2$ and $S_3$ for equivalence.
%
Duality is clearly an involution ($\dual{\dual S} = S$), hence we can
alternatively compute $\dual{S_2}$ and check that $S_1$ is equivalent
to $\dual{S_2}$.
%
For example, to check that $!B;\mu x.(\skipk;!B;x)$ is dual to
$\mu y.(?B;y)$, we compute $\dual{\mu y.(?B;y)}$ to obtain
$\mu y.(!B;y)$, and check that this type is equivalent to
$!B;\mu x.(\skipk;!B;x)$.

% Let $\rho$ denote a map from recursion variables~$x$ into session
% types~$S$, where $\varepsilon$ is empty map, and $\rho[x\mapsto S]$ is
% map extension. The substitution of recursion variable~$x$ by $S'$ in
% $S$, notation $S\subs{S'}x$, is defined appropriately. The
% \emph{duality function} on session types is defined as
% %
% \begin{equation*}
% \dual S = \dualof(S,\varepsilon)
% \end{equation*}
% where
% %
% \begin{align*}
%   \dualof(\skipk,\rho) &= \skipk\\
%   \dualof(!B,\rho) &=\: ?(B\rho)\\
%   \dualof(?B,\rho) &=\: !(B\rho)\\
%   \dualof(S_1;S_2,\rho) &= \dualof(S_1,\rho); \dualof(S_2,\rho)\\
%   \dualof(\oplus\{l_i\colon S_i\}, \rho) &= \&\{l_i\colon \dualof(S_i,\rho)\}\\
%   \dualof(\&\{l_i\colon S_i\}, \rho) &= \oplus\{l_i\colon \dualof(S_i,\rho)\}\\
%   \dualof(\mu x.S, \rho) &= \mu x.\dualof(S, \rho[x\mapsto \mu x.S])
% \end{align*}

% This definition fixes an issue with the (flawed) more conventional
% syntactic definition of
% duality~\cite{bernardi.hennessy:using-contracts-model-session-types}.
% Example. Take $S = \mu x.\mu y.?y;x$. Then
% $\dual T = \mu x \mu y.!(\mu y.?y.S);x$, as opposed to
% $\mu x.\mu y.!y;x$.

% TYPING

\input{fig:typing}

Typing rules in Figure~\ref{fig:typing}.
%
The usual derived rules.
%
\begin{gather*}
  \frac{
    \Gamma_1 \vdash e_1:T_1
    \quad
    \Gamma_2,x:T_1 \vdash e_2:T_2
  }{
    \Gamma_1,\Gamma_2 \vdash \letin x {e_1}{e_2} : T_2
  }
\\
  \frac{
    \Gamma_1 \vdash e_1:T_1
    \quad
    \Gamma_2 \vdash e_2:T_2
    \quad
    \un(T_1)
  }{
    \Gamma_1,\Gamma_2  \vdash e_1;e_2 : T_2
  }
\end{gather*}

% EXAMPLES

\subsection{Examples}
\label{sec:examples}

Streaming a tree on a channel, back and forth.

\lstinputlisting{tree.cfs}

A trace of the type of the channel in function \lstinline|transform|,
where \lstinline|TC| abbreviates \lstinline|TreeChannel|.

\lstinputlisting{trace.cfs}

The arithmetic expression server.

\lstinputlisting{arithmetic-server.cfs}

Now with a datatype to simplify the client's life.

\lstinputlisting{arithmetic-server-data.cfs}

Let's try a SePi-like version. All we need are context-free session
types and type schemas\footnote{Predicative Polymorphism in
  pi-Calculus. Vasco Thudichum Vasconcelos. In 6th Parallel
  Architectures and Languages Europe, volume 817 of LNCS, pages
  425--437. Springer, 1994.}. We witness the extra ``plumbing''
typical of the pi-calculus, but otherwise very little (type only, in
fact) extra basic notions.

\lstinputlisting{arithmetic-server-sepi.cfs}

%%% Local Variables:
%%% mode: latex
%%% TeX-master: "main"
%%% End:
