\section{Conclusion}
\label{sec:conclusion}

Context-free session types extend the expressiveness of regular
session types by generalizing the type structure from
regular to context-free processes. This extension enables
the low-level implementation of type-safe serialization of recursive datatypes
among other examples. 

While we have established decidability of type checking, there
is still work to do towards a practical type checking
algorithm. This algorithm could be based on the algorithm that decides
BPA equivalence, but there may be other alternatives to consider \cite{DBLP:journals/iandc/LanesePSS11}.

We further believe that our approach scales to the serialization of XML
documents, but we leave the elaboration of this connection to future work.


% \vv{Shall we say something about this: ``This is a reasonable simplification for a first study of context free session
% types. There are, however, other decidability results for strong bisimulation of
% the higher-order pi calculus which could perhaps support full context free
% session types with delegation. For example:

%   I. Lanese, J. A. Pérez, D. Sangiorgi, A. Schmitt, 
%   On the expressiveness and decidability of higher-order process calculi,
%   Information and Computation 2011.''}

% \vv{Shall we say something about this: ``Does this mean that the only challenge for a type-checking algorithm is a
% practical algorithm for deciding strong bisimulation? 

% It is difficult to see an obvious answer by observing the fairly complex
% inference rules of figure 6 (the type assignment system), and the paper itself
% does not address this question. Even the challenge of finding a bisimulation
% algorithm is not mentioned before the conclusions.''}

%%% Local Variables:
%%% mode: latex
%%% TeX-master: "main"
%%% End:
