\newpage
\subsection{Properties of session type equivalence}

\subsubsection{Contractivity}
\label{sec:contractivity}


Consider the (illformed) type $W = \mu x.(\skipk; x)$ and the rule
\begin{equation}
  F(R) = \dots {}\cup{} \{ ((\skipk;S_1), S_2) \mid (S_1, S_2) \in R \}\label{eq:1}
\end{equation}
Because the set $R = \{(W,T), ((\skipk; W), T) \mid \cdot \vdash T :: S\}$ is $F$-consistent, every session type would
be equivalent to $W$! With symmetry and transitivity, every session type would be equivalent to
every other.

Fortunately, contractivity rules out types like $W$ so that the $\skipk$-canceling rules like~\eqref{eq:1} can only be
applied a finite number of times between non-skip rules.

\begin{lemma}\label{lemma:S=skip}
  If $S \approx \skipk$, then $S \grmeq \skipk \grmor (S; S) \grmor
  \mu x.S$.
\end{lemma}
\begin{proof}
  The proof rests on the observation that the derivation of $S \approx
  \skipk$ is finite because of contractivity. Thus, induction on the
  derivation yields the result.
\end{proof}

\begin{lemma}\label{lemma:trans-skip}
  If $S_1 \approx \skipk$ and $\skipk \approx S_3$, then $S_1 \approx S_3$.
\end{lemma}
\begin{proof}
  Construct a derivation for $S_1\approx S_3$ by induction on $S_1$
  using its structure according to Lemma~\ref{lemma:S=skip}.

  \textbf{Case $\skipk$.} Immediate by assumption.

  \textbf{Case $\mu x.S_1'$.} By  Lemma~\ref{lemma:S=skip}, $S_1'[\mu
  x.S_1'/x] = S_1'$ so inversion of the $\mu$-left rule yields $S_1'
  \approx \skipk$. By induction, there is a derivation for $S_1'
  \approx S_3$ which can be completed by the $\mu$-left rule to $S_1
  \approx S_3$.

  \textbf{Case $(S_1';S_1'')$ where $S_1' \approx \skipk$ and $S_1''
    \approx \skipk$.} We need to perform an auxiliary induction on
  $S_3$.

  %% This proof step requires the more general skip rules.
  \textbf{Subcase $\skipk$.} Apply rule skip-left-left.

  \textbf{Subcase $(S_3'; S_3'')$ where $\skipk \approx S_3'$ and
    $\skipk \approx S_3''$.} Apply the semicolon rule to the
  inductively constructed proofs for $S_1' \approx S_3'$ and $S_1''
  \approx S_3''$.

  \textbf{Subcase $\mu x. S_3'$.} Use the $\mu$-right rule. Analogous
  to the case for the $\mu$-left rule.
\end{proof}

\begin{lemma}\label{lemma:S=B}
  If $S \approx {!B}$, then $S \grmeq {!B} \grmor (S^{\skipk};
  S)\grmor (S; S^{\skipk}) \grmor \mu x.S$ where $S^{\skipk}$ is
  described in Lemma~\ref{lemma:S=skip}.
\end{lemma}

\begin{lemma}
  If $S_1 \approx {!B}$ and ${!B} \approx S_3$, then $S_1 \approx S_3$.  
\end{lemma}
\begin{proof}
  Construct a derivation for $S_1 \approx S_3$ by induction on $S_1$
  using its structure according to Lemma~\ref{lemma:S=B}.

  \textbf{Case $!B$.} Immediate by assumption.

  \textbf{Case $(S^{\skipk}; S)$ where $S^{\skipk}\approx \skipk$ and
    $S \approx{!B}$.} By induction $S \approx S_3$. Applying rule
  skip-l-l yields $S_1 \approx S_3$.

  \textbf{Case $(S; S^{\skipk})$ where $S^{\skipk}\approx \skipk$ and
    $S \approx{!B}$.} Similar.

  \textbf{Case $\mu x.S_1'$.} By Lemma~\ref{lemma:S=B}, $S_1'[\mu
  x.S_1'/x] = S_1'$, so inversion of the $\mu$-left rule yields $S_1'
  \approx {!B}$. By induction, there is a derivation for $S_1'
  \approx S_3$ which can be completed by the $\mu$-left rule to $S_1
  \approx S_3$.
\end{proof}

\begin{lemma}\label{lemma:S=oplus}
  If $S \approx {\oplus\{l_i\colon S_i\}_{i\in I}}$, then $S \grmeq {\oplus\{l_i\colon S_i'\}_{i\in I}} \grmor (S^{\skipk};
  S)\grmor (S; S^{\skipk}) \grmor \mu x.S$ where $S_i \approx S_i'$ and  $S^{\skipk}$ is
  described in Lemma~\ref{lemma:S=skip}.
\end{lemma}

\begin{lemma}
  If $S_1 \approx {\oplus\{l_i\colon S_i\}_{i\in I}}$ and
  ${\oplus\{l_i\colon S_i\}_{i\in I}} \approx S_3$, then $S_1 \approx
  S_3$.
\end{lemma}

\subsubsection{Reflexivity}
\label{sec:reflexivity}

Let $R = \{ (T, T) \mid \cdot \vdash T :: \kinds \} \cup \{(T'[\mu
x.T' / x], \mu x.T') \mid \cdot \vdash T' :: \kinds \}$. Show that $R \subseteq F(R)$.

Obvious, except for $(\mu x.T', \mu x.T') \in R$, but in this case
$(\mu x.T', \mu x.T') \in F (\{(T'[\mu x.T' / x], \mu x.T')\}) \subseteq F (R)$.

For $(T'[\mu x.T' / x], \mu x.T') \in R$, observe that
$$(T'[\mu x.T' / x], \mu x.T') \in F (\{(T'[\mu x.T' / x], T'[\mu x.T'/x])\}) \subseteq F (R).$$

\subsubsection{Symmetry}
\label{sec:symmetry}

Let $R = \{ (T_2, T_1) \mid T_1 \approx T_2 \}$. Show that $R \subseteq F(R)$.

To consider an exemplary case,
suppose that $((S_1'; S_2'), (S_1; S_2)) \in R$ because $((S_1; S_2) \approx (S_1'; S_2'))$ because
$(S_1 \approx S_1')$ and $(S_2 \approx S_2')$. From the latter two assumptions, we obtain that
$(S_1', S_1) \in R$ and $(S_2', S_2) \in R$, so that $((S_1'; S_2'), (S_1; S_2)) \in F(R)$.

Suppose that $(S', \mu x.S) \in R$ because $\mu x.S \approx S'$ because $S[\mu x.S/x] \approx
S'$. From the latter assumption, we obtain that $(S', S[\mu x.S/x]) \in R$ and hence $(S', \mu x.S)
\in F (R)$. The other case involving $\mu$ is analogous.

\subsubsection{Transitivity}
\label{sec:transitivity}

Let $R = \{ (T, T'') \mid \exists T'. T \approx T' \wedge T' \approx T''\}$. 
Show that $R \subseteq F (R)$.

Observe that, by reflexivity of $\approx$, ${\approx} \subseteq R$.

Suppose that $(\mu x. S, T) \in R$ because $(\mu x.S, T) \in
{\approx}$ and $(T, T) \in {\approx}$ (by reflexivity). Now  $(\mu x.S, T) \in
{\approx}$ because $(S[\mu x.S/x], T) \in {\approx}$ and hence $(S[\mu
x.S/x], T) \in R$, so that $(\mu x. S, T) \in F(R)$. 

More cases omitted.

Consider $(S_1, (\skipk; S_2)) \in {\approx}$ (because  $S_1\approx S_2$)
and $((\skipk; S_2), S_3) \in {\approx}$ (because $S_2 \approx S_3$).
Hence $(S_1, S_3) \in R$. Must show that $(S_1, S_3) \in F (R)$!

But it might be the case that $S_2 = (\skipk; S_2')$ so that, by inverting the same rule, $S_1
\approx S_2'$  and $S_2' \approx S_3$. Fortunately, contractivity tells us that $S_2$ cannot contain 
unguarded recursion, so that an auxiliary induction on $S_2$ or on the proof of contractivity of $S_2$ can be
applied. 

\textbf{Case $S_2 = \skipk$.} In this case, we have $S_1 \approx
\skipk$ and $\skipk \approx S_3$ and by Lemma~\ref{lemma:trans-skip} we
obtain $(S_1, S_3) \in {\approx} = F ({\approx}) \subseteq F (R)$. 

% \textbf{Subcase $S_1 = \skipk$, $S_3 = \skipk$.} Trivally in $F(R)$.

% \textbf{Subcase $S_1 = \skipk$, $S_3 = \mu x.S'$.} It must be that $\skipk \approx S'[\mu x.S'/x]$,
% so $(\skipk, S'[\mu x.S'/x]) \in R$, so $(\skipk, \mu x.S') \in F(R)$.

% \textbf{Subcase $S_1 = \skipk$, $S_3 = (\skipk; S_3')$.} It must be that $(\skipk, S_3') \in
% {\approx} \subseteq R$, so $(\skipk, S_3) \in F (R)$.

% \textbf{Subcase $S_1 = \skipk$, $S_3 = (S_3'; \skipk)$.} Similar.

% \textbf{Subcase $S_1 = \skipk$, $S_3 = ((S_2';S_3'); S_4')$.} It must be that $(\skipk,
% (S_2';(S_3'; S_4'))) \in {\approx} \subseteq R$, so $(\skipk, ((S_2';S_3'); S_4')) \in F(R)$.

% \textbf{Subcase $S_1 = \skipk$, $S_3 = (S_2';(S_3'; S_4'))$.} Similar.

% \textbf{Subcase $S_1 = \mu x.S_1'$, $S_3 = \mu x.S_3'$.}
% It must be that $S_1'[\mu x.S_1'/x] \approx \skipk$ and
% $\skipk \approx S_3'[\mu x.S_3'/x]$.
% By Lemma~\ref{lemma:S=skip}, $x$ does not occur in $S_1'$ or $S_3'$ so
% we have $S_1' \approx \skipk$ and $\skipk \approx S_3'$.

\textbf{Case $S_2 = !B$.} 
In this case, we have $S_1 \approx {!B}$ and ${!B} \approx S_3$ and by
Lemma~\ref{lemma:S=B} we obtain  $(S_1, S_3) \in {\approx} = F
({\approx}) \subseteq F (R)$.

\textbf{Case $S_2 = ?B$.} Analogous.

\newpage
%%% Local Variables:
%%% mode: latex
%%% TeX-master: "main"
%%% End:
