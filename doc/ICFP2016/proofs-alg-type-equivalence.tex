\begin{lemma}[Termination]
  \label{lem:alg-terminates}
  The type equivalence algorithm always terminates on closed session
  types.
\end{lemma}
%
\begin{proof}
  Define the set of \emph{subterms} of a session type $S$,
  $\subterms(S)$, as follows.
  \begin{equation*}
    \{S,\Unfold(S)\} \cup\left\{
      \begin{array}{ll}
        \emptyset & \text{if } \Unfold(S) = \skipk
        \\
        \{S_1\}\cup\subterms(S_2) & \text{if } \Unfold(S) = S_1;S_2
        \\
        \{S_i\mid i\in I\}  & \text{if } \Unfold(S) = \mathsf{\_}\{l_i\colon S_i\}_{i\in I}
      \end{array}
      \right.
  \end{equation*}

  Given an initial goal $\cdot\vdash S^0_1\equiv S^0_2$, let $N$ be
%  the natural number
  $\cardinality{\subterms(S^0_1)} \times
  \cardinality{\subterms(S^0_2)}$.
  % 
  Define the \emph{measure} $\measure$ of an arbitrary goal
  $\Sigma \vdash S_1 \equiv S_2$ as the pair $(N-n,m)$ where~$n$ is
  the number of assumptions in~$\Sigma$ and~$m$ is the sum of the
  nesting of type constructors in~$S_1$ and~$S_2$. Assume $\measure$
  equipped with the usual \emph{lexicographic ordering}.%
  % 
  \footnote{$(a,b)<(a',b')$ if either $a<a'$ or $a=a'$ and $b<b'$.}
  %
  It is straightforward to show that each application of a rule
  strictly decreases $\measure$.

  It remains to show that $\measure$ is well-founded; this follows
  from the fact that $N$ is finite, that $N\ge n$ (below) and that
  $m>0$ (the depth of a term is a positive number). That $N\ge n$ is
  straightforward for all rules (where~$n$ is invariant) except the
  last. For the last rule we have to show that, for each goal
  $\Sigma \vdash S_0\equiv S_1$ arising in the execution of
  $\cdot\vdash S^0_1\equiv S^0_2$, we have:
  %
  \begin{equation*}
    (S_1,S_2) \in \subterms(S_1^0) \times \subterms(S_2^0)
  \end{equation*}
  (To be completeted)
\end{proof}

Now for soundness. We again follow Gay and
Hole~\cite{DBLP:journals/acta/GayH05}.
%
Say that a goal
$S_1 \equiv S'_1,\dots,S_{n-1}\equiv S_{n-1}' \vdash S_n \equiv S_n'$
is \emph{sound} when $S_i \TypeSim S_i'$ for all $1\le i\le n$.

\begin{lemma}
  \label{lem:alg-subgoals}
  If a goal is sound then the conclusion of one of the rules in
  figure~\ref{fig:alg-type-equiv} matches the goal and the new
  subgoals corresponding to the hypotheses are all sound goals.
\end{lemma}
%
\begin{proof}
  Let $\Sigma \vdash S_1 \equiv S_2$ be a sound goal. If
  $S_1 \equiv S_2 \in \Sigma$, then the first axiom applies and there
  are no subgoals.
  
  If $S_i = \alpha;S_i'$ with $i=1,2$, we know by hypothesis that
  $\alpha;S_1' \TypeSim \alpha;S_2'$, and by definition that
  $(\alpha;S_1', \alpha;S_2') \in R$, and by unfolding that
  $(\skipk;S_1',\skipk;S_2') \in R$. The result follows from
  Lemma~\ref{lem:skip-elim}.
  %
  The cases of $S_i = \;!;S_i'$ and $S_i = \;?;S_i'$ are similar.

  In the cases for choice, soundness of the new subgoals follows from
  the definition of~$\TypeSim$.

  The case for the last rule follows from
  Lemma~\ref{lem:unfold-type-sim}.
\end{proof}

\begin{lemma}
  \label{lem:alg-not-false}
  If $S_1 \TypeSim S_2$ then the type equivalence algorithm does not
  return false when applied to $\cdot \vdash S_1 \equiv S_2$.
\end{lemma}
%
\begin{proof}
  Consider all the subgoals produced by the algorithm when given
  $\cdot\vdash S_1\equiv S_2$. From the hypothesis we know that the
  initial goal is sound; by Lemma~\ref{lem:alg-subgoals} all of the
  generated subgoals are sound.  By the same lemma, the algorithm
  either proceeds or returns true.
\end{proof}

\begin{theorem}[Soundness of algorithmic type equivalence]
  \label{thm:alg-soundness}
  If $S_1 \TypeSim S_2$ then $\cdot \vdash S_1 \equiv S_2$  
\end{theorem}
%
\begin{proof}
  By Lemma~\ref{lem:alg-terminates} the algorithm terminates. By
  Lemma~\ref{lem:alg-not-false} the algorithm does not return
  false. Therefore it must return true.
\end{proof}

Now for completeness.

\begin{lemma}
  \label{lem:unfold-preserves-alg-equiv}
  If $\cdot\vdash S_1 \equiv S_2$ then
  $\cdot\vdash \Unfold(S_1) \equiv \Unfold(S_2)$.
\end{lemma}

\begin{theorem}[Completeness of algorithmic type equivalence]
  If $\cdot\vdash S_1 \equiv S_2$ then $S_1 \TypeSim S_2$.
\end{theorem}
%
\begin{proof}
  By Lemma~\ref{lem:unfold-preserves-alg-equiv} it is sufficient to
  consider the case when $S_1$ and $S_2$ are guarded types. We show
  that
  %
  \begin{equation*}
    R = \{(S_1,S_2) \mid \cdot\vdash S_1 \equiv S_2 \text{ and } S_1
    \text{ and } S_2 \text{ are guarded}\}
  \end{equation*}
  %
  is a type simulation.

  Consider case 6 (external choice, $\&$), and assume
  $(S_1,S_2) \in R$ and
  $\Unfold(S_1) = \;\&\{l_i\colon S_i\}_{i\in I}$. Since $S_1$ is
  guarded we have $S_1=\;\&\{l_i\colon S_i\}_{i\in I}$. The only rule
  in the algorithm that applies is the \&-rule, and we get that
  $S_2 = \;\&\{l_j\colon S'_j\}_{j\in J}$. The rule ensures that $I=J$
  and that $\cdot\vdash S_i \equiv S'_i$, for all $i \in I$. Then
  Lemma~\ref{lem:unfold-preserves-alg-equiv} ensures that
  $\cdot\vdash \Unfold(S_i) \equiv \Unfold(S'_i)$, and
  Lemma~\ref{lem:unfold-yields-guarded-types} that $S_i$ and $S'_i$
  are guarded, hence $(S_i',S_i') \in R$ as required.  
\end{proof}

\begin{corollary}
  $\cdot \vdash S_1 \equiv S_2$ if and only if $\S_1 \TypeSim S_2$.
\end{corollary}

%%% Local Variables:
%%% mode: latex
%%% TeX-master: "main"
%%% End:
