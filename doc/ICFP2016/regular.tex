\documentclass[11pt]{article}
\usepackage{amsmath}
\usepackage{stmaryrd}

\newcommand\Sem[1]{\llbracket#1\rrbracket}
\newcommand\LFP{\ensuremath{\mathsf{lfp}}}
\newcommand\SUB{\ensuremath{\mathsf{sub}}}
\newcommand\FREE{\ensuremath{\mathsf{free}}}

\begin{document}
Define $\mu$-regular expressions, $a\in T$ a terminal symbol
\begin{displaymath}
  r,s ::= 0 \mid 1 \mid a \mid r+s \mid r.s \mid \mu x.r \mid x
\end{displaymath}
Semantics of a $\mu$-regular expression.
\begin{align*}
  \Sem{0}\rho &= \emptyset\\
  \Sem{1}\rho &= \{ \varepsilon \}\\
  \Sem{a}\rho &= \{ a \} \\
  \Sem{r+s}\rho &= \Sem{r}\rho \cup \Sem{s}\rho \\
  \Sem{r.s}\rho &= \Sem{r}\rho \cdot \Sem{s}\rho \\
  \Sem{\mu x. r}\rho &= \LFP\, \lambda X. \Sem{r}\rho[x \mapsto X]\\
  \Sem{x}\rho &=\rho(x) 
\end{align*}
Alternative semantics that relies on substitution of symbols by languages.
\begin{align*}
  \Sem{0} &= \emptyset\\
  \Sem{1} &= \{ \varepsilon \}\\
  \Sem{a} &= \{ a \} \\
  \Sem{r+s} &= \Sem{r} \cup \Sem{s} \\
  \Sem{r.s} &= \Sem{r} \cdot \Sem{s} \\
  \Sem{\mu x. r} &= \LFP\, \lambda X. \SUB(x, X, \Sem{r})\\
  \Sem{x}\rho &= \{ x\}
\end{align*}

Define a grammar $G= (N, T, P, S)$ from a $\mu$-regular expression $r$.
Assume that each variable $x$ is bound at most once by a $\mu$ and that $r$ is closed.
\begin{itemize}
\item $N = \{ [s] \mid s \text{ subterm of }r \}$
\item $S = [r]$
\item The set of productions $P$ is given as follows
  \begin{itemize}
  \item no production for nonterminal $[0]$
  \item $[1] \to \varepsilon$
  \item $[a] \to a$, $a\in T$
  \item $[r+s] \to [r] \mid [s]$
  \item $[r.s] \to [r][s]$
  \item $[\mu x.r] \to [r]$
  \item $[x] \to [r]$ where $\mu x.r$ is the unique binding for $x$
  \end{itemize}
\end{itemize}

Let $V = \{ [x] \mid x \in \FREE (s) \}$.
Show that $L (G, [s]) = \Sem{s}$ where we shift $V$ from $N$ to $T$.

Induction on $s$.

\textbf{Case }$0$: $L (G, [0]) = \emptyset = \Sem{0}$.

\textbf{Case }$1$: $L (G, [1]) = \{\varepsilon \} = \Sem{1}$.

\textbf{Case }$a$: $L (G, [a]) = \{ a \} = \Sem{a}$.

\textbf{Case }$r+s$: $L (G, [r+s]) = L (G, [r]) \cup L (G, [s]) = \Sem{r} \cup \Sem{s} = \Sem{r+s}$.

\textbf{Case }$r.s$: $L (G, [r.s]) = L (G, [r]) \cdot L (G, [s]) = \Sem{r} \cdot \Sem{s} = \Sem{r.s}$.

\textbf{Case }$x$: $L (G_V, [x]) = \{ x \} = \Sem{x}$.

\textbf{Case }$\mu x.r$: Let $W = V \cup \{x\}$
\begin{align*}
  L (G_V, [\mu x.r])
  & = L (G_V, [r])\\
  & = \SUB (x, L (G_V, [\mu x.r]), L (G_W, [r])) \\
  & ??? \\
  &= \SUB(x,\Sem{\mu x.r}, \Sem{r}) \\
  &= \SUB(x, \LFP\, \lambda X. \SUB(x, X, \Sem{r}), \Sem{r}) \\
  &= \LFP\, \lambda X. \SUB(x, X, \Sem{r}) \\
  &=  \Sem{\mu x. r}
\end{align*}
Fixpoint induction
\begin{align*}
  \LFP\, \lambda X. \SUB(x, X, \Sem{r}) &\subseteq   L (G_V, [\mu x.r])\\
                                        & \text{if} \\
  \SUB (x, L (G_V, [\mu x.r]), \Sem{r}) & \subseteq L (G_V, [\mu x.r]) \\
                                        &= L (G_V, [r])\\
                                        & = \SUB (x, L (G_V, [\mu x.r]), L (G_W, [r])) \\
                                        & \stackrel{IH}{=} \SUB (x, L (G_V, [\mu x.r]), \Sem{r})
\end{align*}

\clearpage{}
A different approach. Consider a context-free language defined by a system of equations with $\mu$-regular expressions on the right hand sides. Its semantics relies on the environment semantics for expressions.
\begin{align*}
  \mathcal{G} &= \{ x_i = r_i \mid 1\le i\le n \} \\
  \Sem{\mathcal{G}} &= \LFP\ \lambda \rho. [x_i \mapsto \Sem{r_j}\rho]
\end{align*}

Now suppose that $r_j = R[\mu x.r_0]$ where $R$ is a $\mu$-free context and (wlog) $x=x_0$ is different from all $a_i$, $1\le i\le n$. If we introduce a new equation $x_0 = \mu x_0. r_0$, this addition does not change the languages generated in the variables $x_1, \dots, x_n$ because their right hand sides have no free occurrences of $x_0$.

The theorem of Bekic-Leszczylowski is applicable to the augmented system of equations because $\mu$ is interpreted as the least fixpoint. Its application results in a grammar $\mathcal{G}'$ defined by
\begin{align*}
  x_0 & = r_0 \\
  x_1 & = r_1 \\
      & \vdots \\
  x_j & = R[x_0] \\
      & \vdots \\
  x_n & = r_n
\end{align*}
By Bekic-Leszczylowski, $\Sem{\mathcal{G}} (x_i) = \Sem{\mathcal{G}'} (x_i)$, for $1\le i \le n$.

By induction on the number of $\mu$-operators in the original system of equations, we obtain a system which is equivalent to the original with respect to projection on $x_1, \dots, x_n$ and where the right hand sides contain no $\mu$-operators, i.e., a context-free grammar. 

\end{document}
