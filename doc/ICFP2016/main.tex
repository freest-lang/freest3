%\documentclass[preprint]{sigplanconf}
\documentclass{sigplanconf}

% ICFP 2016:
% By Wednesday, March 16 2016, 15:00 (UTC), submit a full paper of at
% most 12 pages (6 pages for an Experience Report), in standard
% SIGPLAN conference format, including figures but excluding
% bibliography. 

% The following \documentclass options may be useful:

% preprint      Remove this option only once the paper is in final form.
% 10pt          To set in 10-point type instead of 9-point.
% 11pt          To set in 11-point type instead of 9-point.
% authoryear    To obtain author/year citation style instead of numeric.

\usepackage{amsmath,amssymb,amsthm}
\usepackage{stmaryrd}
\usepackage{listings,color}
\usepackage{alltt}
\usepackage{flushend}
\usepackage[utf8]{inputenc} % for proper diacritics: Universität, Hüttel, Luís
\usepackage[T1]{fontenc}

%%% Local Variables:
%%% mode: latex
%%% TeX-master: "cfst-inforum18"
%%% End:
      
 
% THEME
\newtcolorbox{mybox}{colback=orange!5!white,colframe=orange!75!black}
\newtcolorbox{myboxazul}{colback=teal!5!white,colframe=teal!75!black}

% TIKZ 
\usetikzlibrary{positioning}
\usetikzlibrary{shapes,arrows}
\usepgfplotslibrary{dateplot}
\tikzstyle{block} = [rectangle, draw, 
    text width=5cm, text centered, rounded corners, minimum height=4em]
\tikzstyle{block2} = [rectangle, draw, 
    text width=10cm, text centered, rounded corners, minimum height=4em]
\tikzstyle{line} = [draw, -latex']

\newenvironment<>{varblock}[2][.9\textwidth]{%
  \setlength{\textwidth}{#1}
  \begin{actionenv}#3%
    \def\insertblocktitle{#2}%
    \par%
    \usebeamertemplate{block begin}}
  {\par%
    \usebeamertemplate{block end}%
  \end{actionenv}}

\newenvironment{changemargin}[3]{%
\begin{list}{}{%
\setlength{\leftmargin}{#1}%
\setlength{\rightmargin}{#2}%
\setlength{\topmargin}{#3}%
}%
\item[]}
{\end{list}}

% session constructors
\newcommand{\intk}{\keyword{int}}
\newcommand{\skipk}{\keyword{skip}}

% The language
\newcommand{\freest}{\textsc{FreeST}}

% notes
\newcommand{\todo}[1]{[{\color{blue}\textbf{#1}}]}

% Keywords
\newcommand{\keyword}[1]{\mathsf{#1}}
\newcommand{\link}{\keyword{lin}}
\newcommand{\unk}{\keyword{un}}

% Kinds
\newcommand\prekind{\upsilon}
\newcommand{\stypes}{\mathcal S}
\newcommand\kinds{\stypes}
\newcommand{\types}{\mathcal T}
\newcommand\kindt{\types}
\newcommand\kindsch{\mathcal C}
\newcommand\kind{\kappa}

% Multiplicity
\newcommand\Un{\ensuremath{\mathbf{u}}} % \infty
\newcommand\Lin{\ensuremath{\mathbf{l}}} % 1 

% Grammars
\newcommand{\grmeq}{\; ::= \;}
\newcommand{\grmor}{\;\mid\;}

% type constructors
\newcommand\tcBase{B}
\newcommand\tcLolli\multimap
\newcommand\tcFun\to
\newcommand\tcBang{\mathop!}

% Keywords for types
\newcommand\kRec{\keyword{rec}}
\newcommand\kForall{\keyword{forall}}

% Types
\newcommand{\tskip}{\keyword{Skip}}
\newcommand\tSemi[2]{#1;#2}
\newcommand\tOut[1]{\tcBang#1}
\newcommand\tIn[1]{?#1}
\newcommand{\tMsg}[1]{\sharp{#1}}
\newcommand\tIChoice[1]{\oplus{#1}}
\newcommand\tEChoice[1]{\&{#1}}
\newcommand{\tChoice}[1]{\star{#1}}
\newcommand{\tData}[1]{[{#1}]}
\newcommand\tUnFun[2]{#1\tcFun#2}
\newcommand\tLinFun[2]{#1\tcLolli#2}
\newcommand\tPair[2]{(#1,\,#2)}
\newcommand\tDatatype[1]{{[#1]}}
\newcommand\tRec[2]{\kRec\,#1\,.\,#2}
%\newcommand\tForall[2]{\forall\,#1\,.\,#2}
%\newcommand\tForall[2]{\kForall\,#1\,=>\,#2}
\newcommand\tForall[2]{\forall\,#1\Rightarrow#2}
% Basic Types
\newcommand{\unite}{()}
\newcommand{\inte}{\keyword{Int}}
\newcommand{\chare}{\keyword{Char}}
\newcommand{\boole}{\keyword{Bool}}

\newcommand\tRecK[2]{\kRec\,#1\,.\,#2}
% Environments
\newcommand{\Empty}{\varepsilon}
\newcommand\emptyEnv{\Empty}
\newcommand\kindEnv{\Delta}
\newcommand\varEnv{\Gamma}

% Variables
\newcommand\vare[1]{#1}
\newcommand\unlete[3]{\keyword{let} \; #1 = #2 \; \keyword{in} \; #3} 

% Applications
\newcommand\appe[2]{#1#2}
\newcommand\tappe[2]{#1[#2]}

% Conditional
\newcommand\conditionale[3]{\keyword{if}\;#1\;\keyword{then}\;#2\;\keyword{else} \; #3}

% Goal
\newcommand\Goal{\vdash}

% Pairs
\newcommand\paire[2]{(#1,#2)}
\newcommand\binlete[4]{\keyword{let}\;#1, #2 = #3\;\keyword{in}\;#4}

% Session Types
\newcommand\newe[1]{\keyword{new}\;#1}
\newcommand\sende[2]{\keyword{send}\;#1\; #2}
\newcommand\sendce[1]{\keyword{send}\;#1}
\newcommand\recve[1]{\keyword{receive}\;#1}
\newcommand\selecte[2]{\keyword{select}\;#1\;{#2}}
\newcommand\matche[2]{\keyword{match}\;#1\;\keyword{with}\;#2}

% Fork
\newcommand\forke[1]{\keyword{fork}\;#1}

% Datatypes
\newcommand{\ctrcte}{C}
\newcommand\casee[2]{\keyword{case}\;#1\;\keyword{of}\;#2}

% Sequents
\newcommand{\isType}[3][\Delta]{{#1} \vdash {#2} : {#3}}
\newcommand{\algkindout}[3][\kindEnv]{{#1} \Alg {#2} \shortrightarrow{ #3}}
\newcommand{\algkindin}[3][\kindEnv]{{#1} \Alg {#2} \shortleftarrow {#3}}
\newcommand{\subkind}[2]{{#1} <: {#2}}
\newcommand{\algtypeout}[4][\kindEnv;\varEnv]{{#1} \Alg {#2} \shortrightarrow {#3};{#4}}
%\newcommand{\algtypein}[4][\kindEnv;\varEnv]{{#1} \Alg {#2}\colon {#3}\shortrightarrow {#4}}
\newcommand{\algtypein}[4][\kindEnv;\varEnv]{{#1} \Alg {#2}\shortleftarrow {#3}; {#4}}
\newcommand{\ctxequiv}[3][\kindEnv]{{#1} \vdash \Equiv{#2}{#3}}
\newcommand{\typeequiv}[3][\kindEnv]{{#1} \vdash \Equiv{#2}{#3}}
\newcommand{\isqualifier}[3][\kindEnv]{{#1} \vdash {#2}\colon{#3}}
\newcommand{\isLin}[2][\kindEnv]{\isqualifier[#1]{#2}\link}
\newcommand{\isUn}[2][\kindEnv]{\isqualifier[#1]{#2}\unk}
\newcommand{\contractive}[2][\kindEnv]{{#1} \vdash_{\textsf c} {#2}}
%\newcommand\Alg{\vdash_{\textsf a}}
\newcommand\Alg{\vdash}

% Operators
\newcommand\Extract[1]{\leadsto_{#1}}% \rightlsquigarrow}
\newcommand{\subs}[3]{[{#1}/{#2}]{#3}}
\newcommand\dual[1]{\overline{#1}}

% Predicates
%\newcommand\Equiv[2]{#1\,\thicksim\,#2}
\newcommand\Equiv[2]{#1\,\sim\,#2}

% Colour

\newcommand{\Blue}[1]{\textcolor{blue}{#1}}
\newcommand{\Red}[1]{\textcolor{red}{#1}}
\newcommand{\Brown}[1]{\textcolor{brown}{#1}}
\newcommand{\highlight}[1]{\Blue{#1}}

% ECLIPSE LOOK

\newcommand\Small{\small}
%\newcommand\Small{\fontsize{7.5}{8}\selectfont} 

\definecolor{darkviolet}{rgb}{0.5,0,0.4}
\definecolor{darkgreen}{rgb}{0,0.4,0.2} 
\definecolor{darkblue}{rgb}{0.1,0.1,0.9}
\definecolor{darkgrey}{rgb}{0.5,0.5,0.5}
\definecolor{lightblue}{rgb}{0.4,0.4,1}

\lstdefinestyle{eclipse}{
  breaklines=true,
  basicstyle=\sffamily\Small,
  emphstyle=\color{red}\bfseries, 
  keywordstyle=\color{darkviolet}\bfseries,
  commentstyle=\color{darkgreen},
  stringstyle=\color{darkblue},
  numberstyle=\color{darkgrey},%\lstfontfamily,
  emphstyle=\color{red},
  % get also javadoc style comments
  morecomment=[s][\color{lightblue}]{/**}{*/},
  %columns=fullflexible, %spaceflexible, %flexible, fullflexible             
  %  escapeinside=`',
  %  escapechar=@,
  showstringspaces=false,
  numbers=left,
  tabsize=2
}

\lstdefinestyle{eclipse-Haskell}{
  breaklines=true,
  basicstyle=\sffamily\Small,
  emphstyle=\color{red}\bfseries, 
  keywordstyle=\color{darkviolet}\bfseries,
  commentstyle=\color{darkgreen},
  stringstyle=\color{darkblue},
  emphstyle=\color{red},
  % get also javadoc style comments
  morecomment=[s][\color{lightblue}]{/**}{*/},
  %columns=fullflexible, %spaceflexible, %flexible, fullflexible             
  %  escapeinside=`',
  %  escapechar=@,
  showstringspaces=false,
  numbers=none,
  tabsize=2
}

\lstdefinelanguage{freest}{
  style=eclipse,
  morekeywords=[1]{Int, Char, Bool, Skip, type, dualof, forall, rec, let, in, if, then, else, new, send, receive, select, fork, case, of, data, match, with, True, False},
  sensitive=true,
  literate=
  {->}{$\rightarrow$}2
  {-o}{$\multimap$}2
  {=>}{$\Rightarrow$}2
  {alpha}{$\alpha$}1,
  breaklines=true,
  morecomment=[l]{--},%
  morecomment=[s]{{-}{-}},%
  morestring=[b]',%
  morestring=[b]",%
  morestring=[s]{`}{`},%
}

\lstset{
  language=freest,
  numbers=none
}
 
%%% Local Variables:
%%% mode: latex
%%% TeX-master: "main"
%%% End:

%%% for the draft
% \renewcommand\vv[1]{}
% \renewcommand\pt[1]{}

\begin{document}
\toappear{}
% \special{papersize=8.5in,11in}
% \setlength{\pdfpageheight}{\paperheight}
% \setlength{\pdfpagewidth}{\paperwidth}

% \conferenceinfo{ICFP'16}{September 19--21, 2016, Nara, Japan} 
% \copyrightyear{2016} 
% \copyrightdata{978-1-nnnn-nnnn-n/yy/mm} 
% \doi{nnnnnnn.nnnnnnn}

% Uncomment one of the following two, if you are not going for the 
% traditional copyright transfer agreement.

%\exclusivelicense                % ACM gets exclusive license to publish, 
                                  % you retain copyright

%\permissiontopublish             % ACM gets nonexclusive license to publish
                                  % (paid open-access papers, 
                                  % short abstracts)

%\titlebanner{banner above paper title}        % These are ignored unless
%\preprintfooter{short description of paper}   % 'preprint' option specified.

\title{Context-Free Session Types}%Towards
%\subtitle{Subtitle Text, if any}

\authorinfo{Peter Thiemann}
           {Universität Freiburg, Germany}
           {thiemann@acm.org}
\authorinfo{Vasco T. Vasconcelos}
           {LaSIGE and University of Lisbon, Portugal}
           {vmvasconcelos@ciencias.ulisboa.pt}

\maketitle

\begin{abstract}
Session types describe structured communication on heterogeneously
typed channels at a high level.
Their tail-recursive structure imposes a protocol that can be
described by a regular language. 
The types of transmitted values are drawn from the underlying
functional language, abstracting from the
details of serializing values of structured data types.

Context-free session types extend session types by allowing nested
protocols that are not restricted to tail recursion. Nested protocols
correspond to deterministic context-free languages. Such protocols are
interesting in their own right, but they are particularly suited to
describe the low-level serialization of tree-structured data in a
type-safe way.  

We establish the metatheory of context-free session types, prove that
they properly generalize standard (two-party) session
types, and take first steps towards type checking by showing
that type equivalence is decidable.
\end{abstract}
% well we  (hope to) achieve the last two items

\category{D.3.3}{Language Constructs and Features}{Concurrent programming structures}
\category{D.3.1}{Formal Definitions and Theory}{}

% general terms are not compulsory anymore, 
% you may leave them out
% \terms
% term1, term2

\keywords
session types, semantics, type checking

\pagestyle{plain}

%  HOW DO WE ORGANISE THE PAPER?

% * 1 - 1.5 pages introduction, perhaps giving some small example already
% * >=2 pages motivation discussing a couple of progressively more complicated examples that highlight that problems and features
% --- do you think a running example is needed? I suppose that technical points can be explained with mini examples made on the spot or by reference to one of the examples from the motivation
% * Then defining formalities
% ** introducing types and type equiv
% ** expressions, statics, and dynamics
% * Then properties (perhaps better interleaved with the formalities)
% ** of types and equivalence
% ** type soundness
% * Then type checking
% ** specification
% ** decidability
% * Conservative extension
% * Related work
% * Conclusion

\section{Introduction}

Session types have been long subject to the shackles of tail
recursion. Regular session-type languages bear the evident advantage
of providing for simple algorithms to check type equivalence and
subtyping. Given two types, a fixed-point construction algorithm such
as the one introduced by Gay and Hole builds, in polynomial time and
space, a bisimulation relating the two types, or decides that no such
relation exists~\cite{DBLP:journals/acta/GayH05}. The scenario darkens
when one decides to let go of tail recursion, for now the fixed-point
construction algorithm does not necessarily terminate.
%
This is one of the main reasons why session types have been confined
to ($\omega$-) regular languages for so many years.

The discipline of conventional (that is, regular) session types
provides guarantees not easily accessible to simpler languages such as
concurrent ML, where channels are unidirectional and transport values
of a fixed size~\cite{DBLP:conf/mcmaster/Reppy93}. Session types, in
turn, provide for the description of richer protocols, epitomised by
the math server~\cite{DBLP:journals/acta/GayH05}, which can be
rendered in the SePi language~\cite{DBLP:conf/sefm/FrancoV13} as
follows:
%
\begin{lstlisting}[morekeywords=end]
MathServer = +{
  Plus: !Int.!Int.?Int.MathServer,
  Gz: !Int.?Bool.MathServer,
  Done: end
}
\end{lstlisting}

Type \lstinline|MathServer| describes the client side of the protocol,
introducing three choices: \lstinline|Plus|, \lstinline|Gz|, and
\lstinline|Done|. A client that chooses the \lstinline|Plus| choice is
supposed to send two integer values and to receive a further integer
(possibly representing the sum of the former two), after which it goes
back to the beginning. The \lstinline|Done| option terminates the
protocol, as described by type \lstinline[morekeywords=end]|end|.

The guarantees introduced by session type systems include the
adherence of the code to the protocol and the related absence of
runtime errors, including race
conditions~\cite{DBLP:conf/esop/HondaVK98}. Some systems further
guarantee progress~\cite{DBLP:conf/concur/CairesP10}. All this, under
a rather expressive type language, that of (regular) session types.

There is one further characteristic of session types that attest for
its flexibility: the ability send channels on channels, often called
delegation. This feature provides for the transmission of complex data
on channels in a typeful manner. Suppose we want to stream a tree
%
\begin{lstlisting}
data Tree = Leaf | Node Int Tree Tree
\end{lstlisting}
%
on a channel. One has to choose between using multiple channels for
sending the tree or incurring on runtime checks to check adherence to
the protocol. In the former scenario, trees are sent on channels of
the form
%
\begin{lstlisting}[morekeywords=end]
type TreeChannel = +{Leaf: end, Node: !Int.!TreeChannel.!TreeChannel.end}
\end{lstlisting}
%
and we see that two new channels must be created and exchanged for
each \lstinline|Node| in a tree, so that $2n+1$ channels are needed to
stream an $n$-\lstinline|Node| tree.
%
In the latter case, tree parts are sent on a single stream, but not
necessarily in a ``tree'' form. A suitable channel type is
%
\begin{lstlisting}[morekeywords=end]
type TreeParts = +{Leaf: TreeParts, Node: !Int.TreeParts, EOS: end}
\end{lstlisting}
%
where \lstinline|EOS| represents the end of stream. In this case tree
\lstinline|Node 1 (Node 2 Leaf Leaf) Leaf| can be streamed as
\lstinline|Node 1 Node 2 Leaf Leaf Leaf EOS|, when visited in a
depth-first manner. It should be easy to see that type
\lstinline|TreeParts| allows streaming many different tree parts that
do not add up to a tree, hence the necessary runtime checks to look
over unexpected parts in the stream.

In 2016, Thiemann and Vasconcelos introduced the concept of
context-free session types and proved type equivalence
decidable~\cite{DBLP:conf/icfp/ThiemannV16}.
%
Context-free session types appear as a natural extension of
conventional (regular) session-types. Syntactically, the changes are
minor: rather than dot (\lstinline|.|), the prefix operator, we use
semi-colon (\lstinline|;|), a new binary operator on types. We also
take the chance to replace \lstinline [morekeywords=end]|end| by
\lstinline|Skip| to make it clear that it does not necessarily
``ends'' a session type, but else merely introduces a mark that can
sometimes be omitted (\lstinline|Skip| is the identity element of the
semi-colon operation).
%
Using the syntax of \freest, a channel that streams a tree can be
written as follows:
%
\begin{lstlisting}
type TreeChannel = +{Leaf: Skip, Node: !Int;TreeChannel;TreeChannel}
\end{lstlisting}

The language \freest, described in the sequel, provides for the best
of both worlds: stream the tree in a \emph{single} channel,
\emph{without} extraneous runtime checks.

% Related work

There are a few experimental programming languages based on session
types, and there are many proposals for encoding session types in
mainstream programming languages. We cannot possible cover them all in
this short abstract; the interested reader is referred to a 2016
survey on behavioural types in programming
languages~\cite{DBLP:journals/ftpl/AnconaBB0CDGGGH16}.  Here, we briefly
discuss prototypical programming languages.

SePi is a programming language based on the pi-calculus whose channels
are governed by regular session types refined with uninterpreted
predicates~\cite{DBLP:conf/sefm/FrancoV13}. The concrete syntax we
choose for \freest{} is heavily influenced by SePi.

The only other implementation of context-free session types we are
aware of is that of Padovani~\cite{DBLP:conf/esop/Padovani17}, a
language that admits equivalence, subtyping and, of particular
interest, inference algorithms. It does however require a structural
alignment between the process code and the session types, enforced by
a \emph{resumption} process operator that explicitly breaks a type
$S;T$. This implies that the monoidal rules proposed by Thiemann and
Vasconcelos are not taken into consideration.

%%% Local Variables:
%%% mode: latex
%%% TeX-master: "main"
%%% End:

%%%%%%%%%%%%%%%%%%%%%%%%%%%%%%%%%%%%%%%%%%%%%%%%%%%%%%%%%%%%%%%%%%%%%%%%%%%%%%%%

\section{Context-Free Session Types in Action}
\label{sec:context-free-session}

To understand the requirements for the metatheory of
context-free session types, we first examine the type derivation of
\lstinline|sendTree| in Listing~\ref{listing:serializing}. Then we
turn to further examples that underline the expressiveness and the
usefulness of context-free session types.

\subsection{Sending Leaves}
\label{sec:sending-leaves}


To typecheck the first alternative of the \lstinline|sendTree|
function for sending leaves, we need to derive type \lstinline|alpha| for the code
fragment
\begin{lstlisting}
  select Leaf c
\end{lstlisting}
given that \lstinline{c : TreeChannel;alpha}. Anticipating the formal
definition in Section~\ref{sec:processes} (Figure~\ref{fig:typing}),
we sketch an informal typing rule for \lstinline|select|, which is
taken verbatim from GV~\cite{DBLP:journals/jfp/GayV10}:
\begin{equation}\label{eq:2}
    \frac{
      \vdash e : \oplus\{l_i\colon S_i\}_{i\in I}
      \quad
      j\in I
      % \quad
      % \cdot\vdash T_i :: \kinds^m
    }{
      \vdash \select {l_j} e \colon S_j
    }
\end{equation}
 The \lstinline|select|
operation expects a branch type $\oplus\{l_i\colon S_i\}$, but we are given the recursive
\lstinline|TreeChannel| type, which has to be unfolded first. Such
unfolding is to be expected in the 
presence of recursive types. As unfolding is not indicated in the term,
we require an \emph{equi-recursive treatment of recursion in types}~\cite{Pierce2002-tpl}.


After unfolding, we obtain
\begin{lstlisting}
  c : oplus{Leaf: skip,
        Node: !int;TreeChannel;TreeChannel};alpha
\end{lstlisting}
This type, a sequence of protocols, is still not in the form expected
by \lstinline|select|. Hence, we further
need to enrich type equivalence to enable us to \emph{commute the
continuation type \lstinline|alpha| inside the branches}.

After commutation, we obtain the typing
\begin{lstlisting}
  c : oplus{Leaf: skip;alpha,
        Node: !int;TreeChannel;TreeChannel;alpha}
\end{lstlisting}
which is finally in a form acceptable to \lstinline|select|. Applying
the typing rule~\eqref{eq:2} yields
\begin{lstlisting}
  select Leaf c : skip;alpha
\end{lstlisting}
At this point, we need to apply the \emph{monoid identity law} (which
also needs to be part of type equivalence) to obtain the desired
outcome.
\begin{lstlisting}
  select Leaf c : alpha
\end{lstlisting}


\subsection{Sending Nodes}
\label{sec:sending-nodes}

We turn to typechecking the second alternative of the
\lstinline|sendTree| function
\begin{lstlisting}
  let c1 = select Node c
      c2 = send x c1
      c3 = sendTree l c2
      c4 = sendTree r c3
  in  c4
\end{lstlisting}
given that 
\begin{lstlisting}
  x : int, l : Tree, r : Tree, c : TreeChannel
\end{lstlisting}

Typechecking the \lstinline|select| operation requires the same steps
as for leaves. We skip over those and note the resulting typing for
\lstinline|c1|.
\begin{lstlisting}
  c1 : !int;TreeChannel;TreeChannel;alpha
\end{lstlisting}
The \lstinline|send| operation just peels off the leading
\lstinline{!int} type, but our typing for \lstinline|c1| glosses over
an important detail, namely the bracketing of the $\scCompose\_\_$
operator. After commuting \lstinline|alpha| inside the branch type and
applying the \lstinline|select| rule, we are actually left with this
type:
\begin{lstlisting}
  c1 : (!int;(TreeChannel;TreeChannel));alpha
\end{lstlisting}
Again, we need to appeal to type equivalence to reassociate the
nesting of the sequence operator, that is, to apply the \emph{monoidal
associativity law}. The resulting type
\begin{lstlisting}
  c1 : !int;((TreeChannel;TreeChannel);alpha)
\end{lstlisting}
is compatible with the typing for \lstinline|send| and we can
proceed with
\begin{lstlisting}
  c2 : (TreeChannel;TreeChannel);alpha
\end{lstlisting}
Again, we need to reassociate:
\begin{lstlisting}
  c2 : TreeChannel;(TreeChannel;alpha)
\end{lstlisting}
At this point, we see the need for \emph{polymorphic recursion}: the
recursive call \lstinline|sendTree l c2| of
\begin{lstlisting}
sendTree : forallbeta.Tree -> TreeChannel;beta -> beta
\end{lstlisting}
must instantiate the type variable \lstinline|beta| to
\lstinline{(TreeChannel;alpha)}. With this instantiation, we obtain
\begin{lstlisting}
  c3 : TreeChannel;alpha
\end{lstlisting}
The second recursive call instantiates \lstinline|beta| to
\lstinline|alpha| (it could be treated monomorphically) and we readily
obtain the desired final outcome, which is equivalent to the outcome
of the first alternative:
\begin{lstlisting}
  c4 : alpha
\end{lstlisting}
In summary, the type system for context-free session types requires
polymorphism with polymorphic recursion.\footnote{This particular example can
be made to work without polymorphic recursion by abstracting the
recursive calls to \lstinline|transform| in a separate function with a
specialized type. However, we argue that it is advantageous to be able
to type the straightforward code that we present.} Furthermore, it relies on a
nontrivial notion of type equivalence that includes unfolding of
equi-recursive types, distributivity of branching over sequencing, and the
monoidal structure of \lstinline|skip| and sequencing (identity and
associativity laws). Our technical treatment of type equivalence in
Sections~\ref{sec:bisimulation} and~\ref{sec:decidability}
relies on a terminating unraveling operation that normalizes the
``head'' of a session type with respect to these notions. 

\subsection{Structure-Preserving Tree Transformation}
\label{sec:remote-tree-transf}

As another example for the expressiveness of context-free session
types, we present client and server code for a remote structure-preserving tree
transformation in Listing~\ref{listing:remote-tree-transformation}. It is based on the same tree datatype as before, but
it introduces a new channel type \lstinline|XformChan| that
receives the transformed node value after sending the old value and
the two subtrees. This code makes use of the \lstinline|receive|
operation that takes a channel of type \lstinline|!int;alpha| and
returns a linear pair of type \lstinline|intotimesalpha|. The pair
must be linear because channels in session-type calculi generally have
linear types to cater for the change of their type at each operation.
\lstinputlisting[float={t},captionpos={b},caption={Remote tree transformation},label={listing:remote-tree-transformation}]{tree.cfs}

The server function \lstinline|transform| demonstrates the use of
\lstinline|receive|. It also uses pattern matching to deconstruct the
linear pairs returned by recursive calls and by receiving integers. No
new issues arise in typing this function compared to
\lstinline|sendTree|. 

The function \lstinline|treeSum| is a suitable client for transformer
channels. It computes the accumulated sum at each tree node, so that
running \lstinline|transform| and
\lstinline|treeSum| concurrently results in a tree where each node value is
replaced by the sum of all node values below. The function
\lstinline|treeSum| takes an argument channel of type
\lstinline|dualof XformChan; alpha| where the \lstinline|dualof|
operator swaps sending and receiving types as usual. The
\lstinline|case| expression is the receiving counterpart of the
\lstinline|select| expression. It receives a label from a channel and
dispatches according to this label. Each branch of the
\lstinline|case| is a function that takes the respective continuation
of the channel and continues the interaction on that channel.

The final definition of \lstinline|go| stitches it all together. Using
\lstinline{new XformChan} it creates a new pair of channels, the types
of which are \lstinline{XFormChan} and its dual, it forks a new
process that runs \lstinline{treeSum} on the server channel, and
finally runs \lstinline{transform} on an example tree and the client channel.

The example also illustrates how channels are
closed. \lstinline|treeSum| (\lstinline|transform|) returns a
\emph{linear pair} of the accumulated sum (transformed tree) and a
depleted channel of type \lstinline|skip|. The function
\lstinline|fst| eliminates the linear pair and returns its first
component, which is possible because the second component is of an
unrestricted type (\lstinline|skip|). It implicitly closes the channel
by discarding it, as the channel end of type \lstinline|skip| can no longer be used for
interaction.

% which is possible because \lstinline|skip| is no
% longer restricted to be linear.

\subsection{Expression Server}
\label{sec:expression-server}

An example that is quite often used in the literature on session types
is an arithmetic server with a type like the one indicated in the
introduction:
\begin{displaymath}
  \sRecvChoice{
    \cgAlt{\mathit{add}}{\sRecv{\intk}\sRecv{\intk}\sSend{\intk}\End}
    ,
    \cgAlt{\mathit{neg}}{\sRecv{\intk}\sSend{\intk}\End}
  }
\end{displaymath}

\lstinputlisting[float={t},captionpos={b},caption={Arithmetic
  expression
  server},label={listing:arithmetic-expression-server}]{arithmetic-server.cfs}
 
Exploiting context-free session types, we can extend the scope of such
a server to receive and process arbitrary well-formed arithmetic
expressions. As an example consider the arithmetic expression server
for terms composed of constants, addition, and multiplication in
Listing~\ref {listing:arithmetic-expression-server}.
%
The implementation of the protocol is straightforward using the
techniques already described.

It is possible to extend this protocol to lazily traverse the term
(Listing~\ref{listing:lazy}). In this case, the server requests from
the client the parts of the term needed to complete the
evaluation. For instance, if a factor in a multiplication is zero, the
server can avoid to even ask for sending the other factor. We
elucidate this idea with a simplified protocol to explore a binary
tree lazily. No new features are required for its realization.

\lstinputlisting[deletekeywords={Left,Right},float={t},captionpos={b},caption={Lazy
tree traversal},label={listing:lazy}]{exploration.cfs}

A client connecting to the server \lstinline|exploreTree| first
connects to the root node of a tree. First it must check whether the
current node is a \lstinline|Leaf| or a \lstinline|Node|. If it is a
\lstinline|Node|, it can further explore the contents: it can ask for
the value or traverse the left subtree or the right subtree as often
as desired. Finally the client sends \lstinline|Exit| to return to the
parent node. 

The type describing this interaction is mutually recursive. The
``inner loop'' described by \lstinline|XploreNodeChan| is
tail-recursive like a regular session type, but the ``outer loop''
corresponding to \lstinline|XploreTreeChan| is not as its invocations
are intertwined with the inner loop.

% Now with a datatype to simplify the client's life.

% \lstinputlisting{arithmetic-server-data.cfs}


% Let's try a SePi-like version. All we need are context-free session
% types and type schemas\footnote{Predicative Polymorphism in
%   pi-Calculus. Vasco Thudichum Vasconcelos. In 6th Parallel
%   Architectures and Languages Europe, volume 817 of LNCS, pages
%   425--437. Springer, 1994.}. We witness the extra ``plumbing''
% typical of the pi-calculus, but otherwise very little (type only, in
% fact) extra basic notions.

% \lstinputlisting{arithmetic-server-sepi.cfs}


% The code for deserialization is analogous to the one for serialization and it can also be typed with
% the type structure explained in the introduction.
% \begin{alltt}
% recvT : Channel -> Tree \(\otimes\) Channel
% recvT c =
%   case c of
%   | Leaf : lambda c0.
%       let (n, c1) = receive (c0) in
%       (Leaf n, c1)
%   | Node : lambda c0.
%       let (r, c1) = recvT c0 in
%       let (l, c2) = recvT c1 in
%       (Node (l, r), c2)
% \end{alltt}


%%% Local Variables:
%%% mode: latex
%%% TeX-master: "main"
%%% End:

\begin{figure}[h!]
  Types:
    \begin{align*}
      S, T &\grmeq \sharp_i U_i \grmor \tend \grmor \mu x . S \grmor x \\
      U, V &\grmeq l \star S . T\\
    \sharp &\grmeq \echoicet{} \grmor \ichoicet{}
    \end{align*}
\end{figure}


%%% Local Variables:
%%% mode: latex
%%% TeX-master: "main"
%%% End:

\section{Processes, Statics, and Dynamics}
\label{sec:processes}

This section introduces our programming language, its static and
dynamic semantics. It shows that typing is preserved by reduction and
concludes by showing that our type system is a conservative extension
of that of a conventional functional session type language.

%%%%%%%%%%%%%%%%%%%%%%%%%%%%%%%%%%%%%%%%%%%%%%%%%%%%%%%%%%%%%%%%%
\subsection{Expressions and Processes}

\begin{figure}[t]
  \begin{align*}
    v \grmeq& \sendk %&& \text{Values}\\
    \grmor \recvk
    \grmor ()
%    \grmor& \text{(other literals)}\\
    \grmor \lambda a.e
    \grmor (v,v)
    \grmor \inject lv
    \\
    e \grmeq& v %&& \text{Expressions}\\
    \grmor a
    \grmor \newk
    \grmor ee
    \grmor \fix ae
    \grmor (e,e)
    \grmor \ink\,l\,e\\
    \grmor& \letin{a,b} e e
    \grmor \forkk\, e
    \grmor \match e {l_i\rightarrow e_i}_{i\in I}\\
%    \grmor& \letin aee\\
    \grmor& \select le
    \grmor \casek\,e\,\ofk\,\{l_i\rightarrow e_i\}_{i\in I}
    \\
    p \grmeq& e %&& \text{Processes}\\
    \grmor p \PAR p
    \grmor (\nu a,b)p
  \end{align*}
  \caption{Values, expressions, and processes}
  \label{fig:processes}
\end{figure}

%%% Local Variables:
%%% mode: latex
%%% TeX-master: "main"
%%% End:

 
Fix a base set of \emph{term variables}, disjoint from those
introduced before. Let $a,b$ range over this set.
%
The syntax of values, expressions, and processes is described in
Figure~\ref{fig:processes}.

\emph{Expressions}, denoted by metavariable $e$, incorporate a
\emph{standard functional core} composed of term variables~$a$,
abstraction introduction $\lambda a.e$ and elimination $ee$, pair
introduction $(e,e)$ and elimination $\letin{a,b} e e$, datatype
introduction $\inject le$ and elimination
$\match e {l_i\rightarrow e_i}_{i\in I}$, as well as a fixed point
construction $\fixk\,e$.
%
Further expressions support the usual \emph{session operators}, in the
form of channel creation $\newk$, message sending $\sendk$ and
receiving $\recvk$, internal choice (or label selection) $\select le$,
and external choice (or branching)
$\case e{\{l_i\rightarrow e_i\}_{i\in I}}$.
%
Concurrency arises from a $\forkk$ operator, spawning
a new process.
%
% In expressions $\match e {l_i\rightarrow e_i}_{i\in I}$ and
% $\casek\,e\,\ofk\,\{l_i\rightarrow e_i\}_{i\in I}$ $I$ is an index set.

\emph{Processes}, denoted by metavariable $p$, are expressions~$e$,
the parallel composition of two processes~$p\PAR q$, and the scope
restriction~$(\nu a,b)p$ of a channel described by its two end
points,~$a$ and~$b$.

%%%%%%%%%%%%%%%%%%%%%%%%%%%%%%%%%%%%%%%%%%%%%%%%%%%%%%%%%%%%%%%%%
\subsection{Operational Semantics}
\label{sec:semantics}

% Bindings, free vars, substitution

The \emph{binding} occurrences for term variables~$a$ and~$b$ are
expressions $\lambda a.e$, $\fixk\,a.e$, and $\letin{a,b}{e_1}{e_2}$.
%, and $\case e{\{l_i\rightarrow e_i\}_{i\in I}}$.
The sets of free and bound
term variables are defined accordingly, and so is the
\emph{capture avoiding substitution} of a variable~$a$ by a value~$v$ in a term~$e$,
denoted by $e\subs va$.

The operational semantics makes use of a \emph{structural congruence}
relation on processes, $\equiv$, defined as the smallest congruence
relation that includes the commutative monoidal rules---binary
operator $\_\PAR\_$ and any value for neutral---and scope extrusion:
%
\begin{equation*}
  (\nu a,b)p \PAR q \equiv (\nu a,b)(p\PAR q) \quad\text{if $a,b$ not free
    in $q$}
\end{equation*}

% op semantics

The operational semantics is call-by-value: expressions are reduced to
values before being ``applied''. The syntax of values is in
Figure~\ref{fig:processes}, and includes the values $\sendk$,
$\recvk$, unit, lambda abstraction, pair of values, and injection of a
value in a sum type. The semantics combines a standard reduction
relation for the functional part with an also standard message passing
semantics of the $\pi$-calculus.
%
The rules are in Figure~\ref{fig:reduction}.

\begin{figure}[t]
  \begin{gather*}
    % FUNCTIONAL
    (\lambda a.e)v \reduces e\subs va
    \qquad
    \letin{a,b}{(u,v)}{e} \reduces e\subs ua\subs vb
    \\
    \match{(\ink\,l_j\,v)}{l_i\rightarrow e_i}  \reduces e_jv
    \qquad
    \fix ae \reduces e \subs{\fix ae}{a}
    \\
    \frac{e_1 \reduces e_2}{E[e_1] \reduces E[e_2]}
    \qquad
    E[\fork e] \reduces E[()] \PAR e
    \\
    E[\newk] \reduces (\nu a,b)E[(a,b)]
    \\ % PROCESSES
    (\nu a,b)(E_1[\send va] \PAR E_2[\recv b])
    \reduces
    (\nu a,b)(E_1[a] \PAR E_2[(v,b)])
    \\
    (\nu a,b)(E_1[\selectk\,l_j\,a] \PAR E_2[\casek\,b\,\ofk\,\{l_i\rightarrow e_i\}])
    \reduces  \qquad \qquad  \qquad\\
    \qquad \qquad \qquad \qquad \qquad \qquad \qquad \qquad (\nu a,b)(E_1[a] \PAR E_2[e_jb])
    \\
    \frac{p \reduces p'}{p\PAR q \reduces p'\PAR q}
    \quad
    \frac{p \reduces p'}{(\new a,b)p \reduces (\new a,b)p'}
    \quad
    \frac{p \equiv q\quad q \reduces q'}{p \reduces q'}
  \end{gather*}
  Context $E_1$ (resp.~$E_2$, resp.~$E$) does not bind~$a$ (resp.~$b$,
  resp.~$a$ and~$b$).
  \\
  Dual $(\nu b,a)$ rules for $\sendk/\recvk$ and $\selectk/\casek$
  omitted.
  \caption{Reduction relation}
  \label{fig:reduction}
\end{figure}

%%% Local Variables:
%%% mode: latex
%%% TeX-master: "main"
%%% End:


The first four axioms are standard in functional call-by-value languages,
and comprise $\beta$-reduction, (linear) pair elimination, data type
elimination, and fixed-point unrolling.
%
The first rule in the figure allows reduction to happen underneath
(functional) evaluation contexts $E$ as defined by the grammar
below.
%
\begin{align*}
  E \grmeq& [] \grmor (E,e) \grmor (v,E) \grmor Ee \grmor vE \grmor \letin{a,b}Ee
  % \letin aEe \grmor
  \\
  \grmor& \casek\,E\,\ofk\,\{l_i\rightarrow e_i\} \grmor \selectk\,l\,E
  \\
  \grmor& \match{E}{l_i\rightarrow e_i} \grmor \ink\,l\,E
\end{align*}

The $\forkk$ operator creates new threads: the expression $\fork{e}$
evaluates to $()$, the unit value, while creating a new thread to run
expression~$e$ concurrently.

The next three axioms in the figure deal with session operations. The
$\newk$ operator creates a new channel. Channels are denoted by their
two end points,~$a$ and $b$ in this case. We require that context~$E$
does not bind variables $a,b$, so that these are bound by the
outermost channel binding, $(\nu a,b)$.

The $\sendk$-$\recvk$ rule captures message passing: the sending
process writes value~$v$ in channel end point~$a$ whereas the
receiving process reads it from channel end point~$b$. That the pair
$a$-$b$ forms the two end points of a channel is captured by the
outermost $(\nu a,b)$ binding. The result of sending a value on channel
end~$a$ is~$a$ itself; that of receiving on~$b$ is the pair
$(v,b)$. In this way both threads are able to use their channel ends
for further interaction. This ``rebinding'' of channel ends provide
for a standard treatment of~$a$ and~$b$ as linear values. It is also
the type system that makes sure that, in a given process, there is
exactly one thread holding a copy of a given channel end, thus
allowing a simplified reduction rule where one finds exactly two
threads underneath channel binder $(\nu a,b)$.

The rule for $\selectk$-$\casek$ is similar in spirit. One thread
selects an $l$-labelled option on a channel end, whereas another offers
a choice on the other channel end. After successful interaction, the
selecting thread is left with its channel end, $a$, whereas the
choice-offering thread is left with an application of the
branch that was selected, $e_j$, to its channel end, $b$.

The remaining three rules are standard in the $\pi$-calculus. The
first two allow reduction underneath parallel composition and scope
restriction; the last incorporates structural congruence in the
reduction relation.

% Example of reduction

As an example consider an expression that creates a new channel, forks
a thread that writes an integer on the channel and reads back its
successor. The original thread, in turn, waits for an integer value
and writes back its successor. We depict the reduction below, where we
make use of the conventional semicolon operator $e_1;e_2$ defined as
$(\lambda a.e_2)e_1$, where $a$ is a variable not occurring free
in~$e_2$. We also write $\letin{a}{e_1}{e_2}$ for $(\lambda a.e_2)e_1$.
%
\begin{gather*}
  \letk\, a,b = \newk\,\ink\,
    \forkk\,(\letin{c}{\send 5a}{\recv c});\qquad \qquad \qquad
    \\\qquad \qquad \qquad \qquad \qquad
    \letin{n,d}{\recv b}{\send{(n+1)}{d}}
  \rightarrow
  \\
  (\nu e,f)\letin{a,b}{(e,f)}{\dots} \rightarrow
  \\
  (\nu e,f) \forkk\,(\letin{c}{\send 5e}{\dots});
    \letin{n,d}{\recv f}{\dots} \rightarrow \rightarrow
  \\
  (\nu e,f) (\letin{c}{\send 5e}{\dots} \PAR \letin{n,d}{\recv f}{\dots})
  \rightarrow
  \\
  (\nu e,f) (\letin{c}{e}{\dots} \PAR \letin{n,d}{(5,f)}{\dots})
  \rightarrow \rightarrow 
  \\
  (\nu e,f) (\recv e \PAR \send{(5+1)}{f})
  \rightarrow
  \\
  (\nu e,f)((e,6) \PAR f)
\end{gather*}

% RUNTIME ERRORS

We complete this section by discussing the notion of \emph{run-time
  errors}. The \emph{subject} of an expression~$e$, denoted by
$\subj(e)$, is~$a$ in the following cases and undefined in all other
cases.
%
\begin{equation*}
  \send ea \qquad
  \recv a \qquad
  \select ea \qquad
  \case ae
\end{equation*}

We say that two expressions~$e_1$ and $e_2$ \emph{agree} on channel
$ab$, denoted $\agree^{ab}( e_1,e_2)$, in the following four cases.
%
\begin{itemize}
\item $\agree^{ab}(\send ea, \recv b)$;
\item $\agree^{ab}(\recv a,\send be)$;
\item $\agree^{ab}(\select{l_j}{a}, \case b{l_i\rightarrow e_i}_{i\in
    I})$ and $j\in I$;
\item $\agree^{ab}(\case a{l_i\rightarrow e_i}_{i\in
    I}, \select{l_j}{b})$ and $j\in I$.
\end{itemize}

A closed process is an \emph{error} if it is structurally congruent to
some process that contains a subexpression or subprocess of one of the
following forms.
%
\begin{enumerate}
\item $\letin{a,b}{v}{e}$ and $v$ is not a pair;
\item $\match{v}{l_i\rightarrow e_i}_{i\in I}$ and
  $v\ne(\ink\,l_j\,v')$ for some $v'$ and some $j\in I$;
\item $E_1[e_1] \PAR E_2[e_2]$ and $\subj(e_1) = \subj(e_2) = a$, where
  neither $E_1$ nor $E_2$ bind~$a$;
\item $(\new a,b)(E_1[e_1] \PAR E_2[e_2] \PAR p)$ and $\subj(e_1)=a$
  and $\subj(e_2)=b$ and $\neg\agree^{ab}(e_1,e_2)$, where $E_1$ does not
  bind~$a$ and $E_2$ does not bind~$b$.
% \item $  (\new a,b)(E_1[\selectk\,l_j\,a] \PAR
%   E_2[\casek\,b\,\ofk\,\{l_i\rightarrow e_i\}_{i\in I}] \PAR p) $
%   where $E_1$ does not bind $a$ and $E_2$ does not bind $b$ and
%   $j\notin I$ (or the same with $a$ and $b$ exchanged).
\end{enumerate}

The first two cases are typical of functional languages with pairs and
datatypes.
%
The third case guarantees that no two threads hold references to the
same channel end (the fact that the process is closed and that the
contexts do not bind variable~$a$ ensure that~$a$ is a channel end).
%
The fourth case says that channel ends agree at all times: if one
thread is ready for sending, then the other is ready for receiving,
and similarly for selection and branching.

%%%%%%%%%%%%%%%%%%%%%%%%%%%%%%%%%%%%%%%%%%%%%%%%%%%%%%%%%%%%%%%%%
\subsection{Type Assignment System}

% DUALITY

\emph{Duality} is a central notion in session types. It allows to
``switch'' the point of view from one end of a channel (say, the
client side) to the other (the server side).
%
The duality function on session types, $\dual S$, is defined as
follows.
%
\begin{gather*}
  \dual\alpha = \alpha 
  \qquad
  \dual\skipk = \skipk
  \qquad
  \dual{!B} = \;?B
  \qquad
  \dual{?B} = \;!B
  \\
  \dual{\&\{l_i\colon S_i\}} = \oplus\{l_i\colon \dual{S_i}\}
  \qquad
  \dual{\oplus\{l_i\colon S_i\}} = \&\{l_i\colon \dual{S_i}\}
  \\
  \dual{S_1;S_2} = \dual{S_1};\dual{S_2}
  \qquad
  \dual{\mu x.S} = \mu x.\dual S
  \qquad
  \dual x = x
\end{gather*}
%
This simple definition is justified by the fact that the types we
consider are first order, thus avoiding a complication known to arise
in the presence of
recursion~\cite{bernardi.hennessy:using-contracts-model-session-types}.

To check whether $S_1$ is dual to $S_2$ we compute $S_3 = \dual{S_1}$
and check $S_2$ and $S_3$ for equivalence.
%
Duality is clearly an involution ($\dual{\dual S} = S$), hence we can
alternatively compute $\dual{S_2}$ and check that $S_1$ is equivalent
to $\dual{S_2}$.
%
For example, to check that $!B;\mu x.(\skipk;!B;x)$ is dual to
$\mu y.(?B;y)$, we compute $\dual{\mu y.(?B;y)}$ to obtain
$\mu y.(!B;y)$, and check that this type is equivalent to
$!B;\mu x.(\skipk;!B;x)$.

We can easily show that duality preserves kinding.

\begin{lemma}
  \label{lem:duality-preserves-kinding}
  If $\Delta \vdash S :: \kind$, then $\Delta \vdash \dual S :: \kind$.
\end{lemma}
%
\begin{proof}
  By rule induction on the premise.
\end{proof}

% TYPING CONTEXT

\begin{figure}[t]
  Context formation, $\Delta \vdash \Gamma$
  %
  \begin{gather*}
    \Delta \vdash \cdot
    \quad
    \frac{
      \Delta \vdash \Gamma
      \quad
      \Delta \vdash T :: k
    }{
      \Delta \vdash \Gamma, x\colon T
    }
  \end{gather*}

  Context splitting, $\Delta \vdash \Gamma = \Gamma \circ \Gamma$
  % 
  \begin{gather*}
    \Delta \vdash \cdot = \cdot \circ \cdot
    \qquad
    \frac{
      \Delta \vdash \Gamma = \Gamma_1 \circ \Gamma_2
      \quad
      \un_\Delta(T)
    }{
      \Delta \vdash \Gamma, x\colon T = (\Gamma_1,x\colon T) \circ (\Gamma_2,x\colon T)
    }
    \\
    \frac{
      \Delta \vdash \Gamma = \Gamma_1 \circ \Gamma_2
      \quad
      \lin_\Delta(T)
    }{
      \Delta \vdash \Gamma, x\colon T = (\Gamma_1,x\colon T) \circ \Gamma_2
    }
    \;\;%\quad
    \frac{
      \Delta \vdash \Gamma = \Gamma_1 \circ \Gamma_2
    \quad
      \lin_\Delta(T)
    }{
      \Delta \vdash \Gamma, x\colon T = \Gamma_1 \circ (\Gamma_2,x\colon T)
    }
  \end{gather*}
  
  \caption{Typing context formation and splitting}
  \label{fig:contexts}
\end{figure}

%%% Local Variables:
%%% mode: latex
%%% TeX-master: "main"
%%% End:


Typing contexts are generated by the following grammar.
%
\begin{equation*}
  \Gamma \grmeq \cdot \grmor \Gamma,a\colon T
\end{equation*}
%
As before we consider contexts up to reordering of their entries.
%
% The UN PREDs
%
The $\un_\Delta$ predicate, on types~$T$ defined on $\Delta$, is an
abbreviation of $\Delta \vdash T :: \prekind^\Unrestricted$. The $\un_\Delta$
predicate is also true of contexts of the form
$x_1\colon T_1,\dots, x_n\colon T_n$ if it is true of all
types~$T_1,\dots,T_n$. We often omit the $\Delta$ in $\un_\Delta$ when
it is clear from the context.

% CONTEXT FORMATION
We expect all types in typing contexts to be well formed, a notion
captured by judgement $\Delta \vdash \Gamma$ whose rules can be found
in Figure~\ref{fig:contexts}.

% CONTEXT SPLITTING
Linear variables must be split between the different subterms to
ensure that each variable is used once. Figure~\ref{fig:contexts}
defines a relation $\Delta \vdash \Gamma = \Gamma_1 \circ \Gamma_2$,
which describes how to split a context $\Gamma$ into two contexts
$\Gamma_1$ and $\Gamma_2$ that will be used in different subterms in
rule premisses~\cite{walker:substructural-type-systems}.

%
% However, unlike conventional contexts ($\Delta$ for example), we
% allow duplicated variables in contexts, but subject to two
% restrictions: if both $a\colon T$ and $a\colon U$ are in $\Gamma$
% then: i) $T=U$ and ii) $\Delta \vdash T :: \kindt^\Unrestricted$.


% TYPING

\begin{figure}[t]
  Typing for expressions, $\Delta;\Gamma \vdash e: T$
  %
  \begin{gather*}
    % 1 _ LAMBDA
    \frac{\Delta \vdash \Gamma \quad \un_\Delta(\Gamma)}{\Delta;\Gamma \vdash () \colon \unitk}
    \quad
    % VAR
    \frac{
      \Delta \vdash \Gamma \quad
      \un_\Delta(\Gamma) \quad
      \Delta \vdash \vec T :: \vec \kind
    }{
      \Delta;\Gamma,a\colon \forall \vec\alpha::\vec\kind. T
      \vdash a : T\subs{\vec T}{\vec \alpha}}
    \\
    % LET-poly
    \frac{
      %\begin{array}{c}
      \Delta, \vec\alpha :: \vec\kind; \Gamma_1 \vdash e_1 : T_1 \quad
      \Delta; \Gamma_2, a : \forall \vec\alpha::\vec\kind. T_1 \vdash
      e_2 : T \quad
      \vec\alpha \notin \Delta
      %\end{array}
    }{
      \Delta;\Gamma_1\circ \Gamma_2 \vdash \letin{a}{e_1}{e_2} : T
    }
    \\
    % FIX
    \frac{
      %\begin{array}{c}
        \Delta, \vec\alpha :: \vec\kind;\Gamma, a\colon  \forall
        \vec\alpha::\vec\kind. T \vdash e : T  \quad
        \un_\Delta(\Gamma) \quad
        \vec\alpha \notin \Delta
      %\end{array}
    }{
      \Delta;\Gamma \vdash \fix ae :  \forall \vec\alpha::\vec\kind .T
    }
    \\
    % Arrow Intro / Elim
    \frac{
      \Delta;\Gamma,a\colon T_1 \vdash e : T_2
      \quad
      \un_\Delta(\Gamma)
      \quad
      \Delta \vdash T_1,T_2 ::\kindt^\Linear
    }{
      \Delta;\Gamma \vdash \lambda a.e : T_1 \rightarrow T_2
    }
    \\
    \frac{
      \Delta;\Gamma,a\colon T_1 \vdash e : T_2::\kindt^\Linear
      \quad
      \Delta \vdash T_1,T_2 ::\kindt^\Linear
    }{
      \Delta;\Gamma \vdash \lambda a.e : T_1 \multimap T_2
    }
    \\
    \frac{
      \Delta;\Gamma_1 \vdash e_1 : T_1 \multimap T_2 \quad \Delta;\Gamma_2
      \vdash e_2 : T_1
    }{
      \Delta;\Gamma_1\circ  \Gamma_2 \vdash e_1 e_2 : T_2
    }
    % \;\;%\quad
    % \frac{
    %   \Delta;\Gamma \vdash e : T_1 \rightarrow T_2
    % }{
    %   \Delta;\Gamma \vdash e : T_1 \multimap T_2
    % }
     \\
    \frac{
      \Delta;\Gamma_1 \vdash e_1 : T_1 \rightarrow T_2 \quad \Delta;\Gamma_2
      \vdash e_2 : T_1
    }{
      \Delta;\Gamma_1\circ  \Gamma_2 \vdash e_1 e_2 : T_2
    }
   \\
    % Tensor Intro / Elim
    \frac{
      \Delta;\Gamma_1 \vdash e_1 : T_1
      \quad
      \Delta;\Gamma_2 \vdash e_2 : T_2
      \quad
      \Delta \vdash T_1,T_2 ::\kindt^\Linear
    }{
      \Delta;\Gamma_1\circ \Gamma_2 \vdash (e_1,e_2) : T_1 \otimes T_2
    }
    \\
    \frac{
      \Delta;\Gamma_1 \vdash e_1 : T_1 \otimes T_2
      \quad
      \Delta;\Gamma_2, a\colon T_1, b\colon T_2 \vdash e_2: U
    }{
      \Delta;\Gamma_1 \circ \Gamma_2 \vdash \letin{a,b}{e_1}{e_2} : U
    }
    \\
    % Variant Intro / Elim
    \frac{
      \Delta;\Gamma \vdash e : T_j \quad j\in I \quad
      \Delta\vdash T_i::\kindt^\Linear
    }{
      \Delta;\Gamma \vdash \inject {l_j} e : [l_i\colon T_i]_{i\in I}
    }
    \\
    \frac{
      \Delta;\Gamma_1 \vdash e : [l_i\colon T_i]
      \quad
      \Delta;\Gamma_2 \vdash e_i : T_i \multimap T
    }{
      \Delta;\Gamma_1\circ \Gamma_2 \vdash \match e {l_i\colon e_i} : T
    }
    % \\
    % % Forall Intro / Elim
    % \frac{
    %   \Delta;\Gamma \vdash e : T
    %   \quad 
    %   \alpha \notin \Delta,\Gamma
    %   \quad 
    %   \Delta,\alpha::\kind \vdash T :: \kindsch^\Linear
    %   \quad
    %   \kind \le \kindt^\Linear
    % }{
    %   \Delta;\Gamma \vdash e : \forall \alpha::\kind.T
    % }
    % \\
    % \frac{
    %   \Delta;\Gamma \vdash e : \forall\alpha::\kind.T_1
    %   \quad
    %   \Delta \vdash T_2::\kind
    % }{
    %   \Delta;\Gamma \vdash e : T_1 \subs{T_2} \alpha
    % }
    \\
    % 2 _ SESSIONS
    % New
    \frac{
      \Delta \vdash \Gamma
      \quad
      \Delta \vdash T::\kinds^\Linear
    }{
      \Delta;\Gamma \vdash \newk : T\otimes \dual T
    }
    \quad
    % Send/Receive
    \frac{
      \Delta \vdash \Gamma
      \quad
      \Delta \vdash T::\kinds^\Linear
    }{
      \Delta;\Gamma\vdash \sendk : B \rightarrow \;!B;T \rightarrow T
    }
    \\
    \frac{
      \Delta \vdash \Gamma
      \quad
      \Delta \vdash T::\kinds^\Linear
    }{
      \Delta;\Gamma \vdash \recvk : \;?B;T \rightarrow (B \otimes T)
    }
    \\
    % Select Intro
    \frac{
      \Delta;\Gamma \vdash e : \oplus\{l_i\colon T_i\}_{i\in I}
      \quad
      j\in I
    }{
      \Delta;\Gamma \vdash \select {l_j} e \colon T_j
    }
    \\
    % Case Intro
    \frac{
      \Delta;\Gamma_1 \vdash e : \&\{l_i\colon T_i\} \quad \Delta;\Gamma_2 \vdash e_i
      : T_i \multimap T
    }{
      \Delta;\Gamma_1\circ  \Gamma_2 \vdash \case e {l_i\colon e_i} : T
    }
    \\
    % 3 _ CONCURRENCY
    % Fork
    \frac{
      \Delta;\Gamma \vdash e : T
      \quad
      \un_\Delta(T)
    }{
      \Delta;\Gamma \vdash \forkk\,e : \unitk
    }
    \quad
    % 4 _ STRUCTURAL (the forall rules are structural as well)
    % Weak, Copy, Equiv
    % \frac{
    %   \Delta;\Gamma \vdash e : T_1 \quad \un_\Delta(T_2) \quad a\notin\Gamma
    % }{
    %   \Delta;\Gamma, a\colon T_2 \vdash e : T_1
    % }
    % \\
    % \frac{
    %   \Delta;\Gamma,a\colon T_1, a\colon T_1 \vdash e : T_2 \quad \un_\Delta(T_1)
    % }{
    %   \Delta;\Gamma, a\colon T_1 \vdash e : T_2
    % }
    % \\
    \frac{
      \Delta;\Gamma \vdash e : T_1 \quad \Delta \vdash T_1 \TypeEquiv T_2
    }{
      \Delta;\Gamma \vdash e : T_2
    }
  \end{gather*}
  %
   Typing for processes, $\Delta;\Gamma \vdash p$
  \begin{gather*}
    \frac{
      \Delta;\Gamma \vdash e : T \quad \un_\Delta(T)
    }{
      \Delta;\Gamma \vdash e
    }
    \\
    \frac{
      \Delta;\Gamma_1 \vdash P_1 \quad \Delta;\Gamma_2 \vdash P_2
    }{
      \Delta;\Gamma_1\circ  \Gamma_2 \vdash p_1 \mid p_2
    }
    \quad
    \frac{
      \Delta;\Gamma, a\colon T, b \colon \dual T \vdash p
            \quad
      \Delta \vdash T::\kinds^\Linear
    }{
      \Delta;\Gamma \vdash (\new a,b)p
    }
  \end{gather*}
  %
  \caption{Typing for expressions and processes}
  \label{fig:typing}
\end{figure}
%%% Local Variables:
%%% mode: latex
%%% TeX-master: "main"
%%% End:


Figure~\ref{fig:typing} contains the typing rules for expressions and for
processes. Judgments for expressions and processes take the usual forms of
$\Delta;\Gamma \vdash e: T$ and 
$\Delta;\Gamma \vdash p$. We describe the rules briefly.

% description of the rules

The first group of rules deals with the functional part of
the language. The rule for the unit value requires a context
free from term variables. If needed, unrestricted term variables are
introduced by an explicit weakening rule. The typing rule for variables
reads the type of the variable from the context. We require that
the term context contains no other entry, and that the type is
well-formed against~$\Delta$, ensuring that types introduced in a
derivation are well-formed. The rule for the fixed point is standard.

The type system comprises rules for the introduction and the
elimination of unrestricted ($\rightarrow$) and linear ($\multimap$)
functions.
%
The elimination rules are standard. In the introduction of linear
functions the term context is split in two parts, one to type the
function~$e_1$, the other to type the argument~$e_2$.
%
% The next rule in the same line allows unrestricted functions to be
% converted into linear functions, so that the rule for function
% elimination may apply.

The next four rules are all standard and provide for the
introduction and the elimination of pairs ($T_1\otimes T_2$) and
variants ($[l_i\colon T_i]$).
%
The rules for the introduction and elimination of type abstraction are
also conventional; the extra premises on kindings are meant to ensure
that types introduced in derivations are well-formed.

We now come to the channel communication rules. The rule for channel
creation introduces a pair of dual session types, one for each end
point. The $\sendk$ operator is a function that expects a value to be
sent~$B$, then a channel on which to send the value~$!B;T$, and
returns the rest of the channel~$T$. The $\recvk$ operator expects a
channel from which a value can be read $?B;T$ and returns a pair
composed of the value and the rest of the channel, $B\otimes T$. The
premises ensure that~$T$ is a session type.
%
The rule for label selection requires that expression~$e$ denotes a
channel offering an internal choice, $\oplus\{l_i\colon
T_i\}$.
Expression $\select {l_j} e$ evaluates to the rest of the channel,
hence its type is~$T_j$.
%
A $\casek$ expression expects a channel offering an external choice,
$\&\{l_i\colon T_i\}$. The expression in each branch must be function
expecting the rest of the channel~$T_i$. All such functions must
produce a value of a common type~$T$, which becomes the type of the
$\casek$ expression.

The rule for $\forkk$ requires the expression to be of an unrestricted
type, for the value the expression evaluates to will never be
consumed.

The last three rules are structural. The first two ---weakening and
copy (or contraction)---manipulate the term context. In both cases
we require the type to be unrestricted. The last rule incorporates
type equivalence in the typing relation.

The rules for processes should be easy to understand. An expression,
when seen as a process must be of an unrestricted type. This implies
that linear resources, channels in particular, are fully consumed. The
rule for parallel composition splits the context in two, using one
part for each process. Finally, the rule for channel creation
introduces two entries in the context, of types dual to each other,
one for each end of the channel.

We complete this section with a result that relates the type system to
the kinding system.

\begin{lemma}[Agreement]
  If $\Delta;a_1\colon T_1,\dots,a_n\colon T_n \vdash e : T_0$, then,
  for all $0\le i\le n$, there are kinds $\kind_i$ such that
  $\Delta \vdash T_i:: \kind_i$.
\end{lemma}
%
\begin{proof}
  By rule induction on the premise using the various kinding
  preservation lemmas (\ref{lem:subs-preserves-kinding},
  \ref{lem:equiv-preserves-kinding}, and
  \ref{lem:duality-preserves-kinding}).
\end{proof}

% The usual derived rules.
% %
% \begin{gather*}
%   \frac{
%     \Gamma_1 \vdash e_1:T_1
%     \quad
%     \Gamma_2,x:T_1 \vdash e_2:T_2
%   }{
%     \Gamma_1,\Gamma_2 \vdash \letin x {e_1}{e_2} : T_2
%   }
% \\
%   \frac{
%     \Gamma_1 \vdash e_1:T_1
%     \quad
%     \Gamma_2 \vdash e_2:T_2
%     \quad
%     \un(T_1)
%   }{
%     \Gamma_1,\Gamma_2  \vdash e_1;e_2 : T_2
%   }
% \end{gather*}

%%%%%%%%%%%%%%%%%%%%%%%%%%%%%%%%%%%%%%%%%%%%%%%%%%%%%%%%%%%%%%%%%
\subsection{Soundness and Type Safety}

The proofs for the two results of this section follow a conventional
approach: we first establish lemmas for strengthening, weakening,
% (weakening is built into the type system) -- Not anymore
substitution, sub-derivation manipulation, and inversion of the typing
relation. Soundness (Theorem~\ref{thm:soundness}) follows by rule
induction on the reduction step, and type safety
(Theorem~\ref{thm:safety}) follow by an analysis of the typing
derivation.

\begin{lemma}[Strengthening]
  \label{lem:strengthening}
  If $\Delta;\Gamma,a\colon T \vdash p$ and~$a$ not free in~$p$, then
  $\Delta;\Gamma \vdash p$ and~$\un_\Delta(T)$.
\end{lemma}
%
\begin{proof}
  By rule induction on the first premise.
\end{proof}

\begin{lemma}[Weakening]
  \label{lem:weakening}
  If $\Delta;\Gamma \vdash p$ and $\Delta \vdash T::\prekind^\Unrestricted$, then
  $\Delta;\Gamma, a\colon T \vdash p$.
\end{lemma}
%
\begin{proof}
  By rule induction on the first premise.
\end{proof}

\begin{lemma}[Congruence]
  \label{lem:congruence}
  If $\Delta;\Gamma \vdash p$ and $p\equiv q$, then
  $\Delta;\Gamma \vdash q$.
\end{lemma}
%
\begin{proof}
  By rule induction on the first premise, using strengthening
  (Lemma~\ref{lem:strengthening}).
\end{proof}

\begin{lemma}[Substitution]
  \label{lem:subs}
  If $\Delta;\Gamma_1, a\colon T_2 \vdash e_1 : T_1$ and
  $\Delta;\Gamma_2 \vdash e_2 : T_2$, then
  $\Delta;\Gamma_1,\Gamma_2 \vdash e_1 \subs{e_2}{a} : T_1$.
\end{lemma}
%
\begin{proof}
  By rule induction on the first premise.
\end{proof}

The following two lemmas are adapted from~\cite{DBLP:journals/jfp/GayV10}.

\begin{lemma}[Sub-derivation introduction]
  \label{lem:derivation-intro}
  If $\mathcal D$ is a derivation of $\Delta;\Gamma \vdash E[e] : T$,
  then there exist $\Gamma_1$, $\Gamma_2$ and $U$ such that
  $\Gamma = \Gamma_1,\Gamma_2$ and $\mathcal D$ has a sub-derivation
  $\mathcal D'$ concluding $\Delta;\Gamma_2 \vdash e : U$ and the
  position of $\mathcal D'$ in $\mathcal D$ corresponds to the
  position of the hole in $E$.
\end{lemma}

\begin{lemma}[Sub-derivation  elimination]
  \label{lem:derivation-elim}
  If
  \begin{itemize}
  \item $\mathcal D$ is a derivation of
    $\Delta;\Gamma_1,\Gamma_2 \vdash E[e] : T$,
  \item $\mathcal D'$ is a sub-derivation of $\mathcal D$
    concluding $\Delta;\Gamma_2 \vdash e : U$,
  \item the position of $\mathcal D'$ in $\mathcal D$ corresponds
    to the position of the hole in $E$,
  \item $\Delta;\Gamma_3 \vdash e_2 : U$,
  \item $\Gamma_1,\Gamma_3$ is defined,
  \end{itemize}
then
  $\Delta;\Gamma_1,\Gamma_3 \vdash E[e_2] : T$.
\end{lemma}

% No more combined rule
% The structural rules (weakening, copy, and $\TypeEquiv$) commute.  We
% can easily show that these three rules can be replaced by a single
% combined rule as follows.
% %
% \begin{equation*}
%   \frac{
%     \Delta;\Gamma_1, \Gamma_2, \Gamma_2 \vdash e : T_1
%     \quad
%     \un_\Delta(\Gamma_2,\Gamma_3)
%     \quad
%     \Delta \vdash T_1 \TypeEquiv T_2
%   }{
%     \Delta;\Gamma_1, \Gamma_2, \Gamma_3 \vdash e : T_2
%   }
% \end{equation*}
% %
% Notice that if we replace, in the weakening rule, the proviso
% $a\notin\Gamma$ by
% $a:U\in\Gamma \Rightarrow \Delta \vdash U\TypeEquiv T$, then copy and
% weakening do not commute anymore.  This combined rule forms the basis
% for the inversion lemma below.

\begin{lemma}[Inversion of the expression typing relation]\
  \label{lem:inversion}

% Cases ordered as required for soundness, which in turn are ordered
% by the reduction rules.

  \begin{itemize}
  \item % app
    If $\Delta;\Gamma \vdash e_1e_2 : T$, then
    $\Gamma = \Gamma_1 \circ \Gamma_2$ with 
    $\Delta;\Gamma_2 \vdash e_2: T_1$ and
    $\Delta \vdash T_2 \TypeEquiv T$ and either
    \begin{itemize}
    \item $\Delta;\Gamma_1 \vdash e_1: T_1 \multimap T_2$; or 
    \item $\Delta;\Gamma_1 \vdash e_1: T_1 \rightarrow T_2$.
    \end{itemize}
  \item % lambda
    If $\Delta;\Gamma \vdash \lambda a.e : T$, then
    $\Delta;\Gamma, a:T_1 \vdash e: T_2$ and
    $\Delta \vdash T_1, T_2 ::\kindt^\Linear$
    either
    \begin{itemize}
    \item $\un_\Delta (\Gamma_1)$ and $\Delta \vdash T_1 \to T_2
      \TypeEquiv T$; or
    \item  $\Delta \vdash T_1 \multimap T_2      \TypeEquiv T$.
    \end{itemize}
  \item % let/2
    If $\Delta;\Gamma \vdash \letin{a,b}{e_1}{e_2} : T$, then
    $\Gamma = \Gamma_1 \circ \Gamma_2$ and
    $\Delta;\Gamma_1 \vdash e_1: T_1 \otimes T_2$ and
    $\Delta \vdash T \TypeEquiv U$ and
    $\Delta;\Gamma_2 ,a\colon T_1,b\colon T_2 \vdash e_2: U$.
  \item % (e1,e2)
    If $\Delta;\Gamma \vdash (e_1, e_2) : T$, then
    $\Gamma = \Gamma_1\circ \Gamma_2$ with 
    $\Delta;\Gamma_1 \vdash e_1: T_1$ and
    $\Delta;\Gamma_2 \vdash e_2: T_2$ and
    $\Delta \vdash T_1, T_2 ::\kindt^\Linear$ and
    $\Delta \vdash T_1\otimes T_2 \TypeEquiv T$.
  \item % match
    If $\Delta;\Gamma \vdash \match{e}{l_i\rightarrow e_i}_{i\in
      I} : T$, then
    $\Gamma =  \Gamma_1 \circ \Gamma_2$ with
    $\Delta; \Gamma_1 \vdash e : [l_i : T_i]$ and
    $\Delta; \Gamma_2 \vdash e_i : U$ and
    $\Delta \vdash U \TypeEquiv T$.
  \item % in
    If $\Delta;\Gamma \vdash \ink\,l_j\,e : T$, then
    $\Delta;\Gamma \vdash e: T_j$ and
    $\Delta \vdash [l_i:T_i]_{i\in I} \TypeEquiv T$ and
    $j\in I$ and
    $\Delta\vdash T_i::\kindt^m$, for all $i\in I$.
  \item % fix
    If $\Delta;\Gamma \vdash \fix ae : T$, then
    $\un_\Delta(\Gamma)$ and
    $\vec\alpha\notin\Delta$ and
    $\Delta, \vec\alpha::\vec\kind;\Gamma,
    a:\forall\vec\alpha::\vec\kind. U \vdash e: U$ and
    $\Delta \vdash \forall\vec\alpha::\vec\kind. U \TypeEquiv T$.
  \item % let
    If $\Delta;\Gamma \vdash \letin{a}{e_1}{e_2} : T$, then
    $\Gamma = \Gamma_1 \circ \Gamma_2$ and
    $\vec\alpha\notin\Delta$ and
    $\Delta, \vec\alpha::\vec\kind; \Gamma_1 \vdash e_1 : T_1$ and
    $\Delta;\Gamma_2,
    a:\forall\vec\alpha::\vec\kind. T_1 \vdash e_2: T_2$ and
    $\Delta \vdash \forall\vec\alpha::\vec\kind. T_2 \TypeEquiv T$.
  \item % fork
    If $\Delta;\Gamma \vdash \fork\,e:T$, then
    $\un_\Delta(U,\Gamma)$ and
    $\Delta;\Gamma \vdash e : U$ and
    $T = \unitk$.
  \item % new
    If $\Delta;\Gamma \vdash \newk : T$,
    then $\un_\Delta(\Gamma)$ and
    $\Delta \vdash T \TypeEquiv S\otimes\dual S$ and
    $\Delta \vdash T::\kinds^m$.
  \item % send
    If $\Delta;\Gamma \vdash \sendk\, e\, a : T$, then
    $\Delta \vdash T \TypeEquiv S$ and
    $\Gamma = \Gamma_1, a\colon T_2$ and
    $\Delta \vdash T_2 \TypeEquiv \;!B;S$ and
    $\Delta;\Gamma_1 \vdash e \colon B$.
  \item % receive
    If $\Delta;\Gamma \vdash \recvk\, a : T$, then
    $\Delta \vdash T \TypeEquiv B \otimes S$ and 
    $\Gamma = \Gamma_1, a\colon T_2$ and
    $\un_\Delta(\Gamma_1)$ and
    $\Delta \vdash T_2 \TypeEquiv \;?B;S$.
  \item % select
    If $\Delta;\Gamma \vdash \select{l_j} a : T$, then
    $\Delta \vdash T \TypeEquiv S_j$ and
    $\Gamma = \Gamma_1, a\colon T_2$ and
    $\un_\Delta(\Gamma_1)$ and $\Delta \vdash T_2 \TypeEquiv
    \oplus \{ l_i\colon S_i \}_{i\in I}$ and $j\in I$.
  \item % case
    If $\Delta;\Gamma \vdash \casek\,b\,\ofk\,\{l_i\rightarrow
    e_i\}_{i\in I} : T$, then
    $\Gamma =  \Gamma_1, b\colon T_2$ and  
    $\Delta \vdash T_2 \TypeEquiv  \& \{ l_i\colon S_i \}$ and
    (for all $i\in I$)
    $\Delta; \Gamma_1 \vdash e_i : S_i \tcLolli T'$ and
    $\Delta \vdash T' \TypeEquiv T$.
  % \item %par
  %   If $\Delta;\Gamma \vdash p \PAR q$, then
  %   $\Gamma = \Gamma_1, \Gamma_2, \Gamma_3, \Gamma_4$ and
  %   $\un_\Delta (\Gamma_3,\Gamma_4)$ and
  %   $\Delta; \Gamma_1, \Gamma_3 \vdash p$ and
  %   $\Delta; \Gamma_2, \Gamma_3 \vdash q$.
  % \item %new
  %   If $\Delta;\Gamma \vdash (\new a,b) p$, then there exists $T$
  %   such that $\Delta; \Gamma, a:T, b:\dual{T} \vdash p$.
  % \item If $\Delta;\Gamma \vdash p$ , then there exists $q$ such that
  %   $p \equiv q$ and $\Delta;\Gamma \vdash q$.
  \end{itemize}
\end{lemma}
%
\begin{proof}
  For each case we consider the derivation ending with the
  corresponding structural rule followed by the combined rule, and
  collect all undischarged assumptions.
\end{proof}

Inversion of the typing relation for processes is obtained by simply
reading the rules for processes bottom-up, since all rules are
syntax-directed.

\begin{theorem}[Soundness]
  \label{thm:soundness}~\\
  If $\Delta;\Gamma \vdash p$ and $p \rightarrow q$, then
  $\Delta;\Gamma \vdash q$.
\end{theorem}
%
\begin{proof}
  By rule induction on the second premise, using the congruence and
  the substitution lemmas, sub-derivation introduction and
  elimination, and inversion
  (Lemmas~\ref{lem:congruence}--\ref{lem:inversion}).

  % Cases by the order the reduction rules appear in the definition
  % (figure X).

  \textbf{Case} the derivation ends with $\beta$: inversion (for
  expressions as processes, application, and abstraction),
  substitution lemma, rule for expressions as processes.

  \textbf{Case} the derivation ends with $\letk$: inversion (for
  expressions as processes, $\letk$, and pairs), substitution lemma
  (twice), weakening and copy rules, rule for expressions as
  processes.

  \textbf{Case} the derivation ends with $\matchk$: inversion (for
  expressions as processes, $\matchk$, and $\ink$), rules for
  application and expressions as processes.

  \textbf{Case} the derivation ends with $\fixk$: inversion (for
  expressions as processes and $\fixk$), substitution lemma, 
  contraction and rule for expressions as processes.

  \textbf{Case} the derivation ends with context: inversion for
  expressions as processes, sub-derivation intro, induction,
  sub-derivation elim, and rule for expressions as processes.

  \textbf{Case} the derivation ends with $\forkk$: inversion for
  expressions as processes, sub-derivation intro, inversion for
  $\forkk$, rule $()$, sub-derivation elimination, combined rule,
  rules for expressions as processes and parallel composition.

  \textbf{Case} the derivation ends with $\newk$: inversion for
  expressions as processes, sub-derivation intro, inversion for
  $\newk$, var axiom, $\otimes$ intro, sub-derivation elimination,
  typing rules for expressions as processes and $\nu$.

  \textbf{Case} the derivation ends with the reduction rule for
  communication: inversion ($\nu$, parallel composition, and
  expressions as processes twice), sub-derivation intro (twice),
  inversion ($\sendk$, $\recvk$), typing rules for variables and
  $\TypeEquiv$, sub-derivation elim (twice), typing rules for
  expressions as processes (twice) and parallel composition and
  weakening and copy, definition of $\dual S$, and typing rule $\nu$.

  \textbf{Case} the derivation ends with the rule for branching:
  similar to the above, but simpler.

  \textbf{Case} the derivation ends with par: inversion for parallel
  composition, induction, typing rule for parallel composition.

  \textbf{Case} the derivation ends with reduction under $\nu$:
  inversion for $\nu$, induction, typing rule for $\nu$.

  \textbf{Case} the derivation ends with $\equiv$: congruence lemma,
  induction.
\end{proof}

% \vv{In the proof above, we could spell out the details of one case
%   (there is some space left)}

We conclude this section with the results on type safety and progress
for the functional sub-language.

\begin{theorem}[Type safety]
  \label{thm:safety}
  If $\Delta;\Gamma \vdash p$, then $p$ is not an error.
\end{theorem}
%
\begin{proof}
  A simple analysis of the typing derivation for the premise. We
  analyse one of the five cases in the definition of error in
  Section~\ref{sec:semantics}, namely $(\new a,b)(E_1[e_1] \PAR
  E_2[e_2] \PAR p)$,
  where $\subj(e_1)=a$ and $\subj(e_2)=b$ and $E_1$ does not bind~$a$
  and $E_2$ does not bind~$b$. We show that $\agree^{ab}(e_1,e_2)$.
 
  The structural typing rules and those for $\newk$ and for parallel
  composition guarantee that
  $\Delta;\Gamma_1,a\colon S \vdash E_1[e_1]$ and
  $\Delta;\Gamma_2,b\colon \dual S \vdash E_2[e_2]$, for some
  $\Delta,\Gamma_1,\Gamma_2$. When $e_1$ is $\send{e'_1}a$,
  sub-derivation introduction and inversion
  (lemmas~\ref{lem:derivation-intro} and~\ref{lem:inversion}) allow to
  conclude that $\Delta \vdash S \TypeEquiv \;!B.S'$,
  hence~$\Delta \vdash \dual S \TypeEquiv \;?B.\dual{S'}$. Of all the
  terms with subject~$b$ only $\recv b$ has a type of the form
  $?B.\dual{S'}$, hence $\agree^{ab}(\send{e'_1}a,\recv b)$.
\end{proof}

\begin{corollary}[Progress for the functional sub-language]
  If $\Delta;\Gamma \vdash (\nu\vec a,\vec b)(E[e] \PAR p)$, then either
  $e\rightarrow e'$ or $e$ is a value or $e$ is of one of the
  following forms: $\send{v}{a}$, $\recv a$, $\select{l}{a}$ or
  $\case{a}{\{l_i\rightarrow e_i\}}$.  
\end{corollary}

It should be easy to see that the full language does not enjoy
progress. Consider two processes exchanging messages on two
different channels as follows.
%
\begin{equation*}
  (\nu a_1,a_2)(\nu b_1,b_2)(\sendk\;5\;a_1; \sendk\;7\;b_1 \mid \recvk\;b_2; \recvk\;a_2)
\end{equation*}
%
The nonbuffered (rendez-vous, synchronous) semantics leads to a
deadlocked situation. A recent survey reviews a few alternatives for
progress on session type
systems~\cite{huttel.lanese.etal:foundations-session-types}.

%%%%%%%%%%%%%%%%%%%%%%%%%%%%%%%%%%%%%%%%%%%%%%%%%%%%%%%%%%%%
\subsection{Conservative Extension}
\label{sec:conservative-extension}

Our system is a conservative extension of previous session type
systems. In those systems, the session type language is restricted to
tail recursion, the $\mu$ operator works with a much simpler notion
of contractivity, and equivalence is defined modulo unfolding.
We take the definitions from the functional session type
calculus~\cite{DBLP:journals/jfp/GayV10} as a blueprint.
The first-order part of the session type language from that paper may
be defined by $S'$ in the following grammar. Henceforth, we call that
language \emph{regular session types}.
\begin{align*}
  S'_X \grmeq& \End \grmor !B.S''_X \grmor ?B.S''_X \grmor \oplus\{l_i\colon {S_i}_X''\} \grmor
         \&\{l_i\colon {S_i}_X''\}
  \\
   \grmor& \mu x. S'_{X\cup\{x\}} \\
  S''_X \grmeq& x\in X \grmor S'_X
\end{align*}

The translation $\Embed{}$ into our system is defined as follows.
\begin{align*}
  \Embed{\End} & = \skipk
  \\
  \Embed{!B.S''} & = !B; \Embed{S''}
  &
  \Embed{?B.S''} & = ?B; \Embed{S''}
  \\
  \Embed{\oplus\{l_i\colon S_i''\}} & = \oplus\{l_i\colon\Embed{S_i''}\}
  &                                      
  \Embed{\&\{l_i\colon S_i''\}} & = \&\{l_i\colon  \Embed{S_i''}\}
  \\
  \Embed{\mu x.S'} & = \mu x. \Embed{S'}
  &
  \Embed{x} & = x
\end{align*}

\begin{lemma}
  For all $S'_\emptyset$, $\cdot \vdash \Embed{S'_\emptyset} :: \kinds^\Linear$.
\end{lemma}
\begin{proof}
  We need to prove a more general property.  Define
  $\GEnv_X = x : \kinds^\Linear \PAR x \in X$ and show that for all
  $S'_X$, $\GEnv_X \vdash \Embed{S'_X} :: \kinds^\Linear$. The proof
  is by straightforward induction.
\end{proof}
\begin{lemma}
  Let $\vdash_{\text{GV}}$ be the typing judgment for expressions from
  Gay and Vasconcelos~\cite{DBLP:journals/jfp/GayV10}.  If
  $\Gamma \vdash_{\text{GV}} e : T$, then
  $\cdot; \Gamma \vdash e : \Embed{T}$.
\end{lemma}


%%% Local Variables:
%%% mode: latex
%%% TeX-master: "main"
%%% End:

\section{Related Work}
\label{sec:related-work}

The system we propose is ultimately rooted in the work of Honda et
al.\ on session types~\cite{DBLP:conf/concur/Honda93,DBLP:conf/parle/TakeuchiHK94,DBLP:conf/esop/HondaVK98}.
%
The particular language of this paper is closely related to the one
proposed by Gay and Vasconcelos~\cite{DBLP:journals/jfp/GayV10}. The
main difference is at the level of semantics: we use a synchronous
semantics in place of a buffered one. We make this choice to simplify
the technical treatment of the operational semantics. We believe that
a buffered semantics can be derived without compromising the most
important properties of the language. At the level of the language,
and in addition to Gay and Vasconcelos, we incorporate variant types
and recursion on functional types. The linear treatment of session
types is identical, including the syntactic distinction of the
two ends of a channel, related by a $\nu$-binding.

The predicative polymorphism we employ is closely related to that of
Bono et al.~\cite{BonoPadovaniTosatto13}, including the kinding system
for type variables. The extra complexity of context-free types lead us
to a more elaborate kinding system, allowing to distinguish session (or
end point) types, from functional types and type schemes (Bono et al.\
rely on different syntactic categories).
%
% Predicative polymorphism for the $\pi$-calculus was introduced by
% Vasconcelos~\cite{DBLP:conf/parle/Vasconcelos94}.
%
A different form of polymorphism---bounded polymorphism on the values
transmitted on channels---was introduced by
Gay~\cite{DBLP:journals/mscs/Gay08} in the realm of session types for
the $\pi$-calculus.

Wadler~\cite{DBLP:journals/jfp/Wadler14} gives a typing preserving
translation of the Gay and Vasconcelos calculus mentioned before to a process
calculus inspired by the work of Caires and
Pfenning~\cite{DBLP:conf/concur/CairesP10}.  The semantics of these
systems, given directly by the cut elimination rules of linear logic,
ensure deadlock freedom. Even though our system ensures progress for
the functional part of the language, the unrestricted interleaving of
channel read/write on multiple channels may lead to deadlocked situations. That is the
price to pay for the flexibility our language offers with respect to
the work of Caires, Pfenning, and Wadler~\cite{DBLP:conf/concur/CairesP10,DBLP:journals/jfp/Wadler14}.

% The Sill language described by Toninho, Caires, and
% Pfenning~\cite{DBLP:conf/esop/ToninhoCP13} 
Functional languages with conventional session
types~\cite{DBLP:journals/jfp/GayV10,DBLP:conf/esop/ToninhoCP13} can
describe type-safe protocols to transmit trees. Doing so
requires a higher-order recursive session type of the following shape:
\begin{lstlisting}
TreeC = oplus{Leaf: end, Node: !int.!TreeC.!TreeC}
\end{lstlisting}
That is, to transmit a node \lstinline|Node(i,t1,t2)| on channel $c$, the originating process first sends the
integer \lstinline|i| on $c$. But then it creates two new channels $c_1$ and $c_2$,
sends their receiving ends on $c$, and closes $c$. Finally, it
recursively transmits \lstinline|t1| on $c_1$ and \lstinline|t2| on
$c_2$.
In comparison, our calculus is intentionally closer to a low level
language: it only supports the transmission of base type values. We
furthermore believe that its run-time implementation is simpler and
more efficient: only one channel is created and used for the
transmission of the tree; thus, it avoids the overhead of multiple
channel creation and channel passing.

%%% Local Variables:
%%% mode: latex
%%% TeX-master: "main"
%%% End:

\section{The bright future of \freest{}}
\label{sec:conclusion}

We have developed a basic compiler for \freest, a concurrent
functional programming with context-free session types, based on the
ideas of Thiemann and Vasconcelos~\cite{DBLP:conf/icfp/ThiemannV16}.
%
There are many possible extensions to the language. We discuss a
few.
%
Support for linear pairs and linear datatypes, as well for polymorphic
datatypes, should not be difficult to incorporate.
%
Because \freest{} compiles to Haskell, a better interoperability is
called for. We plan to add primitive support for lists, and for some
functions in Haskell's prelude, namely rank-1 functions that we will
have to annotate with \freest{} types.
%
We have chosen a buffered semantics with buffers of size one, for ease
of implementation, but we plan to experiment with buffered channels of
arbitrary size by simply replacing the back-end.
%
The original proposal of context-free sessions is based on a
call-by-value operational semantics and we kept that strategy in
\freest. We however plan to experiment with call-by-need, taking
advantage of the back-end in Haskell. 
%
Shared channels allow for multiple readers and multiple writers,
thus introducing (benign) races. There are several proposals in the
literature~\cite{DBLP:journals/pacmpl/BalzerP17,
  DBLP:conf/sefm/FrancoV13,Lindley.Morris_Lightweight.functional.session.types,DBLP:journals/iandc/Vasconcelos12}
on which we may base this extension.
%
The \lstinline|dualof| type operator is present in the SePi
language~\cite{DBLP:conf/sefm/FrancoV13}. Its incorporation in
\freest{} may be complicated by the presence of polymorphic type
variables.

We also plan to extend the expressivity of \freest{} by allowing
messages to convey arbitrary types, as opposed to basic types only.
% session types to be used to send or receive other session types.  For
% this purpose, we intend to enable message operators to be applied not
% only to basic types but to any functional or session type. 
In this wider scope, the type equivalence algorithm for context-free
session types must be intertwined with the type equivalence algorithm
for functional types, for now, the labels of the labeled-transition
system are types themselves.

Last but not least, we plan to incorporate type inference on type
applications in order to allow the automatic identification of the
unifier matching a polymorphic type against a given type. This
unification process should recognize the unifiers of two types up to
type equivalence. However, dealing with type inference on type
applications with recursive types might be challenging, as observed by
Hosoya and Pierce~\cite{DBLP:journals/toplas/PierceT00}.
%in~\cite{hosoya1999good}.

%%% Local Variables:
%%% mode: latex
%%% TeX-master: "main"
%%% End:


\acks % Acknowledgments, if needed.

The authors would like to thank Philip Wadler, Simon Gay, Sam Lindley,
Julian Lange, Hans Hüttel, and Luís Caires for fruitful discussions
and pointers.

% We recommend abbrvnat bibliography style.

\bibliographystyle{abbrvnat}
\bibliography{biblio}

% The bibliography should be embedded for final submission.

% \begin{thebibliography}{}
% \softraggedright

% \bibitem[Smith et~al.(2009)Smith, Jones]{smith02}
% P. Q. Smith, and X. Y. Jones. ...reference text...

% \end{thebibliography}


% \clearpage{}
% \appendix
\section{Transforming recursive types}
\label{sec:transf-recurs-types}


\begin{figure}[t]
  \begin{align*}
    a & ::= \omega && \text{infinite words, only} \\
    & \mid \infty && \text{finite and infinite words} \\
    & \omega \sqsubset \infty \\
    n & ::= 0 \mid 1 && \text{minimum length of trace} \\
    A & ::= \cdot \mid A, x:a && \text{environments} \\
    C & ::= \cdot \mid C, (x,n) && \text{depth of guardedness}
  \end{align*}
  \begin{gather*}
    \frac{}{A \vdash \skipk : \infty; 0; C }
    \quad
    \frac{}{A \vdash {!B} : \infty; 1; C}
    \quad
    \frac{}{A \vdash {?B} : \infty; 1; C}
    \\
    \frac{A \vdash S_1 : a_1;n_1;C_1 \quad A \vdash S_2 : a_2;n_2;C_2}{
      A \vdash (S_1;S_2) : a_1 \sqcap a_2;\seq(a_1)(a_2)}
    \\
    \frac{(\forall i\in I)~A \vdash S_i : a_i; n_i ; C_i}{
      A \vdash \star\{l_i\colon S_i\}_{i\in I} :
      \bigsqcup_i a_i; 
      1; \bigsqcap_i C_i \uparrow 1}
%     \quad
%     \frac{(\forall i\in I)~A \vdash S_i : a_i}{
%       A \vdash \&\{l_i\colon S_i\}_{i\in I} :
%       \bigsqcup_i a_i}
    \\
    \frac{}{A, x : a \vdash x : 1 ; C,(x,0)} \quad
    \frac{A, x : a \vdash S : a; n ; C, (x,1) }{A \vdash \mu x.S : a; n; C}
  \end{gather*}
  \begin{align*}
    \seq (n_1; C_1) (n_2; C_2) & = (n_1 + n_2 ; C_1 \sqcap C_2 \uparrow n_1
    ) \\
    (x, i) \uparrow n &= (x, i + n)
  \end{align*}
  \caption{Inference for infiniteness (and contractivity)}
  \label{fig:inference-infiniteness}
\end{figure}
Figure~\ref{fig:inference-infiniteness} contains an inference system for detecting finite words. It
classifies session types $S$ according to their trace languages. The intent is as follows: If all traces of $S$ are infinite
words in $\Sigma^\omega$, then $A \vdash S : \omega; n; C$ is derivable. If $S$ may admit a finite trace
in $\Sigma^\infty$, then $A \vdash S : \infty; n; C$ is derivable. The result $n$ indicates the
minimum length of a trace generated/accepted by $S$. The $C$ contains pairs of the form $(x,n)$
which indicates a lower bound on the trace produced before $x$ is mentioned recursively. The
formation rule for $\mu x.S$ requires $(x,1)$ which is equivalent to contractivity in $x$.


We write $A \models \rho$ if $\dom (A) = \dom (\rho)$ and for all $x :a \in A$, let $R_x = \rho (x)$
where $R_x \ne \emptyset$, $R_x \subseteq \Sigma^a$, and $\min\{ |t| \mid t\in R_x \}\ge1$.

\begin{lemma}
  If $A \vdash S : a; n; C$ and $A \models \rho$, then $T = \TR (S) \rho \subseteq \Sigma^a$ and
  $\min\{ |t| \mid t \in T\} \ge n$.
\end{lemma}
\begin{proof}
  Induction on $S$. Relies on $\TR (S) \rho \ne \emptyset$, for all $S$.

  \textbf{Case }$\skipk$, $!B$, $?B$: Immediate.

  \textbf{Case }$(S_1;S_2)$: By inversion, $A \vdash S_1: a_1;n_1;C_1$, $A\vdash S_2:a_2;n_2;C_2$, and $a = a_1
  \sqcap a_2$, $(n;C) = \seq (n_1;C_1) (n_2;C_2)$.
  If $a=\omega$, then there are two non-exclusive cases $a_1=\omega$ or $a_2 = \omega$.

  \textbf{Subcase }$a_1 = \omega$: By induction $\TR (S_1) \rho \subseteq \Sigma^\omega$, hence $\TR
  (S_1;S_2)\rho = \TR (S_1)\rho \cdot \TR (S_2)\rho = \TR (S_1)\rho \subseteq \Sigma^\omega$ because $\TR (S_2)\rho$
  is not empty. The condition on $n$ holds trivially because there are no finite traces.

  \textbf{Subcase }$a_1 = \infty$ and $a_2 = \omega$: By induction $\TR (S_2) \rho \subseteq \Sigma^\omega$, hence  $\TR
  (S_1;S_2)\rho = \TR (S_1)\rho \cdot \TR (S_2)\rho \subseteq \Sigma^\omega$ because $\TR (S_1)\rho$
  is not empty. Again, the condition on $n$ holds trivially.

  \textbf{Subcase }$a_1 = a_2 = \omega$: The condition on $n$ holds because $|v\cdot w| = |v|+|w|$.

  \textbf{Case }$\star\{l_i\colon S_i\}$: By inversion, $A \vdash S_i : a_i; n_i; C_i$ and $a = \bigsqcup_i
  a_i$, $n=1$, and $C = \bigsqcap_i C_i \uparrow 1$.
  If $a=\omega$, then $a_i = \omega$, for all $i$. Hence, by induction $\TR (\oplus\{l_i\colon
  S_i\})\rho = \bigcup_i \{L_i\}\cdot\TR (S_i)\rho \subseteq \Sigma^\omega$. The condition on $n$ is
  trivial because each trace has length $\ge1$.

  \textbf{Case }$x$: Immediate by assumption $A \models \rho$.

  \textbf{Case }$\mu x.S$: By inversion, $A, x:a \vdash S:a;n;C,(x,1)$. If $a=\infty$, then any (non-empty)
  extension of $\rho$ satisfies $A, x:\infty \models \rho[x\mapsto Y]$ and the result is immediate by
  induction. 

  If $a= \omega$, then consider
  \begin{align*}
    \TR (\mu x.S)\rho & = \GFP\, \lambda Y. \TR (S)\rho[x\mapsto Y] \\
    & = \TR (S)\rho[x\mapsto \GFP\, \lambda Y. \TR (S)\rho[x\mapsto Y]]
  \end{align*}

  \textbf{Subsidiary lemma:}
  Let $F (Y) = \TR (S)\rho[x\mapsto Y]$ and show that an $F$-consistent set
  cannot contain finite words.

  That is, suppose that $Y \subseteq \TR (S)\rho[x\mapsto Y]$ and there exists some $v \in Y$ of 
  length $|v| = k < \infty$. We show that there must be some $v' \in Y$ with $|v'|\le k-m$ where
  $(x,m)$ is the minimal prefix length for $x$ in $C$.

  Given that result, we argue as follows. Choose $v \in Y$ of finite minimal length $k$. From the
  inversion, we know that $m=1$. Hence, there exists some $v' \in Y$ with length $\le k-1$. From
  this contradiction it follows that $Y$ contains no finite elements.
  
  Proof by induction on the derivation of $A, x:\omega \vdash S : \omega; n; C,(x,m)$.

  \textbf{Case }$\skipk$, $!B$, $?B$: contradiction because they do not derive $\omega$.

  \textbf{Case }$(S_1;S_2)$: Observe that $v \in \TR (S_1;S_2)\rho[x\mapsto Y]$ implies that $v = v_1v_2$ with
  $v_1 \in \TR (S_1)\rho[x\mapsto Y]$ and $v_2 \in \TR (S_2)\rho[x\mapsto Y]$ where $|v_1|, |v_2| \le |v| \le k$.
  By inversion, there are two subcases:

  \textbf{Subcase }$A \vdash S_1:\omega;n_1;C_1,(x,m_1)$ where $m_1\sqsupseteq m$: By induction,
  there exists some $v'\in Y$ with $|v'| \le |v_1| - m_1 \le |v| - m$.

  \textbf{Subcase }$A \vdash S_1:\infty;n_1;C_1,(x,m_1)$ and $A \vdash S_2:\omega;n_2;C_2,(x,m_2)$
  where  $m_1\sqsupseteq m$ and  $m_2 + n_1 \sqsupseteq m$:
  By induction, there exists some $v'\in Y$ with $|v'| \le  |v_2| - m_2 \le |v| - m$ (if $n_1=0$).
  If $n_1=1$, then $|v_2| < |v|$ and we can exploit the inductive hypothesis similarly.

  \textbf{Case }$\star\{l_i\colon S_i\}$: Inversion yields $A \vdash S_i: \omega; n_i;C_i,(x,m_i)$
  and $m =1$.
  In this case, $v = L_iv_i$ (for some $i$) and $v_i \in \TR
  (S_i)\rho[x\mapsto Y]$. By induction, there exists $v'\in Y$ with $|v'|\le |v_i| - m_i \le |v_i|+1
  -1 = |v| -1$.

  \textbf{Case }$x$: In this case, $m=0$ and $v \in \rho (x) = Y$ and $|v|\le |v|-0$.

  \textbf{Case }$x'\ne x$: By inversion, it must be that $x':\omega \in A$. By $A \models \rho$, it
  must be that $v \in \rho (A) \subseteq \Sigma^\omega$, a contradiction.

  \textbf{Case }$\mu x'.S$: By the outer induction, this implies that $v\in\Sigma^\omega$, a contradiction.
\end{proof}

\begin{lemma}
  For each $S$, there is a minimal derivation, where derivations are ordered pointwise on judgments;
  judgments are ordered pointwise on environments $A$ and the $a$-component of the result.
\end{lemma}

\begin{lemma}
  If $A \vdash (S_1; S_2) : \omega; n; C$ is derivable with subderivation $A \vdash S_1 : \omega;
  n_1; C_1$, then $(S_1;S_2) \TypeEquiv S_1$.
\end{lemma}

This lemma enables the direct syntactic transformation of $\mu x.{!B};x;x$ to the equivalent $\mu
x.{!B}; x$. It even applies to proving $\mu x.{!B}; x; S \TypeEquiv \mu x.{!B}; x$ for any $S$.


\section{Obsolete type stuff}
\subsection{Type unfolding}
\label{sec:type-unfolding}

We establish a notion of type unfolding as a step towards
defining type equivalence. The idea of unfolding is to expose the
first nontrivial action of a session type by squeezing out sequences
of $\skipk$s.
Let $A$ range over $\alpha$, $!B$, and $?B$; let $\star$ range over
$\oplus$ and $\&$. 
%
Define the unfolding of a type $T$,  $\Unfold(T)$, recursively by cases on the
structure of~$T$ as follows. 
\begin{enumerate}
\item $\Unfold(\mu x.T) = \Unfold(T\subs{\mu x.T}x)$
\item $\Unfold (S;S') = \left\{%
  \begin{array}{ll}
    \Unfold(S') & \Unfold(S) = \skipk
    \\
    (A; \Norm(S')) & \Unfold(S) = A
    \\
    (S_3; (S_4\fatsemi S')) & \Unfold(S) = (S_3;S_4)
    \\
    \star\{l_i\colon S_i\fatsemi S'\}  & \Unfold (S) = \star\{l_i\colon S_i\}
  \end{array}
  \right.
$
\item $\Unfold(T) = T$, otherwise
\end{enumerate}
We assume two auxiliary definitions
\begin{align*}
  \Norm (S) &=
              \begin{cases}
                S_1 \fatsemi S_2 & S = (S_1; S_2) \\
                S & \text{otherwise}
              \end{cases}
  \\
  S_1 \fatsemi S_2 &=
                     \begin{cases}
                       S_1' \fatsemi (S_1'' \fatsemi S_2) & S_1 =
                       (S_1'; S_1'') \\
                       (S_1; S_2) & \text{otherwise}
                     \end{cases}
\end{align*}

The function $\Unfold$ is well-defined and terminating because we
assume that the body of a recursive type is contractive. The auxiliary
functions $\Norm$ and $\fatsemi$ are both terminating.
The following
lemmas establish well-definedness of $\Unfold$.

\begin{lemma}\label{lemma:app:guarded=skip}
  Suppose that $\GEnv$ does not bind recursion variables and that
  $\sigma$ is a substitution of recursion variables by recursive types.
  If $\GEnv \Contr S : \Guarded$, then $\Unfold (S\sigma) = \skipk$.
\end{lemma}
\begin{proof}
  Induction on $\GEnv \Contr S : \Guarded$.

  \textbf{Case }$\GEnv \Contr \skipk : \Guarded$. Immediate.

  \textbf{Case }$\GEnv \Contr (S_1;S_2) : \Guarded$ because $\GEnv
  \Contr S_1 : \Guarded$ and $\GEnv \Contr S_2 : \Guarded$. By
  induction, $\Unfold (S_1\sigma) = \skipk$, hence $\Unfold ((S_1;S_2)\sigma) =
  \Unfold (S_2\sigma) = \skipk$ by induction.

  \textbf{Case }$\GEnv \Contr \mu x.S : \Guarded$ because $\GEnv
  \Contr S : \Guarded$. In this case, $\Unfold ((\mu x.S)\sigma) = \Unfold
  (S[\mu x.S/x]\sigma) = \skipk$ by induction (for $\sigma' = [\mu x.S/x]\sigma$).
\end{proof}

\begin{definition}
  A \emph{guarded} type has one of the forms below.
  % where $A$ ranges over $\alpha$, $!B$, and $?B$.
\begin{gather*}
  \skipk \quad A \quad A;S' \quad
  \star\{l_i\colon S_i\} %\quad \&\{l_i\colon S_i\}
  \\
  T \to T' \quad T \multimap T' \quad T \otimes T' \quad [l_i : T_i]
  \quad B 
\end{gather*}
\end{definition}
% We say that %
% %
% \footnote{To me $T$ is only a type if $\Theta \vdash T \isOk$ holds,
%   for some $\Theta$ that does not bind recursion variables. To
%   $(B\rightarrow B;\skipk)$ I call a piece of \emph{junk syntax}. This
%   means that ``there is no life outside types'' (the intuitionistic
%   approach). Pragmatically, it means that, whenever we talk of $T$, we
%   need not keep saying ``s.t.\ $\Theta \vdash T \isOk$ for some
%   $\Theta$ that does not bind recursion variables.''. In this , a
%   session type $S$ is a type such that $\Delta \vdash S: \kinds^m$,
%   for some $\Delta$ that does not bind recursion variables.}
% %


\begin{lemma}[Characterization of $\Unfold$]
  \label{lem:app:unfold-yields-guarded-types}
  Suppose that $\GEnv$ does not bind recursion variables and that
  $\GEnv \vdash T :: \kind$ for $\kind \le \kindt^\Unrestricted$, then $\Unfold (T)$ is defined and yields a
  guarded type.

  % result that has one of the following forms.
  % \begin{enumerate}
  % \item $ \skipk$,
  % \item $(\alpha;S')$, $(!B; S')$, $(?B; S')$ for some $\Delta \vdash S' \isOk$,
  % \item $\oplus\{l_i\colon S_i\}$,  $\&\{l_i\colon S_i\}$ for some
  %   $\Delta \vdash S_i \isOk$.
  % \end{enumerate}

  % Furthermore, if $\GEnv \Contr S : \gamma$, then $\Unfold (\mu
  % x.S)$ is defined and yields a guarded type.
\end{lemma}
\begin{proof}
  Induction on the derivation of  $\GEnv \vdash T :: \kind$.

  \textbf{Case }$\GEnv \vdash \skipk :: \kinds^\Unrestricted$.
  In this case, $\Unfold (\skipk) = \skipk$.

  \textbf{Case }$\GEnv \vdash A :: \kind$. $\Unfold (A) = {A}$.

  % \textbf{Case }$\GEnv \vdash !B :: \kinds^\Linear$. $\Unfold (!B) = {!B}$.
  % \textbf{Case }$\GEnv \vdash ?B :: \kinds^\Linear$. $\Unfold (?B) = {?B}$.

  \textbf{Case }$\GEnv \vdash (S_1;S_2) :: \kinds^m$.

  Inversion yields $\GEnv \vdash S_1 :: \kinds^{m_1}$ and  $\GEnv
  \vdash S_2 :: \kinds^{m_2}$. By induction, $S_1' = \Unfold (S_1)$ is
  guarded.

  \textbf{Subcase }$S_1' = \skipk$. In this case, $\Unfold (S_1;S_2) =
  \Unfold (S_2)$ which is guarded by induction.

  \textbf{Subcase }$S_1' = A$. Then $\Unfold (S_1;S_2) = (A; S_2)$
  which is guarded.

  \textbf{Subcase }$S_1' = (A; S_3)$. Hence, $\Unfold (S_1;S_2) = (A; (S_3; S_2))$ is guarded.

  \textbf{Subcase }$S_1' = \star\{l_i\colon S_i\}$: $\Unfold
  (S_1;S_2) = \star\{l_i\colon (S_i; S_2) \}$.

  % \textbf{Subcase }$S_1' = \&\{l_i\colon S_i\}$: $\Unfold
  % (S_1;S_2) = \&\{l_i\colon (S_i; S_2) \}$.

  \textbf{Case }$\GEnv \vdash \star\{l_i\colon S_i\}
  :: \kinds^\Linear$. Immediate.

  % \textbf{Case }$\GEnv \vdash \&\{l_i\colon S_i\}
  % :: \kinds^\Linear$. Immediate.

  % \textbf{Case }$\GEnv \vdash \alpha :: \kind$. Immediate: $\Unfold
  % (\alpha) = \alpha$.

  \textbf{Case }$\GEnv \vdash \mu x. T :: \kind$.
%
  Inversion yields $\GEnv, x:\kind \vdash T :: \kind$ and
  $\GEnv \Contr T : \gamma$. Observe that
  $\Unfold (\mu x.T) = \Unfold (T[\mu x.T/x])$ and
  $\GEnv \Contr T[\mu x.T/x] : \gamma$. By the second claim 
  $\Unfold (T[\mu x.T/x])$ is defined and yields a guarded type using
  $\sigma = [\mu x.T/x]$.

  \textbf{All other cases}: $\Unfold (T) = T$ is guarded.
  
  \textbf{Second claim.}
  % It holds that $\Unfold (\mu x.S) = \Unfold (S[\mu x.S/x])$.
  % Further $\GEnv\setminus x \Contr S : \gamma$ implies that
  % $\GEnv \Contr S[\mu x.S/x] : \gamma$.
  Suppose that  $\GEnv \Contr T :  \gamma$ where $\GEnv$ does not bind
  recursion variables and that  $\sigma$ is a substitution on recursion variables.
  Then  $\Unfold (T\sigma)$ is defined and yields a guarded type.

  The proof is by induction on the derivation of $\GEnv \Contr T :
  \gamma$.

  \textbf{Case }$\skipk$. Immediate.

  \textbf{Case }$A \in \{ {!B}, {?B}, \alpha\}$. $\Unfold (A\sigma) = A$ which is guarded.

  \textbf{Case }$x$ cannot occur because $\GEnv$ 
  contains no assumptions about recursion variables.

  \textbf{Case }$\star\{l_i\colon S_i\}$. $\Unfold
  ((\star\{l_i\colon S_i\})\sigma) = \Unfold (\star\{l_i\colon
  S_i\sigma\}) = \star\{l_i\colon S_i\sigma\}$ which is guarded.

  % \textbf{Case }$\&\{l_i\colon S_i\}$. Analogously.

  \textbf{Case }$(S_1;S_2) : \Productive$ because $S_1 :
  \Productive$. By induction $S_1' = \Unfold (S_1)$ is
  guarded. Proceed by subcases on $S_1'$.

  \textbf{Subcase }$\skipk$. Contradicts $S_1 : \Productive$.

  \textbf{Subcase }$A$. Here, $\Unfold ((S_1;S_2)\sigma) = (A;
  S_1)\sigma$, which is guarded.

  \textbf{Subcase }$(A; S_3)$. $\Unfold
  ((S_1;S_2)\sigma) = (A; (S_3; S_2))\sigma$, which is guarded. 

  \textbf{Subcase }$\star\{l_i\colon S_i\}$. $\Unfold
  ((S_1;S_2)\sigma) = \star\{l_i\colon (S_i; S_2)\sigma\}$ is guarded.

  % \textbf{Subcase }$\&\{l_i\colon S_i\}$. Similar.

  \textbf{Case }$(S_1;S_2) : \Productive$ because $S_1 :
  \Guarded$ and $S_2 : \Productive$. In this case, $\Unfold (S_1) =
  \skipk$ by Lemma~\ref{lemma:guarded=skip} so that the result is
  $\Unfold (S_2\sigma)$, which is guarded by induction on $S_2 : \Productive$.

  \textbf{Case }$\GEnv \Contr \mu x.T : \gamma$ because $\GEnv \Contr
  T : \gamma$. Hence, $\Unfold ((\mu x.T)\sigma) = \Unfold (T\sigma[\mu
  x.T\sigma/x])$. The result follows by induction using $\sigma' = \sigma[\mu
  x.T\sigma/x]$.

  % \textbf{Case }$\GEnv, \alpha : \gamma \Contr \alpha :
  % \gamma$. Immediate.

  \textbf{All remaining cases}: Immediate.
\end{proof}

Next, we consider invariance of kinding and contractivity under
unfolding of recursion anywhere in the type.

\begin{lemma}[Weakening]\label{lemma:app:weakening-kind}
  If $\Delta \vdash T :: \kind$, then
  $\Delta, x\colon \gamma \vdash T :: \kind$ for some $x$ not in
  $\Delta$.
\end{lemma}


\begin{lemma}[Unfolding preserves kinding]
  If $\Delta \vdash T :: \kind$ then $\Delta \vdash \Unfold(T) :: \kind$.
\end{lemma}
%
\begin{proof}
  We only consider the case for a recursive type as the other cases
  are straightforward.
  
  % (Needs adjustment)
  If  $\GEnv \vdash \mu x.T :: \kinds^m$, it must be because  $\GEnv,
  x:\kinds^m \vdash T :: \kinds^m$ and $\GEnv \Contr T : \gamma$.
  We prove by induction on $\GEnv, x:\kinds^m \vdash T :: \kinds^m$ that
  $\GEnv \vdash T[\mu  x. T/x] :: \kinds^m$.

  There are two interesting cases. In the first case, we encounter the
  recursion variable $\GEnv, x::\kinds^m, \GEnv' \vdash x
  :: \kinds^m$. At this point, we have to return $\GEnv, \GEnv' \vdash \mu
  x.T :: \kinds^m$, which is derivable by the initial assumption and
  weakening (Lemma~\ref{lemma:weakening-kind}).

  The other case is a different $\mu$ operator in a judgment $\GEnv,
  x::\kinds^m, \GEnv' \vdash \mu x'. T' :: \kinds^{m'}$. Inversion yields $\GEnv,
  x::\kinds^m, \GEnv', x'::\kinds^{m'} \vdash T'  :: \kinds^{m'}$ and $\GEnv,
  x:\gamma, \GEnv' \Contr T' : \gamma'$. The first part
  can be handled by induction, but the second part requires an
  auxiliary induction to prove that $(\GEnv, \GEnv')
  \Contr T'[\mu x.T/x] : \gamma'$. For this auxiliary induction it is
  sufficient to observe that a successful derivation never reaches a
  recursion variable, so the unrolling does not matter. 
\end{proof}

\begin{lemma}
  \label{lemma:app:unfold-fixpoints}
  If $T$ is a guarded type, then $\Unfold (T) = T$.
\end{lemma}
\begin{proof}
  Cases on $T$.

  \textbf{Case }$\skipk$: Obvious.

  \textbf{Case }$A$: $\Unfold (A) = A$.

  \textbf{Case }$(A; S')$:
  $\Unfold (A;S') = (A; S')$ as $\Unfold (A) = A$.

  \textbf{Case }$\star\{l_i\colon S_i\}$: Immediate.

  \textbf{Remaining cases}: Immediate.
\end{proof}

\begin{lemma}
  \label{lemma:app:unfold-idempotent}
  % For all well-formed types, $
  $\Unfold (\Unfold (T)) = \Unfold (T).$
\end{lemma}
\begin{proof}
  By Lemma~\ref{lem:unfold-yields-guarded-types}, $\Unfold (T)$ is guarded and hence a fixpoint of
  $\Unfold$ by Lemma~\ref{lemma:unfold-fixpoints}.
\end{proof}

\subsection{Type equivalence}
\label{sec:type-equivalence}

We want to define a notion of type equivalence for session types that
only depends on the communication behavior of a process with that
type. To this end, we first define a (weak) labelled transition system
$(\stypes, \Sigma, \LTSderives)$ that captures this behavior. The set
of states is  $\stypes = \{ S \mid \GEnv \vdash S :: \kinds^m \}$
where $\GEnv$ is an arbitrary, fixed kinding environment that binds no
recursion variables. The
actions in this system are drawn from the set $\Sigma = \{ {!B}, {?B}
\mid  B \in \btypes \} \uplus \Tyvars \uplus \{ {!l}, {?l} \mid l
\in \Labels\} $. We use the label $\Silent$ for the \emph{silent
  action} that exhibits no externally observable behavior. The transition
relation $\LTSderives$ is defined by the rule set in Figure~\ref{fig:type-behavior}. We write
$\Wderives$ for the reflexive transitive closure of $\LTSderives[\Silent]$ and $\Wderives[\xi]$ for
the composition $\Wderives \circ \LTSderives[\xi] \circ \Wderives$.

\begin{figure}[tp]
  \begin{gather*}
    {A \LTSderives[A] \skipk }
    \qquad
    {\star\{\overline{l_n:S_n}\} \LTSderives[\star l_i] S_i}
    \\
    \frac{S_1 \LTSderives[\xi] S_1'}{(S_1; S_2) \LTSderives[\xi]
      (S_1';S_2) }
    \qquad
    {(\skipk; S) \LTSderives[\Silent] S}
    \\
    {((S_1;S_2); S_3) \LTSderives[\Silent] (S_1; (S_2; S_3))}
    \\
    {(\star\{\overline{l_n:S_n}\}; S) \LTSderives[\Silent]
      \star\{\overline{l_n:(S_n; S)}\}}
    \\
    { \mu x.S \LTSderives[\Silent] S[\mu x.S/x]}
  \end{gather*}
  \caption{Behavior of a type}
  \label{fig:type-behavior}
\end{figure}
\begin{lemma}\label{lemma:app:unfold-silent}
  If $S' = \Unfold (S)$ is defined, then $S \Wderives S'$.
\end{lemma}
\begin{proof}
  By induction on the number of recursive calls to compute $\Unfold
  (S)$.

  \textbf{Case }$\Unfold (\mu x.S) = \Unfold (S[\mu x.S/x])$: By
  definition, $\mu x.S \LTSderives[\Silent] S[\mu x.S/x] $ and by
  induction $S[\mu x.S/x] \Wderives S'$.

  \textbf{Case }$\Unfold (S_1;S_2)$.

  \textbf{Subcase }$\Unfold (S_1) = \skipk$: By induction, $S_1
  \Wderives \skipk$. By the context rule for behaviors,
  $(S_1; S_2) \Wderives (\skipk; S_2 )$ and by the skip
  rule: $(\skipk; S_2 ) \LTSderives[\Silent] S_2$. Proceed by another
  induction on $S_2$.

  \textbf{Subcase }$\Unfold (S_1) = A$. By induction and the context
  rule.

  \textbf{Subcase }$\Unfold (S_1) = (S_1'; S_1'')$. By induction and
  the context rule, we obtain $(S_1;S_2) \Wderives
  ((S_1'; S_1''); S_2) \LTSderives[\Silent] (S_1'; (S_1''; S_2))$
  where the last step is an application of associativity.

  \textbf{Subcase }$\Unfold (S_1) = \star\{\overline{l_n:S_n} \}$. By
  induction and distributivity.

  \textbf{Remaining cases}. No silent transition needed.
\end{proof}

\begin{lemma}\label{lemma:app:silent-unfold-compatible}
  If $S\LTSderives[\Silent] S'$, then $\Unfold (S) = \Unfold (S')$.
\end{lemma}
\begin{proof}

  \textbf{Case }$\mu x.S \LTSderives[\Silent] S[\mu x.S/x]$:
  $\Unfold (\mu x.S) =  \Unfold (S[\mu x.S/x])$ by definition.

  \textbf{Case }$    {(\star\{\overline{l_n:S_n}\}; S) \LTSderives[\Silent]
    \star\{\overline{l_n:(S_n; S)}\}}$: Immediate by definition of
  $\Unfold$.
  
  \textbf{Case }${((S_1;S_2); S_3) \LTSderives[\Silent] (S_1; (S_2;
    S_3))}$: \\
  Case analysis on the possible outcomes of $\Unfold (S_1)$ and $\Unfold (S_2)$. 

  \textbf{Subcase } $\Unfold (S_1) = \skipk$:
  
  \textbf{Subsubcase }$\Unfold (S_2) = \skipk$:
  
  $\Unfold  (S_1; (S_2;  S_3)) = \Unfold (S_2; S_3) = \Unfold (S_3)$
  and
  $\Unfold ((S_1; S_2); S_3) = \Unfold (S_3)$.

  \textbf{Subsubcase }$\Unfold (S_2) = A$.

  $\Unfold  (S_1; (S_2;  S_3)) = \Unfold (S_2; S_3) = (A; S_3)$
  and
  $\Unfold ((S_1; S_2); S_3) = (A; S_3)$.

  \textbf{Subsubcase }$\Unfold (S_2) = (S_2'; S_2'')$.

  $\Unfold  (S_1; (S_2;  S_3)) = \Unfold (S_2; S_3) = (S_2' ;( S_2''; S_3))$
  and
  $\Unfold ((S_1; S_2); S_3) = (S_2'; (S_2''; S_3))$.

  \textbf{Subsubcase }$\Unfold (S_2) =
  \star\{\overline{l_n:S_n}\}$.

  $\Unfold  (S_1; (S_2;  S_3)) = \Unfold (S_2; S_3) =
  \star\{\overline{l_n:(S_n; S_3)} $
  and
  $\Unfold ((S_1; S_2); S_3) = \star\{\overline{l_n:(S_n; S_3)}$.

  \textbf{Subcase }$\Unfold (S_1) = A$.

  $\Unfold  (S_1; (S_2;  S_3)) =(A; (S_2; S_3))$
  and
  $\Unfold ((S_1; S_2); S_3) = (A; (S_2; S_3))$.

  \textbf{Subcase }$\Unfold (S_1) = (S_1';S_1'')$.

  $\Unfold  (S_1; (S_2;  S_3)) =(S_1'; (S_1''; (S_2; S_3)))$
  and
  $\Unfold ((S_1; S_2); S_3) = (S_1'; (S_1''; (S_2; S_3)))$.

  \textbf{Subcase }$\Unfold (S_1) = \star\{\overline{l_n:S_n}\}$.

  $\Unfold  (S_1; (S_2;  S_3)) =\star\{\overline{l_n:(S_n;
    (S_2\fatsemi S_3))}\} $
  and
  $\Unfold ((S_1; S_2); S_3) = \star\{\overline{l_n:(S_n; (S_2\fatsemi
    S_3)))}\}$.

  \textbf{Case }${(\skipk; S) \LTSderives[\Silent] S}$:
  $\Unfold (\skipk;S) = \Unfold (S) = S'$.

  \textbf{Case }$(S_1; S_2) \LTSderives[\Silent] (S_1';S_2)$
  because $S_1 \LTSderives[\Silent] S_1'$: By induction, $\Unfold
  (S_1) = \Unfold(S_1')$, so that $\Unfold (S_1; S_2) = \Unfold
  (S_1';S_2)$ by definition of unfolding.
\end{proof}
\begin{lemma}\label{lemma:app:silent-unfold}
  If $S \Wderives S'$ and $S'$ is guarded, then $S' = \Unfold (S)$.
\end{lemma}
\begin{proof}
  By induction on the number of silent steps.

  \textbf{Case }$0$. If $S=S'$ is already guarded, then $S$ is a
  fixpoint of $\Unfold$  by Lemma~\ref{lemma:unfold-fixpoints}.

  \textbf{Case }$n>0$. In this case, there is some $S''$ such that $S
  \LTSderives[\Silent] S''$ and $S'' \Wderives S'$ in less than $n$
  steps. Now, $S' = \Unfold (S'') = \Unfold (S)$, the former by
  induction  and the latter by
  Lemma~\ref{lemma:silent-unfold-compatible}.
\end{proof}

% A relation $R \subseteq \stypes \times \stypes$ is a \emph{type
%   simulation} if $(S_1,S_2)\in R$ implies the following conditions:
% %
% \begin{enumerate}
% \item If $\Unfold(S_1) = \skipk$ then $\Unfold(S_2) = \skipk$. 
% \item If $\Unfold(S_1) = (\alpha; S_1')$ then $\Unfold(S_2) =
%   (\alpha;S_2')$ and $(S_1', S_2') \in R$. 
% \item If $\Unfold(S_1) = (!B;S_1')$ then $\Unfold(S_2) = (!B;S_2')$  and $(S_1', S_2') \in R$. 
% \item If $\Unfold(S_1) = (?B;S_1')$ then $\Unfold(S_2) = (?B;S_2')$  and $(S_1', S_2') \in R$. 
% % \item If $\Unfold(S_1) = S_1';S_1''$ then $\Unfold(S_2) = S_2';S_2''$
% %   and both $(S_1',S_2')$  and $(S_1'',S_2'')$ are in $R$.
% \item If $\Unfold(S_1) = \oplus\{l_i\colon S_{1,i}'\}_{i\in I}$ then
%   $\Unfold(S_2) = \oplus\{l_i\colon S_{2,i}'\}_{i\in I}$ and
%   $(S_{1,i}',S_{2,i}')\in R$, for all $i\in I$.
% \item If $\Unfold(S_1) = \&\{l_i\colon S_{1,i}'\}_{i\in I}$ then
%   $\Unfold(S_2) = \&\{l_i\colon S_{2,i}'\}_{i\in I}$ and
%   $(S_{1,i}',S_{2,i}')\in R$, for all $i\in I$.
% \end{enumerate}

\begin{definition}
Define a monotone function~$F$ on $\stypes\times\stypes$ as 
follows. 
%
\begin{align*}
  F (R) &= \{ (S_1, S_2) \mid \Unfold (S_1) = \Unfold (S_2) = \skipk \}
  \\
        &\cup \{ (S_1, S_2) \mid \Unfold (S_1) = \Unfold (S_2) = A \}
  \\
        &\cup \{ (S_1, S_2) \mid
          \begin{array}[t]{@{}l}
            \Unfold (S_1) = (A; S_1'),\\
            \Unfold (S_2) = (A; S_2'), \\
            (S_1', S_2')  \in R \}
          \end{array}
  \\
        &\cup \{ (S_1, S_2) \mid
          \begin{array}[t]{@{}l}
            \Unfold (S_1) = \star\{l_i\colon S_{1,i}'\}_{i\in I}, \\
            \Unfold (S_2) = \star\{l_i\colon S_{2,i}'\}_{i\in I}, \\
            \forall i: (S'_{1,i}, S'_{2,i}) \in R \}
          \end{array}
  % \\
  %       &\cup \{ (S_1, S_2) \mid
  %         \begin{array}[t]{@{}l}
  %           \Unfold (S_1) = \oplus\{l_i\colon S_{1,i}'\}_{i\in I}, \\
  %           \Unfold (S_2) = \oplus\{l_i\colon S_{2,i}'\}_{i\in I}, \\
  %           \forall i: (S'_{1,i}, S'_{2,i}) \in R \}
  %         \end{array}
  % \\
  %       &\cup \{ (S_1, S_2) \mid
  %         \begin{array}[t]{@{}l}
  %           \Unfold (S_1) = \&\{l_i\colon S_{1,i}'\}_{i\in I}, \\
  %           \Unfold (S_2) = \&\{l_i\colon S_{2,i}'\}_{i\in I}, \\
  %           \forall i: (S'_{1,i}, S'_{2,i}) \in R \}
  %         \end{array}
\end{align*}

This function helps define (weak) bisimularity $\TypeEquiv$ for the labelled transition system
$(\stypes, \Sigma, \LTSderives)$ on well-formed session types as a greatest fixpoint: ${\TypeEquiv}
= \GFP (F)$. The 
definition relies on the $\Unfold$ function instead of using silent transitions which is sanctioned
by Lemmas~\ref{lemma:unfold-silent} and~\ref{lemma:silent-unfold}.
\end{definition}

Our goal is to use weak bisimilarity for type
equivalence. To this end, we need to establish that it is indeed an
equivalence relation.\footnote{This development can be extended to a
  relation on $\kindt$, but the extension is entirely standard.}
Reflexivity and symmetry follow directly from the definition. Transitivity requires some work.



\begin{lemma}
  Type  bisimilarity $\TypeEquiv$ is reflexive.
\end{lemma}
% \begin{proof}
%   Obvious.
% \end{proof}

\begin{lemma}
  Type bisimilarity $\TypeEquiv$ is symmetric.
\end{lemma}
% \begin{proof}
%   Obvious.
% \end{proof}

\begin{lemma}
  Type bisimilarity $\TypeEquiv$ is transitive.
\end{lemma}
\begin{proof}
  Let $R  = {\TypeEquiv} \circ {\TypeEquiv}$. Show that $R \subseteq F(R)$, which implies that $R
  \subseteq {\TypeEquiv}$. Observe that ${\TypeEquiv}
  \subseteq R$ because $\TypeEquiv$ is reflexive.

  Suppose that $S_1 \TypeEquiv S_2$ and $S_2 \TypeEquiv S_3$ so that $(S_1, S_3) \in R$.

  \textbf{Case }$\Unfold (S_1) = \skipk$ implies $\Unfold (S_2) =\skipk$, which in turn implies $\Unfold (S_3) =
  \skipk$. Hence, $(S_1, S_3) \in F (\emptyset)$.

  \textbf{Case }$\Unfold (S_1) = A$. Then it must be the case that
  $\Unfold (S_2) = A$ and also $\Unfold (S_3)=A$. Hence, $(S_1, S_3)
  \in F (\emptyset)$.

  \textbf{Case }$\Unfold (S_1) = (A; S_1')$ with $A$ either $\alpha$, $!B$, or $?B$. It must be the
  case that $\Unfold (S_2) = (A; S_2')$ with $(S_1', S_2') \in {\TypeEquiv}$ and further
  $\Unfold (S_3) = (A; S_3')$ with $(S_2', S_3') \in {\TypeEquiv}$. But then $(S_1', S_3') \in R =
  {\TypeEquiv} \circ {\TypeEquiv}$ and hence $(S_1, S_3) \in F (R)$.

  \textbf{Case }$\Unfold (S_1) = \star\{l_i\colon S_{1,i}'\}_{i\in I}$. Then it must be that
  $\Unfold (S_2) = \star\{l_i\colon S_{2,i}'\}_{i\in I}$ with $S_{1,i}' \TypeEquiv S_{2,i}'$, for all
  $i$, and $\Unfold (S_1) = \star\{l_i\colon S_{1,i}'\}_{i\in I}$ with  $S_{2,i}' \TypeEquiv S_{3,i}'$, for all
  $i$. Hence, $(S_{1,i}', S_{3,i}') \in R $, for all $i$, so that $(S_1, S_3) \in F (R)$.
  % \textbf{Case }$\Unfold (S_1) = \&\{l_i\colon S_{1,i}'\}_{i\in I}$. Analogously.
\end{proof}

\begin{lemma}
  \label{lem:app:unfold-type-sim}
  $S \TypeEquiv \Unfold(S)$.
\end{lemma}
\begin{proof}
  Straightforward application of coinduction. We show that $\{ (S, \Unfold (S)) \} \subseteq
  F (\TypeEquiv)$ because 
  of idempotence of $\Unfold$ (Lemma \ref{lemma:unfold-idempotent}) and reflexivity of $\TypeEquiv$.
\end{proof}

\begin{lemma}
  \label{lem:app:skip-elim}
  $\skipk;S_1 \TypeEquiv \skipk;S_2$ iff $S_1 \TypeEquiv S_2$.
\end{lemma}
\begin{proof}
  Immediate because $\Unfold (\skipk; S) = \Unfold (S)$.
\end{proof}

Outside this section, we write $\GEnv \vdash S_1 \TypeEquiv S_2$ to fix the environment $\GEnv$
needed for the formation of $S_1$ and $S_2$. As usual, $\GEnv$ must
not bind recursion variables.


\subsection{Translation to BPA}
\label{sec:translation-bpa}

We define a variant of the unfolding
function for a session type $S$,  $\Unravel(S)$, recursively by cases on the
structure of~$S$. The difference to $\Unfold (S)$ is that the
structure of $S$ is left intact as much as possible.
\begin{enumerate}
\item $\Unravel(\mu x.S) = \Unravel(S[\mu x.S/x])$
\item $\Unravel (S;S') = \left\{%
  \begin{array}{ll}
    \Unravel(S') & \Unravel(S) = \skipk
    \\
    (\Unravel(S); S') & \Unravel(S) \ne \skipk
  \end{array}
  \right.
$
\item $\Unravel (S) = S$ for all other cases
\end{enumerate}


To define the translation to BPA,
we need to show that, for a well-formed session type $S$, $\Unravel
(S)$ is always guarded. That is the output of $\Unravel (S)$ is either
$\skipk$ or it has one
of the following forms:
\begin{align*}
  O &::= A \mid \star\{\overline{l_i:S_i}\} \mid (O; S)
\end{align*}

Now we define the translation of well-formed $S$ to a BPA as
follows. Assume that all recursion variable bindings are unique in the
sense that the set $\{ \mu x_1.S_1, \dots, \mu x_n.S_n\}$ contains all $\mu$-subterms
of $S$ with $S_i : \Productive$ modulo $x_i \equiv \mu x_i.S_i$. 
Define the BPA process equations for $S$ by 
\begin{align*}
  \toBPATop{S} &= \{ \\
  x_0 &= \toBPA{S}, \\
               x_1&= \toBPA{\Unravel (S_1[\mu x_1.S_1/x_1])}, &
                                                                &\dots &
                                                     & \}
\end{align*}
\begin{align*}
  \toBPA{\skipk} &= \varepsilon \\
  \toBPA{A} &= A\\
  \toBPA{S_1;S_2} &= \toBPA{S_1} ; \toBPA{S_2} \\
  \toBPA{\star\{\overline{l_n:S_n}\}} &= (\star l_1;
                                        \toBPA{S_1} + \dots + \star
                                        l_n; \toBPA{S_n}) \\
  \toBPA{\mu x.S} &=
                    \begin{cases}
                      x & S : \Productive \\
                      \varepsilon & S : \Guarded
                    \end{cases}
                   \\
 \toBPA{x} &= x
\end{align*}
It is deliberate that we do \textbf{not} unfold the top-level type $S$
in the defining equation for $x_0$. This equation need not be guarded
because $x_0$ does not appear on the right-hand side of any equation.

Alternative: one could also define the equation extraction by
induction on the kind derivation for $S$ and the right-hand side
extraction by induction on the contractivity judgment. 

\begin{lemma}
  If $\mu x.S$ is a subterm of well-formed $S_0$ with $\Delta \Contr S:\Productive$, then
  $\toBPA{\Unravel (S)}$ is guarded with respect to $\toBPATop{S_0}$. 
\end{lemma}

\begin{proof}
  By Lemma~\ref{lem:unfold-yields-guarded-types}, we know that
  $\Unravel (S)$ yields a term of the form $A$, $A;S'$ or
  $\star\{\overline{l_n:S_n}\}$. Clearly, the translation of a term of
  one of these forms is guarded.

  If $\mu x.S$ has a free variable $x' \ne x$, then
  $\toBPA{\Unravel (S)}$ may have the form $x'$ or $x';S'$.
\end{proof}

It remains to show that $S$ is bisimilar to its
translation. Essentially, we want to prove that the function
$\toBPA{\cdot}$ is a bisimilation when considered as a relation.

\begin{lemma}\label{lemma:app:skip-implies-done}
  If $\Unravel (S) = \skipk$, then $\DONE{S}$.
\end{lemma}
\begin{proof}
  Induction on the number $n$ of recursive calls to $\Unravel$.

  \textbf{Case }$n=0$. $S=\skipk$ and $\DONE{\skipk}$.

  \textbf{Case }$n>0$.

  \textbf{Subcase }$\mu x.S$. $\Unravel (\mu x.S) = \skipk$ 
  because $\Unravel (S[\mu x.S/x]) = \skipk$. By induction,
  $\DONE{S[\mu x.S/x]}$ and by applying the mu-DONE rule $\DONE{\mu
    x.S}$.

  \textbf{Subcase }$S_1;S_2$.
  $\Unravel (S_1;S_2) = \skipk$ because $\Unravel (S_1) =
  \Unravel (S_2) = \skipk$. By induction $\DONE{S_1}$ and
  $\DONE{S_2}$. By rule seq-DONE $\DONE{S_1;S_2}$.
\end{proof}

\begin{lemma}\label{lemma:app:s=unr-s}
  Let $S$ be closed, well-formed. \\
  Then
  $\toBPA{S} \TypeEquiv \toBPA{\Unravel (S)}$.
\end{lemma}
\begin{proof}
  Induction on the number $n$ of recursive calls to $\Unravel$.

  \textbf{Case }$n=0$. In this case, $S$ must be $\skipk$, $A$, or
  $\star\{\overline{l_i:S_i}\}$ and the claim is immediate.

  \textbf{Case }$n>0$. There are two subcases.

  \textbf{Subcase }$\mu x.S$. Then $\toBPA{\mu x.S} = x$ and there is
  an equation $x = \toBPA{\Unravel(S[\mu x.S])}$. Now, $x$ is
  obviously bisimilar to $\toBPA{\Unravel(S[\mu x.S])}$.

  \textbf{Subcase }$S_1;S_2$. If $\Unravel (S_1) = \skipk$, then
  $\DONE{S_1}$ and hence $\DONE{\toBPA{S_1}}$. Furthermore, $\Unravel
  (S_1;S_2) = \Unravel (S_2)$ and, by induction, $\toBPA{S_2}
  \TypeEquiv \toBPA{\Unravel(S_2)}$. The result follows because
  $\toBPA{S_1;S_2} = \toBPA{S_1};\toBPA{S_2} \TypeEquiv \toBPA{S_2}$
  and $\toBPA{\Unravel(S_2)} = \toBPA{\Unravel (S_1;S_2)}$.

  If $\Unravel (S_1) =: S_u \ne \skipk$, then $\Unravel (S_1;S_2) =
  S_u; S_2$.
  By induction, we know that $\toBPA{S_1} \TypeEquiv \toBPA{S_u}$ and
  as bisimilarity is a congruence  $\toBPA{S_1;S_2} \TypeEquiv
  \toBPA{S_u}; \toBPA{S_2} = \toBPA{\Unravel (S_1;S_2)}$.
\end{proof}

\begin{lemma}\label{lemma:app:bisimulation-BPA-forwards}
  Suppose $S$ is a well-formed closed session type.
  If $S \LTSderives S'$, then $\toBPATop{S}
  \LTSderives \toBPATop{S'}$.
\end{lemma}
\begin{proof}
  By induction on  $S \LTSderives S'$.

  \textbf{Case }$A \LTSderives[A] \skipk$.
  In this case $\toBPATop{A} = \{ x_0 = A \} \LTSderives[A] \{ x_0 =
  \varepsilon \} = \toBPATop{\skipk}$.

  \textbf{Case }$\star\{\overline{l_i:S_i}\} \LTSderives[\star l_i]
  S_i$.
  In this case $\toBPATop{\star\{\overline{l_i:S_i}\}} = \{ x_0 =
  (\dots+ \star l_i; \toBPA{S_i} + \dots) \} \LTSderives [\star l_i]
  \{ x_0 = \toBPA{S_i} \} =  \toBPATop{S_i}$.

  \textbf{Case }$\frac{S_1 \LTSderives S_1'}{S_1; S_2
    \LTSderives S_1';S_2}$.
  In this case $\toBPATop{S_1;S_2} = \{ x_0 = E_1;E_2, \dots \}$ where
  $E_i = \toBPA{S_i}$ for $i=1,2$.
  Because $S_1 \LTSderives S_1'$,
  we obtain by induction that $\toBPATop{S_1} = \{ x_0 = E_1, \dots \}
  \LTSderives \toBPATop{S_1'} = \{ x_0 = E_1', \dots
  \}$. Therefore, $\{ x_0 = E_1;E_2, \dots \} \LTSderives \{ x_0 =
  E_1';E_2, \dots \} = \toBPATop{S_1'; S_2}$.

  \textbf{Case }$\frac{\DONE{S_1} \quad S_2 \LTSderives S_2'}{S_1; S_2
    \LTSderives S_2'}$.
  In this case $\toBPATop{S_1;S_2} = \{ x_0 = E_1;E_2, \dots \}$ where
  $E_i = \toBPA{S_i}$ for $i=1,2$.
  It is easy to see that $\DONE{S_1}$ implies $\DONE{\toBPA{S_1}}$,
  that is, $\DONE{E_1}$.
  Because $S_2 \LTSderives S_2'$,
  we obtain by induction that $\toBPATop{S_2} = \{ x_0 = E_2, \dots \}
  \LTSderives \toBPATop{S_2'} = \{ x_0 = E_2', \dots
  \}$. 
  Therefore, $\{ x_0 = E_1;E_2, \dots \} \LTSderives \{ x_0 =
  E_2', \dots \} = \toBPATop{S_2'}$.

  \textbf{Case }$\frac{S[\mu x.S/x] \LTSderives S'}{\mu x.S
    \LTSderives S'}$.
  In this case
  $\toBPATop{\mu x.S} = \{ x_0 = x, x = E, \dots \}$ with $E = \toBPA{\Unravel
    (S[\mu x.S/x])}$.
  By induction, $\toBPATop{S[\mu x.S/x]} \LTSderives
  \toBPATop{S'}$.
  Now $\toBPATop{S[\mu x.S/x]} = \{ x_0 = \toBPA{S[\mu x.S/x]}, \dots
  \}$ which proves the claim because $x_0 \TypeEquiv E$ by
  Lemma~\ref{lemma:s=unr-s}. 
\end{proof}

\clearpage
\begin{lemma}\label{lemma:app:bpa-unr-s}
  Suppose that $\toBPA{\Unravel
    (S)} \LTSderives E'$. Then $S \LTSderives S'$
  and $E' = \toBPATop{S'}$.
\end{lemma}
\begin{proof}
  By induction on the number $n$ of recursive calls of $\Unravel$.

  \textbf{Case }$n=0$.

  \textbf{Subcase }$S=\skipk$. Contradictory.

  \textbf{Subcase }$S=A$. Then $a=A$, $E'=\varepsilon$, and $S' =
  \skipk$.

  \textbf{Subcase }$S = \star\{\overline{l_i:S_i}\}$. Then $a = \star
  l_i$, $E' = \toBPA{S_i}$, and $S' = S_i$.

  \textbf{Case }$n>0$.

  \textbf{Subcase }$S = S_1;S_2$.
  If $\Unravel (S_1) = \skipk$, then $\Unravel (S) = \Unravel
  (S_2)$ with less than $n$ calls. As $\toBPA{\Unravel (S_2)} \LTSderives E'$, induction yields
  that $S_2 \LTSderives S'$ and $E' = \toBPATop{S'}$.
  As $\Unravel (S_1) = \skipk$, we know that $\DONE{S_1}$. Hence,
  ${S_1;S_2} \LTSderives S'$ and $E' = \toBPATop{S'}$.

  If $\Unravel (S_1)\ne \skipk$, then consider $\toBPA{\Unravel({S_1});
    S_2} \LTSderives E'$ because $\toBPA{\Unravel({S_1})} \LTSderives
  E_1'$, so that induction yields some $S_1'$ such that $S_1
  \LTSderives S_1'$ and $E_1' = \toBPATop{S_1'}$.

  \textbf{Subcase }$\mu x.S$.
  $\Unravel (\mu x.S) = \Unravel (S[\mu x.S/x])$ with one less
  invocation. As $\toBPA{\Unravel (S[\mu x.S/x])}
  \LTSderives E'$, induction yields that  $S[\mu x.S/x] \LTSderives
  S'$ with $E' = \toBPATop{S'} $.
\end{proof}

\begin{lemma}\label{lemma:app:bisimulation-BPA-backwards}
  Suppose that $S$ is well-formed and let $\BPAprocess = \toBPATop{S}$
  and $\BPAprocess \LTSderives \BPAprocess'$.

  There is some $S'$ such that $S \LTSderives S'$ and $\BPAprocess' = \toBPATop{S'}$.
\end{lemma}
\begin{proof}
  By induction on $S$.

  \textbf{Case }$\skipk$. Contradictory.

  \textbf{Case }$A$. For $\BPAprocess$, ${A \LTSderives \varepsilon
  }$. Choose $S' = \skipk$.

  \textbf{Case }$\star\{\overline{l_i:S_i}\}$. For $\BPAprocess$,
  ${\sum \overline{\star l_i; \toBPA{S_i}} \LTSderives[\star l_i]
    \toBPA{S_i} }$. Choose $S' = S_i$.

  \textbf{Case }$S_1;S_2$.
  If $\toBPA{S_1} \LTSderives E_1'$,
  then $S_1 \LTSderives S_1'$ and $E_1' = \toBPA{S_1'}$, by induction.
  Now, $\toBPA{S_1;S_2} \LTSderives E_1'; \toBPA{S_2} = \toBPA{S_1';
    S_2}$. Choose $S' = S_1';S_2$.

  If $\DONE{\toBPA{S_1}}$ and $\toBPA{S_2} \LTSderives E_2'$,
  then $\DONE{S_1}$ and $S_2 \LTSderives S_2'$ and $E_2' =
  \toBPA{S_2'}$, by induction. Now, $\toBPA{S_1;S_2} \LTSderives E_2' = \toBPA{S_2'}$. Choose $S' = S_2'$.

  \textbf{Case }$\mu x.S$.
  $\BPAprocess = \toBPATop{\mu x.S} = \{ x_0 = x, x = \toBPA{\Unravel
    (S[\mu x. S/x])} \}$. If $\BPAprocess \LTSderives \BPAprocess'$,
  then it must be because $\toBPA{\Unravel
    (S[\mu x. S/x])} \LTSderives E'$. Use Lemma~\ref{lemma:bpa-unr-s}
  to establish the claim.
\end{proof}

\begin{theorem}
  Suppose that $S$ is well-formed and let $\BPAprocess = \toBPATop{S}$.
  \begin{enumerate}
  \item If $ S \LTSderives[a] S'$, then $\BPAprocess \BPAderives[a]
    \BPAprocess'$ with $\BPAprocess' = \toBPATop{S'}$.
  \item If $\BPAprocess \BPAderives[a] \BPAprocess'$, then $S
    \LTSderives[a] S'$ with  $\BPAprocess' = \toBPATop{S'}$.
  \end{enumerate}
\end{theorem}
\begin{proof}
  By Lemmas~\ref{lemma:bisimulation-BPA-forwards} and~\ref{lemma:bisimulation-BPA-backwards}.
\end{proof}

%%% Local Variables:
%%% mode: latex
%%% TeX-master: "main"
%%% End:


%%% Local Variables:
%%% mode: latex
%%% TeX-master: "main"
%%% End:



\end{document}

%%% Local Variables:
%%% mode: latex
%%% TeX-master: t
%%% End:
