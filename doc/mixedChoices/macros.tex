% The language
\newcommand{\freest}{\textsc{FreeST}}

% notes
\newcommand{\todo}[1]{[{\color{blue}\textbf{#1}}]}

% Keywords
\newcommand{\keyword}[1]{\mathsf{#1}}
\newcommand{\link}{\keyword{lin}}
\newcommand{\unk}{\keyword{un}}

% Kinds
\newcommand{\prekind}{\upsilon}
\newcommand{\kind}{\kappa}
\newcommand{\kinds}{\mathcal S}
\newcommand{\kindt}{\mathcal T}
\newcommand{\kindsch}{\mathcal C}
\newcommand{\kindk}[2]{{#1}^{#2}}

% Multiplicity
\newcommand{\un}{\unk}
\newcommand{\lin}{\link}

% Grammars
\newcommand{\grmeq}{\; ::= \;}
\newcommand{\grmor}{\;\mid\;}

% type constructors
% \newcommand\tcBase{B}
% \newcommand\tcLolli\multimap
% \newcommand\tcFun\to
% \newcommand\tcBang{\mathop!}

% Keywords
\newcommand{\reck}{\keyword{rec}}
\newcommand{\dualofk}{\keyword{dualof}}
\newcommand{\skipk}{\keyword{skip}}
\newcommand{\ifk}{\keyword{if}}
\newcommand{\thenk}{\keyword{then}}
\newcommand{\elsek}{\keyword{else}}
\newcommand{\letk}{\keyword{let}}
\newcommand{\ink}{\keyword{in}}
\newcommand{\newk}{\keyword{new}}
\newcommand{\sendk}{\keyword{send}}
\newcommand{\sendck}{\keyword{send}}
\newcommand{\receivek}{\keyword{receive}}
\newcommand{\selectk}{\keyword{select}}
\newcommand{\matchk}{\keyword{match}}
\newcommand{\withk}{\keyword{with}}
\newcommand{\forkk}{\keyword{fork}}
\newcommand{\casek}{\keyword{case}}
\newcommand{\ofk}{\keyword{of}}
\newcommand{\typek}{\keyword{type}}
\newcommand{\tend}{\keyword{end}}

% Types
\newcommand{\skipt}{\skipk}
\newcommand{\semit}[2]{#1;#2}
\newcommand{\sout}[1]{\mathop!#1}
\newcommand{\sint}[1]{\mathop?#1}
\newcommand{\msgt}[1]{\sharp{#1}}
\newcommand{\choicet}[1]{\star\{#1\}}
\newcommand{\ichoicet}[1]{\oplus{#1}}
\newcommand{\echoicet}[1]{\&{#1}}
\newcommand{\datat}[1]{[{#1}]}
\newcommand{\funt}[3][m]{{#2}\rightarrow_{#1}{#3}}
\newcommand{\unfunt}[2]{#1\rightarrow#2}
\newcommand{\linfunt}[2]{#1\multimap#2}
\newcommand{\pairt}[2]{(#1,\,#2)}
\newcommand{\rect}[2]{\mu{#1}.{#2}}
\newcommand{\forallt}[2]{\forall\,#1\Rightarrow#2}
\newcommand{\dualoft}[1]{\dualofk\,#1}
\newcommand{\unitt}{()}
\newcommand{\intt}{\keyword{Int}}
\newcommand{\chart}{\keyword{Char}}
\newcommand{\boolt}{\keyword{Bool}}


% Environments
\newcommand{\Empty}{\varepsilon}
\newcommand{\splitContext}[2]{#1 \circ #2}

% \newcommand\Delta{\Delta}
% \newcommand\Gamma{\Gamma}

% Expressions
% \newcommand{\vare}[1]{#1}
\newcommand{\unlete}[3]{\letk \; #1 = #2 \; \ink \; #3} 
\newcommand{{\abs}}[4][m]{\lambda{#2}\colon{#3}\rightarrow_{#1}{#4}}
% \newcommand{\appe}[2]{#1#2}
\newcommand{\tappe}[2]{#1[#2]}
\newcommand{\conde}[3]{\ifk\;#1\;\thenk\;#2\;\elsek \; #3}
\newcommand{\paire}[2]{(#1,#2)}
\newcommand{\binlete}[4]{\letk\;#1, #2 = #3\;\ink\;#4}
\newcommand{\newe}[1]{\newk\;#1}
\newcommand{\sende}[1]{\sendk\;#1}
\newcommand{\sendce}[1]{\sendk\;#1}
\newcommand{\recve}[1]{\receivek\;#1}
\newcommand{\selecte}[2]{\selectk\;#1\;{#2}}
\newcommand{\matche}[2]{\matchk\;#1\;\withk\;#2}
\newcommand{\forke}[1]{\forkk\;#1}
\newcommand{\casee}[2]{\casek\;#1\;\ofk\;#2}

% Sequents
\newcommand{\isTypeNoKind}[2][\Delta]{{#1} \vdash {#2}\;\typek}
\newcommand{\synthetise}{\Rightarrow}
\newcommand{\checkagainst}{\Leftarrow}
\newcommand{\isType}[3][\Delta]{{#1} \vdash {#2} : {#3}}
\newcommand{\algkindout}[3][\Delta]{{#1} \Alg {#2} \synthetise{ #3}}
\newcommand{\algkindin}[3][\Delta]{{#1} \Alg {#2} \checkagainst {#3}}
\newcommand{\isSubkind}[2]{{#1} < {#2}}
\newcommand{\isSubType}[2]{{#1} <: {#2}}
\newcommand{\algtypeout}[4][\Delta;\Gamma]{{#1} \Alg {#2} \synthetise {#3};{#4}}
\newcommand{\algtypein}[4][\Delta;\Gamma]{{#1} \Alg {#2}\checkagainst {#3}; {#4}}
\newcommand{\ctxequiv}[3][\Delta]{{#1} \vdash \Equiv{#2}{#3}}
\newcommand{\typeequiv}[3][\Delta]{{#1} \vdash \Equiv{#2}{#3}}
\newcommand{\isLin}[2][\Delta]{\isType[#1]{#2}\link}
\newcommand{\isUn}[2][\Delta]{\isType[#1]{#2}\unk}
\newcommand\Alg{\vdash}
\newcommand{\subt}{<:}
\newcommand{\isContr}[2][\Delta]{{#1}\vdash_{\!\!\mathsf c}{#2}}
\newcommand{\isDone}[1]{{#1}\checkmark}

\newcommand{\codeGen}[4][\varphi]{{#1} \Alg_{#2} {#3} \Rightarrow {#4}}
\newcommand{\orTwo}[2]{{#1}\,||\,{#2}}
\newcommand{\orThree}[3]{{#1}\,||\,{#2}\,||\,{#3}}
\newcommand{\orFour}[4]{{#1}\,||\,{#2}\,||\,{#3}\,||\,{#4}}
\newcommand{\true}{\mbox{\scriptsize True}}
\newcommand{\True}{\mbox{True}}

\newcommand{\algprocin}[2]{{#1} \Alg {#2}}
\newcommand{\translates}[1]{ \llbracket {#1} \rrbracket}

% Labelled transition system
\newcommand{\LTSderives}[1][a]{\stackrel{#1}{\longrightarrow}}

% Operators
\newcommand\extract[1]{\leadsto_{#1}}% \rightlsquigarrow}
\newcommand{\subs}[3]{{#1}[{#2}/{#3}]} % subs #2 for #3 in #1
\newcommand{\dual}[1]{\overline{#1}}
\newcommand{\join}{\sqcup}
\newcommand{\ctxdiff}[4][\Delta]{{#1}\vdash{#2}\div{#3}={#4}}
\newcommand\Equiv[2]{#1\,\sim\,#2}
\newcommand\done[1]{#1 \checkmark}
\newcommand{\unfoldk}{\keyword{unfold}}
\newcommand\TypeEquiv{\sim}

% Theorems
\newtheorem{theorem}{Theorem}
\newtheorem{lemma}{Lemma}

%metafunctions
\newcommand\dom{\mathsf{dom}}
\newcommand\free{\mathsf{free}}

% Processes
\newcommand\prock{\mathsf{proc}}
\newcommand\proc[1]{\mathcal{#1}}

% choice
\newcommand\choice{\mathcal{M}}
\newcommand\choiceSign{\mathop +}
\newcommand\lbl{l}


% Colour

\newcommand{\Blue}[1]{\textcolor{blue}{#1}}
\newcommand{\Red}[1]{\textcolor{red}{#1}}
\newcommand{\Brown}[1]{\textcolor{brown}{#1}}
\newcommand{\highlight}[1]{\Blue{#1}}

% ECLIPSE LOOK

\newcommand\Small{\small}
%\newcommand\Small{\fontsize{7.5}{8}\selectfont} 

\definecolor{darkviolet}{rgb}{0.5,0,0.4}
\definecolor{darkgreen}{rgb}{0,0.4,0.2} 
\definecolor{darkblue}{rgb}{0.1,0.1,0.9}
\definecolor{darkgrey}{rgb}{0.5,0.5,0.5}
\definecolor{lightblue}{rgb}{0.4,0.4,1}

\lstdefinestyle{eclipse}{
  breaklines=true,
  basicstyle=\sffamily\Small,
  emphstyle=\color{red}\bfseries, 
  keywordstyle=\color{darkviolet}\bfseries,
  commentstyle=\color{darkgreen},
  stringstyle=\color{darkblue},
  numberstyle=\color{darkgrey},%\lstfontfamily,
  emphstyle=\color{red},
  % get also javadoc style comments
  morecomment=[s][\color{lightblue}]{/**}{*/},
  %columns=fullflexible, %spaceflexible, %flexible, fullflexible             
  %  escapeinside=`',
  %  escapechar=@,
  showstringspaces=false,
  numbers=left,
  tabsize=2
}

\lstdefinestyle{eclipse-Haskell}{
  breaklines=true,
  basicstyle=\sffamily\Small,
  emphstyle=\color{red}\bfseries, 
  keywordstyle=\color{darkviolet}\bfseries,
  commentstyle=\color{darkgreen},
  stringstyle=\color{darkblue},
  emphstyle=\color{red},
  % get also javadoc style comments
  morecomment=[s][\color{lightblue}]{/**}{*/},
  %columns=fullflexible, %spaceflexible, %flexible, fullflexible             
  %  escapeinside=`',
  %  escapechar=@,
  showstringspaces=false,
  numbers=none,
  tabsize=2
}

\lstdefinelanguage{freest}{
  style=eclipse,
  morekeywords=[1]{Int, Char, Bool, Skip, type, dualof, forall, rec, let, in, if, then, else, new, send, receive, select, fork, case, of, data, match, with, True, False},
  sensitive=true,
  literate=
  {->}{$\rightarrow$}2
  {-o}{$\multimap$}2
  {=>}{$\Rightarrow$}2
  {alpha}{$\alpha$}1,
  breaklines=true,
  morecomment=[l]{--},%
  morecomment=[s]{{-}{-}},%
  morestring=[b]',%
  morestring=[b]",%
  morestring=[s]{`}{`},%
}

\lstdefinelanguage{freestMC}{
  style=eclipse,
  language=freest,
  morekeywords=[1]{choice, end},
  escapeinside=||,
  morecomment=[l]--,%
}

\lstset{
  language=freestMC,
  numbers=none
}
 
%%% Local Variables:
%%% mode: latex
%%% TeX-master: "main"
%%% End:
