\documentclass[10pt]{beamer}

\usetheme{metropolis}

\usepackage{booktabs}
\usepackage{array}
\usepackage[utf8]{inputenc}
\usepackage{pgfplots}
\usepackage{stmaryrd}
\usepackage{mathrsfs}
\usepackage{amsmath}
\usepackage{stmaryrd}
\usepackage{amsfonts}
\usepackage{movie15}
\usepackage{amssymb}
\usepackage{accsupp}
\usepackage{amsthm}
\usepackage{verbatim,cprotect}
\usepackage{listings}
\usepackage{hyperref}
\usepackage{enumerate}
\usepackage{tikz}
\usepackage[percent]{overpic}
\usepackage{tcolorbox}
\usepackage{color}
\usepackage{xspace}
\usepackage{alltt}
\usepackage{flushend}
\usepackage{graphicx}
\usepackage{wrapfig}
\usepackage[underline=false]{pgf-umlsd}

%%% Local Variables:
%%% mode: latex
%%% TeX-master: "cfst-inforum18"
%%% End:
      
 
% THEME
\newtcolorbox{mybox}{colback=orange!5!white,colframe=orange!75!black}
\newtcolorbox{myboxazul}{colback=teal!5!white,colframe=teal!75!black}

% TIKZ 
\usetikzlibrary{positioning}
\usetikzlibrary{shapes,arrows}
\usepgfplotslibrary{dateplot}
\tikzstyle{block} = [rectangle, draw, 
    text width=5cm, text centered, rounded corners, minimum height=4em]
\tikzstyle{block2} = [rectangle, draw, 
    text width=10cm, text centered, rounded corners, minimum height=4em]
\tikzstyle{line} = [draw, -latex']

\newenvironment<>{varblock}[2][.9\textwidth]{%
  \setlength{\textwidth}{#1}
  \begin{actionenv}#3%
    \def\insertblocktitle{#2}%
    \par%
    \usebeamertemplate{block begin}}
  {\par%
    \usebeamertemplate{block end}%
  \end{actionenv}}

\newenvironment{changemargin}[3]{%
\begin{list}{}{%
\setlength{\leftmargin}{#1}%
\setlength{\rightmargin}{#2}%
\setlength{\topmargin}{#3}%
}%
\item[]}
{\end{list}}

% session constructors
\newcommand{\intk}{\keyword{int}}
\newcommand{\skipk}{\keyword{skip}}

% The language
\newcommand{\freest}{\textsc{FreeST}}

% notes
\newcommand{\todo}[1]{[{\color{blue}\textbf{#1}}]}

% Keywords
\newcommand{\keyword}[1]{\mathsf{#1}}
\newcommand{\link}{\keyword{lin}}
\newcommand{\unk}{\keyword{un}}

% Kinds
\newcommand\prekind{\upsilon}
\newcommand{\stypes}{\mathcal S}
\newcommand\kinds{\stypes}
\newcommand{\types}{\mathcal T}
\newcommand\kindt{\types}
\newcommand\kindsch{\mathcal C}
\newcommand\kind{\kappa}

% Multiplicity
\newcommand\Un{\ensuremath{\mathbf{u}}} % \infty
\newcommand\Lin{\ensuremath{\mathbf{l}}} % 1 

% Grammars
\newcommand{\grmeq}{\; ::= \;}
\newcommand{\grmor}{\;\mid\;}

% type constructors
\newcommand\tcBase{B}
\newcommand\tcLolli\multimap
\newcommand\tcFun\to
\newcommand\tcBang{\mathop!}

% Keywords for types
\newcommand\kRec{\keyword{rec}}
\newcommand\kForall{\keyword{forall}}

% Types
\newcommand{\tskip}{\keyword{Skip}}
\newcommand\tSemi[2]{#1;#2}
\newcommand\tOut[1]{\tcBang#1}
\newcommand\tIn[1]{?#1}
\newcommand{\tMsg}[1]{\sharp{#1}}
\newcommand\tIChoice[1]{\oplus{#1}}
\newcommand\tEChoice[1]{\&{#1}}
\newcommand{\tChoice}[1]{\star{#1}}
\newcommand{\tData}[1]{[{#1}]}
\newcommand\tUnFun[2]{#1\tcFun#2}
\newcommand\tLinFun[2]{#1\tcLolli#2}
\newcommand\tPair[2]{(#1,\,#2)}
\newcommand\tDatatype[1]{{[#1]}}
\newcommand\tRec[2]{\kRec\,#1\,.\,#2}
%\newcommand\tForall[2]{\forall\,#1\,.\,#2}
%\newcommand\tForall[2]{\kForall\,#1\,=>\,#2}
\newcommand\tForall[2]{\forall\,#1\Rightarrow#2}
% Basic Types
\newcommand{\unite}{()}
\newcommand{\inte}{\keyword{Int}}
\newcommand{\chare}{\keyword{Char}}
\newcommand{\boole}{\keyword{Bool}}

\newcommand\tRecK[2]{\kRec\,#1\,.\,#2}
% Environments
\newcommand{\Empty}{\varepsilon}
\newcommand\emptyEnv{\Empty}
\newcommand\kindEnv{\Delta}
\newcommand\varEnv{\Gamma}

% Variables
\newcommand\vare[1]{#1}
\newcommand\unlete[3]{\keyword{let} \; #1 = #2 \; \keyword{in} \; #3} 

% Applications
\newcommand\appe[2]{#1#2}
\newcommand\tappe[2]{#1[#2]}

% Conditional
\newcommand\conditionale[3]{\keyword{if}\;#1\;\keyword{then}\;#2\;\keyword{else} \; #3}

% Goal
\newcommand\Goal{\vdash}

% Pairs
\newcommand\paire[2]{(#1,#2)}
\newcommand\binlete[4]{\keyword{let}\;#1, #2 = #3\;\keyword{in}\;#4}

% Session Types
\newcommand\newe[1]{\keyword{new}\;#1}
\newcommand\sende[2]{\keyword{send}\;#1\; #2}
\newcommand\sendce[1]{\keyword{send}\;#1}
\newcommand\recve[1]{\keyword{receive}\;#1}
\newcommand\selecte[2]{\keyword{select}\;#1\;{#2}}
\newcommand\matche[2]{\keyword{match}\;#1\;\keyword{with}\;#2}

% Fork
\newcommand\forke[1]{\keyword{fork}\;#1}

% Datatypes
\newcommand{\ctrcte}{C}
\newcommand\casee[2]{\keyword{case}\;#1\;\keyword{of}\;#2}

% Sequents
\newcommand{\isType}[3][\Delta]{{#1} \vdash {#2} : {#3}}
\newcommand{\algkindout}[3][\kindEnv]{{#1} \Alg {#2} \shortrightarrow{ #3}}
\newcommand{\algkindin}[3][\kindEnv]{{#1} \Alg {#2} \shortleftarrow {#3}}
\newcommand{\subkind}[2]{{#1} <: {#2}}
\newcommand{\algtypeout}[4][\kindEnv;\varEnv]{{#1} \Alg {#2} \shortrightarrow {#3};{#4}}
%\newcommand{\algtypein}[4][\kindEnv;\varEnv]{{#1} \Alg {#2}\colon {#3}\shortrightarrow {#4}}
\newcommand{\algtypein}[4][\kindEnv;\varEnv]{{#1} \Alg {#2}\shortleftarrow {#3}; {#4}}
\newcommand{\ctxequiv}[3][\kindEnv]{{#1} \vdash \Equiv{#2}{#3}}
\newcommand{\typeequiv}[3][\kindEnv]{{#1} \vdash \Equiv{#2}{#3}}
\newcommand{\isqualifier}[3][\kindEnv]{{#1} \vdash {#2}\colon{#3}}
\newcommand{\isLin}[2][\kindEnv]{\isqualifier[#1]{#2}\link}
\newcommand{\isUn}[2][\kindEnv]{\isqualifier[#1]{#2}\unk}
\newcommand{\contractive}[2][\kindEnv]{{#1} \vdash_{\textsf c} {#2}}
%\newcommand\Alg{\vdash_{\textsf a}}
\newcommand\Alg{\vdash}

% Operators
\newcommand\Extract[1]{\leadsto_{#1}}% \rightlsquigarrow}
\newcommand{\subs}[3]{[{#1}/{#2}]{#3}}
\newcommand\dual[1]{\overline{#1}}

% Predicates
%\newcommand\Equiv[2]{#1\,\thicksim\,#2}
\newcommand\Equiv[2]{#1\,\sim\,#2}

% Colour

\newcommand{\Blue}[1]{\textcolor{blue}{#1}}
\newcommand{\Red}[1]{\textcolor{red}{#1}}
\newcommand{\Brown}[1]{\textcolor{brown}{#1}}
\newcommand{\highlight}[1]{\Blue{#1}}

% ECLIPSE LOOK

\newcommand\Small{\small}
%\newcommand\Small{\fontsize{7.5}{8}\selectfont} 

\definecolor{darkviolet}{rgb}{0.5,0,0.4}
\definecolor{darkgreen}{rgb}{0,0.4,0.2} 
\definecolor{darkblue}{rgb}{0.1,0.1,0.9}
\definecolor{darkgrey}{rgb}{0.5,0.5,0.5}
\definecolor{lightblue}{rgb}{0.4,0.4,1}

\lstdefinestyle{eclipse}{
  breaklines=true,
  basicstyle=\sffamily\Small,
  emphstyle=\color{red}\bfseries, 
  keywordstyle=\color{darkviolet}\bfseries,
  commentstyle=\color{darkgreen},
  stringstyle=\color{darkblue},
  numberstyle=\color{darkgrey},%\lstfontfamily,
  emphstyle=\color{red},
  % get also javadoc style comments
  morecomment=[s][\color{lightblue}]{/**}{*/},
  %columns=fullflexible, %spaceflexible, %flexible, fullflexible             
  %  escapeinside=`',
  %  escapechar=@,
  showstringspaces=false,
  numbers=left,
  tabsize=2
}

\lstdefinestyle{eclipse-Haskell}{
  breaklines=true,
  basicstyle=\sffamily\Small,
  emphstyle=\color{red}\bfseries, 
  keywordstyle=\color{darkviolet}\bfseries,
  commentstyle=\color{darkgreen},
  stringstyle=\color{darkblue},
  emphstyle=\color{red},
  % get also javadoc style comments
  morecomment=[s][\color{lightblue}]{/**}{*/},
  %columns=fullflexible, %spaceflexible, %flexible, fullflexible             
  %  escapeinside=`',
  %  escapechar=@,
  showstringspaces=false,
  numbers=none,
  tabsize=2
}

\lstdefinelanguage{freest}{
  style=eclipse,
  morekeywords=[1]{Int, Char, Bool, Skip, type, dualof, forall, rec, let, in, if, then, else, new, send, receive, select, fork, case, of, data, match, with, True, False},
  sensitive=true,
  literate=
  {->}{$\rightarrow$}2
  {-o}{$\multimap$}2
  {=>}{$\Rightarrow$}2
  {alpha}{$\alpha$}1,
  breaklines=true,
  morecomment=[l]{--},%
  morecomment=[s]{{-}{-}},%
  morestring=[b]',%
  morestring=[b]",%
  morestring=[s]{`}{`},%
}

\lstset{
  language=freest,
  numbers=none
}
 
%%% Local Variables:
%%% mode: latex
%%% TeX-master: "main"
%%% End:

%\lstset{identifierstyle=\color{identifierColor}}

\title[Checking the equivalence of context-free session types]{Checking the equivalence of context-free session types}

\date{
\vspace*{1cm}
\begin{center}
	January 2019
\end{center}}
\author[A.Mordido, V.Vasconcelos]{\underline{Andreia Mordido} and Vasco T. Vasconcelos}
\institute[LASIGE, Faculdade de Ci\^encias, ULisboa]{LASIGE, Faculdade de Ci\^encias, Universidade de Lisboa \\\\
}
%\titlegraphic{\hfill\includegraphics[height=1.5cm]{logo.png}}

\begin{document}
\lstset{language=Haskell}

\maketitle

\begin{frame}[fragile]{Motivation}
\vspace*{5mm}

\begin{lstlisting}[language=CFST]
data Tree = Leaf | Node Int Tree Tree
\end{lstlisting}

\begin{lstlisting}[language=CFST]
sendTree Leaf c =
  select Leaf c
sendTree (Node x l r) c =
  let c1 = select Node c
      c2 = send x c1
      c3 = sendTree l c2
      c4 = sendTree r c3
   in c4
\end{lstlisting}
  
\vspace*{-1.8cm}
\hfill \includegraphics[height=4cm]{img/tree}
\end{frame}

%\begin{frame}{Table of contents}
%    \frametitle{Layout}
%    \setbeamertemplate{section in toc}[sections numbered]
%  \tableofcontents[hideallsubsections]
%\end{frame}

\begin{frame}{Plan}

  	\metroset{block=fill}
  	\begin{definition}[Type equivalence problem] 
    	\smallskip 
    	Given any context-free session types $S$ and $T$, the type equivalence problem consists in deciding if types $S$ and $T$ are equivalent, i.e., $S\sim T$.
  	\end{definition}

  	\begin{itemize}
    	\item Implement an algorithm that decides the \emph{type equivalence problem}.
    	\item Prove its soundness and completeness w.r.t. the metatheory of context-free session types proposed by Thiemann and Vasconcelos.
    	\item Provide results on complexity.
  	\end{itemize}
\end{frame}

%\section{Algorithm for checking the equivalence of context-free session types}

\begin{frame}{Algorithm for checking the equivalence of CFST \hfill {\color{mLightBrown}Main stages}}
	\begin{center}
    	\begin{tikzpicture}[node distance = 2.5cm, auto]
      		% Place nodes
      		\node [block2] (typeToGrammar) {{\bf Convert types to a grammar}\\ Translates types into a (finite) set of productions}; 
      		\node [block2,below=5mm of typeToGrammar] (prune) {{\bf Prune unnormed productions}\\ Streamlines the grammar by pruning unnormed productions};%
      		\node [block2,below=5mm of prune] (simplifyExpand) {{\bf Simplify and expand}\\ Alternates between simplification and expansion operations, until reaching a successful branch in the expansion tree or concluding that all branches are unsuccessful };
      		% Draw edges
      		\path [line] (typeToGrammar) -- (prune);
      		\path [line] (prune) -- (simplifyExpand);
   		\end{tikzpicture}
  	\end{center}
\end{frame}


\begin{frame}{Convert types to a grammar}
  
  	\vspace*{-1cm}
  	\hspace*{-3mm}\parbox{8cm}{
  	\vspace*{-2.5cm}
  	  	We consider a finite set of productions $X \xrightarrow{a} \vec Y$ where:
   	\begin{itemize}
     	\item $X, Y$ represent \emph{non-terminal variables}
     	\item $a$ is a \emph{label} in the labelled transition system
   	\end{itemize}
	}\hfill\includegraphics[height=2.8cm]{img/typeToGrammar.png}
	Context-free session types are seen as \emph{simple grammars}, i.e.:
  	\begin{itemize}
		\item context-free grammars in \emph{Greibach normal form} 
		\item s.t. for each $X$ and $a$ there is at most one production $X \xrightarrow{a} \vec Y$
  	\end{itemize}
\end{frame}


\begin{frame}{Convert types to a grammar}

  	\hfill\includegraphics[height=2.8cm]{img/typeToGrammar.png}

  	\vspace*{-3cm}
  	{\color{teal} Example} \\
  
  	\begin{tabular} {l l }
  	  	$S$ &$\triangleq (\mu x. \&\{n\colon x;x;?\,\intk,
      \ell\colon ?\,\intk \});( \mu z. !\,\intk; z;z )$\\\smallskip  
    	$T$ &$\triangleq (\mu y. \&\{n\colon y;y,
      \ell\colon \skipk \}; ?\,\intk); (\mu z. !\,\intk; z)$\\
  	\end{tabular}
  
  	\pause \vspace*{5mm}
  	\begin{tabular}{l l l |l l l}
    	&Productions for $S$& & & Productions for $T$& \\ \hline
    	&\uncover<3-> {{\color{gray}$X_{} \xrightarrow{!\, (\,)} X_1 X_4$}} & & &\uncover<3-> {{\color{gray}$Y_{} \xrightarrow{!\, (\,)} Y_1 Y_3 $} }&\\
    	&$X_1 \xrightarrow{\& n} X_1 X_1 X_2$ & & &$Y_1 \xrightarrow{\& n} Y_1 Y_1 Y_2 $ &\\
    	&$X_1 \xrightarrow{\& \ell} X_3$ &&& $Y_1 \xrightarrow{\& \ell} Y_2 $&\\
    	&$X_2 \xrightarrow{?\, \intk} \varepsilon$&&& $Y_2 \xrightarrow{?\,\intk} \varepsilon$&\\
    	&$X_3 \xrightarrow{?\, \intk} \varepsilon $&&& $Y_3 \xrightarrow{!\,\intk} Y_3$&\\
    	&$X_4 \xrightarrow{!\, \intk} X_4 X_4$ &&&&\\
  	\end{tabular}	
\end{frame}

\begin{frame}{Algorithm for checking the equivalence of CFST \hfill {\color{mLightBrown}Main stages}}

  	\begin{center}
    	\begin{tikzpicture}[node distance = 2.8cm, auto]
      		% Place nodes
      		\node [block2] (typeToGrammar) {{\bf Convert types to a grammar}\\ 
      		{\color{gray}\texttt{LOHaskellCode $\sim$ 100}}\\
      		Soundness {\color{olive}\checkmark}
      		\begin{enumerate}
    	 		\item Conversion of types into BPAs is sound\footnote{ Thiemann and Vasconcelos. Context-free session types. 2016.}.
    	 		\item Conversion of BPAs to Greibach Normal Form without altering the solution is sound\footnote{ Baeten et al. Decidability of bisimulation equivalence for process generating context-free languages.  1993.}.
      		\end{enumerate}
      		}; 
      		\node [block2, below=5mm of typeToGrammar] (prune) {{\bf Prune unnormed productions}};
      		\node [block2, below=5mm of prune] (simplifyExpand) {{\bf Simplify and expand} };
      		\path [line] (typeToGrammar) -- (prune);
      		\path [line] (prune) -- (simplifyExpand);
    	\end{tikzpicture}
  	\end{center}
\end{frame}

\begin{frame} {Prune unnormed productions}
	\hfill\includegraphics[height=2.8cm]{img/prune.png}

	\vspace*{-3.2cm}
	\metroset{block=fill}
	\begin{varblock}[7.8cm]{Definition ((Un)normed terms\footnote{\label{note1} Christensen et al. Bisimulation equivalence is decidable for all CF processes. 1995})}
		\smallskip 
		A term $\alpha$ is \emph{normed} if there is a path from $\alpha$ to $\varepsilon$. Otherwise, it is said to be \emph{unnormed}.
	\end{varblock}

	Christensen, Huttel, and Stirling\footnotemark[\value{footnote}] noted that:\\\smallskip
	\hspace*{1cm}	whenever $\alpha$ is unnormed, $\alpha \sim \alpha \beta$.

	\pause
	\vspace*{-1mm}
	{\color{teal}\rule{3cm}{2pt}}\\
   	{\color{teal} Example} \\\smallskip
	\vspace*{-7mm}

	$$S \triangleq (\mu x. \&\{n\colon x;x;?\,\intk,
      \ell\colon ?\,\intk \});( \mu z. !\,\intk; z;z )$$

	\vspace*{-2mm}
	\hspace*{5mm}\begin{tabular}{l l l }
 		&Productions for $S$&  \\ \hline
 	 	&$X_1 \xrightarrow{\& n} X_1 X_1 X_2$ &\\
 	 	&$X_1 \xrightarrow{\& \ell} X_3$ &\\
 	 	&$X_2 \xrightarrow{?\, \intk} \varepsilon$&\\
 	 	&$X_3 \xrightarrow{?\, \intk} \varepsilon $&\\
 	 	&$X_4 \xrightarrow{!\, \intk} X_4 X_4$ &\\
	\end{tabular}
	\hspace*{1cm} \pause
	\begin{tabular}{l l l }
 		&Productions for \emph{pruned} $S$&  \\ \hline
  		&\hspace*{4mm}$X_1 \xrightarrow{\& n} X_1 X_1 X_2$ & \\
  		&\hspace*{4mm}$X_1 \xrightarrow{\& \ell} X_3$ &\\
 		&\hspace*{4mm}$X_2 \xrightarrow{?\, \intk} \varepsilon$&\\
  		&\hspace*{4mm}$X_3 \xrightarrow{?\, \intk} \varepsilon $&\\
  		&\hspace*{4mm}$X_4 \xrightarrow{!\, \intk} X_4$ &\\
	\end{tabular}
 	\vspace*{2mm}
\end{frame}


\begin{frame}{Algorithm for checking the equivalence of CFST \hfill {\color{mLightBrown}Main stages}}

	\begin{center}
		\begin{tikzpicture}[node distance = 2.8cm, auto]
   			% Place nodes
    		\node [block2] (typeToGrammar) {{\bf Convert types to a grammar}\\ 
    		{\color{gray}\texttt{LOHaskellCode $\sim$ 100}}\\
   			Soundness {\color{olive}\checkmark}
    		}; 
    		\node [block2, below=5mm of typeToGrammar] (prune) {{\bf Prune unnormed productions} \\ 
    		{\color{gray}\texttt{LOHaskellCode $\sim$ 30}}\\
    		Soundness {\color{olive}\checkmark} 
    		\begin{enumerate}
    			\item Pruning unnormed productions is sound\footnote{ Christensen, Huttel, and Stirling. Bisimulation equivalence is decidable for all \\ context-free processes. 1995}
    		\end{enumerate}
    		};
    		\node [block2, below=5mm of prune] (simplifyExpand) {{\bf Simplify and expand} };
    		% Draw edges
    		\path [line] (typeToGrammar) -- (prune);
    		\path [line] (prune) -- (simplifyExpand);
		\end{tikzpicture}
	\end{center}
\end{frame}


\begin{frame} {Simplify and expand}
	\hfill\includegraphics[height=2.8cm]{img/simplifyExpand.png}
	
	\vspace*{-3cm}
	Following the ideas from Hirshfeld, Jan{\v{c}}ar, Moller, \\
	a bisimulation between processes is seen as an \\
	{\bf expansion tree}\footnote{Hirshfeld. Bisimulation trees and the decidability of weak bisimulations. 1997}\footnote{Jan{\v{c}}ar and Moller. Techniques for decidability and undecidability of bisimilarity. 1999}, that alternates between:
	\begin{itemize}
		\item { expansion operations}
		\item { simplification operations	}
	\end{itemize}
	\begin{center}
		\hspace*{-2mm}\includegraphics[height=4cm]{img/expand_simplify}
	\end{center}	
\end{frame}


\begin{frame} {Simplify and expand}
	An expansion tree alternates between:
	\vspace*{-1mm}
	\begin{itemize} 
		\item {\color{mLightBrown} \bf expansion operations} - a single derived node results from the expansion of the parent node. \pause
		\item {\color{teal}\bf simplification operations	}
		\begin{itemize}
			\item {\bf reflexive rule}: omit from a node $N$ any reflexive pair.
			\item {\bf congruence rule}: omit from a node $N$ any pair that belongs to the least congruence containing the ancestors of $N$. \pause
			\item {\bf basic process algebra rules}\footnote{Jan{\v{c}}ar, Moller. Techniques for decidability and undecidability of bisimilarity. 1999}\footnote{ Christensen et al. Bisimulation equivalence is decidable for all CF processes. 1995}\\
			\vspace*{-2mm}
			\hspace*{-2cm}\includegraphics[height=4cm]{img/bpa_new}\smallskip
		\end{itemize}
	\end{itemize}
\end{frame}

\begin{frame}{Simplify and Expand}
	All these transformation rules preserve the \emph{safeness property}:
	\metroset{block=fill}
	\begin{block}{Safeness property\footnotemark[\value{footnote}]}
		\smallskip
		$S\sim T$ iff the expansion tree rooted at $\{(S,T\,)\}$ has a successful branch.	
	\end{block}
	
	The finite witness property holds\footnotemark[\value{footnote}]:
	
	\begin{block}{Finite witness property\footnotemark[\value{footnote}]\footnote{ Christensen et al. Bisimulation equivalence is decidable for all CF processes. 1995}}
		\smallskip
		If $S\sim T$, then there exists a {\bf finite successful branch} in the expansion tree.
	\end{block}
\end{frame}

\begin{frame}
	\begin{tabular} {l l l }
		{\color{teal}\rule{3cm}{2pt}} &  $S$ &$\triangleq (\mu x. \&\{n\colon x;x;?\,\intk,
     	 \ell\colon ?\,\intk \});( \mu z. !\,\intk; z;z )$\\
 		{\color{teal} Example}  &  $T$ &$\triangleq (\mu y. \&\{n\colon y;y,
     	 \ell\colon \skipk \}; ?\,\intk); (\mu z. !\,\intk; z)$
	\end{tabular}
	\vspace*{2mm}
	\includegraphics[width=11.5cm]{img/exemplo-9}\smallskip
\end{frame}

\begin{frame}
	\begin{tabular} {l l l }
  		{\color{teal}\rule{3cm}{2pt}} &  $S$ &$\triangleq (\mu x. \&\{n\colon x;x;?\,\intk,
      	\ell\colon ?\,\intk \});( \mu z. !\,\intk; z;z )$\\
  	{\color{teal} Example}  &  $T$ &$\triangleq (\mu y. \&\{n\colon y;y,
      \ell\colon \skipk \}; ?\,\intk); (\mu z. !\,\intk; z)$
	\end{tabular}
	\vspace*{2mm}
	\includegraphics[width=11.5cm]{img/exemplo-8}\smallskip
\end{frame}


\begin{frame}
	\begin{tabular} {l l l }
  		{\color{teal}\rule{3cm}{2pt}} &  $S$ &$\triangleq (\mu x. \&\{n\colon x;x;?\,\intk,
      \ell\colon ?\,\intk \});( \mu z. !\,\intk; z;z )$\\
  		{\color{teal} Example}  &  $T$ &$\triangleq (\mu y. \&\{n\colon y;y,
      \ell\colon \skipk \}; ?\,\intk); (\mu z. !\,\intk; z)$
	\end{tabular}
	\vspace*{2mm}
	\includegraphics[width=11.5cm]{img/exemplo-7}\smallskip
\end{frame}

\begin{frame}
	\begin{tabular} {l l l }
  		{\color{teal}\rule{3cm}{2pt}} &  $S$ &$\triangleq (\mu x. \&\{n\colon x;x;?\,\intk,
      \ell\colon ?\,\intk \});( \mu z. !\,\intk; z;z )$\\
 		{\color{teal} Example}  &  $T$ &$\triangleq (\mu y. \&\{n\colon y;y,
      \ell\colon \skipk \}; ?\,\intk); (\mu z. !\,\intk; z)$
	\end{tabular}
	\vspace*{2mm}
	\includegraphics[width=11.5cm]{img/exemplo-6}\smallskip
\end{frame}

\begin{frame}
	\begin{tabular} {l l l }
  		{\color{teal}\rule{3cm}{2pt}} &  $S$ &$\triangleq (\mu x. \&\{n\colon x;x;?\,\intk,
      \ell\colon ?\,\intk \});( \mu z. !\,\intk; z;z )$\\
  		{\color{teal} Example}  &  $T$ &$\triangleq (\mu y. \&\{n\colon y;y,
      \ell\colon \skipk \}; ?\,\intk); (\mu z. !\,\intk; z)$
	\end{tabular}
	\vspace*{2mm}
	\includegraphics[width=11.5cm]{img/exemplo-5}\smallskip
\end{frame}

\begin{frame}
	\begin{tabular} {l l l }
  		{\color{teal}\rule{3cm}{2pt}} &  $S$ &$\triangleq (\mu x. \&\{n\colon x;x;?\,\intk,
      \ell\colon ?\,\intk \});( \mu z. !\,\intk; z;z )$\\
  		{\color{teal} Example}  &  $T$ &$\triangleq (\mu y. \&\{n\colon y;y,
      \ell\colon \skipk \}; ?\,\intk); (\mu z. !\,\intk; z)$
	\end{tabular}
	\vspace*{2mm}
	\includegraphics[width=11.5cm]{img/exemplo-4}\smallskip
\end{frame}

\begin{frame}
	\begin{tabular} {l l l }
  		{\color{teal}\rule{3cm}{2pt}} &  $S$ &$\triangleq (\mu x. \&\{n\colon x;x;?\,\intk,
      \ell\colon ?\,\intk \});( \mu z. !\,\intk; z;z )$\\
  		{\color{teal} Example}  &  $T$ &$\triangleq (\mu y. \&\{n\colon y;y,
      \ell\colon \skipk \}; ?\,\intk); (\mu z. !\,\intk; z)$
	\end{tabular}
	\vspace*{2mm}
	\includegraphics[width=11.5cm]{img/exemplo-3}\smallskip
\end{frame}

\begin{frame}
	\begin{tabular} {l l l }
  		{\color{teal}\rule{3cm}{2pt}} &  $S$ &$\triangleq (\mu x. \&\{n\colon x;x;?\,\intk,
      \ell\colon ?\,\intk \});( \mu z. !\,\intk; z;z )$\\
  		{\color{teal} Example}  &  $T$ &$\triangleq (\mu y. \&\{n\colon y;y,
      \ell\colon \skipk \}; ?\,\intk); (\mu z. !\,\intk; z)$
	\end{tabular}
	\vspace*{2mm}
	\includegraphics[width=11.5cm]{img/exemplo-2}\smallskip
\end{frame}

\begin{frame}
	\begin{tabular} {l l l }
  		{\color{teal}\rule{3cm}{2pt}} &  $S$ &$\triangleq (\mu x. \&\{n\colon x;x;?\,\intk,
      \ell\colon ?\,\intk \});( \mu z. !\,\intk; z;z )$\\
  		{\color{teal} Example}  &  $T$ &$\triangleq (\mu y. \&\{n\colon y;y,
      \ell\colon \skipk \}; ?\,\intk); (\mu z. !\,\intk; z)$
	\end{tabular}
	\vspace*{2mm}
	\includegraphics[width=11.5cm]{img/exemplo-1}\smallskip
\end{frame}

\begin{frame}
	\begin{tabular} {l l l }
  		{\color{teal}\rule{3cm}{2pt}} &  $S$ &$\triangleq (\mu x. \&\{n\colon x;x;?\,\intk,
      \ell\colon ?\,\intk \});( \mu z. !\,\intk; z;z )$\\
  		{\color{teal} Example}  &  $T$ &$\triangleq (\mu y. \&\{n\colon y;y,
      \ell\colon \skipk \}; ?\,\intk); (\mu z. !\,\intk; z)$
	\end{tabular}
	\vspace*{2mm}
	\includegraphics[width=11.5cm]{img/example_expansion_tree}\smallskip
\end{frame}

\begin{frame}{Algorithm for checking the equivalence of CFST \hfill {\color{mLightBrown}Main stages}}
	\begin{center}
		\begin{tikzpicture}[node distance = 2.8cm, auto]
    		% Place nodes
    		\node [block2] (typeToGrammar) {{\bf Convert types to a grammar}\\ 
    		{\color{gray}\texttt{LOHaskellCode $\sim$ 100}}\\
    		Soundness {\color{olive}\checkmark}
    		}; 
    		\node [block2, below=5mm of typeToGrammar] (prune) {{\bf Prune unnormed productions} \\ 
    		{\color{gray}\texttt{LOHaskellCode $\sim$ 30}}\\
    		Soundness {\color{olive}\checkmark} 
    		};
    		\node [block2, below=5mm of prune] (simplifyExpand) {{\bf Simplify and expand}\\ 
    		{\color{gray}\texttt{LOHaskellCode $\sim$ 170}}\\
    		Soundness {\color{olive}\checkmark} 
    		\begin{enumerate}
    			\item The \emph{safeness property ensures soundness}\footnote{Jan{\v{c}}ar, Moller. Techniques for decidability and undecidability of bisimilarity.\\ 1999}
   	 		\end{enumerate}
 			};
    	% Draw edges
    		\path [line] (typeToGrammar) -- (prune);
    		\path [line] (prune) -- (simplifyExpand);
		\end{tikzpicture}
	\end{center}
\end{frame}


\begin{frame}{Implementation strategies}
	Implementation choice:
	\begin{itemize}
		\item Breadth-first search on the tree
	\end{itemize}

	\pause
	Strategic options that can enhance performance:
	\begin{itemize}
		\item Instead of looking for a fixed point, iterate the simplification phase
		\item Apply BPA rules wrapped with blocks of reflexive and congruence rules
	\end{itemize}
	\pause
	We showcase runtimes for four scenarios:
	\vspace*{-3mm}

	\begin{center}
		\begin{tabular}{c | c | c | c | c }
 			& \multicolumn{2}{c|}{{\color{teal}{\bf Fixed Point Scenario}}} &\multicolumn{2}{c}{{\color{teal}{\hspace*{2mm}\bf Iterated Scenario\hspace*{2mm}}}}\\
 			& {\enspace\color{mLightBrown}wrapped\enspace} &{\color{mLightBrown}unwrapped}& {\enspace\color{mLightBrown}wrapped\enspace} &{\color{mLightBrown}unwrapped}\\ \hline
 			{\bf Simplification}& \multicolumn{2}{c |}{find fixed point} & \multicolumn{2}{c}{iterate} \\ \hline
  			{\bf BPAs wrapped}&$\checkmark$ & $\times$ & $\checkmark$ & $\times$ \\ \hline
		\end{tabular}
	\end{center}
\end{frame}

\begin{frame}{Running times ($\sim$ 150 tests)}
	\includegraphics[height=6cm]{img/running_time3}

	\fbox{
	\begin{tabular}{l l l }
		{\small Specs: } & {\small MacBook Air }\\
		& {\small 1.8 GHz Intel Core i5} \\
		& {\small 8GB of memory}
	\end{tabular}}
\end{frame}

\begin{frame}{Time and memory allocated \hfill {\color{mLightBrown}Fixed Point}}
	\hspace*{-4mm}\includegraphics[height=5.8cm]{img/fixed_point2}
\end{frame}

\begin{frame}{Time and memory allocated \hfill {\color{mLightBrown}Iterated }}
	\hspace*{-4mm}\includegraphics[height=5.8cm]{img/iterated2} 
\end{frame}


\begin{frame}{Towards completeness...}

	\metroset{block=fill}
	\begin{block}{Finite witness property\footnote{ Christensen et al. Bisimulation equivalence is decidable for all CF processes. 1995}\footnote{Jan{\v{c}}ar, Moller. Techniques for decidability and undecidability of bisimilarity. 1999}}
		\smallskip
		If $S\sim T$, then there exists a {\bf finite successful branch} in the expansion tree.
	\end{block}

	\vspace*{5mm}
	
	Implementation choices aiming to achieve completeness:
	\begin{itemize}
		\item Instead of looking for a fixed point, iterate the simplification phase (there may not be a fixed point)
		\item Double ended enqueue, prepending \emph{promising} nodes, as opposed to queuing all new nodes
	\end{itemize}
\end{frame}


\begin{frame} {Ongoing Work}
	\begin{itemize}
		\item Soundness {\color{olive}\checkmark}
		\item Completeness {\color{orange}?} {\color{gray} ... on our way to achieve it.}
		\item Complexity {\color{orange}?} {\color{gray} ... in practice seems to take not much than parsing!}
		\item Lines of Haskell code: approx.$\setminus$ $300$
	\end{itemize}
	
	{\color{teal}Coming soon: }\\\smallskip
	\textsf{FreeST}, a compiler for context-free session types!\\
	\hspace*{2cm}{\color{gray}(demos on demand)}
\end{frame}

\plain{Thank you!}


\end{document}

%%% Local Variables:
%%% mode: latex
%%% TeX-master: t
%%% End:
