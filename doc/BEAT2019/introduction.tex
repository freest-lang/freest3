\section{Introduction}
\label{sec:introduction}

Session types came to enhance the expressivity of traditional types, enabling to express structured communication on heterogeneously typed channels~\cite{DBLP:conf/concur/Honda93,DBLP:conf/parle/TakeuchiHK94}. Session types have been extended to deal with many realistic situations, e.g., multi-party session types~\cite{DBLP:conf/popl/HondaYC08}, session types for distributed object-oriented programming~\cite{DBLP:conf/popl/GayVRGC10}, session types for programming web services~\cite{DBLP:journals/toplas/CarboneHY12}. Thiemann and Vasconcelos~\cite{thiemann2016context} proposed {\it context-free session types} as an extension of session types by allowing nested protocols that are not restricted to tail recursion. Context-free session types capture the type-safe serialization of recursive datatypes and enable the type-safe implementation of remote operations on recursive datatypes. 

Inspired by the context-free session types' framework, Almeida and
Vasconcelos~\cite{bernardo} proposed a functional programming language
with context-free session types. The usability of such a programming
language is highly dependent on an algorithm to decide type
equivalence. We have developed and implemented an algorithm to decide
type equivalence. Our work capitalizes on the metatheory of
context-free session types proposed by Thiemann and
Vasconcelos~\cite{thiemann2016context}, where type equivalence was
proved to be decidable. Although the decidability of equivalence on
context-free session types has been addressed in the
literature~\cite{DBLP:journals/iandc/ChristensenHS95,janvcar1999techniques,thiemann2016context},
to the best of our knowledge, an algorithm was not yet implemented and
a possible implementation was not obvious.

This document summarizes an ongoing work whose main contributions
stand on the proposal and implementation of an algorithm to decide
type equivalence of context-free session types, prove the soundness of
the algorithm against the declarative definition
in~\cite{thiemann2016context}, study and possibly prove the
completeness of the algorithm, and validate the algorithm on several
meaningful examples.


%%% Local Variables:
%%% mode: latex
%%% TeX-master: "main"
%%% End:
