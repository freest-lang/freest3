\section{Context-free session types}
\label{sec:contextfreesession}

The types we consider build upon some base sets, such as:
\emph{recursion variables}, denoted by $x$, $y$, $z$;
\emph{labels}, denoted by $\ell$; and \emph{base types} denoted by
$B$.  The syntax of types is given by the grammar below.%summarized in
% Figure~\ref{fig:types}.
%
\begin{equation*}
    S,T \grmeq \skipk \mid S;T \grmor \,!B \grmor \,?B \grmor 
       \oplus\{\ell_i\colon S_i\}_{i\in I} \grmor \&\{\ell_i\colon S_i\}_{i\in I} B \grmor
       \mu x.S \grmor x
\end{equation*}
We consider session types: $\skipk$ to represent no communication,
$\,!B$ and $\,?B$ for sending and receiving base type values,
$S;T$ for the sequence of actions,
% denoted by $S_1$ followed by those denoted by $S_2$;
$\oplus\{\ell_i\colon T_i\}_{i\in I}$ and
$\&\{\ell_i\colon T_i\}_{i\in I}$ for internal and external choice,
and $\mu x.T$ and $x$ for recursion.

% \begin{figure}[h!]
% \vspace*{-4mm}
%   \begin{align*}
%     T \grmeq& \skipk \mid T;T \grmor \,!B \grmor \,?B \grmor 
%      \oplus\{\ell_i\colon T_i\}_{i\in I} \grmor \&\{\ell_i\colon T_i\}_{i\in I} B \grmor T\to T \grmor T\multimap T\\
%     & T \otimes T \grmor [\ell_i\colon T_i]_{i\in I} \grmor \mu x.T \grmor x \grmor \forall\alpha.T  \grmor \alpha 
%   \end{align*}
%   \vspace*{-4mm}
%   \caption{Syntax of context-free session types}
%   \label{fig:types}
% \end{figure}

Type equivalence is based on the notion of bisimulation, as
in~\cite{thiemann2016context}. According to this approach, two session
types are equivalent if they exhibit the same behavior. We consider a
labelled transition system where $A$ ranges over $\alpha$, $\,!B$, and
$\,?B$, $\star$ ranges over $\oplus$ and $\&$, and $a$ ranges over
both $A$ and $\star l$.
%
The labelled transition system is given the set of types as
\emph{states}, the set of \emph{actions} ranged over by $a$, and the
transition relation $\LTSderives$ defined by the rules below. The
transition relation makes use of an auxiliary judgment $\DONE{S}$ that
characterizes ``terminated'' session types that exhibit no further
action~\cite{DBLP:journals/jacm/AcetoH92} .
%
% \begin{figure}[h!]
% \vspace*{-4mm}
  \begin{gather*}
    \DONE{\skipk}
    \qquad
    \frac{\DONE{S_1} \quad \DONE{S_2}}{\DONE{S_1; S_2}}
    \qquad
    \frac{\DONE{S[\mu x.S/x]}}{\DONE{\mu x.S}}
    \qquad
    {A \LTSderives[A] \skipk }
    \qquad
    {\star\{l_i\colon S_i\}_{i\in I} \LTSderives[\star l_j] S_j}
    \\
    \frac{S_1 \LTSderives S_1'}{S_1; S_2 \LTSderives S_1';S_2}
    \quad\;\;
    \frac{\DONE{S_1} \quad S_2 \LTSderives S_2'}{S_1; S_2 \LTSderives S_2'}
    \quad\;\;
    \frac{S[\mu x.S/x] \LTSderives S'}{\mu x.S \LTSderives S'}
  \end{gather*}
%   \vspace*{-4mm}
%   \caption{Labelled transition system for context free session types}
%   \label{fig:lts}
% \end{figure}

\emph{Type bisimilarity}, $\sim$, is defined in the usual way from the
labelled transition system.
% , and is extended to the remaining type constructors as described in Figure~\ref{fig:type-equivalence-lifted}.

% \begin{figure}[h!]
% \vspace*{-4mm}
%   \begin{gather*}
%     \frac
%     {T \TypeEquiv T'}
%     { \forall \alpha.T \TypeEquiv \forall \alpha. T'}
%     \qquad
%     \alpha \TypeEquiv \alpha
%     \qquad
%     B \TypeEquiv B
%     \qquad
%     \frac
%     { T_1 \TypeEquiv T_1' \quad T_2 \TypeEquiv T_2'}
%     {T_1 \to T_2 \TypeEquiv T_1' \to T_2'}
%     \qquad
%     \frac
%     { T_1 \TypeEquiv T_1' \quad T_2 \TypeEquiv T_2'}
%     { T_1 \multimap T_2 \TypeEquiv T_1' \multimap T_2'}
%     \\\smallskip
%     \frac
%     {T_1 \TypeEquiv T_1' \quad T_2 \TypeEquiv T_2'}
%     { T_1 \otimes T_2 \TypeEquiv T_1' \otimes T_2'}
%     \qquad
%     \frac
%     {(\forall i\in I)~ T_i \TypeEquiv T_i'}
%     { [l_i:T_i]_{i\in I} \TypeEquiv [l_i:T_i']_{i\in I}}
%     \qquad
%     \frac
%     { T\subs{\mu x.T}x \TypeEquiv T'}
%     { \mu x.T \TypeEquiv T'}
%     \qquad
%     \frac
%     { T \TypeEquiv T'\subs{\mu x.T'}x}
%     { T \TypeEquiv \mu x.T'}
%   \end{gather*}
%   \vspace*{-4mm}
%   \caption{Type equivalence lifted to all types}
%   \label{fig:type-equivalence-lifted}
% \end{figure}

%%% Local Variables:
%%% mode: latex
%%% TeX-master: "main"
%%% End:
