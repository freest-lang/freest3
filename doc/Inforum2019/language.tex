\section{A linguagem de programação}

Esta secção introduz a linguagem apresentando a sua sintaxe, 
semântica e sistema de tipos. 

A linguagem que propomos é funcional apresentando uma sintaxe 
bastante semelhante à do Haskell acrescida de primitivas para 
criação de canais e de envio e receção de dados nos mesmos. 
Esta linguagem é caracterizada pela existência da
primitiva \lstinline|choice|, que permite
enumerar as operações de trocas de mensagens 
que poderão ser escolhidas.

A troca de mensagens dá-se em canais de comunicação síncronos e 
bidirecionais. Cada canal pode ser descrito pelas suas duas 
extremidades (\textit{endpoints}) e são caracterizados por tipos 
que descrevem a sequência de mensagens que passam no canal.
Os processos podem escrever numa das extremidades do canal ou ler na outra.

% \begin{figure}[t]
  \begin{align*}
    \tcBase \grmeq & \inte \grmor \, \chare \grmor \, \boole \grmor \, \unite  \grmor \, l && \text{Tipos básicos}\\
    T \grmeq       & \tskip \grmor \tChoice{T}{T} \grmor \tSemi{T}{T}  \grmor \,\tOut{\tcBase} \grmor \,\tIn{\tcBase} \grmor \tRec{x}{T} \grmor x && \text{Tipos} 
    % 
  \end{align*}
%   \hrulefill
%   \caption{Sintaxe dos tipos}
%   \label{fig:types}
% \end{figure}


%%% Local Variables:
%%% mode: latex
%%% TeX-master: "main"
%%% End:


A figura \ref{fig:types} apresenta os tipos da linguagem.
Os tipos básicos formam um subconjunto daqueles existentes no Haskell, 
inteiros, caracteres, booleanos, e ainda o \lstinline"Unit" e as etiquetas
das escolhas, $l$.
Os restantes tipos são compostos pelo 
operador de escolha $\tChoice{\_}{\_}$,
pelo operador de sequenciação 
$\tSemi{\_}{\_}$, pela sua unidade $\tskip$, pelos tipos que representam o envio $\tOut{B}$, 
a receção $\tIn{B}$, 
pelos tipos recursivos $\tRecK{x}{T}$ e pelas variáveis \textit{x}.
