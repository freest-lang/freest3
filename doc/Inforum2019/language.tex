\section{A linguagem de programação \mixedchoice}

% Esta secção introduz a linguagem apresentando a sua sintaxe, 
% semântica e sistema de tipos. 

A linguagem que propomos é funcional apresentando uma sintaxe bastante
semelhante à do Haskell acrescida de primitivas para criação de canais
e de envio e receção de dados nos mesmos.  Construída sobre a
linguagem \freest~\cite{2019freest}, esta linguagem é caracterizada
pela primitiva \lstinline|choice|, primitiva essa que permite enumerar
as operações de trocas de mensagens que poderão ser escolhidas.

A troca de mensagens dá-se em canais de comunicação síncronos e
bidirecionais. Cada canal pode ser descrito pelas suas duas
extremidades (\textit{endpoints}) e é caracterizado por tipos que
descrevem a sequência de mensagens que passam no canal. %  Os processos
% podem escrever numa das extremidades do canal ou ler na outra.
% %
A figura abaixo apresenta os tipos para canais (tipos de sessão) da
linguagem.
%
% \begin{figure}[t]
  \begin{align*}
    \tcBase \grmeq & \inte \grmor \, \chare \grmor \, \boole \grmor \, \unite  \grmor \, l && \text{Tipos básicos}\\
    T \grmeq       & \tskip \grmor \tChoice{T}{T} \grmor \tSemi{T}{T}  \grmor \,\tOut{\tcBase} \grmor \,\tIn{\tcBase} \grmor \tRec{x}{T} \grmor x && \text{Tipos} 
    % 
  \end{align*}
%   \hrulefill
%   \caption{Sintaxe dos tipos}
%   \label{fig:types}
% \end{figure}


%%% Local Variables:
%%% mode: latex
%%% TeX-master: "main"
%%% End:


Os tipos básicos formam um subconjunto daqueles existentes no Haskell,
inteiros, caracteres, booleanos, e ainda o tipo \textit{unit}
\lstinline|()| e as etiquetas (marcas) $l$ das escolhas.  Os tipos de
sessão são compostos pelo operador de escolha $\tChoice{\_}{\_}$, pelo
operador de sequenciação $\tSemi{\_}{\_}$, pela sua unidade $\tskip$,
pelos tipos que representam o envio $\tOut{B}$, a receção $\tIn{B}$,
pelos tipos recursivos $\tRecK{x}{T}$ e pelas variáveis de tipo
\textit{x}. Nesta sintaxe, o tipo \lstinline|CanalInt| escreve-se
%
\lstinline|rec x. ?EOS + !Int;x|.

%%% Local Variables:
%%% mode: latex
%%% TeX-master: "main"
%%% End:
