% This is samplepaper.tex, a sample chapter demonstrating the
% LLNCS macro package for Springer Computer Science proceedings;
% Version 2.20 of 2017/10/04
%
\documentclass[runningheads]{llncs}

\usepackage[portuguese]{babel}
\usepackage[utf8]{inputenc}  % for proper diacritics
\usepackage[T1]{fontenc}
\usepackage{color}
\usepackage {listings}
\usepackage{tikz}
\usepackage{wrapfig}
\usepackage{hyperref}
\usepackage{amsmath,amssymb}
\usepackage{alltt}
\usepackage{flushend}
\usepackage{graphicx}

% If you use the hyperref package, please uncomment the following line
% to display URLs in blue roman font according to Springer's eBook style:
\renewcommand\UrlFont{\color{blue}\rmfamily}

\usepackage[portuguese]{babel}
\usepackage[utf8]{inputenc}  % for proper diacritics
\usepackage[T1]{fontenc}
\usepackage {listings}
\usepackage{tikz}
% \usepackage{xcolor}
% Links
\usepackage{hyperref}

\usepackage{amsmath,amssymb}
%\usepackage{stmaryrd}
\usepackage{listings,color}
\usepackage{alltt}
\usepackage{flushend}



%\usepackage[T1]{fontenc}

% \lstdefinestyle{Go}{	
% 	keywordstyle=[1]\bfseries,
% 	basicstyle=\footnotesize\ttfamily,	
% 	numberstyle=\tiny,
% 	numbersep=5pt,
% 	breaklines=true,
% 	%prebreak=\raisebox{0ex}[0ex2][0ex]{\ensuremath{\hookleftarrow}},
% 	showstringspaces=false,
% 	upquote=true,
% 	tabsize=3,
% 	frame=tb,
% 	morekeywords={go,make,chan,int,import,main,func,for,select,case,string},
%       }

\lstdefinelanguage{Golang}%
  {morekeywords=[1]{package,import,func,type,struct,return,defer,panic,%
     recover,select,var,const,iota,},%
   morekeywords=[2]{string,uint,uint8,uint16,uint32,uint64,int,int8,int16,%
     int32,int64,bool,float32,float64,complex64,complex128,byte,rune,uintptr,%
     error,interface},%
   morekeywords=[3]{map,slice,make,new,nil,len,cap,copy,close,true,false,%
     delete,append,real,imag,complex,chan,},%
   morekeywords=[4]{for,break,continue,range,goto,switch,case,fallthrough,if,%
     else,default,},%
   morekeywords=[5]{Println,Printf,Error,Print,},%
   sensitive=true,%
   morecomment=[l]{//},%
   morecomment=[s]{/*}{*/},%
   morestring=[b]',%
   morestring=[b]",%
   morestring=[s]{`}{`},%
}

      
\lstdefinelanguage{SePi}%
  {morekeywords=[1]{type,integer,string,boolean,new, select, assume, assert},%
   sensitive=true,%
   morecomment=[l]{//},%
   morecomment=[s]{/*}{*/},%
   morestring=[b]',%
   morestring=[b]",%
   morestring=[s]{`}{`},%
 }

\lstdefinelanguage{CFST}%
{
  morekeywords=[1]{Int, Char, Bool, Skip, forall, rec, let, in, if, then, else, new, send, receive,
    select, fork, case, of, data, match, with},%  
  sensitive=true,%
  literate={->}{{$\rightarrow$}}1,%
   breaklines=true,
   morecomment=[l]{--},%
   morecomment=[s]{{-}{-}},%
   morestring=[b]',%
   morestring=[b]",%
   morestring=[s]{`}{`},%
 }

 

% notes
\newcommand{\todo}[1]{[{\color{blue}\textbf{#1}}]}

% Keywords
\newcommand{\keyword}[1]{\mathsf{#1}}

% Prekinds

\newcommand\prekind{\upsilon}

\newcommand{\stypes}{\mathcal S}
\newcommand\kinds{\stypes}

\newcommand{\types}{\mathcal T}
\newcommand\kindt{\types}

\newcommand\kindsch{\mathcal C}

% Multiplicity
\newcommand\Un{\ensuremath{\mathbf{u}}} % \infty
\newcommand\Lin{\ensuremath{\mathbf{l}}} % 1 

% Kinds
\newcommand\kind{\kappa}

% Grammars
\newcommand{\grmeq}{\; ::= \;}
\newcommand{\grmor}{\;\mid\;}

% type constructors
\newcommand\tcBase{B}
\newcommand\tcLolli\multimap
\newcommand\tcFun\to
\newcommand\tcBang{\mathop!}

% Keywords for types
\newcommand\kRec{\keyword{rec}}


% Types
\newcommand{\tskip}{\keyword{Skip}}
\newcommand\tSemi[2]{#1;#2}
\newcommand\tOut[1]{\tcBang#1}
\newcommand\tIn[1]{?#1}
\newcommand\tIChoice[1]{\oplus#1}
\newcommand\tEChoice[1]{\&#1}
\newcommand\tUnFun[2]{#1\tcFun#2}
\newcommand\tLinFun[2]{#1\tcLolli#2}
\newcommand\tPair[2]{#1\otimes#2}
\newcommand\tDatatype[1]{{[#1]}}
\newcommand\tRec[2]{\mu\,#1\,.\,#2}
\newcommand\tForall[2]{\forall\,#1\,.\,#2}

\newcommand\tRecK[2]{\kRec\,#1\,.\,#2}
% Environments
% \newcommand\emptyEnv{\cdot}
\newcommand\emptyEnv{\varepsilon}
\newcommand\kindEnv{\Delta}
\newcommand\varEnv{\Gamma}

% Language
% Expressions
% Language Types
\newcommand{\unite}{\keyword{Unit}}
\newcommand{\inte}{\keyword{Int}}
\newcommand{\chare}{\keyword{Char}}
\newcommand{\boole}{\keyword{Bool}}

% Variables
\newcommand\vare[1]{#1}
\newcommand\unlete[3]{\keyword{let} \; #1 = #2 \; \keyword{in} \; #3} 

% Applications
\newcommand\appe[2]{#1#2}
\newcommand\tappe[2]{#1[#2]}

% Conditional
\newcommand\conditionale[3]{\keyword{if}\;#1\;\keyword{then}\;#2\;\keyword{else} \; #3}

% Pairs
\newcommand\paire[2]{(#1,#2)}
\newcommand\binlete[4]{\keyword{let}\;#1, #2 = #3\;\keyword{in}\;#4}

% Session Types
\newcommand\newe[1]{\keyword{new}\;#1}
\newcommand\sende[2]{\keyword{send}\;#1\; #2}
\newcommand\recve[1]{\keyword{receive}\;#1}
\newcommand\selecte[1]{\keyword{select}\;#1}
\newcommand\matche[2]{\keyword{match}\;#1\;\keyword{with}\;#2}

% Fork
\newcommand\forke[1]{\keyword{fork}\;#1}

% Datatypes
\newcommand{\ctrcte}{C}
\newcommand\casee[2]{\keyword{case}\;#1\;\keyword{of}\;#2}

% Goal
\newcommand\Alg{\vdash_{a}}

% Equivalent
\newcommand\Equiv[2]{#1\,\thicksim\,#2}


%%% Local Variables:
%%% mode: latex
%%% TeX-master: "cfst-inforum18"
%%% End:
      

\title{Uma linguagem de programação com escolhas mistas em tipos de sessão}
\titlerunning{Uma linguagem de programação com escolhas mistas em tipos de sessão}
\author{Bernardo Almeida, Andreia Mordido e Vasco T. Vasconcelos}
%\authorrunning{F. Author et al.}

\institute{LASIGE, Faculdade de Ciências, Universidade de Lisboa, Portugal}

\begin{document}

\maketitle

\begin{abstract}

A abstração e formalização da comunicação inerente a 
sistemas de software é fundamental para a confiabilidade dos 
sistemas.
Neste trabalho propomos um sistema de tipos que enriquece
os tipos de sessão, permitindo uma 
\emph{escolha mista} de trocas de mensagens em cada operação.
No formalismo que propomos, um servidor deverá ser capaz de escolher se
pretende continuar a receber mensagens do cliente ou se pretende
enviar uma determinado valor. Este trabalho está a decorrer.

%  Sistemas de software distribuídos apresentam por vezes padrões de
%  comunicação intensivos cuja complexidade dificulta a sua
%  codificação.  
%  Os tipos de sessão foram propostos para responder a esta
%  necessidade, permitindo definir protocolos na forma de tipos que
%  representam interações corretas do sistema e que garantem
%  propriedades tais como a inexistência de erros na comunicação e de
%  situações de impasse.  
%  Os tipos de sessão tradicionais são descritos por linguagens
%  regulares, permitindo, por exemplo, a definição de protocolos com
%  a estrutura de uma lista mas não em forma de árvore.
%  Neste artigo apresenta-se uma linguagem de programação concorrente,
%  explicitamente tipificada, onde os processos comunicam
%  exclusivamente por troca de mensagens e cujos protocolos são definidos
%  por tipos de sessão livres do contexto.

\keywords{Concorrência \and Troca de mensagens \and Tipos de sessão \and Escolhas mistas}
\end{abstract}

\section{Motivação}

Os tipos de sessão foram propostos para responder à necessidade de
formalização de trocas de mensagens, permitindo definir protocolos na
forma de tipos que representam interações corretas do sistema e que
garantem propriedades tais como a inexistência de erros na comunicação
e de situações de impasse~\cite{ref-lang-primitives}. Contudo, a
expressividade dos tipos de sessão está ainda aquém de todos os
desafios de comunicação que encontramos nos sistemas: não permitem em
particular misturar \textit{input} e \textit{output} na mesma escolha.

Considere-se o seguinte problema.
%
\begin{quotation}
	Dado $k\in\mathbb{Z}$, qual o maior inteiro $n$ tal que
	$\sum_{i=1}^n i < k$?
\end{quotation}

Imagine-se um processador capaz de gerar sequências de números
inteiros mas sem a capacidade de os somar. Imagine-se a um
segundo processador capaz de somar números inteiros, mas incapaz de os
gerar. Ambos os processadores estão aptos para a utilização de
comunicação por canais.
%
Este exemplo simplista retrata o caso em que a interação entre
os processos se dá necessariamente nos dois sentidos e ambos 
os processos deverão
estar aptos, leia-se \emph{corretamente tipados}, para responder 
às necessidades da comunicação.

Tirando partido das restrições computacionais dos nossos dois
processadores, podemos tentar resolver o problema dos seguinte modo:
%
\begin{itemize}
\item Um dos processos, \lstinline|produtor|, gera a sequência de
  números: 1,2,3,\dots e envia-a num canal, até receber no mesmo
  canal uma notificação para terminar o envio. A notificação vem na
  forma de uma marca \lstinline|EOS| (\textit{end-of-stream});
\item O outro processo, \lstinline|consumidor|, vai somando os números
  que recebe os números que recebe enquanto a soma for menor do que
  $n$. Neste ponto envia a marca \lstinline|EOS|.
\item O programa principal cria o canal e lança os dois processos.
\end{itemize}

O canal de comunicação segue o  protocolo delineado acima. Quando
visto do lado do \lstinline|produtor| o canal toma o tipo
%
\begin{lstlisting}
  type CanalInt = ?EOS + !Int;CanalInt
\end{lstlisting}
%
permitindo a cada momento a leitura (\lstinline|?|) da marca
\lstinline|EOS| ou a escrita (\lstinline|!|) de um inteiro. No caso da
leitura o protocolo volta ``ao início''.
%
Quando visto do lado do \lstinline|consumidor| o canal toma o tipo
obtido do acima trocando as operações de leitura pelas de escrita, e
vice-versa. Abreviamos esse tipo com \lstinline|dualof CanalInt|.

\begin{lstlisting}
produtor : CanalInt -> Int -> ()
produtor c i = choice {
  (c, EOS) = receive -> (),
  c = send c n -> consumidor c (i+1)

consumidor : dualof CanalInt -> Int -> Int -> Int -> Int
consumidor c s n k = if n >= n
then choice { c = send EOS -> n--1 }
else choice { (c, m) = receive c -> produtor c (s+m) (n+1) k }

main : Int
main = let
  k = 1000
  (p, c) = new CanalInt in
  fork produtor p 1;
  consumidor c 0 0 k
\end{lstlisting}

O construtor linguístico que manipula escolhas é \lstinline|choice|:
entre as chavetas encontramos \emph{um ou mais} padrões. Os padrões de
\textit{output} levam a palavra reservada \lstinline|send| e os de
\textit{input} \lstinline|receive|. Todas estas primitivas devolvem o
canal (\lstinline|c| no exemplo) onde a interação deve continuar. No
caso da receção, a operação \lstinline|receive| um segundo elemento: o
valor lido. Finalmente, no caso de receção de marcas,
(\lstinline|EOS|, por exemplo), permitimo-nos usar
\textit{pattern-matching}.
%
A escolha do \lstinline|consumidor| é degenerada, quer no ramo
\lstinline|then| como no ramo \lstinline|else|. Poderiamos neste caso
eliminar a palavra reservada \lstinline|choice|.

O programa principal cria um novo canal através da primitiva
\lstinline|new|. O resultado é um par descrevendo as duas extremidades
do canal, \lstinline|p| e \lstinline|c|, destinado a cada um dos
processos.
%
A primitiva \lstinline|fork| lança uma nova \textit{thread} destinada
a correr o processo produtor. O resultado do programa principal é o
resultado do processo \lstinline|consumidor|.

%%% Local Variables:
%%% mode: latex
%%% TeX-master: "main"
%%% End:
t
% \input{implementacao}
\section{Equivalência de tipos}

Um compilador para uma linguagem de programação com tipos de sessão
com escolhas mistas, como os que propomos, tem de estar munido de um
algoritmo de verificação de equivalência de tipos.

Os tipos de sessão que estendemos
neste trabalho não são só tipos de sessão regulares.
Note-se que a sintaxe que propomos estende também
tipos de sessão livres do contexto.
Assim, o algoritmo de verificação de equivalência de tipos 
não é óbvio e deverá basear-se em \emph{Basic Process Algebras}
(BPA), como provado por Thiemann e Vasconcelos em~\cite{ref-cfst}.
Neste trabalho pretendemos tirar partido do algoritmo de verificação
de tipos livres de contexto que propomos em~\cite{type-equiv}.
Este algoritmo começa por traduzir tipos livres de contexto
em gramáticas simples (i.e.\ gramáticas determinísticas
em Greibach Normal Form) e, depois de simplificar as 
gramáticas, decide a existência de uma bisimulação entre as gramáticas
através de uma árvore de expansão. 
A árvore de expansão consiste numa
sucessão de passos de expansão e de simplificação 
sobre os símbolos da gramática, de acordo com 
as produções da gramática e com algumas regras de simplificação
propostas por Caucal, Christensen,
H\"uttel, Stirling, Jan\v car, and Moller
~\cite{caucal1986decidabilite,DBLP:journals/iandc/ChristensenHS95,janvcar1999techniques}

Os tipos de sessão com escolhas mistas têm
uma expressividade idêntica à expressividade das gramáticas
simples. Assim, a redução do algoritmo a estas gramáticas
permite a sua incorporação num compilador
para uma linguagem de programação com estes tipos mais expressivos.

%%% Local Variables:
%%% mode: latex
%%% TeX-master: "main"
%%% End:


\paragraph{Agradecimentos.}

Este trabalho foi apoiado em parte pela FCT através do projeto
Confident, PTDC/EEI-CTP/4503/2014, e da Unidade de Investigação LASIGE,
UID/CEC/00408/2019.

\bibliographystyle{splncs04}
\bibliography{bibliography}

\end{document}
%%% Local Variables:
%%% mode: latex
%%% TeX-master: t
%%% End:
