% This is samplepaper.tex, a sample chapter demonstrating the
% LLNCS macro package for Springer Computer Science proceedings;
% Version 2.20 of 2017/10/04
%
\documentclass[runningheads]{llncs}

\usepackage[portuguese]{babel}
\usepackage[utf8]{inputenc}  % for proper diacritics
\usepackage[T1]{fontenc}
\usepackage{color}
\usepackage {listings}
\usepackage{tikz}
\usepackage{wrapfig}
\usepackage{hyperref}
\usepackage{amsmath,amssymb}
\usepackage{alltt}
\usepackage{flushend}
\usepackage{graphicx}

% If you use the hyperref package, please uncomment the following line
% to display URLs in blue roman font according to Springer's eBook style:
\renewcommand\UrlFont{\color{blue}\rmfamily}

\usepackage[portuguese]{babel}
\usepackage[utf8]{inputenc}  % for proper diacritics
\usepackage[T1]{fontenc}
\usepackage {listings}
\usepackage{tikz}
% \usepackage{xcolor}
% Links
\usepackage{hyperref}

\usepackage{amsmath,amssymb}
%\usepackage{stmaryrd}
\usepackage{listings,color}
\usepackage{alltt}
\usepackage{flushend}



%\usepackage[T1]{fontenc}

% \lstdefinestyle{Go}{	
% 	keywordstyle=[1]\bfseries,
% 	basicstyle=\footnotesize\ttfamily,	
% 	numberstyle=\tiny,
% 	numbersep=5pt,
% 	breaklines=true,
% 	%prebreak=\raisebox{0ex}[0ex2][0ex]{\ensuremath{\hookleftarrow}},
% 	showstringspaces=false,
% 	upquote=true,
% 	tabsize=3,
% 	frame=tb,
% 	morekeywords={go,make,chan,int,import,main,func,for,select,case,string},
%       }

\lstdefinelanguage{Golang}%
  {morekeywords=[1]{package,import,func,type,struct,return,defer,panic,%
     recover,select,var,const,iota,},%
   morekeywords=[2]{string,uint,uint8,uint16,uint32,uint64,int,int8,int16,%
     int32,int64,bool,float32,float64,complex64,complex128,byte,rune,uintptr,%
     error,interface},%
   morekeywords=[3]{map,slice,make,new,nil,len,cap,copy,close,true,false,%
     delete,append,real,imag,complex,chan,},%
   morekeywords=[4]{for,break,continue,range,goto,switch,case,fallthrough,if,%
     else,default,},%
   morekeywords=[5]{Println,Printf,Error,Print,},%
   sensitive=true,%
   morecomment=[l]{//},%
   morecomment=[s]{/*}{*/},%
   morestring=[b]',%
   morestring=[b]",%
   morestring=[s]{`}{`},%
}

      
\lstdefinelanguage{SePi}%
  {morekeywords=[1]{type,integer,string,boolean,new, select, assume, assert},%
   sensitive=true,%
   morecomment=[l]{//},%
   morecomment=[s]{/*}{*/},%
   morestring=[b]',%
   morestring=[b]",%
   morestring=[s]{`}{`},%
 }

\lstdefinelanguage{CFST}%
{
  morekeywords=[1]{Int, Char, Bool, Skip, forall, rec, let, in, if, then, else, new, send, receive,
    select, fork, case, of, data, match, with},%  
  sensitive=true,%
  literate={->}{{$\rightarrow$}}1,%
   breaklines=true,
   morecomment=[l]{--},%
   morecomment=[s]{{-}{-}},%
   morestring=[b]',%
   morestring=[b]",%
   morestring=[s]{`}{`},%
 }

 

% notes
\newcommand{\todo}[1]{[{\color{blue}\textbf{#1}}]}

% Keywords
\newcommand{\keyword}[1]{\mathsf{#1}}

% Prekinds

\newcommand\prekind{\upsilon}

\newcommand{\stypes}{\mathcal S}
\newcommand\kinds{\stypes}

\newcommand{\types}{\mathcal T}
\newcommand\kindt{\types}

\newcommand\kindsch{\mathcal C}

% Multiplicity
\newcommand\Un{\ensuremath{\mathbf{u}}} % \infty
\newcommand\Lin{\ensuremath{\mathbf{l}}} % 1 

% Kinds
\newcommand\kind{\kappa}

% Grammars
\newcommand{\grmeq}{\; ::= \;}
\newcommand{\grmor}{\;\mid\;}

% type constructors
\newcommand\tcBase{B}
\newcommand\tcLolli\multimap
\newcommand\tcFun\to
\newcommand\tcBang{\mathop!}

% Keywords for types
\newcommand\kRec{\keyword{rec}}


% Types
\newcommand{\tskip}{\keyword{Skip}}
\newcommand\tSemi[2]{#1;#2}
\newcommand\tOut[1]{\tcBang#1}
\newcommand\tIn[1]{?#1}
\newcommand\tIChoice[1]{\oplus#1}
\newcommand\tEChoice[1]{\&#1}
\newcommand\tUnFun[2]{#1\tcFun#2}
\newcommand\tLinFun[2]{#1\tcLolli#2}
\newcommand\tPair[2]{#1\otimes#2}
\newcommand\tDatatype[1]{{[#1]}}
\newcommand\tRec[2]{\mu\,#1\,.\,#2}
\newcommand\tForall[2]{\forall\,#1\,.\,#2}

\newcommand\tRecK[2]{\kRec\,#1\,.\,#2}
% Environments
% \newcommand\emptyEnv{\cdot}
\newcommand\emptyEnv{\varepsilon}
\newcommand\kindEnv{\Delta}
\newcommand\varEnv{\Gamma}

% Language
% Expressions
% Language Types
\newcommand{\unite}{\keyword{Unit}}
\newcommand{\inte}{\keyword{Int}}
\newcommand{\chare}{\keyword{Char}}
\newcommand{\boole}{\keyword{Bool}}

% Variables
\newcommand\vare[1]{#1}
\newcommand\unlete[3]{\keyword{let} \; #1 = #2 \; \keyword{in} \; #3} 

% Applications
\newcommand\appe[2]{#1#2}
\newcommand\tappe[2]{#1[#2]}

% Conditional
\newcommand\conditionale[3]{\keyword{if}\;#1\;\keyword{then}\;#2\;\keyword{else} \; #3}

% Pairs
\newcommand\paire[2]{(#1,#2)}
\newcommand\binlete[4]{\keyword{let}\;#1, #2 = #3\;\keyword{in}\;#4}

% Session Types
\newcommand\newe[1]{\keyword{new}\;#1}
\newcommand\sende[2]{\keyword{send}\;#1\; #2}
\newcommand\recve[1]{\keyword{receive}\;#1}
\newcommand\selecte[1]{\keyword{select}\;#1}
\newcommand\matche[2]{\keyword{match}\;#1\;\keyword{with}\;#2}

% Fork
\newcommand\forke[1]{\keyword{fork}\;#1}

% Datatypes
\newcommand{\ctrcte}{C}
\newcommand\casee[2]{\keyword{case}\;#1\;\keyword{of}\;#2}

% Goal
\newcommand\Alg{\vdash_{a}}

% Equivalent
\newcommand\Equiv[2]{#1\,\thicksim\,#2}


%%% Local Variables:
%%% mode: latex
%%% TeX-master: "cfst-inforum18"
%%% End:
      

\title{Uma linguagem de programação com escolhas mistas em tipos de sessão}
\titlerunning{Uma linguagem de programação com escolhas mistas em tipos de sessão}
\author{Bernardo Almeida, Andreia Mordido e Vasco T. Vasconcelos}
%\authorrunning{F. Author et al.}

\institute{LASIGE, Faculdade de Ciências, Universidade de Lisboa, Portugal}

\begin{document}

\maketitle

\begin{abstract}

A abstração e formalização da comunicação inerente a 
sistemas de software é fundamental para a confiabilidade dos 
sistemas.
Neste trabalho propomos um sistema de tipos que enriquece
os tipos de sessão, permitindo uma 
\emph{escolha mista} de trocas de mensagens em cada operação.
No formalismo que propomos, um servidor deverá ser capaz de escolher se
pretende continuar a receber mensagens do cliente ou se pretende
enviar uma determinado valor. Este trabalho está a decorrer.

%  Sistemas de software distribuídos apresentam por vezes padrões de
%  comunicação intensivos cuja complexidade dificulta a sua
%  codificação.  
%  Os tipos de sessão foram propostos para responder a esta
%  necessidade, permitindo definir protocolos na forma de tipos que
%  representam interações corretas do sistema e que garantem
%  propriedades tais como a inexistência de erros na comunicação e de
%  situações de impasse.  
%  Os tipos de sessão tradicionais são descritos por linguagens
%  regulares, permitindo, por exemplo, a definição de protocolos com
%  a estrutura de uma lista mas não em forma de árvore.
%  Neste artigo apresenta-se uma linguagem de programação concorrente,
%  explicitamente tipificada, onde os processos comunicam
%  exclusivamente por troca de mensagens e cujos protocolos são definidos
%  por tipos de sessão livres do contexto.

\keywords{Concorrência \and Troca de mensagens \and Tipos de sessão \and Escolhas mistas}
\end{abstract}

\section{Introdução}

Os tipos de sessão foram propostos para responder à necessidade
de formalização de trocas de mensagens, permitindo definir protocolos 
na forma de tipos que representam interações corretas do sistema e que 
garantem propriedades tais como a inexistência de erros na comunicação 
e de situações de impasse. Contudo, a expressividade dos tipos de sessão
está ainda aquém de todos os desafios de comunicação que encontramos nos 
sistemas.

Considere-se que temos um processo que consegue gerar sequências de números
inteiros mas não tem capacidade de os somar. Por outro lado, temos um
segundo processo que é capaz de somar números inteiros, mas não consegue
gerá-los. Ambos os processos estão aptos para a utilização de comunicação
por canais. Considere-se então o seguinte problema:

\begin{example}
	Dado $k\in\mathbb{Z}$, qual o maior inteiro $n$ tal que
	$\sum_{i=1}^n i < k$?
\end{example}

Este exemplo simplista retrata o caso em que a interação entre
os processos se dá necessariamente nos dois sentidos e ambos 
os processos deverão
estar aptos, leia-se \emph{corretamente tipados}, para responder 
às necessidades da comunicação.

Neste trabalho propomos uma linguagem de programação
com tipos capazes de modelar um cenário como o descrito acima.


\paragraph{Agradecimentos.}

Este trabalho foi apoiado em parte pela FCT através do projeto
Confident, PTDC/EEI-CTP/4503/2014, e da Unidade de Investigação LASIGE,
UID/CEC/00408/2019.

\bibliographystyle{splncs04}
\bibliography{bibliography}

\end{document}