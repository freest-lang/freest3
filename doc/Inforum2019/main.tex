% This is samplepaper.tex, a sample chapter demonstrating the
% LLNCS macro package for Springer Computer Science proceedings;
% Version 2.20 of 2017/10/04
%
\documentclass[runningheads]{llncs}

\usepackage[portuguese]{babel}
\usepackage[utf8]{inputenc}  % for proper diacritics
\usepackage[T1]{fontenc}
\usepackage{color}
\usepackage {listings}
\usepackage{tikz}
\usepackage{wrapfig}
\usepackage{hyperref}
\usepackage{amsmath,amssymb}
\usepackage{alltt}
\usepackage{flushend}
\usepackage{graphicx}

% If you use the hyperref package, please uncomment the following line
% to display URLs in blue roman font according to Springer's eBook style:
\renewcommand\UrlFont{\color{blue}\rmfamily}

\usepackage[portuguese]{babel}
\usepackage[utf8]{inputenc}  % for proper diacritics
\usepackage[T1]{fontenc}
\usepackage {listings}
\usepackage{tikz}
% \usepackage{xcolor}
% Links
\usepackage{hyperref}

\usepackage{amsmath,amssymb}
%\usepackage{stmaryrd}
\usepackage{listings,color}
\usepackage{alltt}
\usepackage{flushend}



%\usepackage[T1]{fontenc}

% \lstdefinestyle{Go}{	
% 	keywordstyle=[1]\bfseries,
% 	basicstyle=\footnotesize\ttfamily,	
% 	numberstyle=\tiny,
% 	numbersep=5pt,
% 	breaklines=true,
% 	%prebreak=\raisebox{0ex}[0ex2][0ex]{\ensuremath{\hookleftarrow}},
% 	showstringspaces=false,
% 	upquote=true,
% 	tabsize=3,
% 	frame=tb,
% 	morekeywords={go,make,chan,int,import,main,func,for,select,case,string},
%       }

\lstdefinelanguage{Golang}%
  {morekeywords=[1]{package,import,func,type,struct,return,defer,panic,%
     recover,select,var,const,iota,},%
   morekeywords=[2]{string,uint,uint8,uint16,uint32,uint64,int,int8,int16,%
     int32,int64,bool,float32,float64,complex64,complex128,byte,rune,uintptr,%
     error,interface},%
   morekeywords=[3]{map,slice,make,new,nil,len,cap,copy,close,true,false,%
     delete,append,real,imag,complex,chan,},%
   morekeywords=[4]{for,break,continue,range,goto,switch,case,fallthrough,if,%
     else,default,},%
   morekeywords=[5]{Println,Printf,Error,Print,},%
   sensitive=true,%
   morecomment=[l]{//},%
   morecomment=[s]{/*}{*/},%
   morestring=[b]',%
   morestring=[b]",%
   morestring=[s]{`}{`},%
}

      
\lstdefinelanguage{SePi}%
  {morekeywords=[1]{type,integer,string,boolean,new, select, assume, assert},%
   sensitive=true,%
   morecomment=[l]{//},%
   morecomment=[s]{/*}{*/},%
   morestring=[b]',%
   morestring=[b]",%
   morestring=[s]{`}{`},%
 }

\lstdefinelanguage{CFST}%
{
  morekeywords=[1]{Int, Char, Bool, Skip, forall, rec, let, in, if, then, else, new, send, receive,
    select, fork, case, of, data, match, with},%  
  sensitive=true,%
  literate={->}{{$\rightarrow$}}1,%
   breaklines=true,
   morecomment=[l]{--},%
   morecomment=[s]{{-}{-}},%
   morestring=[b]',%
   morestring=[b]",%
   morestring=[s]{`}{`},%
 }

 

% notes
\newcommand{\todo}[1]{[{\color{blue}\textbf{#1}}]}

% Keywords
\newcommand{\keyword}[1]{\mathsf{#1}}

% Prekinds

\newcommand\prekind{\upsilon}

\newcommand{\stypes}{\mathcal S}
\newcommand\kinds{\stypes}

\newcommand{\types}{\mathcal T}
\newcommand\kindt{\types}

\newcommand\kindsch{\mathcal C}

% Multiplicity
\newcommand\Un{\ensuremath{\mathbf{u}}} % \infty
\newcommand\Lin{\ensuremath{\mathbf{l}}} % 1 

% Kinds
\newcommand\kind{\kappa}

% Grammars
\newcommand{\grmeq}{\; ::= \;}
\newcommand{\grmor}{\;\mid\;}

% type constructors
\newcommand\tcBase{B}
\newcommand\tcLolli\multimap
\newcommand\tcFun\to
\newcommand\tcBang{\mathop!}

% Keywords for types
\newcommand\kRec{\keyword{rec}}


% Types
\newcommand{\tskip}{\keyword{Skip}}
\newcommand\tSemi[2]{#1;#2}
\newcommand\tOut[1]{\tcBang#1}
\newcommand\tIn[1]{?#1}
\newcommand\tIChoice[1]{\oplus#1}
\newcommand\tEChoice[1]{\&#1}
\newcommand\tUnFun[2]{#1\tcFun#2}
\newcommand\tLinFun[2]{#1\tcLolli#2}
\newcommand\tPair[2]{#1\otimes#2}
\newcommand\tDatatype[1]{{[#1]}}
\newcommand\tRec[2]{\mu\,#1\,.\,#2}
\newcommand\tForall[2]{\forall\,#1\,.\,#2}

\newcommand\tRecK[2]{\kRec\,#1\,.\,#2}
% Environments
% \newcommand\emptyEnv{\cdot}
\newcommand\emptyEnv{\varepsilon}
\newcommand\kindEnv{\Delta}
\newcommand\varEnv{\Gamma}

% Language
% Expressions
% Language Types
\newcommand{\unite}{\keyword{Unit}}
\newcommand{\inte}{\keyword{Int}}
\newcommand{\chare}{\keyword{Char}}
\newcommand{\boole}{\keyword{Bool}}

% Variables
\newcommand\vare[1]{#1}
\newcommand\unlete[3]{\keyword{let} \; #1 = #2 \; \keyword{in} \; #3} 

% Applications
\newcommand\appe[2]{#1#2}
\newcommand\tappe[2]{#1[#2]}

% Conditional
\newcommand\conditionale[3]{\keyword{if}\;#1\;\keyword{then}\;#2\;\keyword{else} \; #3}

% Pairs
\newcommand\paire[2]{(#1,#2)}
\newcommand\binlete[4]{\keyword{let}\;#1, #2 = #3\;\keyword{in}\;#4}

% Session Types
\newcommand\newe[1]{\keyword{new}\;#1}
\newcommand\sende[2]{\keyword{send}\;#1\; #2}
\newcommand\recve[1]{\keyword{receive}\;#1}
\newcommand\selecte[1]{\keyword{select}\;#1}
\newcommand\matche[2]{\keyword{match}\;#1\;\keyword{with}\;#2}

% Fork
\newcommand\forke[1]{\keyword{fork}\;#1}

% Datatypes
\newcommand{\ctrcte}{C}
\newcommand\casee[2]{\keyword{case}\;#1\;\keyword{of}\;#2}

% Goal
\newcommand\Alg{\vdash_{a}}

% Equivalent
\newcommand\Equiv[2]{#1\,\thicksim\,#2}


%%% Local Variables:
%%% mode: latex
%%% TeX-master: "cfst-inforum18"
%%% End:
      

\title{Uma linguagem de programação com escolhas mistas em tipos de sessão}
\titlerunning{Uma linguagem de programação com escolhas mistas em tipos de sessão}
\author{Bernardo Almeida, Andreia Mordido e Vasco T. Vasconcelos}
%\authorrunning{F. Author et al.}

\institute{LASIGE, Faculdade de Ciências, Universidade de Lisboa, Portugal}

\begin{document}

\maketitle

\begin{abstract}
  A abstração e formalização da comunicação inerente a sistemas de
  software é fundamental para a confiabilidade de sistemas
  concorrentes.  Neste trabalho propomos um sistema de tipos que
  enriquece os tipos de sessão, permitindo \emph{escolha mistas} entre
  \textit{input} e \textit{output}.
  %
  Na linguagem propomos, \mixedchoice, os processos podem, a cada
  momento, decidir entre ler e escrever num dado canal.
  %
  Um sistema tipos de sessão garante ainda assim que os processos
  seguem o protocolo.

  \keywords{Concorrência \and Troca de mensagens \and Tipos de sessão \and Escolhas mistas}
\end{abstract}

\section{Motivação}

Os tipos de sessão foram propostos para responder à necessidade de
formalização de trocas de mensagens, permitindo definir protocolos na
forma de tipos que representam interações corretas do sistema e que
garantem propriedades tais como a inexistência de erros na comunicação
e de situações de impasse~\cite{ref-lang-primitives}. Contudo, a
expressividade dos tipos de sessão está ainda aquém de todos os
desafios de comunicação que encontramos nos sistemas: não permitem em
particular misturar \textit{input} e \textit{output} na mesma escolha.

Considere-se o seguinte problema.
%
\begin{quotation}
	Dado $k\in\mathbb{Z}$, qual o maior inteiro $n$ tal que
	$\sum_{i=1}^n i < k$?
\end{quotation}

Imagine-se um processador capaz de gerar sequências de números
inteiros mas sem a capacidade de os somar. Imagine-se a um
segundo processador capaz de somar números inteiros, mas incapaz de os
gerar. Ambos os processadores estão aptos para a utilização de
comunicação por canais.
%
Este exemplo simplista retrata o caso em que a interação entre
os processos se dá necessariamente nos dois sentidos e ambos 
os processos deverão
estar aptos, leia-se \emph{corretamente tipados}, para responder 
às necessidades da comunicação.

Tirando partido das restrições computacionais dos nossos dois
processadores, podemos tentar resolver o problema dos seguinte modo:
%
\begin{itemize}
\item Um dos processos, \lstinline|produtor|, gera a sequência de
  números: 1,2,3,\dots e envia-a num canal, até receber no mesmo
  canal uma notificação para terminar o envio. A notificação vem na
  forma de uma marca \lstinline|EOS| (\textit{end-of-stream});
\item O outro processo, \lstinline|consumidor|, vai somando os números
  que recebe os números que recebe enquanto a soma for menor do que
  $n$. Neste ponto envia a marca \lstinline|EOS|.
\item O programa principal cria o canal e lança os dois processos.
\end{itemize}

O canal de comunicação segue o  protocolo delineado acima. Quando
visto do lado do \lstinline|produtor| o canal toma o tipo
%
\begin{lstlisting}
  type CanalInt = ?EOS + !Int;CanalInt
\end{lstlisting}
%
permitindo a cada momento a leitura (\lstinline|?|) da marca
\lstinline|EOS| ou a escrita (\lstinline|!|) de um inteiro. No caso da
leitura o protocolo volta ``ao início''.
%
Quando visto do lado do \lstinline|consumidor| o canal toma o tipo
obtido do acima trocando as operações de leitura pelas de escrita, e
vice-versa. Abreviamos esse tipo com \lstinline|dualof CanalInt|.

\begin{lstlisting}
produtor : CanalInt -> Int -> ()
produtor c i = choice {
  (c, EOS) = receive -> (),
  c = send c n -> consumidor c (i+1)

consumidor : dualof CanalInt -> Int -> Int -> Int -> Int
consumidor c s n k = if n >= n
then choice { c = send EOS -> n--1 }
else choice { (c, m) = receive c -> produtor c (s+m) (n+1) k }

main : Int
main = let
  k = 1000
  (p, c) = new CanalInt in
  fork produtor p 1;
  consumidor c 0 0 k
\end{lstlisting}

O construtor linguístico que manipula escolhas é \lstinline|choice|:
entre as chavetas encontramos \emph{um ou mais} padrões. Os padrões de
\textit{output} levam a palavra reservada \lstinline|send| e os de
\textit{input} \lstinline|receive|. Todas estas primitivas devolvem o
canal (\lstinline|c| no exemplo) onde a interação deve continuar. No
caso da receção, a operação \lstinline|receive| um segundo elemento: o
valor lido. Finalmente, no caso de receção de marcas,
(\lstinline|EOS|, por exemplo), permitimo-nos usar
\textit{pattern-matching}.
%
A escolha do \lstinline|consumidor| é degenerada, quer no ramo
\lstinline|then| como no ramo \lstinline|else|. Poderiamos neste caso
eliminar a palavra reservada \lstinline|choice|.

O programa principal cria um novo canal através da primitiva
\lstinline|new|. O resultado é um par descrevendo as duas extremidades
do canal, \lstinline|p| e \lstinline|c|, destinado a cada um dos
processos.
%
A primitiva \lstinline|fork| lança uma nova \textit{thread} destinada
a correr o processo produtor. O resultado do programa principal é o
resultado do processo \lstinline|consumidor|.

%%% Local Variables:
%%% mode: latex
%%% TeX-master: "main"
%%% End:

\section{A linguagem de programação}
\lstset{language=CFST, style=eclipse}

Esta secção introduz a linguagem apresentando exemplos, a sua sintaxe e semântica e o sistema de tipos. 

A linguagem que propomos é funcional e, como tal, apresenta uma sintaxe bastante semelhante à do Haskell acrescida com primitivas para criação de canais e de envio e receção de dados nos mesmos. Vamos considerar como exemplo ao longo desta secção, o envio de um tipo de dados estruturado em forma de árvore num canal.

A única primitiva de comunicação que a linguagem disponibiliza é troca de mensagens. As mensagens são trocadas em canais de comunicação síncronos e bidirecionais. Cada canal pode ser descrito pelas suas duas extremidades (\textit{endpoints}) e são caracterizados por tipos que descrevem a sequência de mensagens que passam no canal.
Os processos podem escrever numa das pontas do canal ou ler na outra.

Na seguinte figura \ref{fig:types} estão presentes os tipos que estão disponíveis na linguagem.

\begin{figure}[h!]
  \begin{align*}
    \tcBase \grmeq & \inte \grmor \, \chare \grmor \, \boole \grmor \, \unite && \text{Tipos básicos}\\
    T \grmeq       & \tskip \grmor \tSemi{T}{T} \grmor \,\tOut{\tcBase} \grmor \,\tIn{\tcBase} && \text{Tipos}\\
    \grmor         & \tIChoice\{l_i\colon T_i\}_{i\in I} \grmor \tEChoice\{l_i\colon T_i\}_{i\in I} \\ 
    \grmor         & \tcBase \grmor \tUnFun{T}{T} \grmor \tLinFun{T}{T}\\   
    \grmor         & \tPair{T}{T} \grmor \tDatatype{l_i\colon T_i}_{i\in I} \grmor \tRec{\alpha}{T} \grmor \alpha\\
    \kindsch \grmeq & T \grmor \tForall{\alpha}{\kindsch}  && \text{Esquemas de tipos}
    % 
  \end{align*}
\end{figure}


%%% Local Variables:
%%% mode: latex
%%% TeX-master: "cfst"
%%% End:

Os tipos básicos representados na figura \ref{fig:types} são alguns dos existentes no Haskell, inteiros, caracteres, booleanos e ainda o tipo Unit (``()''). Os restantes tipos são compostos pelo operador de sequenciação $\tSemi{\_}{\_}$, pela sua unidade $\tskip$, pelos tipos que representam o envio $\tOut{B}$, a receção $\tIn{B}$, escolhas internas e externas, $\tIChoice{\{l_i\colon T_i\}}_{i\in I}$ e $\tEChoice{\{l_i\colon T_i\}}_{i\in I}$ respetivamente, funções lineares e \textit{unrestricted}, $\tLinFun{T}{T}$ e $\tUnFun{T}{T}$ e ainda pares $\tPair{T}{T}$, tipos de dados $\tDatatype{l_i\colon T_i}_{i\in I}$, tipos recursivos $\tRecK{x}{T}$ e variáveis \textit{x}.

Deste modo, um tipo \lstinline"!Int;?Bool" é um tipo que descreve uma ponta de um canal que espera fazer sequencialmente as operações de enviar um inteiro, receber um booleano e de seguida termina sem mais nenhuma interação.

\subsection{Exemplo: Transmitir uma árvore binária num canal}
\label{sec:example}
Para realizar este exemplo, vamos definir o tipo de dados que representa uma árvore binária:

\begin{lstlisting}
  data Tree = Leaf | Node Int Tree Tree
\end{lstlisting}

O tipo de sessão que descreve o envio da árvore é \lstinline"rec x . +{Leaf: Skip, Node: !Int;x;x}". Este, apresenta uma escolha interna ($\tIChoice{\{l_i\colon T_i\}}_{i\in I}$) que contempla duas opções: \lstinline"Leaf" e \lstinline"Node". No ramo \lstinline"Leaf: Skip" temos o valor \lstinline"Skip" que representa a unidade (não há comunicação). No ramo \lstinline"Node: !Int;x;x" podemos observar a chamada recursiva (\lstinline"rec") do tipo para que seja possível enviar as duas subárvores que este define.

Assim sendo, a função que envia árvores binárias num canal tem o seguinte tipo:

\begin{lstlisting}
  sendTree :: forall a => Tree -> (rec x . +{LeafC : Skip, NodeC: !Int;x;x}); a -> a
\end{lstlisting}


No caso em que se envia uma \lstinline"Leaf", é necessário escolher o ramo através da operação \lstinline"select Leaf" que espera um canal que tenha tipo na forma $\tIChoice{\{l_i\colon T_i\}}_{i\in I}$.
Por outro lado, quando se envia um \lstinline"Node", selecionamos o ramo certo: \lstinline"select Node" e ficamos com o tipo $\tSemi{\tOut{\inte}}{\tSemi{x}{x}}$, assim sendo, primeiro enviamos o inteiro v no canal c ($\sende{v}{c}$) de seguida, é necessário fazer uma chamada recursiva à função para as duas subárvores.

\begin{lstlisting}
  sendTree[rec x . +{LeafC : Skip, NodeC: !Int;x;x}] l c1
  sendTree[Skip] r c2
\end{lstlisting}

O polimorfismo, presente no tipo da função raramente é considerado com tipos de sessão. No entanto, como se pode observar no exemplo, este aparece de forma bastante natural, visto que, o envio de uma árvore generaliza o envio de um único valor, que é polimórfico.

As chamadas a funções polimórficas, são na forma \lstinline"sendTree [Skip]" porque nesta linguagem é necessário especificar o tipo de que as variáveis (neste caso \textit{a}) vão adotar na chamada recursiva.

A função que recebe a árvore binária é análoga mas com o tipo de sessão dual \lstinline"rec x . &{Leaf: Skip, Node: ?Int;x;x}" que, em vez de impor a seleção de um dos ramos (\lstinline"select"), oferece uma escolha $\matche{tree}{\{l_i\,\to\,S_i\}_{i\in I}}$ dos mesmos.
\begin{lstlisting}  
  receiveTree c =
    match c with
      LeafC c1 -> (Leaf, c1)
      NodeC c1 ->
        let x, c2 = receive c1 in
        let left, c3 = receiveTree [rec x.&{LeafC: Skip, NodeC: ?Int;x;x}] c2 in
        let right, c4 = receiveTree [Skip] c3 in
        (Node x left right, c4)
\end{lstlisting}


\subsection{Expressões}
A figura \ref{fig:expressions} apresenta a sintaxe para as expressões da linguagem. As expressões básicas (para os tipos básicos) são inteiros (ex: 1), caracteres (ex: 'a'), booleanos (True e False) e ainda o tipo Unit (``()'').
As aplicações, operações de envio e receção em canais e expressões para escolhas foram brevemente descritas na secção \ref{sec:example}

Das restantes expressões, é importante realçar as operações de criação de canais e de fios de execução (\textit{threads}).
A operação de criação de canais \lstinline"new T", devolve um par com as duas extremidades do canal que são descritas pelo tipo T. Mais precisamente, a primeira extremidade (primeiro elemento do par) é descrita pelo tipo T e a segunda extremidade (segundo elemento) pelo seu dual (\textbf{dualof} T).
A expressão \lstinline"fork e" é responsável por criar um novo fio de execução onde a expressão e vai ser executada, esta operação devolve $\unite$.

\begin{figure}[ht]
  \begin{align*}
    e \grmeq & \unite \grmor \inte \grmor \chare \grmor \boole && \text{Expressões básicas}\\
    \grmor & \vare{x} \grmor \unlete{x}{e}{e} && \text{Variáveis}\\
    \grmor & \appe{e}{e} \grmor \tappe{e}{T} && \text{Applicações}\\
    \grmor & \conditionale{e}{e}{e} && \text{Condicional}\\
    \grmor & \paire{e}{e} \grmor \binlete{x}{y}{e}{e} && \text{Pares}\\
    %
    \grmor & \newe{T} \grmor \sende{e}{e} \grmor \recve{e} && \text{Operações de comunicação}\\
    \grmor & \selecte{e} \grmor \matche{e}{\{l_i\;\to\;e_i\}_{i\in I}} \\
    \grmor & \forke{e}  && \text{Fork}\\
    \grmor & \ctrcte \grmor \casee{e}{\{C_i\;\to\;e_i\}_{i\in I}} && \text{Tipos de dados}\\
    %
  \end{align*}
  \hrulefill
  \caption{Sintaxe das expressões}
  \label{fig:expressions}
\end{figure}


%%% Local Variables:
%%% mode: latex
%%% TeX-master: "cfst-inforum18"
%%% End:

\subsection{Validação}

Para garantir que o sistema está isento de erros a nossa fase de validação contempla com um sistema de \textit{kinding} para assegurar a boa formação dos tipos, uma verificação de tipos onde se verifica se todas as expressões têm o tipo esperado e uma parte que define quando é que dois tipos são equivalentes.

\subsubsection{Sistema de \textit{kinding}}
O sistema de \textit{kinding} tem como objetivo assegurar a boa formação dos tipos. Para além de verificar se os tipos são bem formados classifica-os nas categorias de tipos de sessão ou tipos gerais. Associa ainda multiplicidades aos tipos (linear ou \textit{unrestricted}).
Por exemplo, o tipo $\tOut{\inte}$ é bem formado e é um tipo de sessão linear, isto é, só pode ser utilizado uma vez. Por outro lado o tipo $\tRec{x}{\tSemi{a}{x}}$ é mal formado se a variável \textit{a} não estiver no ambiente de \textit{kinding} (onde estão todas as associações de variáveis com os respetivos \textit{kinds}).

\subsubsection{Verificação de tipos}
\label{sec:typecheck}
Para fazermos a verificação de tipos utilizámos um sistema bidirecional, isto é, distinguimos duas relações:
\begin{itemize}
\item Dado um contexto e uma expressão, sintetizar o tipo T: \begin{figure}[h!]
%\centering
  \begin{gather*}
    \frac{\overbrace{\text{...}}^{\text{Específico para cada expressão e}}}
    {\underbrace{\kindEnv;\varEnv \Goal \text{e}}_{\text{In}} \rightarrow
      \underbrace{T;\varEnv}_{\text{Out}}}
    %
  \end{gather*}
  %\hrulefill
%  \caption{Sintetizar um tipo T a partir de uma expressão e}
  \label{fig:synthesize}
\end{figure}



%%% Local Variables:
%%% mode: latex
%%% TeX-master: "cfst-inforum18"
%%% End:
\item Dado um contexto, uma expressão e um tipo verificar o tipo da expressão de encontro ao tipo esperado:
\begin{figure}[h!]
%\centering
  \begin{gather*}
    \frac{\kindEnv;\varEnv_1 \vdash U;\varEnv_2 \qquad \kindEnv \vdash \Equiv{U}{T}}
         {\underbrace{\kindEnv;\varEnv_1 \vdash \, e \colon T}_{\text{Input}} \rightarrow \underbrace{\varEnv_2}_{\text{Output}}}
         % 
  \end{gather*}
  %\hrulefill
  \caption{Verificar de encontro a um tipo}
  \label{fig:check-against}
\end{figure}


%%% Local Variables:
%%% mode: latex
%%% TeX-master: "cfst-inforum18"
%%% End:

\end{itemize}

\subsubsection{Equivalencia de tipos}

Determinar se dois tipos são equivalentes apresenta diversos desafios. O artigo apresentado por Christensen et. al \cite{decidable-CFP-bisimilarity} procura mostrar que a equivalência de tipos é decidível para processos independentes do contexto, contudo não define diretamente um algoritmo.

O algoritmo que utilizamos está baseado nas ideias de Jančar et. al \cite{bisimilarity} em que traduzimos um tipo de sessão numa gramática independente do contexto e onde geramos uma árvore de expansão. Este algoritmo, tal como está implementado ainda não é completo, visto que, há tipos que sabemos equivalentes mas que o algoritmo diz não serem. Este algoritmo é ainda trabalho que está em progresso.

% que o algoritmo tal como implementado ainda nao é completo (há tipos de sabemos equivalentes, mas q o algoritmo diz nao serem); que estamos a trabalhar no assunto. 

\subsection{Geração de código}

A linguagem que apresentamos é \textit{call-by-value} e o Haskell (código gerado) é \textit{call-by-name}, isto é, apenas computa as expressões que são passadas como argumento quando esta são utilizadas em vez de computar antes de chamar a função. Para resolver esta questão, tirámos partido da extensão da linguagem Haskell \textit{BangPatterns} que, se usar-mos um sinal de ! antes de cada parâmetro força a avaliação do mesmo. Uma função \lstinline"fun x = e" quando traduzida fica \lstinline"fun !x = e".

O código gerado relativo às operações de comunicação presentes na figura \ref{fig:expressions} foi implementado recorrendo a uma \textit{MVar} que é uma zona de memória mutável que apenas tem dois estados, ou vazio ou um valor do tipo t. Tem duas operações fundamentais \textit{putMVar} para escrever nessa zona de memória (operação de \lstinline"send") e \textit{takeMVar} para ler dessa zona (operação de \lstinline"receive").

Uma \textit{MVar} t têm um tipo associado (t) que é sempre o mesmo para cada \textit{MVar}. Como esta é utilizada para a implementação dos canais de comunicação necessitamos que o seu tipo possa variar para que seja possível enviar, por exemplo, um valor inteiro seguido de um booleano (\lstinline"!Int;!Bool"). Para que o sistema de tipos do Haskell não verifique os tipos que circulam nos canais utilizámos uma primitiva do Haskell que converte um valor de um tipo noutro. Apesar de insegura, esta primitiva não apresenta qualquer problema porque fazemos a nossa própria verificação de tipos (secção \ref{sec:typecheck}).

O código gerado está escrito na linguagem Haskell e derivado disso, as operações de envio ($\sende{v}{c}$), receção ($\recve{c}$), criação de canais ($\newe{T}$) e criação de fios de execução ($\forke{e}$) são forçosamente operações sobre um mónade, mais precisamente, são operações de IO.

Este facto representa uma dificuldade que é decidir quando traduzir as expressões para código de um mónade ou não. A cada momento da tradução ($[\![e]\!]$) apenas temos disponível uma expressão, que sabemos se é monádica ou não, mas não temos qualquer informação acerca das expressões anteriores (na mesma função). Se, num dado momento, encontramos uma expressão que origina código monádico, teríamos de retraduzir o código anterior para estar também na forma monádica.

A nossa aproximação foi resolver este problema antes de aplicar a função de tradução, isto é, anotar a árvore sintática com valores booleanos que representam o estado que é esperado de cada expressão. Antes disso, é necessário percorrer as expressões para saber que funções são monádicas (têm alguma expressão monádica), isto porque, as chamadas a funções são apenas aplicações de variáveis.

Assim sendo, geramos código para cada expressão com base na tabela \ref{tab:monad} que contempla o valor esperado e o encontrado pela função de tradução a cada momento da geração de código.

% Assim sendo, após a primeira iteração obtemos uma estrutura com as funções e um valor booleano associado que indica se são monádicas. Após esta fase, vamos anotar a árvore sintática e terminamos este processo com os valores esperados para cada nó da árvore.

% Neste momento, a função de tradução, para cada expressão, tem informação acerca do valor monádico esperado e do encontrado (na própria função de tradução).

% A tabela \ref{tab:monad} contempla as combinações dos valores monádicos encontrados e esperados assim como o código que deve ser gerado em cada um dos casos.

\begin{table}
\begin{center}
  \begin{tabular}[ht!]{| c | c | c |}
    \hline  
    \quad Valor esperado \quad&\quad Valor encontrado \quad&\quad Código gerado \quad\\\hline
    False & False & e \\
    True & False & \lstinline"return e" \\
    True & True & e \\
    False & True & \lstinline"e >>= x -> x" \\
    \hline
  \end{tabular}
  \vspace{0.2cm}
  \caption{Tabela para geração de código monádico}
  \label{tab:monad}
\end{center}
\end{table}

% Nos casos em o valor esperado e o valor encontrado são iguais a tradução deve ser literal, no caso em que o valor esperado é true e o valor encontrado é false a tradução deve ser \lstinline"return e" para tornar o valor que é suposto ser monádico num valor monádico. No último caso, em que o valor esperado é false e o valor encontrado é true devemos, ao traduzir, retirar o valor do monad: \lstinline"e >>= x -> x".
Por exemplo, o código fonte para o envio de uma árvore seria:

\begin{lstlisting}
  case tree of
    Node x l r ->
      let w1 = select NodeC tree in
      let w2 = send x w1 in
      let w3 = sendTree[rec x . +{LeafC : Skip, NodeC: !Int;x;x}] l w2 in
      let w4 = sendTree[Skip] r w3 in
      w4
\end{lstlisting}

e o código gerado:
\begin{lstlisting}
  case tree of
   Node x l r ->
    _send "NodeC" tree >>=
    \w1 -> _send x w1 >>=
    \w2 -> ((sendTree l) w2) >>=
    \w3 -> ((sendTree r) w3) >>=
    \w4 -> return w4 
\end{lstlisting}

Note-se que, como são enviados e recebidos valores o código encontra-se num monáde de IO. 


\subsection{Testes}

Foram realizados diversos tipos de testes, com diversas ferramentas para aferir a robustez do compilador. As ferramentas que utilizámos foram o \textit{HSpec}\footnote{\url{http://hspec.github.io/}} e o \textit{HUnit}\footnote{\url{https://github.com/hspec/HUnit}} para desenvolver os testes e o \textit{HPC}\footnote{\url{https://wiki.haskell.org/Haskell_program_coverage}} para aferir a cobertura dos testes. Foram feitos testes unitários para testar diversas funções isoladamente e ainda testes com programas, nos quais escrevemos um programa na linguagem apresentada e verificamos se o seu resultado é o esperado.



%%% Local Variables:
%%% mode: latex
%%% TeX-master: "cfst-inforum18"
%%% End:

\section{Equivalência de tipos}

Um compilador para uma linguagem de programação com tipos de sessão
com escolhas mistas, como os que propomos, tem de estar munido de um
algoritmo de verificação de equivalência de tipos.

Os tipos de sessão que estendemos
neste trabalho não são só tipos de sessão regulares.
Note-se que a sintaxe que propomos estende também
tipos de sessão livres do contexto.
Assim, o algoritmo de verificação de equivalência de tipos 
não é óbvio e deverá basear-se em \emph{Basic Process Algebras}
(BPA), como provado por Thiemann e Vasconcelos em~\cite{ref-cfst}.
Neste trabalho pretendemos tirar partido do algoritmo de verificação
de tipos livres de contexto que propomos em~\cite{type-equiv}.
Este algoritmo começa por traduzir tipos livres de contexto
em gramáticas simples (i.e.\ gramáticas determinísticas
em Greibach Normal Form) e, depois de simplificar as 
gramáticas, decide a existência de uma bisimulação entre as gramáticas
através de uma árvore de expansão. 
A árvore de expansão consiste numa
sucessão de passos de expansão e de simplificação 
sobre os símbolos da gramática, de acordo com 
as produções da gramática e com algumas regras de simplificação
propostas por Caucal, Christensen,
H\"uttel, Stirling, Jan\v car, and Moller
~\cite{caucal1986decidabilite,DBLP:journals/iandc/ChristensenHS95,janvcar1999techniques}

Os tipos de sessão com escolhas mistas têm
uma expressividade idêntica à expressividade das gramáticas
simples. Assim, a redução do algoritmo a estas gramáticas
permite a sua incorporação num compilador
para uma linguagem de programação com estes tipos mais expressivos.

%%% Local Variables:
%%% mode: latex
%%% TeX-master: "main"
%%% End:

\section{A implementação}

A implementação de um compilador com escolhas mistas, que estendem os tipos de sessão
independentes do contexto, deverá seguir a mesma linha de ideias que as presentes no compilador da linguagem FreeST\cite{2019freest,2019freest-inforum}.

A fase de validação é, também, composta por dois sistemas bidirecionais:
\begin{itemize}
\item \textbf{Um sistema de verificação de géneros} (\textit{kinding}) que assegura que a boa formação dos tipos, incluindo as escolhas mistas.
\item \textbf{Um sistema de verificação de tipos} cujas relações permitem, sintetizar o tipo de uma expressão e verificar se uma expressão tem o tipo esperado. 
\end{itemize}
A introdução de escolhas mistas não invalida nenhuma das duas relações.
Uma escolha (mista) necessita que os todos os seus componentes sejam tipos de sessão lineares. A verificação de tipos garante que os padrões de uma \lstinline|choice| respeitam os padrões do tipo do canal.

O ambiente de execução (\textit{runtime system}) é bastante compacto e é, essencialmente, composto pelas primitivas de criação de novos fios de execução (\lstinline|fork|) e primitivas para a manipulação de canais (\lstinline|send|, \lstinline|receive|, \lstinline|choice|). O compilador gera código Haskell que posteriormente será compilado através do GHC, o compilador convencional do Haskell.

%%% Local Variables:
%%% mode: latex
%%% TeX-master: "main"
%%% End:


\paragraph{Agradecimentos.}

Este trabalho foi apoiado em parte pela FCT através do projeto
Confident, PTDC/EEI-CTP/4503/2014, e da Unidade de Investigação LASIGE,
UID/CEC/00408/2019.

\bibliographystyle{splncs04}
\bibliography{bibliography}

\end{document}
%%% Local Variables:
%%% mode: latex
%%% TeX-master: t
%%% End:
