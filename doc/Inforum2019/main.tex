% This is samplepaper.tex, a sample chapter demonstrating the
% LLNCS macro package for Springer Computer Science proceedings;
% Version 2.20 of 2017/10/04
%
\documentclass[runningheads]{llncs}

\usepackage[portuguese]{babel}
\usepackage[utf8]{inputenc}  % for proper diacritics
\usepackage[T1]{fontenc}
\usepackage{color}
\usepackage {listings}
\usepackage{tikz}
\usepackage{wrapfig}
\usepackage{hyperref}
\usepackage{amsmath,amssymb}
\usepackage{alltt}
\usepackage{flushend}
\usepackage{graphicx}

% If you use the hyperref package, please uncomment the following line
% to display URLs in blue roman font according to Springer's eBook style:
\renewcommand\UrlFont{\color{blue}\rmfamily}

%%% Local Variables:
%%% mode: latex
%%% TeX-master: "cfst-inforum18"
%%% End:
      
 
% THEME
\newtcolorbox{mybox}{colback=orange!5!white,colframe=orange!75!black}
\newtcolorbox{myboxazul}{colback=teal!5!white,colframe=teal!75!black}

% TIKZ 
\usetikzlibrary{positioning}
\usetikzlibrary{shapes,arrows}
\usepgfplotslibrary{dateplot}
\tikzstyle{block} = [rectangle, draw, 
    text width=5cm, text centered, rounded corners, minimum height=4em]
\tikzstyle{block2} = [rectangle, draw, 
    text width=10cm, text centered, rounded corners, minimum height=4em]
\tikzstyle{line} = [draw, -latex']

\newenvironment<>{varblock}[2][.9\textwidth]{%
  \setlength{\textwidth}{#1}
  \begin{actionenv}#3%
    \def\insertblocktitle{#2}%
    \par%
    \usebeamertemplate{block begin}}
  {\par%
    \usebeamertemplate{block end}%
  \end{actionenv}}

\newenvironment{changemargin}[3]{%
\begin{list}{}{%
\setlength{\leftmargin}{#1}%
\setlength{\rightmargin}{#2}%
\setlength{\topmargin}{#3}%
}%
\item[]}
{\end{list}}

% session constructors
\newcommand{\intk}{\keyword{int}}
\newcommand{\skipk}{\keyword{skip}}

% The language
\newcommand{\freest}{\textsc{FreeST}}

% notes
\newcommand{\todo}[1]{[{\color{blue}\textbf{#1}}]}

% Keywords
\newcommand{\keyword}[1]{\mathsf{#1}}
\newcommand{\link}{\keyword{lin}}
\newcommand{\unk}{\keyword{un}}

% Kinds
\newcommand\prekind{\upsilon}
\newcommand{\stypes}{\mathcal S}
\newcommand\kinds{\stypes}
\newcommand{\types}{\mathcal T}
\newcommand\kindt{\types}
\newcommand\kindsch{\mathcal C}
\newcommand\kind{\kappa}

% Multiplicity
\newcommand\Un{\ensuremath{\mathbf{u}}} % \infty
\newcommand\Lin{\ensuremath{\mathbf{l}}} % 1 

% Grammars
\newcommand{\grmeq}{\; ::= \;}
\newcommand{\grmor}{\;\mid\;}

% type constructors
\newcommand\tcBase{B}
\newcommand\tcLolli\multimap
\newcommand\tcFun\to
\newcommand\tcBang{\mathop!}

% Keywords for types
\newcommand\kRec{\keyword{rec}}
\newcommand\kForall{\keyword{forall}}

% Types
\newcommand{\tskip}{\keyword{Skip}}
\newcommand\tSemi[2]{#1;#2}
\newcommand\tOut[1]{\tcBang#1}
\newcommand\tIn[1]{?#1}
\newcommand{\tMsg}[1]{\sharp{#1}}
\newcommand\tIChoice[1]{\oplus{#1}}
\newcommand\tEChoice[1]{\&{#1}}
\newcommand{\tChoice}[1]{\star{#1}}
\newcommand{\tData}[1]{[{#1}]}
\newcommand\tUnFun[2]{#1\tcFun#2}
\newcommand\tLinFun[2]{#1\tcLolli#2}
\newcommand\tPair[2]{(#1,\,#2)}
\newcommand\tDatatype[1]{{[#1]}}
\newcommand\tRec[2]{\kRec\,#1\,.\,#2}
%\newcommand\tForall[2]{\forall\,#1\,.\,#2}
%\newcommand\tForall[2]{\kForall\,#1\,=>\,#2}
\newcommand\tForall[2]{\forall\,#1\Rightarrow#2}
% Basic Types
\newcommand{\unite}{()}
\newcommand{\inte}{\keyword{Int}}
\newcommand{\chare}{\keyword{Char}}
\newcommand{\boole}{\keyword{Bool}}

\newcommand\tRecK[2]{\kRec\,#1\,.\,#2}
% Environments
\newcommand{\Empty}{\varepsilon}
\newcommand\emptyEnv{\Empty}
\newcommand\kindEnv{\Delta}
\newcommand\varEnv{\Gamma}

% Variables
\newcommand\vare[1]{#1}
\newcommand\unlete[3]{\keyword{let} \; #1 = #2 \; \keyword{in} \; #3} 

% Applications
\newcommand\appe[2]{#1#2}
\newcommand\tappe[2]{#1[#2]}

% Conditional
\newcommand\conditionale[3]{\keyword{if}\;#1\;\keyword{then}\;#2\;\keyword{else} \; #3}

% Goal
\newcommand\Goal{\vdash}

% Pairs
\newcommand\paire[2]{(#1,#2)}
\newcommand\binlete[4]{\keyword{let}\;#1, #2 = #3\;\keyword{in}\;#4}

% Session Types
\newcommand\newe[1]{\keyword{new}\;#1}
\newcommand\sende[2]{\keyword{send}\;#1\; #2}
\newcommand\sendce[1]{\keyword{send}\;#1}
\newcommand\recve[1]{\keyword{receive}\;#1}
\newcommand\selecte[2]{\keyword{select}\;#1\;{#2}}
\newcommand\matche[2]{\keyword{match}\;#1\;\keyword{with}\;#2}

% Fork
\newcommand\forke[1]{\keyword{fork}\;#1}

% Datatypes
\newcommand{\ctrcte}{C}
\newcommand\casee[2]{\keyword{case}\;#1\;\keyword{of}\;#2}

% Sequents
\newcommand{\isType}[3][\Delta]{{#1} \vdash {#2} : {#3}}
\newcommand{\algkindout}[3][\kindEnv]{{#1} \Alg {#2} \shortrightarrow{ #3}}
\newcommand{\algkindin}[3][\kindEnv]{{#1} \Alg {#2} \shortleftarrow {#3}}
\newcommand{\subkind}[2]{{#1} <: {#2}}
\newcommand{\algtypeout}[4][\kindEnv;\varEnv]{{#1} \Alg {#2} \shortrightarrow {#3};{#4}}
%\newcommand{\algtypein}[4][\kindEnv;\varEnv]{{#1} \Alg {#2}\colon {#3}\shortrightarrow {#4}}
\newcommand{\algtypein}[4][\kindEnv;\varEnv]{{#1} \Alg {#2}\shortleftarrow {#3}; {#4}}
\newcommand{\ctxequiv}[3][\kindEnv]{{#1} \vdash \Equiv{#2}{#3}}
\newcommand{\typeequiv}[3][\kindEnv]{{#1} \vdash \Equiv{#2}{#3}}
\newcommand{\isqualifier}[3][\kindEnv]{{#1} \vdash {#2}\colon{#3}}
\newcommand{\isLin}[2][\kindEnv]{\isqualifier[#1]{#2}\link}
\newcommand{\isUn}[2][\kindEnv]{\isqualifier[#1]{#2}\unk}
\newcommand{\contractive}[2][\kindEnv]{{#1} \vdash_{\textsf c} {#2}}
%\newcommand\Alg{\vdash_{\textsf a}}
\newcommand\Alg{\vdash}

% Operators
\newcommand\Extract[1]{\leadsto_{#1}}% \rightlsquigarrow}
\newcommand{\subs}[3]{[{#1}/{#2}]{#3}}
\newcommand\dual[1]{\overline{#1}}

% Predicates
%\newcommand\Equiv[2]{#1\,\thicksim\,#2}
\newcommand\Equiv[2]{#1\,\sim\,#2}

% Colour

\newcommand{\Blue}[1]{\textcolor{blue}{#1}}
\newcommand{\Red}[1]{\textcolor{red}{#1}}
\newcommand{\Brown}[1]{\textcolor{brown}{#1}}
\newcommand{\highlight}[1]{\Blue{#1}}

% ECLIPSE LOOK

\newcommand\Small{\small}
%\newcommand\Small{\fontsize{7.5}{8}\selectfont} 

\definecolor{darkviolet}{rgb}{0.5,0,0.4}
\definecolor{darkgreen}{rgb}{0,0.4,0.2} 
\definecolor{darkblue}{rgb}{0.1,0.1,0.9}
\definecolor{darkgrey}{rgb}{0.5,0.5,0.5}
\definecolor{lightblue}{rgb}{0.4,0.4,1}

\lstdefinestyle{eclipse}{
  breaklines=true,
  basicstyle=\sffamily\Small,
  emphstyle=\color{red}\bfseries, 
  keywordstyle=\color{darkviolet}\bfseries,
  commentstyle=\color{darkgreen},
  stringstyle=\color{darkblue},
  numberstyle=\color{darkgrey},%\lstfontfamily,
  emphstyle=\color{red},
  % get also javadoc style comments
  morecomment=[s][\color{lightblue}]{/**}{*/},
  %columns=fullflexible, %spaceflexible, %flexible, fullflexible             
  %  escapeinside=`',
  %  escapechar=@,
  showstringspaces=false,
  numbers=left,
  tabsize=2
}

\lstdefinestyle{eclipse-Haskell}{
  breaklines=true,
  basicstyle=\sffamily\Small,
  emphstyle=\color{red}\bfseries, 
  keywordstyle=\color{darkviolet}\bfseries,
  commentstyle=\color{darkgreen},
  stringstyle=\color{darkblue},
  emphstyle=\color{red},
  % get also javadoc style comments
  morecomment=[s][\color{lightblue}]{/**}{*/},
  %columns=fullflexible, %spaceflexible, %flexible, fullflexible             
  %  escapeinside=`',
  %  escapechar=@,
  showstringspaces=false,
  numbers=none,
  tabsize=2
}

\lstdefinelanguage{freest}{
  style=eclipse,
  morekeywords=[1]{Int, Char, Bool, Skip, type, dualof, forall, rec, let, in, if, then, else, new, send, receive, select, fork, case, of, data, match, with, True, False},
  sensitive=true,
  literate=
  {->}{$\rightarrow$}2
  {-o}{$\multimap$}2
  {=>}{$\Rightarrow$}2
  {alpha}{$\alpha$}1,
  breaklines=true,
  morecomment=[l]{--},%
  morecomment=[s]{{-}{-}},%
  morestring=[b]',%
  morestring=[b]",%
  morestring=[s]{`}{`},%
}

\lstset{
  language=freest,
  numbers=none
}
 
%%% Local Variables:
%%% mode: latex
%%% TeX-master: "main"
%%% End:


\title{Uma linguagem de programação com escolhas mistas em tipos de sessão}
\titlerunning{Uma linguagem de programação com escolhas mistas em tipos de sessão}
\author{Bernardo Almeida, Andreia Mordido e Vasco T. Vasconcelos}
%\authorrunning{F. Author et al.}

\institute{LASIGE, Faculdade de Ciências, Universidade de Lisboa, Portugal}

\begin{document}

\maketitle

\begin{abstract}
  A abstração e formalização da comunicação inerente a sistemas de
  software é fundamental para a confiabilidade de sistemas
  concorrentes.  Neste trabalho propomos um sistema de tipos que
  enriquece os tipos de sessão, permitindo \emph{escolha mistas} entre
  \textit{input} e \textit{output}.
  %
  Na linguagem propomos, os processos podem, a cada momento, decidir
  entre ler e escrever num dado canal.
  %
  Um sistema tipos de sessão garante ainda assim que os processos
  seguem o protocolo.

  \keywords{Concorrência \and Troca de mensagens \and Tipos de sessão \and Escolhas mistas}
\end{abstract}

\section{Introdução}

Os tipos de sessão foram propostos para responder à necessidade
de formalização de trocas de mensagens, permitindo definir protocolos 
na forma de tipos que representam interações corretas do sistema e que 
garantem propriedades tais como a inexistência de erros na comunicação 
e de situações de impasse. Contudo, a expressividade dos tipos de sessão
está ainda aquém de todos os desafios de comunicação que encontramos nos 
sistemas.

Considere-se que temos um processo que consegue gerar sequências de números
inteiros mas não tem capacidade de os somar. Por outro lado, temos um
segundo processo que é capaz de somar números inteiros, mas não consegue
gerá-los. Ambos os processos estão aptos para a utilização de comunicação
por canais. Considere-se então o seguinte problema:

\begin{example}
	Dado $k\in\mathbb{Z}$, qual o maior inteiro $n$ tal que
	$\sum_{i=1}^n i < k$?
\end{example}

Este exemplo simplista retrata o caso em que a interação entre
os processos se dá necessariamente nos dois sentidos e ambos 
os processos deverão
estar aptos, leia-se \emph{corretamente tipados}, para responder 
às necessidades da comunicação.

Neste trabalho propomos uma linguagem de programação
com tipos capazes de modelar um cenário como o descrito acima.

%%% Local Variables:
%%% mode: latex
%%% TeX-master: "main"
%%% End:

\section{A Linguagem}
\lstset{language=CFST}
Esta secção introduz a linguagem apresentando exemplos, a sua sintaxe e semântica e o sistema de tipos. 

A linguagem que propomos é funcional e, como tal, apresenta uma sintaxe bastante semelhante à do Haskell acrescida com primitivas para criação de canais e de envio e receção de dados nos mesmos. Vamos considerar como exemplo ao longo desta secção, o envio de um tipo de dados estrururado em forma de árvore num canal.

A única primitiva de comunicação que a linguagem disponibiliza é troca de mensagens. As mensagens são trocadas em canais de comunicação síncronos e bidireccionais. Cada canal pode ser descrito pelas suas duas extremidades (\textit{endpoints}) e são caracterizados por tipos que descrevem a sequência de mensagens que passam no canal.
Os processos podem escrever numa das pontas do canal ou ler na outra.

Na seguinte figura \ref{fig:types} estão presentes os tipos que estão disponíveis na linguagem.

% \begin{figure}[t]
  \begin{align*}
    \tcBase \grmeq & \inte \grmor \, \chare \grmor \, \boole \grmor \, \unite  \grmor \, l && \text{Tipos básicos}\\
    T \grmeq       & \tskip \grmor \tChoice{T}{T} \grmor \tSemi{T}{T}  \grmor \,\tOut{\tcBase} \grmor \,\tIn{\tcBase} \grmor \tRec{x}{T} \grmor x && \text{Tipos} 
    % 
  \end{align*}
%   \hrulefill
%   \caption{Sintaxe dos tipos}
%   \label{fig:types}
% \end{figure}


%%% Local Variables:
%%% mode: latex
%%% TeX-master: "main"
%%% End:


Os tipos básicos representados na figura \ref{fig:types} são alguns dos existentes no Haskell, inteiros, caracteres, booleanos e ainda o tipo Unit (``()''). Os restantes tipos são compostos a partir do operador de sequenciação $\tSemi{\_}{\_}$ da sua unidade $\tskip$, os tipos que representam o envio $\tOut{B}$, a receção $\tIn{B}$, escolhas internas e externas, $\tIChoice{\{l_i\colon T_i\}}_{i\in I}$ e $\tEChoice{\{l_i\colon T_i\}}_{i\in I}$ respectivamente, funções lineares e \textit{unrestricted}, $\tLinFun{T}{T}$ e $\tUnFun{T}{T}$ e ainda pares $\tPair{T}{T}$, tipos de dados $\tDatatype{l_i\colon T_i}_{i\in I}$, tipos recursivos $\tRecK{x}{T}$ e variáveis \textit{x}.

Deste modo, um tipo \lstinline"!Int;?Bool;Skip" é um tipo que descreve uma ponta de um canal que espera fazer sequencialmente as operações de enviar um inteiro, receber um booleano e de seguida termina sem mais nenhuma interação.

\subsection{Exemplo: Enviar uma árvore binária num canal}
\label{sec:example}
Para realizar este exemplo, vamos definir o tipo de dados que representa uma árvore binária:

\begin{lstlisting}
  data Tree = Leaf | Node Int Tree Tree
\end{lstlisting}

O tipo de sessão que descreve o envio da árvore é \lstinline"rec x . +{Leaf: Skip, Node: !Int;x;x}". Este, apresenta uma escolha interna ($\tIChoice{\{l_i\colon T_i\}}_{i\in I}$) que contempla duas opções: \lstinline"Leaf" e \lstinline"Node". No ramo \lstinline"Leaf: Skip" temos o valor \lstinline"Skip" que representa a unidade (um canal vazio). No ramo \lstinline"Node: !Int;x;x" podemos observar a chamada recursiva (\lstinline"rec") do tipo para que seja possível enviar as duas subárvores que este define.

Assim sendo, a função que envia árvores binárias num canal tem o seguinte tipo:

\begin{lstlisting}
  sendTree :: forall a => Tree -> (rec x . +{LeafC : Skip, NodeC: !Int;x;x}); a -> a
\end{lstlisting}


No caso em que se envia uma \lstinline"Leaf", é necessário escolher o ramo através da operação \lstinline"select Leaf" que espera um canal que tenha tipo na forma $\tIChoice{\{l_i\colon T_i\}}_{i\in I}$.
Por outro lado, quando se envia um \lstinline"Node", selecionamos o ramo certo: \lstinline"select Node" e ficamos com o tipo $\tSemi{\tOut{\inte}}{\tSemi{x}{x}}$, assim sendo, primeiro enviamos o inteiro v no canal c ($\sende{v}{c}$) de seguida, é necessário fazer uma chamada recursiva à função para as duas subárvores.

\begin{lstlisting}
  sendTree[rec x . +{LeafC : Skip, NodeC: !Int;x;x}] l c1
  sendTree[Skip] r c2
\end{lstlisting}

O polimorfismo, presente no tipo da função raramente é considerado com tipos de sessão. No entanto, como se pode observar no exemplo, este aparece de forma bastante natural, visto que, o envio de uma árvore generaliza o envio de um único valor, que é polimorfico.

As chamadas a funções polimorficas, são na forma \lstinline"sendTree [Skip]" porque nesta linguagem é necessário especificar o tipo de que as variáveis (neste caso \textit{a}) vão adotar na chamada recursiva.

A função que recebe a árvore binária é análoga mas com o tipo de sessão dual \lstinline"rec x . &{Leaf: Skip, Node: ?Int;x;x}" que, em vez de impor a seleção de um dos ramos (\lstinline"select"), oferefece um escolha $\matche{tree}{\{l_i\,\to\,S_i\}_{i\in I}}$.
\begin{lstlisting}  
  receiveTree c =
    match c with
      LeafC c1 -> (Leaf, c1)
      NodeC c1 ->
        let x, c2 = receive c1 in
        let left, c3 = receiveTree [rec x.&{LeafC: Skip, NodeC: ?Int;x;x}] c2 in
        let right, c4 = receiveTree [Skip] c3 in
        (Node x left right, c4)
\end{lstlisting}


\subsection{Expressões}
A figura \ref{fig:expressions} apresenta a sintaxe para as expressões da linguagem. As expressões básicas (para os tipos básicos) são inteiros (ex: 1), caracteres (ex: 'a'), booleanos (True e False) e ainda o tipo Unit (``()'').
As aplicações, operações de envio e receção em canais e expressões para escolhas foram brevemente descritas na secção \ref{sec:example}

Das restantes expresões, é importante realçar as operações de criação de canais e de fios de execução (\textit{threads}).
A operação de criação de canais \lstinline"new T", devolve um par com as duas extremidades do canal que são descritas pelo tipo T. Mais precisamente, a primeira extremidade (primeiro elemento do par) é descrita pelo tipo T e a segunda extremidade (segundo elemento) pelo seu dual (\textbf{dualof} T).
A expressão \lstinline"fork e" é responsável por criar um novo fio de execução onde a expressão e vai ser executada, esta operação devolve $\unite$.

\begin{figure}[t]
  \begin{align*}
    e \grmeq & \unite \grmor x \grmor c \grmor \text{True} \grmor \text{False} && \text{Expressões básicas}\\
    \grmor & \vare{x} \grmor \unlete{x}{e}{e} && \text{Variáveis e let}\\
    \grmor & \appe{e}{e} \grmor \tappe{e}{T} && \text{Aplicações}\\
    \grmor & \conditionale{e}{e}{e} && \text{Condicional}\\
    \grmor & \paire{e}{e} \grmor \binlete{x}{y}{e}{e} && \text{Pares}\\
    %
    \grmor & \newe{T} \grmor \sende{e}{e} \grmor \recve{e} && \text{Operações de comunicação}\\
    \grmor & \selecte{e} \grmor \matche{e}{\{l_i\;\to\;e_i\}_{i\in I}} \\
    \grmor & \forke{e}  && \text{Fork}\\
    \grmor & \ctrcte \grmor \casee{e}{\{C_i\;\to\;e_i\}_{i\in I}} && \text{Tipos de dados}\\
    %
  \end{align*}
  \hrulefill
  \caption{Sintaxe das expressões}
  \label{fig:expressions}
\end{figure}


%%% Local Variables:
%%% mode: latex
%%% TeX-master: "cfst-inforum18"
%%% End:


\subsection{Verificação de tipos... (Validation)}
\todo{Kinding + typechecking + type equiv}

\subsection{CodeGen}




%%% Local Variables:
%%% mode: latex
%%% TeX-master: "cfst-inforum18"
%%% End:

\section{Equivalência de tipos}

Um compilador para uma linguagem de programação com
tipos de sessão com escolhas mistas,
como os que propomos, deverá basear-se num algoritmo de 
verificação de equivalência de tipos. 

Os tipos de sessão que estendemos
neste trabalho não são só tipos de sessão regulares.
Note-se que a sintaxe que propomos estende também
tipos de sessão livres do contexto.
Assim, o algoritmo de verificação de equivalência de tipos 
não é óbvio e deverá basear-se em \emph{Basic Process Algebras}
(BPA), como provado por Thiemann e Vasconcelos em~\cite{ref-cfst}.
Neste trabalho pretendemos tirar partido do algoritmo de verificação
de tipos livres de contexto que propomos em~\cite{type-equiv}.
Este algoritmo começa por traduzir tipos livres de contexto
em gramáticas simples (i.e.\ gramáticas determinísticas
em Greibach Normal Form) e, depois de simplificar as 
gramáticas, decide a existência de uma bisimulação entre as gramáticas
através de uma árvore de expansão. 
A árvore de expansão consiste numa
sucessão de passos de expansão e de simplificação 
sobre os símbolos da gramática, de acordo com 
as produções da gramática e com algumas regras de simplificação
propostas por Caucal, Christensen,
H\"uttel, Stirling, Jan\v car, and Moller
~\cite{caucal1986decidabilite,
  DBLP:journals/iandc/ChristensenHS95,janvcar1999techniques}

Os tipos de sessão com escolhas mistas têm
uma expressividade idêntica à expressividade das gramáticas
simples. Assim, a redução do algoritmo a estas gramáticas
permite a sua incorporação num compilador
para uma linguagem de programação com estes tipos mais expressivos.

%%% Local Variables:
%%% mode: latex
%%% TeX-master: "main"
%%% End:

\section{A implementação}

A implementação de um compilador com escolhas mistas, que estendem os tipos de sessão
independentes do contexto, deverá seguir as mesmas diretrizes do compilador da linguagem FreeST\cite{2019freest,2019freest-inforum}.

A fase de validação é, também, composta por dois sistemas bidirecionais:
\begin{itemize}
\item \textbf{Um sistema de verificação de géneros} (\textit{kinding}) que assegura que a boa formação dos tipos, incluindo as escolhas mistas.
\item \textbf{Um sistema de verificação de tipos} cujas relações permitem, sintetizar o tipo de uma expressão e verificar se uma expressão tem o tipo esperado. 
\end{itemize}
A introdução de escolhas mistas não invalida nenhuma das duas relações.
Uma escolha necessita que todos os seus componentes sejam tipos de sessão lineares e a verificação de tipos garante que os padrões de uma \lstinline|choice| respeitam o tipo do canal.

O ambiente de execução (\textit{runtime system}) é bastante compacto e é, essencialmente, composto pelas primitivas de criação de novos fios de execução (\lstinline|fork|) e primitivas para a manipulação de canais (\lstinline|send|, \lstinline|receive|, \lstinline|choice|). O compilador gera código Haskell que posteriormente será compilado através do GHC, o compilador convencional do Haskell.

%%% Local Variables:
%%% mode: latex
%%% TeX-master: "main"
%%% End:


\paragraph{Agradecimentos.}

Este trabalho foi apoiado em parte pela FCT através do projeto
Confident, PTDC/EEI-CTP/4503/2014, e da Unidade de Investigação LASIGE,
UID/CEC/00408/2019.

\bibliographystyle{splncs04}
\bibliography{bibliography}

\end{document}
%%% Local Variables:
%%% mode: latex
%%% TeX-master: t
%%% End:
