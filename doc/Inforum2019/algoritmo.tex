\section{Equivalência de tipos}

Um compilador para uma linguagem de programação com tipos de sessão
com escolhas mistas, como os que propomos, tem de estar munido de um
algoritmo de verificação de equivalência de tipos.

Os tipos de sessão que estendemos
neste trabalho não são só tipos de sessão regulares.
Note-se que a sintaxe que propomos estende também
tipos de sessão livres do contexto.
Assim, o algoritmo de verificação de equivalência de tipos 
não é óbvio e deverá basear-se em \emph{Basic Process Algebras}
(BPA), como provado por Thiemann e Vasconcelos em~\cite{ref-cfst}.
Neste trabalho pretendemos tirar partido do algoritmo de verificação
de tipos livres de contexto que propomos em~\cite{type-equiv}.
Este algoritmo começa por traduzir tipos livres de contexto
em gramáticas simples (i.e.\ gramáticas determinísticas
em Greibach Normal Form) e, depois de simplificar as 
gramáticas, decide a existência de uma bisimulação entre as gramáticas
através de uma árvore de expansão. 
A árvore de expansão consiste numa
sucessão de passos de expansão e de simplificação 
sobre os símbolos da gramática, de acordo com 
as produções da gramática e com algumas regras de simplificação
propostas por Caucal, Christensen,
H\"uttel, Stirling, Jan\v car, and Moller
~\cite{caucal1986decidabilite,DBLP:journals/iandc/ChristensenHS95,janvcar1999techniques}

Os tipos de sessão com escolhas mistas têm
uma expressividade idêntica à expressividade das gramáticas
simples. Assim, a redução do algoritmo a estas gramáticas
permite a sua incorporação num compilador
para uma linguagem de programação com estes tipos mais expressivos.

%%% Local Variables:
%%% mode: latex
%%% TeX-master: "main"
%%% End:
