\section{Equivalência de tipos}

Dada a natureza dos tipos da linguagem \mixedchoice{}, a noção de
equivalência mais apropriada é uma bisimulação.
%
Um compilador para esta linguagem tem de estar munido de um algoritmo
de verificação de equivalência de tipos, isto é tem de ser capaz de
decidir se dois tipos são bisimilares.

Os tipos de sessão que utilizamos na linguagem \mixedchoice{} não
apresentam necessariamente recursividade terminal, são
\emph{independentes do contexto}.
%
O algoritmo de verificação de equivalência de tipos baseia-se em
\emph{Basic Process Algebras} (BPA), seguindo as ideias de Thiemann e
Vasconcelos~\cite{ref-cfst}.  \mixedchoice{} tira partido do algoritmo
de verificação de tipos livres de contexto proposto pelos
autores~\cite{type-equiv}.  Este algoritmo começa por traduzir tipos
livres de contexto em gramáticas simples (i.e.\ gramáticas
determinísticas em Greibach Normal Form) e, depois de simplificar as
gramáticas, decide a existência de uma bisimulação entre as gramáticas
através de uma árvore de expansão.  A árvore de expansão consiste numa
sucessão de passos de expansão e de simplificação sobre os símbolos da
gramática, de acordo com as produções da gramática e com algumas
regras de simplificação propostas por Caucal, Christensen, H\"uttel,
Stirling, Jan\v car, and Moller
~\cite{caucal1986decidabilite,DBLP:journals/iandc/ChristensenHS95,janvcar1999techniques}

Os tipos de sessão com escolhas mistas têm uma expressividade idêntica
à expressividade das gramáticas simples. Assim, a tradução de tipos de
sessão com escolhas múltiplas em gramáticas simples permite a
utilização do algoritmo desenvolvido para \freest~\cite{type-equiv}, e
por sua vez a incorporação num compilador para uma linguagem de
programação com estes tipos mais expressivos.

%%% Local Variables:
%%% mode: latex
%%% TeX-master: "main"
%%% End:
