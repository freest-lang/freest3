% Colour

\newcommand{\Blue}[1]{\textcolor{blue}{#1}}
\newcommand{\Red}[1]{\textcolor{red}{#1}}
\newcommand{\Brown}[1]{\textcolor{brown}{#1}}
\newcommand{\highlight}[1]{\Blue{#1}}

% ECLIPSE LOOK

\newcommand\Small{\small}
%\newcommand\Small{\fontsize{7.5}{8}\selectfont} 

\definecolor{darkviolet}{rgb}{0.5,0,0.4}
\definecolor{darkgreen}{rgb}{0,0.4,0.2} 
\definecolor{darkblue}{rgb}{0.1,0.1,0.9}
\definecolor{darkgrey}{rgb}{0.5,0.5,0.5}
\definecolor{lightblue}{rgb}{0.4,0.4,1}

\lstdefinestyle{eclipse}{
  breaklines=true,
  basicstyle=\sffamily\Small,
  emphstyle=\color{red}\bfseries, 
  keywordstyle=\color{darkviolet}\bfseries,
  commentstyle=\color{darkgreen},
  stringstyle=\color{darkblue},
  numberstyle=\color{darkgrey},%\lstfontfamily,
  emphstyle=\color{red},
  % get also javadoc style comments
  morecomment=[s][\color{lightblue}]{/**}{*/},
  %columns=fullflexible, %spaceflexible, %flexible, fullflexible             
  %  escapeinside=`',
  %  escapechar=@,
  showstringspaces=false,
  numbers=left,
  tabsize=2
}

\lstdefinestyle{eclipse-Haskell}{
  breaklines=true,
  basicstyle=\sffamily\Small,
  emphstyle=\color{red}\bfseries, 
  keywordstyle=\color{darkviolet}\bfseries,
  commentstyle=\color{darkgreen},
  stringstyle=\color{darkblue},
  emphstyle=\color{red},
  % get also javadoc style comments
  morecomment=[s][\color{lightblue}]{/**}{*/},
  %columns=fullflexible, %spaceflexible, %flexible, fullflexible             
  %  escapeinside=`',
  %  escapechar=@,
  showstringspaces=false,
  numbers=none,
  tabsize=2
}

%\usepackage[T1]{fontenc}

% \lstdefinestyle{Go}{	
% 	keywordstyle=[1]\bfseries,
% 	basicstyle=\footnotesize\ttfamily,	
% 	numberstyle=\tiny,
% 	numbersep=5pt,
% 	breaklines=true,
% 	%prebreak=\raisebox{0ex}[0ex2][0ex]{\ensuremath{\hookleftarrow}},
% 	showstringspaces=false,
% 	upquote=true,
% 	tabsize=3,
% 	frame=tb,
% 	morekeywords={go,make,chan,int,import,main,func,for,select,case,string},
%       }

\lstdefinelanguage{Golang}%
  {morekeywords=[1]{package,import,func,type,struct,return,defer,panic,%
     recover,select,var,const,iota,},%
   morekeywords=[2]{string,uint,uint8,uint16,uint32,uint64,int,int8,int16,%
     int32,int64,bool,float32,float64,complex64,complex128,byte,rune,uintptr,%
     error,interface},%
   morekeywords=[3]{map,slice,make,new,nil,len,cap,copy,close,true,false,%
     delete,append,real,imag,complex,chan,},%
   morekeywords=[4]{for,break,continue,range,goto,switch,case,fallthrough,if,%
     else,default,},%
   morekeywords=[5]{Println,Printf,Error,Print,},%
   sensitive=true,%
   morecomment=[l]{//},%
   morecomment=[s]{/*}{*/},%
   morestring=[b]',%
   morestring=[b]",%
   morestring=[s]{`}{`},%
}

      
\lstdefinelanguage{SePi}%
  {morekeywords=[1]{type,integer,string,boolean,new, select, assume, assert},%
   sensitive=true,%
   morecomment=[l]{//},%
   morecomment=[s]{/*}{*/},%
   morestring=[b]',%
   morestring=[b]",%
   morestring=[s]{`}{`},%
 }

\lstdefinelanguage{CFST}{
  style=eclipse,
  morekeywords=[1]{Int, Char, Bool, Skip, forall, rec, let, in, if, then, else, new, send, receive,
    select, fork, case, of, data, match, with, True, False, type,choice},%  
  sensitive=true,%
  literate={->}{{$\rightarrow$}}1,%
   breaklines=true,
   morecomment=[l]{--},%
   morecomment=[s]{{-}{-}},%
   morestring=[b]',%
   morestring=[b]",%
   morestring=[s]{`}{`},%
 }

\lstset{
  language=CFST,
  numbers=none
}
 

% notes
\newcommand{\todo}[1]{[{\color{blue}\textbf{#1}}]}

% Keywords
\newcommand{\keyword}[1]{\mathsf{#1}}

% Prekinds

\newcommand\prekind{\upsilon}

\newcommand{\stypes}{\mathcal S}
\newcommand\kinds{\stypes}

\newcommand{\types}{\mathcal T}
\newcommand\kindt{\types}

\newcommand\kindsch{\mathcal C}

% Multiplicity
\newcommand\Un{\ensuremath{\mathbf{u}}} % \infty
\newcommand\Lin{\ensuremath{\mathbf{l}}} % 1 

% Kinds
\newcommand\kind{\kappa}

% Grammars
\newcommand{\grmeq}{\; ::= \;}
\newcommand{\grmor}{\;\mid\;}

% type constructors
\newcommand\tcBase{B}
\newcommand\tcLolli\multimap
\newcommand\tcFun\to
\newcommand\tcBang{\mathop!}

% Keywords for types
\newcommand\kRec{\keyword{rec}}
\newcommand\kForall{\keyword{forall}}

% Types
\newcommand{\tskip}{\keyword{Skip}}
\newcommand\tSemi[2]{#1;#2}
\newcommand\tChoice[2]{#1 \oplus #2}
\newcommand\tOut[1]{\tcBang#1}
\newcommand\tIn[1]{?#1}
%\newcommand\tIChoice[1]{\oplus#1}
%\newcommand\tEChoice[1]{\&#1}
\newcommand\tUnFun[2]{#1\tcFun#2}
\newcommand\tLinFun[2]{#1\tcLolli#2}
\newcommand\tPair[2]{(#1,\,#2)}
\newcommand\tDatatype[1]{{[#1]}}
\newcommand\tRec[2]{\kRec\,#1\,.\,#2}
%\newcommand\tForall[2]{\forall\,#1\,.\,#2}
%\newcommand\tForall[2]{\kForall\,#1\,=>\,#2}
\newcommand\tForall[2]{\kForall\,#1\Rightarrow#2}

\newcommand\tRecK[2]{\kRec\,#1\,.\,#2}
% Environments
% \newcommand\emptyEnv{\cdot}
\newcommand\emptyEnv{\varepsilon}
\newcommand\kindEnv{\Delta}
\newcommand\varEnv{\Gamma}

% Language
% Expressions
% Language Types
\newcommand{\unite}{()}
\newcommand{\inte}{\keyword{Int}}
\newcommand{\chare}{\keyword{Char}}
\newcommand{\boole}{\keyword{Bool}}

% Variables
\newcommand\vare[1]{#1}
\newcommand\unlete[3]{\keyword{let} \; #1 = #2 \; \keyword{in} \; #3} 

% Applications
\newcommand\appe[2]{#1#2}
\newcommand\tappe[2]{#1[#2]}

% Conditional
\newcommand\conditionale[3]{\keyword{if}\;#1\;\keyword{then}\;#2\;\keyword{else} \; #3}

% Pairs
\newcommand\paire[2]{(#1,#2)}
\newcommand\binlete[4]{\keyword{let}\;#1, #2 = #3\;\keyword{in}\;#4}

% Session Types
\newcommand\newe[1]{\keyword{new}\;#1}
\newcommand\sende[2]{\keyword{send}\;#1\; #2}
\newcommand\recve[1]{\keyword{receive}\;#1}
\newcommand\selecte[1]{\keyword{select}\;#1}
\newcommand\matche[2]{\keyword{match}\;#1\;\keyword{with}\;#2}

% Fork
\newcommand\forke[1]{\keyword{fork}\;#1}

% Datatypes
\newcommand{\ctrcte}{C}
\newcommand\casee[2]{\keyword{case}\;#1\;\keyword{of}\;#2}

% Goal
\newcommand\Alg{\vdash_{a}}
\newcommand\Goal{\vdash}

% Equivalent
\newcommand\Equiv[2]{#1\,\thicksim\,#2}

%%% Local Variables:
%%% mode: latex
%%% TeX-master: "main"
%%% End:
