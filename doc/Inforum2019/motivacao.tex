\section{Introdução}

Os tipos de sessão foram propostos para responder à necessidade
de formalização de trocas de mensagens, permitindo definir protocolos 
na forma de tipos que representam interações corretas do sistema e que 
garantem propriedades tais como a inexistência de erros na comunicação 
e de situações de impasse. Contudo, a expressividade dos tipos de sessão
está ainda aquém de todos os desafios de comunicação que encontramos nos 
sistemas.

Considere-se que temos um processo que consegue gerar sequências de números
inteiros mas não tem capacidade de os somar. Por outro lado, temos um
segundo processo que é capaz de somar números inteiros, mas não consegue
gerá-los. Ambos os processos estão aptos para a utilização de comunicação
por canais. Considere-se então o seguinte problema:

\begin{quote}
	Dado $k\in\mathbb{Z}$, qual o maior inteiro $n$ tal que
	$\sum_{i=1}^n i < k$?
\end{quote}

Este exemplo simplista retrata o caso em que a interação entre
os processos se dá necessariamente nos dois sentidos e ambos 
os processos deverão
estar aptos, leia-se \emph{corretamente tipados}, para responder 
às necessidades da comunicação.

% Neste trabalho propomos uma linguagem de programação
% com tipos capazes de modelar um cenário como o descrito acima.

Tirando partido das restrições computacionais dos nossos dois
processos, podemos tentar resolver o problema dos seguinte modo:
%
\begin{itemize}
\item Um dos processos, \lstinline|produtor|, gera a sequência de
  números: 1,2,3,\dots e envia-os num canal, até receber no mesmo
  canal uma notificação para terminar o envio. A notificação vem na
  forma de uma marca \lstinline|EOS| (\textit{end-of-stream});
\item O outro processo, \lstinline|consumidor|, vai somando os recebe
  os números que recebe enquanto a soma for menor do que $n$. Neste
  ponto envia a marca \lstinline|EOS|.
\item O programa principal cria o canal e lança os dois processos.
\end{itemize}

O canal de comunicação segue o  protocolo delineado acima. Quando
visto do lado do \lstinline|produtor| o canal toma o tipo
%
\begin{lstlisting}
  type CanalInt = ?EOS + !Int;CanalInt
\end{lstlisting}
%
permitindo a cada momento a leitura (\lstinline|?|) da marca
\lstinline|EOS| ou a escrita (\lstinline|!|) de um inteiro. No caso da
leitura o protocolo volta ``ao início''.
%
Quando visto do lado do \lstinline|consumidor| o canal toma o tipo
obtido do acima trocando as operações de leitura pelas de escrita, e
vice-versa. Abreviamos esse tipo com \lstinline|dualof CanalInt|.

\begin{lstlisting}
produtor : CanalInt -> Int -> ()
produtor c i = choice {
  (c, EOS) =receive -> (),
  c = send c n -> consumidor c (i+1)

consumidor : dualof CanalInt -> Int -> Int -> Int -> Int
consumidor c s n k = if n >= n
then choice { c = send EOS -> n--1
else choice { (c, m) = receive c -> produtor c (s+m) (n+1) k

main : Int
main = let
  k = 1000
  (r, w) = new CanalInt in
  fork produtor r 1;
  consumidor w 0 0 k
\end{lstlisting}




%%% Local Variables:
%%% mode: latex
%%% TeX-master: "main"
%%% End:
