\section{Introdução}

Os tipos de sessão foram propostos para responder à necessidade
de formalização de trocas de mensagens, permitindo definir protocolos 
na forma de tipos que representam interações corretas do sistema e que 
garantem propriedades tais como a inexistência de erros na comunicação 
e de situações de impasse. Contudo, a expressividade dos tipos de sessão
está ainda aquém de todos os desafios de comunicação que encontramos nos 
sistemas.

Considere-se que temos um processo que consegue gerar sequências de números
inteiros mas não tem capacidade de os somar. Por outro lado, temos um
segundo processo que é capaz de somar números inteiros, mas não consegue
gerá-los. Ambos os processos estão aptos para a utilização de comunicação
por canais. Considere-se então o seguinte problema:

\begin{example}
	Dado $k\in\mathbb{Z}$, qual o maior inteiro $n$ tal que
	$\sum_{i=1}^n i < k$?
\end{example}

Este exemplo simplista retrata o caso em que a interação entre
os processos se dá necessariamente nos dois sentidos e ambos 
os processos deverão
estar aptos, leia-se \emph{corretamente tipados}, para responder 
às necessidades da comunicação.

Neste trabalho propomos uma linguagem de programação
com tipos capazes de modelar um cenário como o descrito acima.

%%% Local Variables:
%%% mode: latex
%%% TeX-master: "main"
%%% End:
