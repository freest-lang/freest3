\section{A implementação}

A implementação de um compilador com escolhas mistas, que estendem os tipos de sessão
independentes do contexto, deverá seguir a mesma linha de ideias do compilador da linguagem FreeST\cite{2019freest,2019freest-inforum}.

A fase de validação é, também, composta por dois sistemas bidirecionais:
\begin{itemize}
\item \textbf{Um sistema de verificação de géneros} (\textit{kinding}) que assegura que a boa formação dos tipos, incluindo as escolhas mistas.
\item \textbf{Um sistema de verificação de tipos} cujas relações permitem, sintetizar o tipo de uma expressão e verificar se uma expressão tem o tipo esperado. 
\end{itemize}
A introdução de escolhas mistas não invalida nenhuma das duas relações.
Uma escolha (mista) necessita que os todos os seus componentes sejam tipos de sessão lineares. A verificação de tipos garante que os padrões de uma \lstinline|choice| respeitam os padrões do tipo do canal.

O ambiente de execução (\textit{runtime system}) é bastante compacto e é, essencialmente, composto pelas primitivas de criação de novos fios de execução (\lstinline|fork|) e primitivas para a manipulação de canais (\lstinline|send|, \lstinline|receive|, \lstinline|choice|). O compilador gera código Haskell que posteriormente será compilado através do GHC, o compilador convencional do Haskell.

%%% Local Variables:
%%% mode: latex
%%% TeX-master: "main"
%%% End:
