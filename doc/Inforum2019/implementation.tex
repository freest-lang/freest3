\section{A implementação}

A implementação de um compilador para a linguagem \mixedchoice,
equipada com tipos de sessão, segue as mesmas diretrizes do compilador
da linguagem FreeST\cite{2019freest,2019freest-inforum}, uma linguagem
com tipos de sessão independentes do contexto mas sem escolhas
múltiplas.

Assim,  fase de validação é composta por dois sistemas bidirecionais:
\begin{description}
\item[Um sistema de verificação de géneros] (\textit{kinding}) que
  assegura  a boa formação dos tipos, incluindo as escolhas mistas.
\item[Um sistema de verificação de tipos] cujas relações permitem
  sintetizar o tipo de uma expressão e verificar se uma expressão tem
  o tipo esperado.
\end{description}
A introdução de escolhas mistas não invalida nenhuma das duas
relações.  Um tipo escolha necessita que todos os seus componentes
sejam tipos de sessão lineares e a verificação de tipos garante que os
padrões de uma \lstinline|choice| respeitam o tipo do canal.

O ambiente de execução (\textit{runtime system}) é bastante compacto e
é, essencialmente, composto pelas primitivas de criação de novos fios
de execução (\lstinline|fork|) e primitivas para a manipulação de
canais (\lstinline|send|, \lstinline|receive|, \lstinline|choice|). O
compilador gera código Haskell que posteriormente será compilado
através do GHC, o compilador convencional do Haskell.
%
Estamos neste momento a modificar o compilador de \freest{} de modo a
acomodar escolhas múltiplas.

%%% Local Variables:
%%% mode: latex
%%% TeX-master: "main"
%%% End:
