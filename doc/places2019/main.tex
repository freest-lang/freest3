\documentclass[submission,copyright,creativecommons]{eptcs}

\providecommand{\event}{PLACES 2019}

\usepackage[utf8]{inputenc}
\usepackage[T1]{fontenc}
\usepackage{listings}

\title{FreeST: context-free session types in a functional language}
\author{
  Bernardo Almeida
  \and
  Andreia Mordido
  \and
  Vasco T. Vasconcelos
  \institute{LASIGE, Faculdade de Ciências, Universidade de Lisboa, Portugal}
}
\def\titlerunning{FreeST}
\def\authorrunning{Almeida, Mordido \& Vasconcelos}

\begin{document}
\maketitle

Session types have been long subject to the shackles of tail
recursion. Regular languages have the evident advantage of simple
algorithms for checking type equivalence or subtyping. Given two
types, the fixed-point construction algorithm introduced by Gay and
Hole~\cite{DBLP:journals/acta/GayH05} builds in polynomial time a
bisimulation relating the two types, or decides that no such relation
exists. The scenario complicates a little when when we let go of tail
recursion, for now the fixed-point construction algorithm does not
necessary terminate.

This is one of the reasons why session types have been confined to
($\omega$-) regular languages for so many years. In 2016, Thiemann and
Vasconcelos introduced a functional programming language equipped with
context-free session types and proved type equivalence
decidable~\cite{DBLP:conf/icfp/ThiemannV16}.

\paragraph{Acknowledgements}

This work was supported by FCT through project Confident, ref.\
PTDC/EEI-CTP/4503/2014, and the LASIGE Research Unit, ref.\
UID/CEC/00408/2019.

\bibliography{biblio}
\bibliographystyle{eptcs}

\end{document}

%%% Local Variables:
%%% mode: latex
%%% TeX-master: t
%%% End:
