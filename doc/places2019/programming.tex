\section{Programming in FreeST}
\label{sec:programming}

\freest{} is a basic implementation of the language introduced by
Thiemann and Vasconcelos~\cite{DBLP:conf/icfp/ThiemannV16}. Here we
provide a gentle introduction to the language.

We have chosen the concrete syntax to be aligned with that of Haskell,
as much as possible. A \freest{} script is a collection of type
abbreviations, datatype and function (or value) declarations. Function
\lstinline|start| starts it all.

Our first example serializes a tree object on a channel. We need a
datatype for trees, for which we take a binary tree with integer
values on the internal nodes:
%
\begin{lstlisting}
data Tree = Leaf | Node Int Tree Tree
\end{lstlisting}

We also need a channel capable of transmitting a tree. The type below
describes a channel end that not only sends a tree, but for each
\lstinline|Node| also receives back an integer value.
%
\begin{lstlisting}
type TreeC = +{Leaf: Skip, Node: !Int;TreeC;TreeC;?Int}
\end{lstlisting}

The \lstinline|+| type constructor introduces an internal choice (the
process in possession of the channel end chooses) with two
alternatives, labelled \lstinline|Leaf| and \lstinline|Node|. These
two labels should not be confused with those of the datatype
introduced above. \lstinline|Leaf| states that no further interaction
is possible on the channel (denoted by \lstinline|Skip|); the
\lstinline|Node| branch writes an integer value, followed by two
trees, and terminates by reading an integer value
(\lstinline|!Int;TreeC;TreeC;?Int|).

The writer process, \lstinline|transform|, writes a tree on a given
channel. The channel is of type \lstinline|TreeC;alpha|, for
\lstinline|alpha| a type variable. It returns a \lstinline|Tree| and
the residual of the input channel, of type \lstinline|alpha|. The type
of \lstinline|transform| is \emph{polymorphic}; different calls to the
function use different values for \lstinline|alpha|.

\begin{lstlisting}
transform: forall alpha => Tree -> TreeC;alpha -> (Tree, alpha)
\end{lstlisting}

For each \lstinline|Node| in the input tree, \lstinline|transform|
reads an integer from the channel and returns a tree isomorphic to the
input where each integer in nodes is read from the channel.
%
The \lstinline|transform| function performs a \lstinline|case|
analysis on the \lstinline|Tree| constructor (either \lstinline|Leaf|
or \lstinline|Node|). In the former case, it \lstinline|select|s the
\lstinline|Leaf| choice and returns a pair composed of the original
tree and the residual of the channel. In the latter case, the function
\lstinline|select|s the \lstinline|Node| choice, then sends the
integer value in the node, followed by the two subtrees (via recursive
calls). Finally, reads an integer \lstinline|y| from the channel,
assembles a tree with \lstinline|y| at the \lstinline|Node| and
returns this tree together with the residual of the channel.
%
\begin{lstlisting}
transform tree c =
  case tree of
    Leaf ->
      (Leaf, select Leaf c)
    Node x l r ->
      let c   = select Node c in
      let c   = send x c in
      let l,c = transform l c in
      let r,c = transform r c in
      let y,c = receive c in
      (Node y l r, c)
\end{lstlisting}

Notice that the language requires a continuous rebinding of channel
\lstinline|c|, for its type changes at each interaction, as in Gay and
Vasconcelos~\cite{DBLP:journals/jfp/GayV10}. Also \freest{} infers the
appropriate types that instantiate \lstinline|alpha| in the two
recursive calls, namely, \lstinline|TreeC;?Int;alpha| and
\lstinline|?Int;alpha|.

The reader process, \lstinline|treeSum|, reads a tree from a given
channel, writes back on the channel the sum of the elements in the
tree and returns this sum. This process sees the channel from the
other end: rather than performing an internal choice (\lstinline|+|),
it performs an external choice (\lstinline|&|), rather than writing
(\lstinline|!|), it reads (\lstinline|?|), and rather than reading, it
writes. We could write its type as
\lstinline|rec x. &{Leaf: Skip, Node: ?Int;x;x;!Int}|,
but the \lstinline|dualof| operator provides an handy abbreviation.
%
\begin{lstlisting}
treeSum: forall alpha => dualof TreeC;alpha -> (Int, alpha)
\end{lstlisting}

Rather than performing a case analysis on a \lstinline|Tree|, function
\lstinline|treeSum| \lstinline|match|es the channel against its two
possible choices (\lstinline|Leaf| and \lstinline|Node|). In the
former case the function returns \lstinline|0| and the residual
channel. In the latter, the function reads an integer from the
channel, then reads two subtrees (via recursive calls) and sends on
the channel the sum of the values in the subtree. It finally returns
this exact sum, together with the channel residual.
%
\begin{lstlisting}
treeSum c =
  match c with
    Leaf c ->
      (0, c)
    Node c ->
      let x, c = receive c in
      let l, c = treeSum c in
      let r, c = treeSum c in
      let c    = send (x + l + r) c in
      (x + l + r, c)
\end{lstlisting}

Again notice the continuous rebinding of channel \lstinline|c| and the
omission of two types that instantiate variable \lstinline|alpha| in
the two recursive calls, namely \lstinline|dualof TreeC;!Int;alpha|
and \lstinline|!Int;alpha|.

Function \lstinline|start| completes the script. It starts by creating
a \lstinline|new| channel. This channel constructor takes a type
\lstinline|T| and returns a pair of channel ends of type
%
\lstinline|(T,dualof T)|. Then the \lstinline|start| function
\lstinline|fork|s a new thread to compute the \lstinline|treeSum|. In
the main thread, it \lstinline|transform|s a given tree
(\lstinline|aTree|). Function \lstinline|treeSum| uses the
\lstinline|r| end of the channel and \lstinline|transform| uses
\lstinline|w|, the other end. In these calls, both functions are
(implicitly) applied to type \lstinline|Skip|. This means that both
channel ends \lstinline|r| and \lstinline|w| are both consumed to
\lstinline|Skip|, and hence can be safely discarded. In addition to
the residual of channel end \lstinline|w|, function
\lstinline|transform| also returns a new tree \lstinline|t|, which
becomes the result of the function.
%
\begin{lstlisting}
start: Tree
start =
  let w,r = new TreeC in
  let u   = fork (treeSum r) in
  let t,w = transform aTree w in
  t
\end{lstlisting}

The two diagrams below show a tree transformation, from tree
\lstinline|aTree| into tree \lstinline|t|. \todo{sketch the diagram}

\begin{lstlisting}
aTree = Node 1 (Node 2 Leaf (Node 3 Leaf (Node 4 Leaf (Node 5 Leaf Leaf)))) (Node 6 Leaf (Node 7 Leaf (Node 8 Leaf Leaf)))
\end{lstlisting}

\begin{lstlisting}
t = Node 36 (Node 14 Leaf (Node 12 Leaf (Node 9 Leaf (Node 5 Leaf Leaf)))) (Node 21 Leaf (Node 15 Leaf (Node 8 Leaf Leaf)))
\end{lstlisting}



%%% Local Variables:
%%% mode: latex
%%% TeX-master: "main"
%%% End:
