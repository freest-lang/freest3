\section{\freest{} requires a type equivalence checker}
\label{sec:equivalence}

Checking type equivalence is the main challenge of the
compiler. If, on the one hand, the algorithm should be sound and 
complete, on the other hand it should have a runtime compatible
with the efficiency needs of a compiler.
However, the recursive structure underlying context-free 
session types often make bisimulations between two types infinite. 

In~\cite{typeEquivalence} we propose an algorithm to decide
the equivalence of context-free session types. 
It has three distinct stages.
%
The \emph{first stage} builds a context-free grammar in Greibach
Normal Formal (GNF)---in fact a simple grammar---from a context-free
session type in a way that bisimulation is preserved.  
%
The \emph{second stage} prunes the grammar by removing unreachable
symbols in unnormed sequences of non-terminal symbols. This stage
builds on the result of Christensen, H\"uttel, and 
Stirling~\cite{DBLP:journals/iandc/ChristensenHS95}.
%
The \emph{third stage} constructs an expansion tree, by alternating
between expansion and simplification steps.  This last stage uses
ideas on the expansion operations proposed by Janc\v ar, Moller, and
Hirshfeld~\cite{hirshfeld1996bisimulation,janvcar1999techniques}, and
ideas on the simplification rules proposed by Caucal, Christensen,
H\"uttel, Stirling, Janc\v ar, and Moller
~\cite{caucal1986decidabilite,
  DBLP:journals/iandc/ChristensenHS95,janvcar1999techniques}.  
We use breadth-first search on the expansion tree and the 
algorithm terminates as soon as it reaches an empty node---case
in which it decides positively---or it fails to expand a node---case
in which it decides negatively.  
The
finite representation of bisimulations of BPA transition
graphs~\cite{caucal1986decidabilite, DBLP:journals/iandc/ChristensenHS95} 
is thus paramount for the soundness and completeness of the algorithm,
which we prove. 

Although the branching nature of the expansion tree confers an exponential 
complexity to the algorithm, we propose heuristics that allow constructing 
the relation in a reasonable time. We have benchmarked the algorithm 
on a test suite of carefully crafted pair of types comprising valid and invalid 
equivalences, for a total os 138 tests. By running the tests on the base and
optimized algorithms, we observe an improvement of more than 
$11,000,000\%$ on runtime and memory allocated: it goes from 4445.38 
seconds and 8,259,115 Mb memory allocated
with the base algorithm to an average of 0.04
seconds for the running time and 62 Mb of allocated memory with the 
optimized algorithm.


