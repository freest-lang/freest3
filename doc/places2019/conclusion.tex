\section{The bright future of \freest{}}
\label{sec:conclusion}

We have developed a basic compiler for \freest, a concurrent
functional programming with context-free session types, based on the
ideas of Thiemann and Vasconcelos~\cite{DBLP:conf/icfp/ThiemannV16}.
%
There are many possible extensions to the language. We discuss a
few. Shared channels allow for multiple readers and multiple writers,
thus introducing (benign) races. There are several proposals in the
literature~\cite{DBLP:journals/pacmpl/BalzerP17,
  DBLP:conf/sefm/FrancoV13,Lindley.Morris_Lightweight.functional.session.types,DBLP:journals/iandc/Vasconcelos12}
on which we may base this extension.
%
Because \freest{} compiles to Haskell, a better interoperability is
called for. We plan to add primitive support for lists, and for some
functions in Haskell's prelude, rank-1 functions that do not collide
with linearity constraints.
%
We have chosen a synchronous semantics for the communication
primitives, but we plan to experiment with buffered channels by simply
replacing the back-end.
%
The original proposal of context-free sessions is based on a
call-by-value operational semantics and we kept that strategy in
\freest. We however plan to experiment with call-by-need, taking
advantage of the back-end in Haskell. We also intend to allow messages
and choices to contain any functional or session type, rather than
containing only basic types. To achieve this,
the type equivalence algorithm for context-free session 
types should be intertwined with the type equivalence
algorithm for functional types, which is a challenging task.
Finally, we plan to incorporate type inference on type 
applications in order to allow the automatic identification of the
unifier matching a polymorphic type with the next type.

%%% Local Variables:
%%% mode: latex
%%% TeX-master: "main"
%%% End:
