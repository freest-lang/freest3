\section{Evaluation}
\label{sec:evaluation}

% \vv{estes testes incluem o tempo de parsing? nao deveriam. deveriam só
%   contar o tempo da chamada à funcao bisimilar/equivalent.}

We implemented the algorithm skteched in Listings~\ref{lst:toGrammar}
to~\ref{lst:enhanced} in 300 lines of Haskell and used the Glasgow
Haskell Compiler, GHC version 8.6.3, from which we have obtained the
results we present in this section.  Evaluation was conducted on a Mac
mini equipped with a 3.6 GHz Intel Core i3, 8 GB of memory, running
MacOS 10.14.3.

Once the proposals for improvement of the algorithm were established,
we have benchmarked the algorithm on a test suite of carefully crafted
pair of types. These tests comprise valid and invalid equivalences,
for a total of 138 tests. We have profiled our program for the time
and memory allocated during the tests. For this purpose, we have used
GHC's profiling feature, that maintains a cost-centre stack to keep
track of the incurred costs. The results are presented in
figure~\ref{fig:results}.

\begin{figure}[h]
  \includegraphics[height=4.8cm]{img/run_time}
  \quad 
  \includegraphics[height=4.8cm]{img/memory_alloc}	
  \caption{Running times (on the left) and memory allocated (on the
    right) when checking the bisimilarity of context-free session
    types on a test suite of 150 tests}
  \label{fig:results}
\end{figure}

The running times and memory allocated are presented in
Figure~\ref{fig:results}, exhibit an improvement on more than
12,000,000\%. The running time of example in~\eqref{ex:chaotic} was
brought down to 0.008 seconds. For this reasons, our
proposal for an algorithm to check the equivalence of context-free
session types stands on adapting the simplification stage to enable
double-ended enqueueing and the computation of a fixed point at the
simplification phase Listing~\ref{lst:enhanced} presents an enhanced
version of the simplification stage coping the new proposals.

%%% Local Variables:
%%% mode: latex
%%% TeX-master: "main"
%%% End:
