\section{Evaluation}
\label{sec:evaluation}

% \vv{estes testes incluem o tempo de parsing? nao deveriam. deveriam só
%   contar o tempo da chamada à funcao bisimilar/equivalent.}

We implemented the algorithm skteched in Listings~\ref{lst:toGrammar}
to~\ref{lst:enhanced} in 300 lines of Haskell and used the Glasgow
Haskell Compiler, GHC version 8.6.3, from which we have obtained the
results we present in this section.  Evaluation was conducted on a Mac
mini equipped with a 3.6 GHz Intel Core i3, 8 GB of memory, running
MacOS 10.14.3.

Once the proposals for improvement of the algorithm were established,
we have benchmarked the algorithm on a test suite of carefully crafted
pair of types. These tests comprise valid and invalid equivalences,
for a total of 138 tests. We have profiled our program for the
time and memory allocated during the tests. For this purpose,
we have used GHC's profiling feature, that maintains a cost-centre stack
to keep track of the incurred costs. We ran the tests 10 times,
kept a record of the run time and memory allocated for each run,
discarded the best and worst values obtained and, then, we have
measured the average of the remaining values. The results are
depicted in figure~\ref{fig:results}.

\begin{figure}[h]
	\includegraphics[height=5cm]{img/run_time}	\enspace
	\includegraphics[height=5cm]{img/memory_alloc}
	\caption{Test results: running times (on the left) and
	memory allocated (on the right) checking the equivalence
	of context-free session types in 138 tests.}
	\label{fig:results}
\end{figure}

For the base algorithm, proposed in listing~\ref{lst:algorithm},
we have obtained an average running time of about 4445,38 seconds
and 8259115 Mb memory allocated. From the moment we introduced
optimizations in the algorithm the results have improved remarkably:
iterating the simplification phase in the search for a fixed point
allowed to reduce the value of the running time to 110,35 seconds and
the memory allocated to 164745 Mb, whereas the implementation of the
double-ended queue allowed to reduce the running time to 18,45 seconds
and the allocated memory to 29298. The combination of both exhibit an
improvement on more than 12,000,000\% of the base case, by
achieving an average of 0,04 seconds for the running time and 62 Mb
of allocated memory.

We should also highlight that, we have run example in~\eqref{ex:chaotic} 
with the improved algorithm, in a battery os 100 runs, and obtained an
average running time of 0.008 seconds. 

The heuristic we proposed actually circumvents the exponential complexity 
inherent to the expansion tree, thus allowing to obtain running times that 
are manifestly small, thus allowing the use of this algorithm as an integral 
part of a compiler, as we had intended from the beginning.

%%% Local Variables:
%%% mode: latex
%%% TeX-master: "main"
%%% End:
