\section{Context-free session types}
\label{sec:contextfreesession}

The types we consider build upon a denumerable set of \emph{type
  variables}, denoted by $X,Y,Z$, and a set \emph{type labels},
denoted by $\ell$. We assume given a set of base types $B$ that
include the $\unitk$ type. The syntax of types is given by the grammar
below.
%
\begin{gather*}
  S,T \grmeq \skipk \mid S;T \grmor \sharp B \grmor 
  \star\{\ell_i\colon S_i\}_{i\in I} \grmor \mu X.S \grmor X
  \\
  \sharp \grmeq {}! \grmor {}? 
  \qquad \qquad
  \star  \grmeq \oplus \grmor {}\&
  \qquad \qquad
  a \grmeq \sharp B \grmor \star l \grmor X
\end{gather*}

%  We consider a labelled transition system where $A$ ranges
% over $\alpha$, $\,!B$, and $\,?B$, $\star$ ranges over $\oplus$ and
% $\&$, and $a$ ranges over both $A$ and $\star l$.
%
The labelled transition system (LTS) for context-free session types is
given by the set of types as \emph{states}, the set of \emph{actions}
ranged over by $A$, and the \emph{transition relation} $\LTSderives$
defined by the rules below~\cite{thiemann2016context}.
% in figure~\ref{fig:lts}.
The transition relation makes
use of an auxiliary judgment $\DONE{S}$ that characterizes terminated
states: session types that exhibit no further
action~\cite{DBLP:journals/jacm/AcetoH92} .
%
%\begin{figure}
  \begin{gather*}
    \DONE{\skipk}
    \quad
    \frac{\DONE{S} \quad \DONE{T}}{\DONE{S; T}}
    \quad
    \frac{\DONE{S[\mu X.S/X]}}{\DONE{\mu X.S}}
    \\
%    {a \LTSderives \skipk }
    X \LTSderives[X] \skipk
    \quad
    \sharp B \LTSderives[\sharp B] \skipk
    \quad
    \star\{l_i\colon S_i\} \LTSderives[\star l_j] S_j
    \quad
    \frac{S \LTSderives S'}{S; T \LTSderives S';T}
    \quad
    \frac{\DONE{S} \quad T \LTSderives T'}{S; T \LTSderives T'}
    \quad
    \frac{S[\mu X.S/X] \LTSderives S'}{\mu X.S \LTSderives S'}
  \end{gather*}
% \caption{Labelled transition system for context-free session types~\cite{thiemann2016context}.}
% \label{fig:lts}
% \end{figure}

% Type equivalence for context-free session types is based on the notion
% of bisimulation~\cite{thiemann2016context}.

\emph{Type bisimulation}, $\TypeEquiv$, is defined in the usual way from the
labelled transition system. \vv{cite Sangiorgi's book?}

%%% Local Variables:
%%% mode: latex
%%% TeX-master: "main"
%%% End:
