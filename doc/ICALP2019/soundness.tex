\section{Soundness and Completeness}
\label{sec:soundness}

In this section we prove that our algorithm is sound and complete 
with respect to the meta-theory of context-free session types proposed 
by Thiemann and Vasconcelos~\cite{thiemann2016context}.

We start by showing that the bisimulation relation proposed by 
Thiemann and Vasconcelos, $\TypeEquiv$, is equivalent to the 
bisimulation relation obtained from the productions, $\ProdEquiv$. 
Then, based on results from Caucal~\cite{caucal1986decidabilite}, Christensen, 
H{\"{u}}ttel, Stirling~\cite{DBLP:journals/iandc/ChristensenHS95}, Jan{\v{c}}ar 
and Moller~\cite{janvcar1999techniques}, we conclude that our algorithm 
is sound and complete.

\subsection{The bisimilarities coincide}

In the following lemma we prove that the initial (dummy) production 
does not affect equivalence checking.
\begin{lemma}
	Given two session types $S$ and $T$, 
	\[ X_{S} \sim X_{T}  \text{ if and only if } 
	\text{\lstinline{toGrammar S}} \sim \text{\lstinline{toGrammar T}}.\]
\end{lemma}

\begin{proof}
	By~\eqref{initial_prod}, we have:
	\[X_S \rightarrow \enspace \initialProd\enspace \text{(\lstinline{toGrammar S})}\]
    \[X_T \rightarrow \enspace \initialProd\enspace \text{(\lstinline{toGrammar T})}\]

Hence, $X_S\sim X_T$ if and only if 
\lstinline{toGrammar S} $\sim$ \lstinline{toGrammar T}.
\end{proof}

We start by proving that terminated session types
do not originate new productions.

\begin{lemma}
\label{lemma:terminated_session}
	Given a context-free session type $S$, 
	if \DONE{S} then \lstinline{toGrammar S} does not add any 
	production and returns \lstinline{[ ]}.
\end{lemma}

\begin{proof}
	The proof is done by induction on the structure of a terminated
	context-free session type $S$:
	\begin{description}
		\item[Case $S\triangleq \skipk$,] then \lstinline{toGrammar} does not 
		add any production and returns \lstinline{[ ]}.
		\item[Case $S\triangleq S_1;S_2$] for terminated session types $S_1$ and
		$S_2$, \lstinline{toGrammar S} recursively adds the productions from 
		\lstinline{toGrammar S1} and \lstinline{toGrammar S2}. But $S_1$ and 
		$S_2$ are terminated session types, hence no production is added and
		\lstinline{[ ]} is returned.
		\item[Case $S\triangleq \mu x. S$,] where $S[\mu x.S/x] \checkmark$,
		\lstinline{toGrammar S} adds the productions from \linebreak
		\lstinline{toGrammar (subs (Var y) x t)}. But $S[\mu x.S/x]$ is a terminated
		session, thus \lstinline{toGrammar S} returns does not add 
		any production and returns \lstinline{[ ]}.
	\end{description}
\end{proof}

Now we prove that any transition in the \LTS\ has a 
corresponding transition derived from the set of productions.

\begin{lemma}
Given context-free session types $S,S'$ and a label $\ell$,
	\[ \text{if } S \LTSderives[\ell] S' \text{ then } 
	\text{\lstinline{toGrammar S}} \rightarrow \enspace 
	\ell \enspace \vec Y, \]
	where \lstinline{toGrammar S'} is prefix of $\vec Y$.
\end{lemma}

\begin{proof}
The proof is done by induction on the structure of the labelled 
transition system (Fig.~\ref{lts}):
\begin{itemize}
	\item If $S\triangleq \skipk$, then $S$ is a terminated 
	      session type, with no further transition. Similarly, 
	      \lstinline{toGrammar S} returns \lstinline{[ ]} and 
	      does not add any production.
	\item If $S\triangleq A$, where $A$ ranges over $!B$ and $?B$, 
	      then $S  \LTSderives[A] \skipk$. On the other hand, 
	      \lstinline{toGrammar S} returns a fresh variable $X$ and 
	      inserts a production $X\rightarrow A$. 
	\item If $S \triangleq \alpha$, then $S   \LTSderives[\alpha] \skipk$. 
	      On the other hand, since $\alpha$ does not contain non-terminal 
	      symbols, \lstinline{toGrammar S} returns a fresh variable $X$ and 
	      inserts a production $X\rightarrow \alpha$.
	\item If $S\triangleq \star\{l_i\colon S_i\}_{i\in I}$ then, for each 
          $j\in I$, $S \LTSderives[\star l_j] S_j$. On the other hand, 
          \lstinline{toGrammar S} returns a fresh variable 
          $X$ and, recursively, inserts a production 
          $X\rightarrow \star l_j \enspace \text{(\lstinline{toGrammar Sj})}$
          for each $j\in I$.
	\item If $S\triangleq \mu x.S$ and $S[\mu x.S/x] \LTSderives[a] S'$, 
	      then $\mu x.S \LTSderives[a] S'$. By induction hypothesis,
	      the corresponding production in \lstinline{toGrammar (subs (Var y) x t)} 
	      is recursively added by \lstinline{toGrammar S} in the form
	      $X \rightarrow a\enspace  \text{(\lstinline{toGrammar S'})}$,
	      where $X$ is a fresh variable that is, then, returned by 
	      \lstinline{toGrammar S}.
	\item If $S\triangleq T;U$ and $T \LTSderives[a] T'$ then $S \LTSderives[a] T';U$. 
	      By induction hypothesis, \lstinline{toGrammar T} adds the production 
	      $X\rightarrow a \enspace \text{(\lstinline{toGrammar T'})}$, where 
	      $X$ is a fresh variable.
		  We notice that \lstinline{toGrammar T} $= X \, \vec X_T$ for some 
		  sequence of non-terminal symbols $\vec X_T$. By congruence, we have: 
		  \[\text{\lstinline{toGrammar S}} = X \, \vec X_T \, \text{(\lstinline{toGrammar U})} 
		  \rightarrow a \enspace \text{(\lstinline{toGrammar T'})} \, \vec X_T  \, 
		  \text{(\lstinline{toGrammar U})} .\]
	\item If $S\triangleq T;U$, $T$ is a terminated session, with no further 
	      action, and $U \LTSderives[a] U'$ then, using the \LTS\ we have
	      $T;U \LTSderives[a] U'$. On the other hand, by induction hypothesis and 
	      using Lemma~\ref{lemma:terminated_session}:
	    \begin{itemize}
			\item \lstinline{toGrammar T}  returns \lstinline{[ ]},
			\item \lstinline{toGrammar U}  adds a production 
			$X \rightarrow a \enspace \text{\lstinline{toGrammar U'}}$, where $X$
			is a fresh variable.
		\end{itemize}
		We notice that \lstinline{toGrammar U} $= X \, \vec X_U$ for some sequence of 
		non-terminal symbols $\vec X_U$. Hence, by congruence, we have a 
		transition 
		\[\text{\lstinline{toGrammar S}} = \, \text{\lstinline{[ ]}} \, X \, 
		\vec X_U \rightarrow \enspace a \enspace \text{(\lstinline{toGrammar U'})} \, 
		\vec X_U.\]
\end{itemize}
\end{proof}

Conversely, we prove that any transition derived from the productions 
has a corresponding labelled transition in the \LTS.

\begin{lemma}
Given a context-free session type $S$ and a label $\ell$,
	\[ \text{if } \text{\lstinline{toGrammar S}} \rightarrow \enspace \ell \enspace 
	 \vec Y \text{ then } S \LTSderives[\ell] S', \text{ for some context-free session type $S'$}.\]
\end{lemma}

\begin{proof}
	The proof is done by induction on the structure of $S$:
	\begin{description}
		\item[Case $S \triangleq \skipk$:] \lstinline{toGrammar S} does not 
		     add any production, and \DONE{S}.
		\item[Case $S \triangleq A$:] from \lstinline{toGrammar S} the 
		     production $Y\rightarrow A$ is added and, from the \LTS, we derive
		     $S \LTSderives[A] \skipk$, where $A$ ranges over $!B$, $?B$, 
		     and $\alpha$.
		\item[Case $S\triangleq \star \{\ell_i : S_i\}_{i\in I}$:] 
		     \lstinline{toGrammar} recursively adds 
		     $\text{\lstinline{toGrammar S}} \rightarrow \star \ell_j\, 
		     \text{(\lstinline{toGrammar Sj})}$, 
		     for each $j\in I$. In the \LTS\ we also have 
		     $S \LTSderives[\star \ell_j] S_j$, for each $j\in I$.
		\item[Case $S\triangleq \mu x.S$:] \lstinline{toGrammar S} adds, 
		     recursively, all productions from \linebreak \lstinline{toGrammar (subs (Var y) x t)}. 
		     Analogously, in the \LTS\ side, any transition 
		     $S[\mu x.S/x] \LTSderives[a] S'$ leads to a transition 
		     $\mu x.S \LTSderives[a] S'$.
		\item[Case $S \triangleq T;U$:]   \lstinline{toGrammar S} recursively adds 
		     all productions from   \lstinline{toGrammar T} and from  
		     \lstinline{toGrammar U}. Hence, if 
		     $\text{(\lstinline{toGrammar S})} \rightarrow \, \ell \, \vec Y$ then,
		     either: 
		     \begin{itemize}
		     	\item $\text{(\lstinline{toGrammar T}}) \rightarrow \, \ell \, \vec Y$ and,	
		     	      in this case, by induction hypothesis $T \LTSderives[\ell] T'$ and 
		     	      from the \LTS\ we derive $S \LTSderives[\ell] T';U$;
		     	\item \DONE{T} and $\text{(\lstinline{toGrammar U}}) \rightarrow \, 
		     	      \ell \, \vec Y$ and, by induction hypothesis $U \LTSderives[\ell] U'$
		     	      and from the \LTS\ we derive $S \LTSderives[\ell] U'$.
		     \end{itemize}
	\end{description}
\end{proof}

Having proved that any labelled transition in the LTS has a corresponding
transition in the grammar and vice-versa, the following theorem is now 
immediate.

\begin{theorem}
\label{cfst_vs_grammar}
	Given two context-free session types $S_1, S_2$,
	\[ S_1 \TypeEquiv S_2 \text{ if and only if } X_{S_1} \ProdEquiv X_{S_2}. \]
\end{theorem}

\subsection{\textit{Unnormedness} is preserved}

To prove that the pruning stage is in accordance with the results 
from Christensen et al., we now observe that unnormed non-terminal symbols 
corresponding to (un)normed types are (un)normed. These results follow 
immediately from the previous results. 

\begin{corollary}
	Given a context-free session type $S$, $|S| = |\text{\lstinline{toGrammar S}}|$.
\end{corollary}

\begin{corollary}
	A context-free session type $S$ is unnormed if and only if 
	$X_S$ is unnormed.
\end{corollary}

\subsection{The expansion tree is correct}
%
%Let us start by proving a small lemma, whose ultimate purpose 
%stands on proving that all nodes excluded by the filtering rule
%would lead to unsuccessful branches.
%
%\begin{lemma}
%\label{lemma:filtering}
%	Let $(\vec X, \vec Y)$ be a pair in node $N$ of the expansion tree. 
%	If $|\vec X| \neq |\vec Y|$ then  $\vec X \not\ProdEquiv \vec Y$.
%\end{lemma}
%
%\begin{proof}
%	Assume that $n = |\vec X| < |\vec Y|$. This means that:
%	\begin{equation}
%	\label{pathX}
%		\vec X \rightarrow \ell_1 \vec X_1 \rightarrow \cdots 
%		\rightarrow \ell_n \rightarrow \varepsilon.
%	\end{equation}
%	Now assume that $\vec Y$ has an expansion sequence whose 
%	labels coincide with those from $\vec X$ (otherwise, we would immediately
%	have $\vec X \not\ProdEquiv \vec Y$). Since $\vec Y > n$, there should 
%	exist a label $\ell_{n+1}$ such that:
%	\[\vec Y \rightarrow \ell_1 \vec Y_1 \rightarrow \cdots 
%	\rightarrow \ell_n \vec Y_n\rightarrow \ell_{n+1} \vec Y_n 
%	\rightarrow \cdots\]
%	Since our grammar is simple, \eqref{pathX} is the unique path from $\vec X$
%	through labels $\ell_1, \ldots, \ell_n$. Hence, the $(n+1)$-th expansion of
%	$\vec X$ with label $\ell_{n+1}$ would fail and we conclude that
%	$\vec X \not\ProdEquiv \vec Y$.
%\end{proof}

In this section we focus on the correctness of our algorithm, 
presented in Listing~\ref{lst:algorithm}. 
The \emph{safeness property} is paramount to prove soundness:

\begin{proposition} [Safeness Property~\cite{janvcar1999techniques}]
\label{prop:safeness}
	$\vec X \ProdEquiv \vec Y$ if and only if the expansion tree rooted 
	at $\{(\vec X, \vec Y)\}$ has a successful branch.
\end{proposition}

\begin{proof}
	The reflexive, congruence and \BPA\ rules were proved to 
	preserve the safeness property~\cite{janvcar1999techniques}.
%	On the other hand, as observed in~\cite{janvcar1999techniques},
%	the union of nodes along a successful branch is a relation $R$ 
%	such that $R\subseteq \ProdEquiv$. Hence, any pair $(\vec X, \vec Y)$ 
%	occurring along a successful branch is such that $\vec X \ProdEquiv \vec Y$,
%	which, by Lemma~\ref{lemma:filtering}, means that $|\vec X|=|\vec Y|$.
%	So, the filtering rule would node exclude any node in the successful branch
%	and, then, also preserved the safeness property.
\end{proof}

\begin{theorem}
	If Algorithm~\ref{lst:algorithm} returns \textsf{true} on input 
	$(S_1,S_2)$, then $X_{S_1} \ProdEquiv X_{S_2}$.
\end{theorem}

\begin{proof}
	The algorithm returns \textsf{true} on input $(S_1,S_2)$ whenever 
	it reaches a finite successful branch in the expansion tree rooted 
	at $\{(X_{S_1}, X_{S_2})\}$. Since all rules preserve the safeness 
	property, if $\{(X_{S_1}, X_{S_2})\}$ has a (finite) successful 
	branch then $X_{S_1} \ProdEquiv X_{S_2}$.
\end{proof}

Using Theorem~\ref{cfst_vs_grammar}, the soundness of our algorithm is 
immediate:

\begin{theorem}
	Algorithm~\ref{lst:algorithm} is sound with respect to the meta-theory 
	of context-free session types, i.e., if it returns \textsf{true} then $S_1 \TypeEquiv S_2$.
\end{theorem}

Having observed that the safeness property was paramount for soundness, 
we now notice that the \emph{finite witness property} is of utmost 
importance to prove completeness.

\begin{proposition} [Finite Witness Property~\cite{janvcar1999techniques}]
\label{finite_witness}
	If $\vec X \ProdEquiv \vec Y$, then the expansion tree rooted at 
	$\{(\vec X, \vec Y)\}$ has a finite successful branch.
\end{proposition}

\begin{proof}
	The reflexive, congruence and \BPA\ rules were proved to ensure 
	the finite witness property~\cite{janvcar1999techniques}. 
\end{proof}

\begin{theorem}
	Algorithm~\ref{lst:algorithm} is complete with respect to the meta-theory 
	of context-free session types, i.e., if $S_1 \TypeEquiv S_2$ then 
	the algorithm returns \textsf{true}.
\end{theorem}

\begin{proof}
	Assuming that $S_1 \TypeEquiv S_2$, by Theorem~\ref{cfst_vs_grammar}, we 
	have $X_{S_1} \ProdEquiv X_{S_2}$. Hence, the finite witness property 
	ensures the existence of a finite successful branch on the expansion 
	tree rooted at $\{(X_{S_1},  X_{S_2})\}$. Since our algorithm traverses 
	the expansion tree using breadth-first search we will, eventually, 
	reach the finite successful branch and conclude the equivalence positively.
\end{proof}
