\section{Optimisations}
\label{sec:optimisations}

Armed with the results in section~\ref{sec:algorithm}, we decided to
benchmark the algorithm on a test suite of carefully crafted pair of
types (more on this in section~\ref{sec:evaluation}). During this 
process we came across a pair of types,
\begin{equation}
\label{ex:chaotic}
\begin{aligned}
  S &\triangleq \mu x . \&\{ \mathsf{Add}\colon x;x; !\,\intk,
  \mathsf{Const}\colon ?\,\intk;!\intk,
  \mathsf{Mult}\colon x;x;!\,\intk\}
  \\
  T &\triangleq \mu x . \&\{ \mathsf{Add}\colon x;x,
  \mathsf{Const}\colon ?\,\intk,
  \mathsf{Mult}\colon x;x\}; !\,\intk
\end{aligned}
\end{equation}
%
on which function \lstinline|bisimilar| took 4379.98 seconds (that is
one hour and forty minutes) to terminate. This is certainly not a
reasonable running time for an algorithm to be included in a
compiler. Hence, we looked into ways to improve the running
times. Among the different optimisations that we tried, two stand
out:
\begin{enumerate}
\item Iterate the simplification stage until a fixed point is reached;
\item Use a double-ended queue where promising children are prepended
  rather than appended.
\end{enumerate}

If, on the one hand, we believed that the computation of the expansion
tree could be speeded up by extending the simplification phase, on the
other hand we suspected that a double-ended queue would allow
prioritizing nodes with potential to reach an empty node faster.  The
latter is obtained by prepending nodes that are already empty or,
either, whose pairs $(\vec X, \vec Y)$ are such that $|\vec X|\leq 1$
and $|\vec Y| \leq 1$.  For the former, we need to check that a fixed
point exists and, hence, for a given node $N$, we are able to compute
the \emph{simplest} children nodes derived from $N$ using the
simplification rules.


\begin{lstlisting}[
  caption={Haskell code for the improved simplification step (replaces
    function \lstinline|simplify| in Listing~\ref{lst:algorithm})},
  label={lst:enhanced},
  captionpos=b]
simplify :: Productions -> Node -> Ancestors -> NodeQueue -> NodeQueue
simplify ps n a q = foldr enqueueNode (Queue.dequeue q) nas
  where nas = findFixedPoint ps (Set.singleton (n,a))

enqueueNode :: (Node,Ancestors) -> NodeQueue -> NodeQueue
enqueueNode (n,a) q
 | maxLength n <= 1 = Queue.prepend (n,a) q
 | otherwise        = Queue.append (n,a) q

findFixedPoint :: Productions -> Set.Set (Node,Ancestors) -> 
                    Set.Set (Node,Ancestors)
findFixedPoint ps nas
  | nas == nas' = nas
  | otherwise   = findFixedPoint ps nas'
  where nas' = if allNormed ps
               then foldr (apply ps) nas [reflex,congruence,bpa2]
               else foldr (apply ps) nas [reflex,congruence,bpa1,bpa2]
\end{lstlisting}

\begin{theorem}
\label{thm:fixed_point}
	The simplification function that results from applying the reflexive,
	congruence, and \BPA\ rules, has a fixed point in the complete partial
	ordered set \lstinline{Set (Node,Ancestors)}, where the set of ancestors
	is supposed to be fixed and equal to $A$.
\end{theorem}

Throughout the proof of Theorem~\ref{thm:fixed_point}, we will abuse 
notation and denote the application of simplification rules
to nodes or to elements of $\text{\lstinline{Set (Node,Ancestors)}}$
in the same way, when no ambiguities arise.

\begin{proof}
	Consider the order 
	$\leqSets$ defined in $\text{\lstinline{Set (Node,Ancestors)}}
	\times \text{\lstinline{Set (Node,Ancestors)}}$ by:
	$S_1 \leqSets S_2$ if $|S_1| \leq |S_2|$ and
	exists an injective map $\sigma : S_1 \rightarrow S_2$ s.t.\  
	$\sigma(N_1,A) = (N_2,A)$ with $N_2\subseteq N_1$.\smallskip
	
	\noindent\textbf{$\leqSets$ is a partial order:} The proof that
	$\leqSets$ is reflexive and transitive is straightforward. 
	To prove that it is 
	antisymmetric, assume that $S_1\leqSets S_2$ and $S_2 \leqSets S_1$.
	This means that $|S_1|=|S_2|$ and, furthermore, the maps 
	$\sigma_1 : S_1 \rightarrow S_2$ and $\sigma_2 : S_2 \rightarrow S_1$
	are bijective. Notice that $\sigma_1\circ \sigma_2$  
	is the identity map, otherwise we could consider $(N,A)\in S_2$
	where $N$ is minimal w.r.t.\ inclusion and s.t.\
	$(\sigma_1\circ \sigma_2)(N,A) \neq (N,A)$, i.e.,
	$(\sigma_1\circ \sigma_2)(N,A) = (N',A)$ with $N'\subseteq N$ 
	for some $(N',A)\in S_2$;
	due to the minimality of $N$, we would have 
	$(\sigma_1\circ \sigma_2)(N',A) = (N',A)$, which would contradict the
	injectivity of $\sigma_1\circ \sigma_2$. Since
	$\sigma_1 (N,A) = (N',A)$ is such that $N'\subseteq N$, we shall 
	have $\sigma_1 (N,A) = (N,A)$. Hence, $S_1=S_2$.\smallskip
	
	\noindent\textbf{The simplification function is order-preserving:}
	To prove that the reflexive rule preserves the order, let $S_1$ and 
	$S_2$ be s.t.\ $S_1\leqSets S_2$ and let us prove that 
	$\text{\lstinline{reflex}} S_1\leqSets\text{\lstinline{reflex}}S_2$.
	%Start noticing that $|S| = |\text{\lstinline{reflex}} S|$, hence
	%the number of children nodes is preserved by using the reflexive
	%rule, let us analyze each case. 
	Let $(N,A)\in \text{\lstinline{reflex}} S_1$
	and notice that there exists $(N_1,A)\in S_1$, such that
	$\text{\lstinline{reflex}} N_1 = N $, and so,
	in $S_2$ there is $(N_2,A)=\sigma(N_1,A)$ s.t. $N_2\subseteq N_1$. 
	Since $N_2\subseteq N_1$, we have
	$\text{\lstinline{reflex}} N_2 \subseteq
    \text{\lstinline{reflex}} N_1 = N$.
	The same reasoning applies to prove that if $S_1\leqSets S_2$ then 
	$\text{\lstinline{congruence} } S_1\leqSets\text{\lstinline{congruence} }S_2$.
	
	To prove that \lstinline{bpa1} preserves the order,
	note that 
	$S \subseteq \text{\lstinline{bpa1} }S$. Assume that 
	$S_1\leqSets S_2$, let $(N,A)\in \text{\lstinline{bpa1} }S_1$, 
	and denote by $(N_1,A)\in S_1$ and $(\vec X,\vec Y)\in N_1$  
	the node and the pair whose simplification 
	led to $(N,A)$. We know that exists $(N_2,A)\in S_2$
	s.t.\ $N_2 \subseteq N_1$. If $(\vec X,\vec Y)\in N_2$,
	then the \lstinline{bpa1} simplification of $N_2$ with
	the pair $(\vec X,\vec Y)$ generates 
	$(N',A)\in \text{\lstinline{bpa1} }S_2$ such that 
	$N'\subseteq N$. On the other hand, if 
	$(\vec X,\vec Y)\not \in N_2$, then $N_2\subseteq N$ 
	and, since  $S_2 \subseteq \text{\lstinline{bpa1} }S_2$,
	$(N_2,A)\in \text{\lstinline{bpa1} }S_2$ is such that
	$N_2\subseteq N$.
	The same reasoning applies to \lstinline{bpa2}. 
	
	Having proved
	that each simplification function preserves the order, and
	since the simplification procedure results from the successive 
	application of these rules, we have proved that the simplification
	function also preserves the order.\smallskip
	
	\noindent\textbf{$(\text{\lstinline{Set (Node,Ancestors)}}, \leqSets)$
	is a lattice:} Given 
	$S_1, S_2\in \text{\lstinline{Set (Node,Ancestors)}}$,
	$S_1 \cup S_2$ is an upper bound and
	$S_1 \cap S_2$ is a lower bound for $S_1$ and $S_2$.\smallskip
	
	\noindent\textbf{$(\text{\lstinline{Set (Node,Ancestors)}}, \leqSets)$
	is a complete lattice:} Given 
	$\mathcal{B}\subseteq \text{\lstinline{Set (Node,Ancestors)}}$:
	$\bigcup_{S\in \mathcal{B}} S$ is an upper bound and
	$\bigcap_{S\in \mathcal{B}} S$ is a lower bound for the sets in 
	$\mathcal{B}$.
	
	Using the fixed point theorem from Tarski~\cite{tarski1955lattice}, 
	we conclude that the simplification function has a
	fixed point in $\text{\lstinline{Set (Node,Ancestors)}}$.
\end{proof}

The optimizations we propose aim to improve the performance of the algorithm, 
however the branching nature of the expansion tree promotes an exponential 
complexity, since each simplification step (potentially) generates a number of 
nodes polynomial in the size of input, and the same simplification phase may, in 
the worst case, be iterated a linear number of times in the size of the input.
Nevertheless, the heuristic we present seems to succeed very well in practice, as 
we will see in the next section.


%%% Local Variables:
%%% mode: latex
%%% TeX-master: "main"
%%% End:
