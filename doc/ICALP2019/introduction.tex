\section{Motivation}
\label{sec:introduction}

Session types enhance the expressivity of traditional types for
programming languages by enabling expressing structured communication
on heterogeneously typed
channels~\cite{DBLP:conf/concur/Honda93,DBLP:conf/esop/HondaVK98,DBLP:conf/parle/TakeuchiHK94}.
%
Traditional session types are \emph{regular}, in the sense that the
sequences of communication actions admitted by a type are in the union
of a regular language (for finite executions) and an $\omega$-regular
language (for infinite executions).
%
Introduced by Thiemann and Vasconcelos, context-free session types
liberate traditional session types from the shackles of tail
recursion, allowing for example, the safe serialization of arbitrary
recursive datatypes~\cite{thiemann2016context}.

If the algorithm aspects of the type equivalence for regular session
types are well known (Gay and Hole show authored an algorithm to
decide subtyting~\cite{DBLP:journals/acta/GayH05}, from which type
equivalence can be decided), the same does not apply to context-free
session types.

In the aforementioned work, Thiemann and Vasconcelos showed that the
equivalence of context-free session types is decidable, by reducing
the problem to the verification of bisimulation for Basic Process
Algebra (BPA) which, in turn, was proved decidable by Christensen,
H{\"{u}}ttel, and Stirling~\cite{DBLP:journals/iandc/ChristensenHS95}.
%
Even if the equivalence problem for context-free session types is
known to be decidable, to date, no algorithm has been proposed for the
effect.
%
Similarly, even if the problem of deciding the equivalence of BPA
terms has been addressed in the
literature~\cite{DBLP:journals/iandc/ChristensenHS95,janvcar1999techniques,thiemann2016context},
to the best of our knowledge, no practical algorithm was ever
proposed.

We present an algorithm to decide the equivalence of context-free
session types, practical to the point that it may be included in any
compiler, an exercise that we conducted in parallel~\cite{freeST}.
%
The main contributions of this work are:
%
\begin{itemize}
\item The proposal and implementation of an algorithm to decide type
  equivalence of context-free session types (in the process we also
  provide an algorithm to decide the equivalence of simple
  context-free grammars),
\item a proof of soundness and completeness of the algorithm against
  the declarative definition,
\item A study of the complexity of the algorithm,
\item The exploration of several optimizations that cut the running
  time in 12,000,000\%.
%\item validation of the algorithm on several meaningful examples.
\end{itemize}

The rest of the paper is organized as follows: context-free session
types in Section~\ref{sec:contextfreesession}, the algorithm in
Section~\ref{sec:algorithm}, the main results in
Section~\ref{sec:soundness}, optimizations in
Section~\ref{sec:optimisations}, evaluation in
Section~\ref{sec:evaluation}, and related work and conclusions in
Section~\ref{sec:conclusion}.

% Thiemann and
% Vasconcelos~\cite{thiemann2016context} proposed {\it context-free
%   session types} as an extension of session types by allowing nested
% protocols that are not restricted to tail recursion. Context-free
% session types capture the type-safe serialization of recursive
% datatypes and enable the type-safe implementation of remote operations
% on recursive datatypes.

% Inspired by the context-free session types' framework, Almeida and
% Vasconcelos~\cite{bernardo} proposed a functional programming language
% equipped with context-free session types.  Such a programming language
% is highly dependent on an algorithm to decide type equivalence. We
% have developed and implemented an algorithm to decide type
% equivalence. Our work capitalizes on the metatheory of context-free
% session types proposed by Thiemann and
% Vasconcelos~\cite{thiemann2016context}, where type equivalence was
% proved to be decidable. Although the decidability of equivalence on
% context-free session types has been addressed in the
% literature~\cite{DBLP:journals/iandc/ChristensenHS95,janvcar1999techniques,thiemann2016context},
% to the best of our knowledge, no algorithm was ever implemented.
% % and a possible implementation was not obvious.

%%% Local Variables:
%%% mode: latex
%%% TeX-master: "main"
%%% End:
