\section{Type equivalence algorithm}
\label{sec:algorithm}

% The main purpose of this work is to present and implement an algorithm to decide the type equivalence problem. In this section, we briefly describe our proposed algorithm.
We start by recalling the type equivalence problem for context-free 
session types.

\begin{quote}
  Given context-free session types $S$ and $T$, the type equivalence
  problem consists in deciding if types $S$ and $T$ are equivalent,
  i.e., $S \TypeEquiv T$.
\end{quote}

In this section we propose an algorithm to check the equivalence of 
context-free session types. This algorithm has three main stages: 
it starts by converting types into grammars, then streamlines the grammar
by pruning out unreachable symbols from the productions and, finally, 
explores an expansion tree, alternating 
between simplification and expansion operations, until either finding 
an empty node---case in which we decide equivalence positively---or 
failing to expand a node---case in which we decide equivalence negatively.

\subsection{Convert types to a grammar}
\label{subsec:typeToGrammar}

We will consider context-free session types as being represented by a 
finite set of productions, built upon the labelled transition system (\LTS)
presented in Figure~\ref{lts}. In this line, a context-free session type 
is seen as a finite set of productions of the form $X\rightarrow a \vec Y$, 
where $X$ range over a finite set of \emph{non-terminal symbols}, $\vec Y$ 
is a sequence of non-terminal symbols, and $a$ is a \emph{terminal symbol}
ranging over the set of labels in the \LTS, extended with an initial (dummy) 
label $!\,(\enspace)$. When no distinction is required, we refer to $X, Y, a$ 
simply as \emph{symbols}.

Given a context-free session type $S$, the set of productions $\mathcal{P}$
is defined recursively on the structure of $S$. We start by considering 
a non-terminal symbol $X_S$ to represent $S$ and let the set of productions 
$\mathcal{P}$ be initialized as the singleton with the initial production: 
\begin{equation}
\label{initial_prod}
	X_S \rightarrow \enspace \initialProd\enspace \mathsf{toGrammar}(\mathsf{freshVar}(\,),S) 	
\end{equation}
where $\mathsf{toGrammar}$ is defined recursively as presented 
in Algorithm~\ref{alg:toGrammar} and:
\begin{itemize}
	\item $\mathsf{freshVar}$ returns a fresh non-terminal variable;
	\item $\mathsf{addProduction}$ inserts the given production in $\mathcal{P}$;
	\item $\mathsf{insertVisited}$ marks a non-terminal symbol as visited;
	\item $\mathsf{isVisited}$ identifies whether the given non-terminal symbol was visited;
	\item $\mathsf{subst}(x,X,S)$ replaces the occurrences of $x$ by $X$ in $S$;
	\item $\vec X \vec Y$ represents the concatenation of the sequences of symbols $\vec X$ and $\vec Y$;
	\item $\varepsilon$ represents the empty sequence of symbols.
\end{itemize}

\begin{algorithm}
\caption{Algorithm to convert context-free session types into a grammar. Pseudocode.}
\label{alg:toGrammar}
\begin{algorithmic}
\\
\State \textsf{toGrammar} (\_, $\skipk$) = 
\State \qquad \Return{$\varepsilon$}\smallskip
\State \textsf{toGrammar} ($X$, $A$) = \Comment{$A$ ranges over $?\, \boolk$, $!\, \boolk$}
\State \qquad \textsf{addProduction} ($X \rightarrow A$)
\State \qquad \Return{$X$}\smallskip
\State \textsf{toGrammar} ($X$, $\alpha$) = 
\State \qquad\textbf{if} \textsf{isVisited} ($\alpha$) \textbf {then}
\State \qquad\qquad \Return{$\alpha$}
\State \qquad\textbf{else}
\State \qquad\qquad \textsf{addProduction} ($X \rightarrow \alpha$)
\State \qquad\qquad \Return{$X$}\smallskip
\State \textsf{toGrammar} ($X$, $\star \{\ell_i : S_i \}_{j\in I}$) =  \Comment{$\star$ ranges over $\oplus$, $\&$}
\State \qquad \textbf{for} {$j\in I$} \textbf{do} 
\State \qquad \qquad \textsf{addProduction} ($X \rightarrow \enspace \star \ell_j \enspace \mathsf{toGrammar}(\mathsf{freshVar}(\,), S_j)$)
\State \qquad \qquad \Return{$X$}\smallskip
\State \textsf{toGrammar} (\_, $S;T$) =
\State \qquad $\vec X_S$ $\gets$ \textsf{toGrammar} (\textsf{freshVar} (\,), $S$)
\State \qquad $\vec X_T$ $\gets$ \textsf{toGrammar} (\textsf{freshVar} (\,), $T$)
\State \qquad \Return{$\vec X_S \, \vec X_T$}\smallskip
\State \textsf{toGrammar} ($X$, $\mu x.S$) =
\State \qquad \textsf{insertVisited} $X$\smallskip
\State \qquad $\vec X_S$ $\gets$ \textsf{toGrammar} ($X$, $\mathsf{subst}(x,X,S)$)
\State \qquad \textbf{if} \textsf{null} $\vec X_S$ \textbf{then} 
\State \qquad \qquad \Return{$\varepsilon$}
\State \qquad \textbf{else} 
\State \qquad \qquad \Return{$X$}\smallskip
\end{algorithmic}
\end{algorithm}

\todo{AM}{example productions}


Notice that this recursive procedure always terminates and the resulting 
set of productions $\mathcal{P}$ is finite, because the recursion is always 
on subterms, except for recursive types, where we mark visited symbols, thus 
ensuring that the process terminates with a finite set of productions.

Due to the deterministic nature of the labelled transition system, context-free 
session types are always converted into \emph{simple grammars}~\cite{baeten1993decidability}, i.e., 
context-free grammars in Greibach normal form such that, for each non-terminal 
symbol $X$ and terminal symbol $a$, there is at most one production of the form 
$X\rightarrow a \enspace \vec Y$.

We notice that given two context-free session types $S$ and $T$, we can obtain 
a set of productions $\mathcal{P}$ for both, by ensuring that fresh variables
do not overlap.

The bisimulation relation, $\ProdEquiv$, is defined in the usual way from the 
set of productions. Throughout the text we drop the subscript $\mathcal{P}$ when 
it is clear from context.

\subsection{Prune unnormed productions}
\label{subsec:prune}

Using the same approach as~\cite{DBLP:journals/iandc/ChristensenHS95} on 
the definition of (un)normed processes, we say that a sequence of symbols 
$\vec Y$ is \emph{normed} if there are labels $\ell_1,\ldots, \ell_k$ 
such that:
\[\vec Y \rightarrow \ell_1\enspace Y_1 \rightarrow \cdots \rightarrow \ell_k 
\rightarrow \varepsilon.\]
$\vec Y$ is said to be \emph{unnormed} when it is not normed. If $\vec Y$ 
is normed, we define its \emph{norm} as: 
\[ | \vec Y | = \underset{k}{\mathsf{min}} \{\vec Y \rightarrow \ell_1\enspace Y_1 
\rightarrow \cdots \rightarrow \ell_k \rightarrow \varepsilon \}.\]

As observed by Christensen et al.~\cite{DBLP:journals/iandc/ChristensenHS95}, 
any unnormed sequence of symbols $\vec Y$ is equivalent to its concatenation 
with any other symbol:
\begin{equation}
\label{unnormed}
\text{ if } \vec Y \text{ is unnormed, then } \vec Y \sim \vec Y X.
\end{equation}

Upon the definition of the productions underlying the context-free session types,
our algorithm builds upon~\eqref{unnormed} to prune out unreachable symbols in
unnormed sequences of symbols in $\mathcal{P}$. The Haskell code 
is presented in Listing~\ref{lst:prune}.

\begin{lstlisting}[caption={Haskell code for the stage of pruning unnormed productions},label={lst:prune},captionpos=b]
type Transitions = Map.Map Label [TypeVar]
type Productions = Map.Map TypeVar Transitions

prune :: Productions -> Productions
prune p = Map.map (Map.map (pruneWord p)) p

pruneWord :: Productions -> [TypeVar] -> [TypeVar]
pruneWord _ [] = []
pruneWord p (x:xs)
  | normed p x = x : pruneWord p xs
  | otherwise  = [x]

normed :: Productions -> TypeVar -> Bool
normed p x = normedWord p Set.empty [x]

normedWord :: Productions -> Set.Set TypeVar -> [TypeVar] -> Bool
normedWord _ _ []     = True
normedWord p v (x:xs) =
  x `Set.notMember` v &&
  or (map (normedWord p v') (Map.elems (transitions p (x:xs))))
  where v' = if or $ map (x `elem`) (Map.elems (transitions p [x])) 
                then Set.insert x v else v
\end{lstlisting}

\todo{AM}{pruning example}

\subsection{Expansion Tree}
\label{subsec:expand}

We recall that, given two context-free session types $S$ and $T$, our main goal 
is to decide whether these types are equivalent or not. For this purpose, 
the algorithm we propose starts by applying Algorithm~\ref{alg:toGrammar} 
to convert $S$ and $T$ into a grammar containing the productions derived 
from them. Afterwards, the algorithm in Listing~\ref{lst:prune} is used to 
streamline the grammars, by pruning unnormed sequences of symbols.
Throughout this section we focus on the third and last stage of the algorithm.

We use the notion of \emph{expansion tree} proposed by Jan{\v{c}}ar 
and Moller~\cite{janvcar1999techniques} as as adaption of the former idea by 
Hirshfeld~\cite{hirshfeld1996bisimulation}. We say a set of pairs $N'$ is an 
\emph{expansion} of $N$ if $N'$ is a minimal set such that: for every pair 
$(\vec X, \vec Y) \in N$,
\begin{itemize}
	\item if $\vec X \rightarrow a \enspace\vec X'$ then $\vec Y \rightarrow 
		  a \enspace\vec Y'$ with $(\vec X',\vec Y')\in N'$;
	\item if $\vec Y \rightarrow a \enspace\vec Y'$ then $\vec X \rightarrow 
	      a \enspace\vec X'$ with $(\vec X',\vec Y')\in N'$.
\end{itemize}

Given a set of productions, an \emph{expansion tree} is composed by nodes 
labelled by pairs of sequences of symbols. The naive proposal for an
expansion tree considers that any children node is obtained by expansion 
from its parent node. Nevertheless, as Jan{\v{c}}ar and Moller 
observed, this would often lead to infinite expansion trees. Hence,
we follow the proposal in~\cite{janvcar1999techniques} and let the 
expansion tree alternate between simplification and expansion operations
until either finding an empty node---case in which we
decide equivalence positively---or failing to expand a node---case in
which we decide equivalence negatively.

We say that a branch
is \emph{successful} if it is infinite or finishes in an empty node, 
otherwise it is said to be \emph{unsuccessful}.

\paragraph{Expansion stage:}
In the expansion stage, each node $N$ derives a single children node, 
obtained as an expansion of $N$. As we are dealing with simple grammars, 
no branching is expected in the expansion tree at this stage.
Before proceeding to the simplification stage, the sequences of symbols 
labelling the children nodes are pruned, in accordance with~\eqref{unnormed}.

\paragraph{Simplification stage:} The simplification stage consists on the application of the following \emph{simplification rules}:
\begin{itemize}
	\item {\bf Reflexive rule:} omit from a node $N$ any reflexive pair;
	\item {\bf Congruence rule:}  omit from a node $N$ any pair that 
	      belongs to the least congruence containing the ancestors of $N$;
	\item {\bf Basic Process Algebra (\BPA) rules:} \todo{AM}{add details}
	\item {\bf Filtering rule:} remove any node containing a pair
	       $(\vec X, \vec Y)$ such that $|\vec X|\neq |\vec Y|$.
\end{itemize}

Notice that, contrarily to the remaining (simplification and expansion) operations, 
\BPA\ rules promote branching on the expansion tree. The number of children nodes generated by these rules is finite~\cite{DBLP:journals/iandc/ChristensenHS95}.

\todo{performance enhanced through fixed point approach}. 

Notice that the sibling nodes do not exclude the (often) infinite branch 
resulting from successive expansions.

\subsection{Algorithm to check the equivalence of context-free session types}

The algorithm to decide the equivalence of context-free session 
types capitalizes on the previous algorithms. To avoid getting stuck in an 
infinite branch of the expansion tree, we have implemented 
a breadth-first search on the expansion tree. Upcoming nodes are stored
in a queue.

Given two context-free session types \lstinline|t| and \lstinline|u|,
the algorithm sketched in Listing~\ref{lst:algorithm} returns \textsf{true}
if the types are equivalent and \textsf{false} otherwise. 
This algorithm capitalizes on the previous algorithms, is written in Haskell 
and should now be self explanatory.

\begin{lstlisting}[caption={Algorithm to check the equivalence of context-free session types. Haskell code.},label={lst:algorithm},captionpos=b]
type Node = Set.Set ([TypeVar], [TypeVar])
type Ancestors = Node
type NodeQueue = Queue.Queue (Node, Ancestors)
type NodeTransformation = Productions -> Ancestors -> Node -> Set.Set 

equivalent :: KindEnv -> Type -> Type -> Bool
equivalent k t u =
  isSessionType k t &&
  isSessionType k u &&
  expansionTree (prune p) [x] [y]
  where Grammar [x, y] p = toGrammar [t, u]

expansionTree :: Productions -> NodeQueue -> Bool
expansionTree g q
  | Queue.isEmpty q   = False
  | Set.null n        = True
  | otherwise         = case expandNode g n of
      Nothing  -> expansionTree' g (Queue.dequeue q)
      Just n' -> n' = = Set.fromList [([],[])] ||
                expansionTree g (simplify g (Set.union a n) 
                             (pruneNode g n') q)
  where (n,a) = Queue.front q

simplify :: Productions -> Ancestors ->  Node -> NodeQueue -> NodeQueue
simplify g a n q = foldr enqueueNode (Queue.dequeue q) m'
  where m' = findFixedPoint g (Set.singleton (n,a))

findFixedPoint :: Productions -> Set.Set (Node,Ancestors) -> Set.Set (Node,Ancestors)
findFixedPoint g nas
  | nas == nas' = nas
  | otherwise = findFixedPoint g nas'
  where nas' = 
      if allNormed g 
         then foldr (apply g) nas [reflex, congruence, bpa2, filtering]
         else foldr (apply g) nas [reflex, congruence, bpa1, filtering]
\end{lstlisting}


%Any algorithm must build upon some representation of the context-free
%session types. For this purpose, we recall that Jan{\v{c}}ar and
%Moller~\cite{janvcar1999techniques} noted that any process can be seen
%as a (possible infinite) process algebra graph (PA-graph).  Such a
%graph can be finitely represented by a finite set of productions
%$X \LTSderives \alpha$, where $X$ ranges over a finite set of
%\emph{term variables} $V$, $a$ ranges over the set of actions
%described above, and $\alpha$ ranges over the terms of the free
%algebra over $V$ generated by the semi-colon operator.
%The vertices of the PA-graph are then represented by terms~$\alpha$ of
%the algebra, whereas the edges are given by the production rules.
%
%We build on the notion of expansion trees, only that in our case we
%can work with \emph{expansion lists}, given the deterministic nature
%of transitions in context-free session types.
%% \footnote{PA-graphs for context-free session types are such that,
%%   given a variable $X$ and an action $a$, we cannot derive two
%%   distinct terms in the algebra.}
%%
%Nodes in this list are sets of pairs of terms~$\alpha$.
%%
%Given two session types $S$ and $T$, the first step of the algorithm
%builds the PA-graph and the initial node $\{(X,Y)\}$, where~$X$ is the
%term variable corresponding to~$S$ and~$Y$ to~$T$. This step further
%normalizes $\alpha$-terms by removing terms preceded by unnormed
%terms, given that $\alpha \sim \alpha\beta$ whenever $\alpha$ is
%unnormed~\cite{janvcar1999techniques}.
%%
%The algorithm then alternates between simplification and expansion
%operations, until either finding an empty node---case in which we
%decide equivalence positively---or failing to expand a node---case in
%which we decide equivalence negatively.
%
%The simplification procedure builds on  Jan{\v{c}}ar and
%Moller rules, namely reflexivity and BPA1.
%%
%The algorithm, written in Haskell, should be self explanatory.
%%
%\begin{lstlisting}
%bisim :: Graph -> Ancestors -> Node -> Bool
%bisim g a n
% | Set.null n' = True
% | otherwise   = case expandNode g n' of
%    Nothing  -> False
%    Just n'' -> bisim g (Set.union n' a) n''
% where n' = simplify g a n
%
%simplify :: Graph -> Ancestors -> Node -> Node
%simplify g a n = foldr (apply g a) n [reflex, bpa1]
%
%apply :: Graph -> Ancestors -> NodeTransformation -> Node -> Node
%apply g a trans = Set.foldr (\p n -> Set.union (trans g a p) n) Set.empty
%
%type Graph = Map.Map TermVar (Map.Map Label [TermVar])
%type Node = Set.Set ([TermVar], [TermVar])
%type Ancestors = Node
%type NodeTransformation = Graph -> Ancestors -> ([TermVar],[TermVar]) -> Node
%\end{lstlisting}
%
%
%We showcase this procedure on a couple of examples. %Details can be
%% found in the Appendix.
%%
%% \begin{example}
%%   \label{ex:example1}
%Consider two types $S$ and $T$:
%%
%\begin{align*}
%  S &\triangleq \mu x. (\oplus\{\ell\colon !\,\intk; \&\{\ell\colon x,
%      m\colon \skipk \} , m\colon !\, \boolk; \&\{\ell\colon x,
%      m\colon \skipk \} \}); \mu x. (\&\{\ell\colon\skipk, m\colon
%      ?\,\boolk; x \})
%  \\
%  T &\triangleq \mu x. (\oplus\{\ell\colon !\,\intk , m\colon !\,
%      \boolk\}; \&\{\ell\colon x, m\colon \mu
%      y. (\&\{\ell\colon\skipk, m\colon ?\,\boolk; y \})\})
%\end{align*}
%%
%The associated PA-graph is given by the following productions, from the initial node $\{(X,Y)\}$:
%\begin{center}
%\begin{tabular}{ |l l |l l| }
%  \hline
%  \multicolumn{4}{|c|}{Productions for $S$ and $T$} \\
%  \hline
%  \hline
%\multicolumn{2}{|c|}{$S$}&  \multicolumn{2}{|c|}{$T$} \\ \hline
%$X \xrightarrow{\oplus \ell} X_1 X_2 X_5$ & $X_4 \xrightarrow{\& \ell} X$ & $Y \xrightarrow{\oplus \ell} Y_1 Y_3$ & $Y_3 \xrightarrow{\& m} Y_4$ \\
%$X \xrightarrow{\oplus m} X_3 X_4 X_5$& $X_4 \xrightarrow{\& m} \varepsilon$ & $Y \xrightarrow{\oplus m} Y_2 Y_3$ & $Y_4 \xrightarrow{\& \ell} \varepsilon$\\
%$X_1 \xrightarrow{!\, \intk} \varepsilon$& $X_5 \xrightarrow{\& \ell} \varepsilon$& $Y_1 \xrightarrow{!\intk} \varepsilon$ & $Y_4 \xrightarrow{\& m} Y_5 Y_4$\\
%$X_2 \xrightarrow{\& \ell} X$& $X_5 \xrightarrow{\& m} X_6 X_5$& $Y_2 \xrightarrow{!\boolk} \varepsilon$ & $Y_5 \xrightarrow{?\,\boolk} \varepsilon$\\
%$X_2 \xrightarrow{\& m} \varepsilon$& $X_6 \xrightarrow{?\,\boolk} \varepsilon$& $Y_3 \xrightarrow{\& \ell} Y$ &\\
%$X_3 \xrightarrow{!\, \boolk} \varepsilon$&&&\\
%  \hline
%\end{tabular}
%\end{center}
%
%%
%\noindent
%We get the expansion list below for $S\sim T$.  \smallskip
%\par\noindent
%\begin{tikzcd}[cells={nodes={draw=black}}, column sep=large]
%   (X,Y)  \ar[r,"expand"]
%&   (X_1X_2X_5,Y_1Y_3), (X_3X_4X_5, Y_2Y_3)  \ar[r,"expand"]
%&   (X_2X_5,Y_3), (X_4X_5, Y_3)   
%\end{tikzcd} 
%$\xrightarrow{\mathit{expand}}$\\\\
%\begin{tikzcd}[cells={nodes={draw=black}}, column sep=large]
%& (XX_5,Y),(X_5,Y_4) \ar[r,dashed,"bpa1"]
%& (X_5,\varepsilon),(X_5,Y_4)  \ar[r,"expand"]
%&(X_6X_5,Y_5Y_4),(\varepsilon,\varepsilon) 
%\end{tikzcd} 
%$\xdashrightarrow{\enspace\mathit{reflex}\enspace}$ \\\\
%\begin{tikzcd}[cells={nodes={draw=black}}, column sep=large]
%& (X_6X_5,Y_5Y_4) \ar[r,"expand"]
%& \enspace(X_5,Y_4)\enspace \ar[r,dashed,"bpa1"]
%&\emptyset
%\end{tikzcd}
%\enspace  $\checkmark$\\\smallskip
%
%Now consider the types $R$ and $U$:
%%
%\begin{align*}
%  R \triangleq \mu x.\&\{\ell\colon ?\,\boolk;x, m\colon ?\,\intk;x;x\}
%      &&
%  U \triangleq \mu x.\&\{\ell\colon ?\,\boolk, m\colon ?\,\intk;x;x\}
%\end{align*}
%%
%The associated PA-graph is given by the following productions, from the initial node $\{(Z,W)\}$:
%\begin{center}
%\begin{tabular}{ |l l |l l| }
%  \hline
%  \multicolumn{4}{|c|}{Productions for $R$ and $T$} \\
%  \hline
%  \hline
%\multicolumn{2}{|c|}{$R$}&  \multicolumn{2}{|c|}{$U$} \\ \hline
%$Z \xrightarrow{\& \ell} Z_1 Z$ & $Z_1 \xrightarrow{?\, \boolk} \varepsilon$ &$W \xrightarrow{\& \ell} W_1$ & $W_1 \xrightarrow{?\, \boolk} \varepsilon$\\
%$Z \xrightarrow{\& m} Z_2 Z Z$ & $Z_2 \xrightarrow{?\, \intk} \varepsilon$& $W \xrightarrow{\& m} W_2 W W$ & $W_2 \xrightarrow{?\, \intk} \varepsilon$\\
%  \hline
%\end{tabular}
%\end{center}
%%
%The expansion list is as follows:\smallskip
%\par\noindent
%\begin{tikzcd}[cells={nodes={draw=black}}, column sep=large]
%  (Z,W) \ar[r,"expand"]
%& (Z_1 Z,W_1), \atop(Z_2 Z Z, W_2 W W) \ar[r,"expand"]
%& (Z,\varepsilon), (ZZ, WW) \ar[r,dashed,"bpa1"] 
%& (Z,\varepsilon), (W,\varepsilon) 
%\end{tikzcd} 
%\enspace  $\times$\smallskip
%
%Our algorithm uses the simplification rules of  Jan{\v{c}}ar and
%Moller~\cite{janvcar1999techniques} and implements the derivation flow on
%a simplified version of PA graphs. For these reasons, we
%believe that our implementation is sound. Furthermore, we have tested
%our algorithm in more than one hundred examples and we have not
%obtained any false positive nor any true negative result. This
%reinforces our conviction that this algorithm should be sound (and
%possibly complete).

%%% Local Variables:
%%% mode: latex
%%% TeX-master: "main"
%%% End:
