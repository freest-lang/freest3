\section{Complexity Analysis}
\label{sec:complexity}

Throughout this section, we carry out an analysis of the
improved version of algorithm \lstinline{bisimilar},
that involves the algorithms from Listings~\ref{lst:toGrammar},
\ref{lst:prune}, \ref{lst:algorithm}, and the improved version
of the simplification stage in Listing~\ref{lst:enhanced}.


\todo{AM}{complexity analysis for toGrammar (and prune)}

Despite the great branching ability of this expansion tree, 
we will prove throughout this section that it has polynomial 
size on the length of the root. For this, consider $X_S$ and
$X_T$ the initial non-terminal symbols for $S$ and $T$, let 
$\mathcal{P}$ denote the set of productions for $S$ and $T$ and 
also:
\begin{itemize}
	\item $S$ represent the maximum size of right-hand size of
		  the productions in $\mathcal{P}$;
	\item $L$ stand for the maximum number of labels within the choice 
		  operators in $\mathcal{P}$;
	\item $N_0^+$ and $N_0^-$ represent the maximum and minimum values
		  of the norms of $X_S$ and $X_T$, respectively, where we 
		  assume $|X|=\infty$ whenever $X$ is unnormed.
\end{itemize}

We will analyze the complexity of each simplification or expansion 
step in the expansion tree. For this, we now fix some notation, to 
refer to the number ``elements'' (read: nodes, pairs, length, etc.)
at the $i^{th}$ level of the expansion tree:
\begin{itemize}
	\item $a_i$ denotes the number of ancestors;
	\item $n_i$ denotes the number of nodes;
	\item $p_i$ refers to the maximum number of pairs in each node
				at level $i$;
	\item $s_i^+$ and $s_i^-$ denotes the maximum and minimum 
		  number of symbols in the sequences composing each pair;
	\item $\rho_i^+$ and $\rho_i^-$ denotes the maximum and minimum
		  norms occurring among all pairs.
\end{itemize}

Observe that in the root we have: $a_0=0$, $n_0=1$, $p_0=1$,
$s_0^+=s_0^-=1$, $\rho_0^+=N_0^+$, $\rho_0^-=N_0^-$. Now assume 
that at the $i^{th}$ level we succeed on an expansion step to
level $i+1$;
we would have: $a_{i+1}=a_1+1$, $n_{i+1}=n_i$, $p_{i+1}=p_i\cdot L$,
$s_{i+1}^+=s_i^++S$, $s_{i+1}^-=s_i^-+S$, $\rho_{i+1}^+=\rho_i^+-1$ 
is $\rho_i^+$ is finite and  $\rho_{i+1}^+ = \infty$ otherwise.

Similarly, notice that when applying \lstinline{bpa2} from level $i$
to level $i+1$, we have: $a_{i+1}=a_i$, $n_{i+1} = n_i\cdot p_i + n_i$,
$p_{i+1}=p_i+1$, $s_{i+1}^+=s_i^+$, $s_{i+1}^-=1$, 
$\rho_{i+1}^\pm\leq \rho_i^\pm$ if $\rho_i^\pm$ are finite. 
Finally, when applying \lstinline{bpa1} from level $i$
to level $i+1$, we have: $a_{i+1}=a_i$, $n_{i+1} = p_i \cdot a_i+n_i$,
$p_{i+1}=p_i+1$, $s_{i+1}^- \leq s_i^- - 1$.

Before proceeding, just notice that \lstinline{reflex} and 
\lstinline{congruence} do not affect the complexity analysis, hence 
we will omit them from this analysis.
So far, we have made a simple analysis on the complexity of each 
simplification and expansion operation, and we conclude that
these operations promote a polynomial increase on the size of the
expansion tree. 

However, the proposed algorithm, alternates between expansion 
operations and iterations of the simplification phase until 
reaching a fixed point. We will carry out this analysis by 
splitting into two cases: the case in which all symbols 
in the productions are normed, and the case where some
of them are unnormed.\\

\noindent\textbf{Case 1:} In the simplest case where all symbols
occurring in $\mathcal{P}$ are normed, the expansion stage should 
terminate because at each iteration step, from level $i$ to level
$i+1$, we have $\rho_{i+1}^+ < \rho_i^+$. Furthermore, in the 
simplification stage, \lstinline{bpa2} also promotes the reduction 
in the maximum norm, hence the expansion procedure should terminate.
On the other hand, and since the application of \lstinline{bpa2} from 
level $i$ to level $i+1$ also ensures that $\rho_{i+1}^+ < \rho_i^+$,
the simplification stage can only iterate at most 
$(n_i\cdot p_i)\cdot (n_{i+1}\cdot p_{i+1})\ldots (n_{i+\rho_i^+}\cdot
p_{i+\rho_i^+})$ times. Still, the number of nodes in $n_{i+\rho_i^+}$, 
the number of pairs $p_{i+\rho_i^+}$ are all of polynomial size on the 
size of the root.\\

\noindent\textbf{Case 2:} 





























