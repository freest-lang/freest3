\section{Complexity Analysis}
\label{sec:complexity}

\todo{AM}{complexity analysis for toGrammar (and prune)}

\subsection{Complexity analysis}

Now that we have proved the correctness of the algorithm,
we proceed with an analysis on the complexity of the algorithm.
Due to the branching nature of the expansion tree, 
there is no hope on obtaining a polynomial bound.
However, we can prove that the finite witness has polynomial size on the 
length of the context-free session types $S$ and $T$, whenever
$S\TypeEquiv T$. 
And, thus, we can conclude that the problem of deciding the equivalence 
of context-free session types is in \textsf{NP}.

With this purpose, assume that $S\TypeEquiv T$, and let $\pi_0$
denote a minimal (finite) successful branch in the expansion tree
(it exists, by theorems~\ref{thm:typeToProd} and \ref{prop:finite_witness}).
Consider $X_S$ and $X_T$ the initial non-terminal symbols for 
$S$ and $T$, let $\mathcal{P}$ denote the set of productions for 
$S$ and $T$ and also:
\begin{itemize}
	\item $G$ represent the maximum number of symbols in the RHS
	      of productions in $\mathcal{P}$;
	\item $C$ stand for the maximum number of labels within the choice 
		  operators in $\mathcal{P}$;
	\item $N_0^+$ and $N_0^-$ represent the maximum and minimum values
		  of the norms of $X_S$ and $X_T$, respectively, where we 
		  assume $|X|=\infty$ whenever $X$ is unnormed.
\end{itemize}

Given a node $N$ in the finite successful branch, with $P_N$ pairs,
maximum length $L_N$ among the pairs, notice that the size of $N$,
$|N|$, might be written as $|N|\leq 2\cdot P_N \cdot L_N$. Furthermore,
let $\rho_N$ denote the maximum norm within the pairs at node $N$,
and assume the maximum
length of all pairs within the ancestors is given by $L_A$. Applying
the operations of the successful branch at node $N$ we have:
\begin{itemize}
\item an expansion operation replaces $N$ by $N'$, where
	  $p_{N'}=C\cdot P_N$, $L_{N'}=L_N+G$ and, thus,
	  $|N'|\leq 2\cdot(C\cdot |N|)\cdot (L_N+G) \sim \mathcal{O}(|N|^3)$;
\item \lstinline|reflex| and \lstinline|congruence| replace $N$ by $N'$, 
	  where $|N'|\leq|N|$;
\item \lstinline|bpa1| replaces $N$ by $N'$, where
	  $p_{N'}=p_N+1$, $L_N=\max\{L_N,L_A\}$ and, thus,
	  $|N'|\leq 2\cdot(P_N+1)\cdot L_N\sim \mathcal{O}(|N|)$;
\item \lstinline|bpa2| replaces $N$ by $N'$, where
	  $p_{N'}=p_N+1$, $L_N\leq \max\{L_N-1,\rho_N-1\}$ and, thus,
	  $|N'|\leq 2\cdot(P_N+1)\cdot L_N\sim \mathcal{O}(|N|)$;
\end{itemize}

This means that, the simplification and expansion operations in the successful
branch only lead to a cubic increase in the size of the nodes. On
the other hand, we now conclude that the number of expansions and
simplifications along the successful branch is polynomial in the 
size of the input and, hence, this branch as a polynomial size.

Christensen et al.~\cite{DBLP:journals/iandc/ChristensenHS95} proved
that at the application of \BPA\ rules, the choice of the minimal branch
would ensure a strict decrease on the size of the pairs, where the 
size of a pair $(\vec X, \vec Y)$ was defined as 
$\mathsf{size}(\vec X, \vec Y) = \max\{s(\vec X), s(\vec Y)\}$, 
where for an unnormed $X_0$, $s(\vec X X_0) = s(\vec X ) = |\vec X|$. 

Hence, we have two cases to analyse:

\noindent\textbf{Case 1:} In the simplest case where all symbols
occurring in $\mathcal{P}$ are normed, the expansion stage 
promotes a strict decrease on the norm of the node and, hence,
the expansion stage should terminate in, at most, $N_0^+$ steps. 
Similarly, the application of \lstinline{bpa2} 
on a given node $N$ iterates at most 
$P_N\cdot \rho_N$. 
 \smallskip

\noindent\textbf{Case 2:} In the case where there are unnormed symbols
in the productions, the \BPA\ rules also apply to a node $N$ at most
$P_N\cdot \rho_N$.
As shown by Christensen et 
al.~\cite{DBLP:journals/iandc/ChristensenHS95}.
For a given pair $(X_0\vec X, Y_0,\vec Y)$:
\begin{enumerate}
	\item \label{bullet:un,un}  if $X_0$ and $Y_0$ are unnormed, then 
		  $\vec X = \varepsilon$
		  and $\vec Y = \varepsilon$ and the conclusion on its equivalence
		  follows right after the next expansion operation;
	\item if $X_0$ is unnormed and $Y_0$ is normed, then $\vec X = \varepsilon$
		  and the procedure will either terminate as soon as
		  \begin{itemize}
		  	\item $Y_0\vec Y\LTSderives[\vec a] \varepsilon$, and takes 
		  		  $|Y_0\vec Y|$ expansion steps,
		  	\item $Y_0\vec Y\LTSderives[\vec a] Y_0\vec Y'$, case in which
		  		  there will be a successful branch using 
		  		  \lstinline{bpa1}~\cite{DBLP:journals/iandc/ChristensenHS95}, or
		  	\item $Y_0\vec Y$ reaches an unnormed variable 
		  		  (reduces to~\ref{bullet:un,un}.);
		  \end{itemize}
	\item if $X_0$ and $Y_0$ are normed, let 
		  $\rho^* = \min \{|X_0|, |Y_0|\}$. After $\rho^*$ expansion
		  operations, we will reach a pair $(X_0'\vec{X'}, Y_0',\vec{Y'})$
		  where $X_0$ is unnormed and $Y_0$ is (un)normed (or vice-versa)
		  and are reduced to the previous cases.
\end{enumerate}


We conclude that, in both cases, a minimal finite successful branch would
have size $\mathcal{O}(|N|^4)$. The following result summarizes this
analysis.

%We start by analyzing the complexity of each simplification or expansion 
%step in the expansion tree. For this, we now fix some notation, to 
%refer to the number ``elements'' (read: nodes, pairs, length, etc.)
%at the $i^{th}$ level of the expansion tree:
%\begin{itemize}
%	\item $a_i$ denotes the number of ancestors;
%	\item $n_i$ denotes the number of nodes;
%	\item $p_i$ refers to the maximum number of pairs in each node
%				at level $i$;
%	\item $s_i^+$ and $s_i^-$ denotes the maximum and minimum 
%		  number of symbols in the sequences composing each pair;
%	\item $\rho_i^+$ and $\rho_i^-$ denotes the maximum and minimum
%		  norms occurring among all pairs.
%\end{itemize}
%
%Observe that in the root we have: $a_0=0$, $n_0=1$, $p_0=1$,
%$s_0^+=s_0^-=1$, $\rho_0^+=N_0^+$, $\rho_0^-=N_0^-$. And now, let us proceed
%with this analysis by exploring the structure of the expansion tree. Assume 
%that at the $i^{th}$ level we succeed on an expansion step to
%level $i+1$;
%we would have: $a_{i+1}=a_i+n_i$, $n_{i+1}\leq n_i$, 
%$p_{i+1}\leq p_i\cdot L$,
%$s_{i+1}^+\leq s_i^++S$, $s_{i+1}^-\leq s_i^-+S$, $\rho_{i+1}^+=\rho_i^+-1$ 
%if $\rho_i^+$ is finite and  $\rho_{i+1}^+ = \infty$ otherwise.
%
%Similarly, notice that when applying \lstinline{bpa2} from level $i$
%to level $i+1$, we have: $a_{i+1}=a_i$, $n_{i+1} \leq n_i\cdot p_i + n_i$,
%$p_{i+1} \leq p_i+1$, $s_{i+1}^+=s_i^+$, $s_{i+1}^-=1$, 
%$\rho_{i+1}^\pm\leq \rho_i^\pm$ if $\rho_i^\pm$ are finite. 
%Finally, when applying \lstinline{bpa1} from level $i$
%to level $i+1$, we have: $a_{i+1}=a_i$, $n_{i+1} \leq p_i \cdot a_i+n_i$,
%$p_{i+1}\leq p_i+1$, $s_{i+1}^- \leq s_i^- - 1$.
%Before proceeding, just notice that \lstinline{reflex} and 
%\lstinline{congruence} do not affect the complexity analysis, hence 
%we will omit them from this analysis.

%So far, we have made a simple analysis on the complexity of each 
%simplification and expansion operations, and we conclude that
%these operations promote a polynomial increase on the size of the
%expansion tree. 
%However, the proposed algorithm, alternates between expansion 
%operations and iterations of the simplification phase until 
%reaching a fixed point. We will carry out this analysis by 
%splitting into two cases: the case in which all symbols 
%in the productions are normed, and the case where some
%of them are unnormed.\smallskip

\begin{theorem}
	Whenever $S\TypeEquiv T$, the finite branch as polynomial length 
	on the size of $S$ and $T$.
\end{theorem}

We can now formulate the main complexity result:

\begin{theorem}
	The algorithm to check the equivalence of context-free 
	session types presented in Listings~\ref{lst:toGrammar},
	\ref{lst:prune}, \ref{lst:algorithm}
	is in \textsf{NP}.
\end{theorem}


























